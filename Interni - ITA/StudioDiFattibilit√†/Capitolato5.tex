\documentclass[StudioDiFattibilità.tex]{subfiles}
\begin{document}
\chapter{Capitolato C5 - IronWorks}
\section{Descrizione generale}
Il capitolato C5 propone lo sviluppo di un software
di costruzione di diagrammi UML di “robustezza” con la relativa generazione di codice Java.
\section{Obiettivo finale}
L'obiettivo finale consiste nella creazione di un'applicazione in grado di disegnare i diagrammi UML di "robustezza"(robustness diagram) e di generatore il codice nel linguaggio di programmazione Java del diagramma creato. Permettendo cosi di aumentando la velocità di produzione e la qualità del software prodotto.
\section{Tecnologie richieste}
\begin{itemize}
	\item \textbf{HTML}, \textbf{CSS}, \textbf{Javascript}, per il lato client;
	\item \textbf{Java o Javascript}, per il lato server.
\end{itemize}
\section{Valutazione finale}
Il capitolato è stato accolto positivamente dai membri del gruppo 353, poiché non sembrava particolarmente impegnativo ed era interessante l'idea di poter avere una base di partenza "standardizzata" per la fase di avvio dello sviluppo di nuove applicazioni. Il capitolato non è stato scelto in quanto non presenta tecnologie innovative che possono fornire nuove capacità e conoscenze.

\end{document}