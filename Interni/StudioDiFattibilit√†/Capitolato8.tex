\documentclass[StudioDiFattibilità.tex]{subfiles}
\begin{document}
\chapter{Capitolato C8 - TuTourSelf}
\section{Descrizione generale}
Il capitolato C8 propone lo sviluppo di un web app denominata TuTourSelf che permette agli artisti indipendenti di tutto il mondo di organizzare in poco tempo il proprio tour comunicando direttamente con i locali disponibili.
\section{Obiettivo finale}
L’obiettivo è la realizzazione di questo portale web che possa permettere la creazione di una community di artisti e locali/spazi nell'intento di rendere l’accordo tra le due parti semplice, rapido e sicuro. L'app dovrà permettere all'artista la ricerca e la selezione dei locali disponibili ad ospitare la sua performance secondo una data e un'ora stabilita, gestendo l’organizzazione dell'evento con il gestore del locale e il pagamento finale.
\section{Tecnologie richieste}
\begin{itemize}
	\item \textbf{HTML}, \textbf{CSS}, \textbf{Javascript}: stack standard per lo sviluppo frontend;
	\item \textbf{React}: libreria open source \citGloss{JavaScript} per la creazione di interfacce grafiche e la gestione delle interazioni in ambito web.
\end{itemize}
\section{Valutazione finale}
Il gruppo 353 ha valutato positivamente la proposta poiché il progetto non sembrava particolarmente impegnativo ed era stato posto in un bel contesto applicativo. Inoltre era allettante l'idea di collaborare con una giovane start up e con una particolare libertà nella scelta delle tecnologie per lo sviluppo \citGloss{back-end}.\\
Dopo un'attenta analisi è stato deciso di scartare questo capitolato perché il progetto non era molto stimolante rispetto alle altre proposte, in quanto consiste nello sviluppo di un applicativo simile a molti altri già presenti nel mercato e l'unica nuova tecnologia acquisita sarebbe stata \citGloss{React}.
\end{document}