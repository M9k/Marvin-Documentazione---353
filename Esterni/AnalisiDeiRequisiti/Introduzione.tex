\documentclass[AnalisiDeiRequisiti.tex]{subfiles}

\begin{document}

\chapter{Introduzione}
\section{Scopo del documento}
Il presente documento ha come scopo quello di fornire una descrizione completa e precisa di tutti i \citGloss{requisiti} individuati e dei casi d'uso riguardanti il progetto Marvin.\\
Tali informazioni sono state estrapolate dal capitolo, dagli incontri tra il gruppo \gruppo\ e dai verbali concessi da \Proponente.

\scopoProdotto

\glossExpl

\section{Riferimenti}

\subsection{Riferimenti normativi}

\begin{itemize}
	\item \textbf{\ndp \vruno};
	\item \textbf{Capitolato d'appalto C6}\\ \nURI{www.math.unipd.it/~tullio/IS-1/2017/Progetto/C6.pdf};
	\item \textbf{Verbale di incontro esterno} \textit{"VER-2017-11-22"} avvenuto tra i componenti del gruppo e il proponente Red Babel in data 22 novembre 2017 attraverso la piattaforma Slack;
	\item \textbf{Verbale di incontro esterno} \textit{"VER-2017-12-08"} avvenuto tra i componenti del gruppo e il proponente Red Babel in data 8 dicembre 2017 attraverso la piattaforma Skype.
	%TODO: completare
\end{itemize}

\subsection{Riferimenti informativi}
\begin{itemize}
	\item \textbf{Presentazione del capitolato C6}\\ \nURI{www.math.unipd.it/~tullio/IS-1/2017/Progetto/C6p.pdf};
	\item \textbf{Lucidi didattici utilizzati durante il corso di Ingegneria del Software}\\ \nURI{www.math.unipd.it/~tullio/IS-1/2017/}
	\begin{itemize}
		\item Diagrammi dei casi d'uso;
		\item Analisi dei requisiti.
	\end{itemize}
\end{itemize}

\end{document}
