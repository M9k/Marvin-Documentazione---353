\documentclass[AnalisiDeiRequisiti.tex]{subfiles}

\begin{document}

\chapter{Introduzione}
\section{Scopo del documento}
Il presente documento ha come scopo quello di fornire una descrizione completa e precisa di tutti i requisiti individuati e dei casi d'uso riguardanti il progetto Marvin.\\
Tali informazioni sono state estrapolate dal capitolo, dagli incontri tra il team 353 e dai verbali concessi da Red Babel.

\section{Scopo del prodotto}
Lo scopo del prodotto è la creazione di una DApp accessibile da interfaccia web ed utilizzabile assieme al plugin MetaMask che consenta una gestione di un sistema che simuli le funzionalità base di UniPD.\\
Il front-end del sistema consisterà in una applicazione web realizzata con React e Redux, mentre il back-end sarà composto da un insieme di smart contract scritti in linguaggio Solidity che andranno a girare su una rete Ethereum.\\

\section{Riferimenti}

\subsection{Riferimenti normativi}

\begin{itemize}
	\item \textbf{Norme di progetto} v1.0.0;\\
	\item \textbf{Capitolato d'appalto} \textit{"Marvin: dimostratore di Uniweb su Ethereum"}, disponibile alla consultazione attraverso il link web \url{www.math.unipd.it/~tullio/IS-1/2017/Progetto/C6.pdf};\\
	\item \textbf{Verbale di incontro esterno} \textit{"Verbale171122"} con i componenti del gruppo e il proponente Red Babel avvenuto in data 22 novembre 2017.
	%TODO: completare
\end{itemize}

\subsection{Riferimenti informativi}
\begin{itemize}
	\item \textbf{Presentazione del capitolato} \textit{"C6p.pdf"}, disponibile  alla consultazione attraverso il link web \url{www.math.unipd.it/~tullio/IS-1/2017/Progetto/C6p.pdf};
	\item \textbf{Lucidi didattici utilizzati durante il corso di Ingegneria del Software}, disponibili alla consultazione attraverso il link web \url{www.math.unipd.it/~tullio/IS-1/2017/}.
\end{itemize}

\end{document}