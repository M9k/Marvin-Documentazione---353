\documentclass[StudioDiFattibilità.tex]{subfiles}
\begin{document}
\chapter{Capitolato C6 - Marvin}
\section{Descrizione generale}
Il capitolato C6 si pone l'obbiettivo di sviluppare una piattaforma web che simuli le funzionalità base offerte da \citGloss{Uniweb}, utilizzando al posto del database la rete \citGloss{Ethereum} attraverso l'uso di \citGloss{smart contract}.\\
L'iterazione con il sistema avviene da parte dell'università stessa, dei professori e degli alunni e dovrà essere possibile l'aggiunta di corsi, l'assegnazione dei professori, la pubblicazione degli esiti degli esami e l'accettazione o meno dei voti.
\section{Obiettivo finale}
L'obbiettivo finale è la creazione di un sito web che possa accedere attraverso il plugin Metamask, disponibile per Google \citGloss{chrome} e Mozilla \citGloss{firefox}, alla rete Ethereum ed interagire con smart contract al fine di simulare alcuni aspetti della piattaforma Uniweb riguardanti i corsi, gli esami e l'accettazione dei voti.\\
Tutta la business logic verrà eseguita sulla rete Ethereum, lasciando il compito di interfacciamento ad essa al sito web, vincolo che implica la necessità di dover porre controlli e restrizioni direttamente sugli smart contract.\\
Il sistema dovrà funzionare su una simulazione locale della \citGloss{blockchain} e sulla rete di test Ropsten, in quanto la sua implementazione sulla rete ufficiale Ethereum richiede una spesa economica.
\section{Tecnologie richieste}
\begin{itemize}
	\item \textbf{Solidity}, linguaggio orientato agli oggetti per la scrittura di smart contract.
	\item \textbf{React}, libreria open source \citGloss{JavaScript} per il supporto dello strato di presentazione nell'ambito web.
	\item \textbf{Redux}, libreria open source Javascript per la gestione degli stati di \citGloss{React} o AngularJS.
	\item \textbf{Sass CSS}, estensione del linguaggio \citGloss{CSS} che permette l'utilizzo di funzionalità aggiuntive.
	\item \textbf{Redux-minimal}, Boilerplate comprendente \citGloss{React}, Redux, \citGloss{SCSS} e svariati altri componenti che facilitano sviluppo e svolgimento di test.
	\item \textbf{Truffle}, framework per lo sviluppo di smart contract su rete Ethereum.
	\item \textbf{Web3}, API Javascript per effettuare chiamate remote a un nodo Ethereum.
	\item \textbf{Metamask}, plugin per browser che permette il collegamento a un nodo di una rete \citGloss{Ethereum}, compresi i nodi locali e la rete Ropsten.
\end{itemize}
\section{Valutazione finale}
Tutti i membri del gruppo hanno accolto in maniera largamente positiva l'esposizione del capitolato, soprattutto grazie alle numerose tecnologie innovative presenti, in particolare l'utilizzo di smart contract, la rete Ethereum e la libreria React.\\
Un altro aspetto accolto positivamente è il fatto di dover sviluppare una \citGloss{DAPP} (applicazione distribuita) a fine non unicamente finanziario, tipologia di impiego delle blockchain ancora poco diffusa che potrebbe avere un'enorme importanza nel futuro.\\
Sono comunque emerse alcune possibili problematiche, soprattutto riguardo il futuro della rete Ethereum, che attualmente risulta molto instabile come estensione e si teme che possa essere lasciata in secondo piano rispetto a blockchain che non implementano smart contract come Bitcoin e Litecoin.\\
Un ultimo aspetto negativo legato a questo capitolato è l'estrema innovazione di alcune tecnologie, soprattutto riguardanti la blockchain, che risultano ancora poco documentate se non parzialmente incomplete di definizioni formali riguardanti il loro l'utilizzo, implicando l'assenza di best practices da seguire.
\end{document}