\documentclass[VER-2018-02-19.tex]{subfiles}
\begin{document}
\taburowcolors[2] 2{tableLineOne .. tableLineTwo}
\tabulinesep = ^3mm_2mm
\chapter{Riunione}
\section{Informazioni generali}
\begin{itemize}
	\item \textbf{Motivo della riunione:} \`{E} stata indetta questa riunione per fare il punto della situazione sulle modifiche da fare ai documenti in seguito a dei chiarimenti avuti in mattinata ad un incontro con il professore Vardanega. \\
	Si è poi discusso riguardo il \textit{Proof of Concept} dopo le indicazioni fornite dalla Proponente durante la chiamata Skype avvenuta il \nData{18}{02}{2018}.
	\item \textbf{Luogo e data:} Padova, \nData{19}{02}{2018};
	\item \textbf{Orario:} 10.15 - 11.45;
	\item \textbf{Partecipanti:} Ogni membro del gruppo \gruppo.
\end{itemize}


\chapter{Resoconto}

\section{Argomenti}
Di seguito sono riportati i punti dell'ordine del giorno che sono stati discussi:
\begin{enumerate}
	\item \textbf{Proof of Concept:} dopo la creazione del wireframe, sono state creati degli scheletri per le parti di React e dei contratti in Solidity. Si è discusso sulla parte di interfaccia tra le due, vale a dire Redux, chiarendo alcuni dubbi espressi alla Proponente durante la chiamata Skype;
	\item \textbf{Incontro con professore Vardanega:} tutte le modifiche apportate fino ad ora sono risultate corrette rispetto ai chiarimenti avuti in mattinata.\\
	Durante l'incontro è poi stata posta una domanda sul linguaggio Solidity e la sua rappresentazione in UML. Il professore Vardanega ha risposto che siccome UML è stato progettato quando si pensava che la programmazione si sarebbe fermata agli oggetti, non è facile rappresentare tutto attraverso di esso. Quindi ci ha invitato a riflettere su come poterlo fare, e in caso di difficoltà, ci ha invitato a contattare il professore Cardin.\\
	Sempre in seguito all'incontro, è stato deciso di controllare settimanalmente il numero di ore lavorate da tutti i membri del gruppo così da poter tenere meglio sotto controllo i costi del progetto rispetto al preventivo.
\end{enumerate}

\section{Tracciamento delle decisioni}
\begin{table}[H]
	\begin{center}
		\begin{tabu} to \textwidth {
				>{\centering}m{0.25\linewidth}  
				>{\centering\arraybackslash}m{0.7\linewidth}
			}
			\tableHeaderStyle
			\textbf{Codice} & \textbf{Decisione} \\
			VER-2018-02-19.1 & Controllo settimanale delle ore lavorate \\
			VER-2018-02-19.2 & Contattare il professore Cardin in caso di difficoltà con la rappresentazione di Solidity in UML \\
		\end{tabu}
		\caption{Tracciamento delle decisioni del verbale}
	\end{center}
\end{table}
\end{document}