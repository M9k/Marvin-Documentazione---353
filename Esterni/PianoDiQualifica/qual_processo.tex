\documentclass[PianoDiQualifica.tex]{subfiles}

\begin{document}

\chapter{Qualità di processo}

\section{Scopo} 
Per garantire la qualità del prodotto è necessario perseguire la qualità dei processi che lo definiscono.
Per raggiungere questo obiettivo, si è deciso di seguire il principio di miglioramento continuo (\citGloss{PDCA}) e di adottare lo standard ISO/IEC 15504 denominato SPICE (Software Process Improvement and Capability Determination).

\section{Processi}

\subsection{PROC001 - Pianificazione di progetto, impostazione e controllo di processi}
Il macro-processo ha lo scopo di produrre dei piani di sviluppo per il progetto, comprendenti la scelta del modello di ciclo di vita del prodotto, descrizioni delle attività e dei compiti da svolgere, pianificazione temporale del lavoro e dei costi da sostenere, allocazione di compiti e responsabilità e misurazioni per rilevare lo stato del progetto rispetto alle pianificazioni prodotte.

\subsubsection{Obiettivi}
Lo sviluppo del progetto dovrà porre particolare attenzione a rispettare dei particolari obiettivi:
\begin{itemize}
	\item \textbf{Budget:} Utilizzando le metriche descritte nella sezione seguente, si deve tenere sempre controllato l'utilizzo del budget, al fine di non avere scarti eccessivi con il costo preventivato;
	\item \textbf{Task:} Porre attenzione alla pianificazione dei task e al loro completamento, assicurandosi che seguano la metodologia del miglioramento continuo, affinché tutti i compiti ne traggano vantaggio;
	\item \textbf{Formazione personale:} Avere accortezza che ogni membro del gruppo abbia un livello di preparazione adatto all'esecuzione dei task assegnati, al fine di evitare ritardi sul calendario;
	\item \textbf{Calendario:} Assicurare una pianificazione adatta ai compiti da svolgere, per evitare scostamenti dal budget preventivato;
	\item \textbf{Standard:} Definire standard di processo ogni qualvolta sia possibile, per facilitare il lavoro in gruppo e favorire un incremento continuo.
\end{itemize}

Vengono utilizzate le seguenti metriche definite nelle \ndp in \S{D.1}:
\begin{itemize}
	\item \textlink{MPS001TAB}{MPS001}{MPS001} Schedule Variance (SV);
	\item \textlink{MPS002TAB}{MPS002}{MPS002} Budget Variance (BV).
\end{itemize}

\subsection{PROC002 - Verifica software}
Il processo punta a verificare se qualsiasi elemento del sistema soddisfa completamente i \citGloss{requisiti} ad esso assegnati.
\subsubsection{Obiettivi}
Per poter definire delle \citGloss{baseline} per lo sviluppo del software, è necessario che il codice venga sempre verificato:
\begin{itemize}
	\item \textbf{Commenti al codice:} Ogni unità di codice dovrà essere sufficientemente commentata affinché sia ritenuta verificabile;
	\item \textbf{Prevenzione di bug:} Accertarsi per quanto possibile che ogni unità di codice non sia affetta da bug prima dell'utilizzo.
\end{itemize}
Vengono utilizzate le seguenti metriche definite nelle \ndp in \S{D.1}:
\begin{itemize}
	\item \textlink{MPS003TAB}{MPS003}{MPS003} Function coverage;
	\item \textlink{MPS004TAB}{MPS004}{MPS004} Statement coverage;
	\item \textlink{MPS005TAB}{MPS005}{MPS005} \citGloss{Branch} coverage;
	\item \textlink{MPS006TAB}{MPS006}{MPS006} Lines coverage
\end{itemize}

	\subsection{PROC003 - Gestione rischi}
Il processo mira ad identificare nuovi rischi e a monitorare e ridurre la possibilità dell'insorgere di questi durante l'attività di progetto.
\subsubsection{Obiettivi:}
\begin{itemize}
	\item \textbf{Individuare rischi Fase:} Ad ogni nuova fase del progetto verranno analizzati possibili rischi, cercando delle soluzioni automatiche per diminuire l'occorrenza di questi;
	\item \textbf{Analisi:} I rischi saranno gestiti con una prima analisi che dovrà fornire uno strumento o procedura automatica per ridurre o prevenire le cause scatenanti di questo rischio.
\end{itemize}
Vengono utilizzate le seguenti metriche definite nelle \ndp in \S{D.1}:
\begin{itemize}
	\item \textlink{MPS007TAB}{MPS007}{MPS007} Indisponibilità servizi esterni;
	\item \textlink{MPS008TAB}{MPS008}{MPS008} Rischi non previsti.
\end{itemize}

\subsection{PROC004 - Gestione Test}	
Per garantire una gestione efficacie dell'analisi dinamica è necessario stabilire delle misurazioni sull'esecuzione di essa.
Le misure elencate qui sotto sono consigliate dal blog di una azienda che produce prodotti di riferimento per la gestione dei test, le informazioni possono essere trovate nei riferimenti informativi.
Vengono utilizzate le seguenti metriche definite nelle \ndp in \S{D.1}:
\begin{itemize}
	\item \textlink{MTSA001TAB}{MTSA001}{MTSA001} Percentuale di test case passati;
	\item \textlink{MTSA002TAB}{MTSA002}{MTSA002} Percentuale di test case falliti;
	\item \textlink{MTSA003TAB}{MTSA003}{MTSA003} Tempo medio del team di sviluppo per la risoluzione di errori;
	\item \textlink{MTSA004TAB}{MTSA004}{MTSA004} Efficienza della progettazione dei test;
	\item \textlink{MTSA005TAB}{MTSA005}{MTSA005} Contenimento dei difetti;
	\item \textlink{MTSH001TAB}{MTSH001}{MTSH001} Percentuale di difetti sistemati;
	\item \textlink{MTSH002TAB}{MTSH002}{MTSH002} Copertura dei test eseguiti;
	\item \textlink{MTSH003TAB}{MTSH003}{MTSH003} Copertura dei requisiti;
	\item \textlink{MTSH004TAB}{MTSH004}{MTSH004} Difetti per requisito;	
\end{itemize}

\subsection{PROC005 - Versionamento e Build}
Per garantire la correttezza per costruzione, verranno monitorati i \citGloss{commit}, sia per numero che per contenuto. Inoltre le \citGloss{build} effettuate con lo strumento di integrazione continua \citGloss{Travis} sono l'indicatore della correttezza e aderenza alle norme del codice.
\subsubsection{Obiettivi:}
\begin{itemize}
	\item \textbf{Commit frequenti:} I commit effettuati dai membri gruppo devono essere frequenti affinché il codice sia sempre aggiornato.
	Questo permette ad ogni componente del gruppo di comprendere come procede lo sviluppo potendo anche dare consigli;
	\item \textbf{Commit con poche modifiche:} I commit effettuati dai membri del gruppo devono contenere poche modifiche, affinché il ciclo di vita del codice sia facilmente monitorabile e verificabile;
	\item \textbf{Build positive:} Ogni commit deve portare al successo della build, per evitare la propagazione di errori.
\end{itemize}
Vengono utilizzate le seguenti metriche definite nelle \ndp in \S{D.1}:
\begin{itemize}
	\item \textlink{MPS009TAB}{MPS009}{MPS009} Media numero commit per settimana;
	\item \textlink{MPS010TAB}{MPS010}{MPS010} Media numero build Travis per settimana;
	\item \textlink{MPS011TAB}{MPS011}{MPS011} Percentuale build Travis superate.
\end{itemize}

\subsection{Tabella delle metriche}
% ID METRICA
% M --> metrica
% PS--> processo
% (lettera per macro categoria, se esistono)
% 00n --> numero incrementale
Questa tabella indica i \textbf{range} di accettazione e di ottimalità per le metriche utilizzate per la qualità di processo:
\begin{table}[H]
	\begin{center}
		\begin{tabu} to \textwidth {
				>{\centering}m{0.1\linewidth}
				>{\centering}m{0.4\linewidth}
				>{\centering}m{0.2\linewidth} 
				>{\centering\arraybackslash}m{0.2\linewidth}
			}
			\tableHeaderStyle
			
			\textbf{ID} & \textbf{Nome} & \textbf{Range di accettazione} & \textbf{Range di ottimalità}\\
			
			\multicolumn{4}{c}{\textbf{PROC001}}\\
			\textlink{MPS001}{MPS001TAB}{MPS001} & Schedule Variance & $ \geq -5 \ giorni $ & $ \geq 0 \ giorni $ \\
			\textlink{MPS002}{MPS002TAB}{MPS002} & Budget Variance & $ \geq -3 \ giorni $ & $ \geq 0 \ giorni$ \\
			
			\hline
			\multicolumn{4}{c}{\textbf{PROC002}}\\
			\textlink{MPS003}{MPS003TAB}{MPS003} & Function coverage & $ \geq 95\% $ & $ 100\% $\\
			\textlink{MPS004}{MPS004TAB}{MPS004} & Statement coverage &  $ \geq 95\% $& $ 100\% $\\
			\textlink{MPS005}{MPS005TAB}{MPS005} & \citGloss{branch} coverage & $ \geq 90\% $ & $ 100\% $\\
			\textlink{MPS006}{MPS006TAB}{MPS006} & Condition coverage & $ \geq 95\% $ & $ 100\% $\\

			\hline
			\multicolumn{4}{c}{\textbf{PROC003}}\\
			\textlink{MPS007}{MPS007TAB}{MPS007} & Indisponibilità servizi esterni & $ -3 \ giorni $ & $ 0 \ giorni$\\
			\textlink{MPS008}{MPS008TAB}{MPS008} & Rischi non previsti & $ -5 $ & $ 0 $\\ 		
			
			\hline
			\multicolumn{4}{c}{\textbf{PROC005}}\\
			\textlink{MPS009}{MPS009TAB}{MPS009} & Media commit per settimana & 20 & 30 \\
			\item \textlink{MPS010}{MPS010TAB}{MPS010} & Media build Travis per settimana & 20 & 30 \\
			\item \textlink{MPS011}{MPS011TAB}{MPS011} & Percentuale build superate & $ \geq 75\% $ & $ \geq 85\% $ \\
			
				
		\end{tabu}
		\caption{Tabella delle metriche della qualità di processo}
		\vspace{-1em}
	\end{center}
\end{table}

\subsubsection{Tabella delle metriche sui test PROC0004}
Questa tabella indica i \textbf{range} di accettazione e ottimalità per le misure effettuate sui test, per le misure effettuate su tutti i tipi di test saranno presenti due valori per range: TM, per i test di modulo, e TH, per i test ad alto livello.\\
Una piccola legenda per facilitare la lettura della tabella: \\
\begin{itemize}
	\item \textbf{PRO:} percentuale requisiti obbligatori;
	\item \textbf{PRF:} percentuale requisiti facoltativi;
	\item \textbf{xh:} dove x è un numero, ore.
\end{itemize}

\begin{table}[H]
	\begin{center}
		\begin{tabu} to \textwidth {
				>{\centering}m{0.15\linewidth}
				>{\centering}m{0.45\linewidth}
				>{\centering}m{0.18\linewidth} 
				>{\centering\arraybackslash}m{0.17\linewidth}
			}
			\tableHeaderStyle
			\textbf{ID} & \textbf{Nome} & \textbf{Range di accettazione} & \textbf{Range di ottimalità}\\
			\multicolumn{4}{c}{\textbf{Per tutti i test}}\\
			\textlink{MTSA001}{MTSA001TAB}{MTSA001} & Percentuale di test case passati & 100\%-100\%(TM) PRC-100\%(TH) & 100\%-100\%(TM) 100\%-100\%(TH) \\ 
			\textlink{MTSA002}{MTSA002TAB}{MTSA002} & Percentuale di test case falliti & 0\%-0\%(TM) 0\%-PRF(TH) & 0\%-0\%(TM) 0\%-0\%(TH) \\
			\textlink{MTSA003}{MTSA003TAB}{MTSA003} & Tempo medio del team di sviluppo per la risoluzione di errori& 0h-4h(TM) 0h-4h(TH) & 0h-2h(TM) 0h-2h(TH) \\
			\textlink{MTSA004}{MTSA004TAB}{MTSA004} & Efficienza della progettazione dei test & 0.5h-3h & 1h-2h \\
			%\textlink{MTSA005}{MTSA005TAB}{MTSA005} & Tempo medio per il testing dei bug fix & ? & ? \\
			\textlink{MTSA005}{MTSA005TAB}{MTSA005} & Contenimento dei difetti & 60\%-100\% & 80\%-100\% \\
			\multicolumn{4}{c}{\textbf{Per i test ad alto livello}}\\
			\textlink{MTSH001}{MTSH001TAB}{MTSH001} & Percentuale di difetti sistemati & 90\%-100\% & 98\%-100\% \\
			\textlink{MTSH002}{MTSH002TAB}{MTSH002} & Copertura dei test eseguiti & 80\%-100\% & 95\%-100\%  \\
			\textlink{MTSH003}{MTSH003TAB}{MTSH003} & Copertura dei requisiti & PRO-100\% & 100\%-100\% \\
			\textlink{MTSH004}{MTSH004TAB}{MTSH004} & Difetti per requisito & Variabile in base alla gravità del difetto & 0  \\
			%\textlink{MTSH005}{MTSH005TAB}{MTSH005} & Tasso di iniezione dei difetti & ? & ? \\			
		\end{tabu}
		\caption{Tabella delle metriche per i test}
		\vspace{-1em}
	\end{center}
\end{table}

\end{document}