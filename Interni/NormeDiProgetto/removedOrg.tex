
\subparagraph{Altri strumenti:}
		\begin{itemize}
			\item \textbf{Google Drive:} è stato scelto per la condivisione veloce di file come guide, documentazioni, riferimenti utili e documenti informali tramite
			Google Docs. Google Docs permette la creazione di documenti
			maneggevoli quando è necessaria una modifica da parte di molti membri
			del gruppo (ad esempio nella fase di approvazione degli scheletri dei
			documenti); in questo Google Docs si rivela estremamente flessibile,
			visualizzando in tempo reale le modificheche sono apportate al documento,
			permettendo inoltre di aggiungere commenti al testo.
			\item \textbf{Google Calendar:} è stato scelto per la gestione degli eventi ed
			impegni del gruppo, come riunioni interne, revisioni o incontri con il committente. Fornisce inoltre un \citGloss{Bot} integrabile con \citGloss{Slack}.
		\end{itemize}
	
	\subsubsection{Documentazione}
	\subparagraph{\LaTeX:} per quanto riguarda la stesura della documentazione, si è deciso di utilizzare \LaTeX. \\
		\'{E}	stato scelto questo linguaggio di \citGloss{markup}, nonostante
		non sempre sia di immediata comprensione, per le sue potenzialità quando
		si tratta di riferimenti all'interno del documento, tabelle, numerazioni,
		note, organizzazione delle pagine, creazione di template. inoltre
		la possibilità di creare comandi personalizzati, caratteristica utile nella
		stesura della documentazione prodotta nel corso del progetto. Di fronte a
		queste caratteristiche così positive, altre soluzioni possibili (come utilizzare
		LibreOffice o Google Docs) sono state immediatamente scartate.
	\subparagraph{Editor:} l'editor consigliato per scrivere la documentazione con \LaTeX è
		TexStudio in quanto software libero aggiornato e multi piattaforma. Esso
		permette il completamento automatico dei comandi, la codifica in UTF-8 e
		dispone di un visualizzatore PDF integrato.
	\subparagraph{Diagrammi UML:} per la modellazione dei diagrammi UML è stato deciso di utilizzare 	\emph{UML Designer} in quanto supporta UML 1.x ed UML 2.x. Permette la modellazione dei casi d'uso, i 	diagrammi delle classi, degli oggetti, delle attività	e di sequenza.
	\subparagraph{Tracciamento:} per il tracciamento dei requisiti e dei casi d'uso si è deciso di utilizzare \emph{Trender}, un software che permette di tener traccia delle dipendenze dei vari componenti durante lo sviluppo del progetto.
	
	\subsubsection{Ambiente di sviluppo}
	\subparagraph{Sistemi operativi:} i membri del gruppo 353 possono operare indistintamente sui tre principali sistemi operativi (Windows, Mac OS X e Linux), in quanto esistono frameworks per tutte le piattaforme che sono compatibili con le librerie di sviluppo interessate.
	\subparagraph{Browsers:} \citGloss{Metamask}, il plugin che permette ad un \citGloss{browser} l'accesso alla rete \citGloss{Ethereum}, è attualmente disponibile solo per \citGloss{Google Chrome} o \citGloss{Firefox} e si rende quindi necessario l'utilizzo di uno tra questi.
	\subsubsection{Ambiente di verifica}
	\subparagraph{Documentazione:} per quanto riguarda la verifica dei documenti, TexStudio non
		ha di default il dizionario per il controllo ortografico italiano. \'{E}
		stato quindi preparato un pacchetto da scompattare all'interno della cartella
		"dictionaries" dentro la directory di installazione di TexStudio in modo da
		poter selezionare come lingua preferita l'italiano nella sezione "Controllo
		Linguistico" delle Configurazioni di TexStudio.
