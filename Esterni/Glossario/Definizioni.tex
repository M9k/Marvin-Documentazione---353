\usepackage{glossaries}
\makeglossaries

%\newglossaryentry{nome}{
%	name={nome},
%	description={descrizione}
%}
%INIZIO SLACK
\newglossaryentry{Slack}{
	name={Slack},
	description={Applicazione di messaggistica istantanea pensata per la collaborazione tra i membri di uno o più gruppi di lavoro. Offre la possibilità di creare canali privati tematizzati, gruppi privati e chat dirette. Tutto ciò che è presente in Slack può essere cercato, che sia un file, una conversazione o le persone stesse. Inoltre può essere esteso mediante l'uso di applicazioni di terze parti}
}
%FINE SLACK
%INIZIO BOT
\newglossaryentry{Bot}{
	name={Bot},
	description={Programma che sostituisce un umano nello svolgere una attività usualmente ripetitiva, fornisce agli utenti servizi e informazioni in modo completamente automatizzato}
}
%FINE BOT
%INIZIO SKYPE
\newglossaryentry{Skype}{
	name={Skype},
	description={Software di messaggistica istantanea e VoIP, cioè permette di effettuare chiamate e videochiamate utilizzando la rete Internet}
}
%FINE SKYPE
%INIZIO ASANA
\newglossaryentry{Asana}{
	name={Asana},
	description={Applicazione web e mobile ideata per tracciare ed organizzare il lavoro di un gruppo. Permette la creazione di task e la loro assegnazione, la creazione di workspace che contengno i progetti e l'aggiunta e rimozione di commenti}
}
%FINE ASANA
%INIZIO INSTAGANTT
\newglossaryentry{Instagantt}{
	name={Instagantt},
	description={Applicazione integrabile con Asana che permette di creare, modificare e visualizzare diagrammi di Gantt sincronizzati con le attività programmate in Asana}
}
%FINE INSTAGANTT
%INIZIO GANTT
\newglossaryentry{Gantt}{
	name={Diagrammi di Gantt},
	description={Strumento di supporto alla gestione dei progetti. \`{E} costituito partendo da un asse orizzontale a rappresentazione dell'arco temporale totale del progetto, suddiviso in fasi incrementali, e da un'asse verticale a rappresentazione delle mansioni o attività che costituiscono il progetto}
}
%FINE GANTT
%INIZIO REQUISITI
\newglossaryentry{Requisiti}{
	name={Requisiti},
	description={Funzionalità che il prodotto software deve offrire}
}
%FINE REQUISITI
%INIZIO STANDUPS
\newglossaryentry{Standups}{
	name={Stand-up meeting },
	description={Alcune metodologie di sviluppo del software utilizzano brevi riunioni tra i partecipanti al progetto per fare il punto della situazione, definiti stand-up meeting. A turno vengono esposti problemi e dubbi nati durante il lavoro di sviluppo e conseguentemente coordinati gli sforzi per risolvere i problemi individuati}
}
%FINE STANDUPS
%INIZIO GITHUB
\newglossaryentry{GitHub}{
	name={GitHub},
	description={Servizio di web hosting per lo sviluppo di progetti software che usa il sistema di controllo di versione Git}
}
%FINE GITHUB

%INIZIO BRANCH
\newglossaryentry{Branch}{
	name={branch},
	description={Ramo di lavoro, favorisce uno sviluppo non lineare e incrementale, offrendo la possibilità di sviluppare degli incrementi sui branch che verranno poi uniti al branch principale, chiamato master}
}
%FINE BRANCH
%INIZIO REPOSITORY
\newglossaryentry{Repository}{
	name={Repository},
	description={Ambiente di un sistema informativo in cui vengono gestiti i metadati attraverso tabelle relazionali. L'insieme di tabelle, regole e motori di calcolo tramite cui si gestiscono i metadati prende il nome di metabase}
}
%FINE REPOSITORY
%INIZIO UNIWEB
\newglossaryentry{Uniweb}{
	name={Uniweb},
	description={Portale dell'Università degli Studi di Padova disponibile all'indirizzo web \nURI{https://uniweb.unipd.it/Home.do} che permette a studenti di gestire la propria carriera universitaria, come visuailizzare il libretto e iscriversi agli appelli d'esame e ai docenti di predisporre le liste e di inserire i risultati degli appelli d'esame}
}
%FINE UNIWEB
%INIZIO ANNO
\newglossaryentry{Anno}{
	name={Anno},
	description={Con il termine anno, o anno accademico, si definisce un periodo di tempo della durata approssimativamente di un anno nel quale l'università svolge le proprie attività}
}
%FINE ANNO
%INIZIO CORSO
\newglossaryentry{Corso}{
	name={Corso},
	description={Con il termine corso, o corso di laurea, si definisce un percorso di studi che porta ad una determinata laurea}
}
%FINE CORSO
%INIZIO ESAME
\newglossaryentry{Esame}{
	name={Esame},
	description={Con il termine esame, o esame universitario, si definisce un insieme di lezioni frontali ed eventualmente laboratori che veicolano delle informazioni agli studenti e si concludono con degli appelli}
}
%FINE ESAME
%INIZIO BLOCKCHAIN
\newglossaryentry{Blockchain}{
	name={Blockchain},
	description={Registro pubblico e distribuito delle transizioni e dei contratti sulla quale si basa una cryptovaluta. Quando utilizzata nei documenti allegati si fa sempre riferimento alla blockchain alla base di Ethereum}
}
%FINE BLOCKCHAIN
%INIZIO BLOCCO
\newglossaryentry{Blocco}{
	name={Blocco},
	description={Parte di una blockchain che contiene le transizioni. Per essere aggiunto alla blockchain deve avere l'approvazione da parte dei miner}
}
%FINE BLOCCO
%INIZIO MINING
\newglossaryentry{Mining}{
	name={Mining},
	description={Processo con il quale si va a confermare la validità dei blocchi e quindi ad inserirli nella blockchain, in caso di esito negativo si procede all'eliminazione del blocco non ancora confermato}
}
%FINE MINING
%INIZIO MININGFARM
\newglossaryentry{Mining farm}{
	name={Mining farm},
	description={Insieme, solitamente di grosse dimensione, di hardware dedicato ad attività di mining}
}
%FINE MININGFARM
%INIZIO MINER
\newglossaryentry{Miner}{
	name={Miner},
	description={Entità che svolge attività di mining, può essere utilizzato per indicare l'hardware oppure la persona che possiede una mining farm}
}
%FINE MINER
%INIZIO TRANSAZIONE
\newglossaryentry{Transazione}{
	name={Transazione},
	description={Messaggio che richiede ai miner una operazione di alterazione della blockchain, come ad esempio l'interazione con un contratto che ne modifica lo stato oppure un trasferimento di Ether. Il costo dell'operazione viene pagato in Gas}
}
%FINE TRANSAZIONE
%INIZIO GAS
\newglossaryentry{Gas}{
	name={Gas},
	description={Valuta per il pagamento delle transazioni, con un valore in Ether variabile a seconda dello stato della blockchain. Può indicare anche il costo per l'esecuzione di un contratto o di una parte di esso all'interno di una EVM}
}
%FINE GAS
%INIZIO CHIAVEPRIVATA
\newglossaryentry{Chiave privata}{
	name={Chiave privata},
	description={Anche abbreviata in chiave, è una stringa esadecimale di 65 caratteri utilizzata per firmare i messaggi, in modo da garantirne l'autenticità}
}
%FINE CHIAVEPRIVATA

%INIZIO METAMASK
\newglossaryentry{Metamastk}{
	name={Metamask},
	description={Plug-in browser che consente un accesso alla blockchain Ethereum e che gestisce le chiavi}
}
%FINE METAMASK
%INIZIO VIRTUALDOM
\newglossaryentry{Virtual DOM}{
	name={Virtual DOM},
	description={Astrazione del DOM indipendente dal browser che consente operazioni veloci senza necessità di manipolare realmente la pagina}
}
%FINE VIRTUALDOM
%INIZIO DOM
\newglossaryentry{DOM}{
	name={DOM},
	description={Document Object Model, astrazione strutturata degli elementi e del testo presenti in una pagina HTML}
}
%FINE DOM
%INIZIO CONTROLLODIVERSIONE
\newglossaryentry{Controllo di versione}{
	name={Controllo di versione},
	description={Software utilizzato per gestire lo sviluppo di programmi o documenti, mantenendo traccia di tutte le modifiche effetuate e permettendo così di analizzare la storia di un progetto}
}
%FINE CONTROLLODIVERSIONE
%INIZIO IDE
\newglossaryentry{IDE}{
	name={IDE},
	description={Integrated development environment, in italiano ambiente di sviluppo integrato, è uno strumento software che offre un editor di testo accompagnato da tutti gli strumenti utili al programmatore per scrivere efficacemente codice}
}
%FINE IDE
%INIZIO APACHELICENSE
\newglossaryentry{Apache License}{
	name={Apache License},
	description={in italiano licenza Apache, è una licenza di software libero non copyleft scritta dall'Apache Software Foundation}
}
%FINE APACHELICENSE
%INIZIO COPYLEFT
\newglossaryentry{Copyleft}{
	name={Colpyleft},
	description={Aggettivo indicante una licenza che non permette la ridistribuzione del software o di versioni modificate con vincoli aggiuntivi rispetto a quelli indicati dalla licenza stessa}
}
%FINE COPYLEFT
%INIZIO MILESTONE
\newglossaryentry{Milestone}{
	name={Milestone},
	description={Termine inglese traducibile con pietra miliare, indica il raggiungimento di obbiettivi stabiliti con importanza strategica. Coincide con una o più baseline e può derivare da un obbligo contrattuale o da un'opportunità decisa dal gruppo}
}
%FINE MILESTONE
%INIZIO BASELINE
\newglossaryentry{Baseline}{
	name={Baseline},
	description={Base verifica e di appoggio dalla quale non si può retrocedere. \`{E} collocata in una repository ed utilizzata per i successivi incrementi}
}
%FINE BASELINE
%INIZIO STUB
\newglossaryentry{Stub}{
	name={Stub},
	description={Componente passivo fittizzio che serve a simulare una parte del sistema in modo tale da testare una porzione di codice singolarmente}
}
%FINE STUB
%INIZIO DRIVER
\newglossaryentry{Driver}{
	name={Driver},
	description={Componente attivo fittizzio che serve a pilotare l'esecuzione di un codice in modo tale da permetterne il test e l'eventuale individuazione di errore}
}
%FINE DRIVER
%INIZIO DAPP
\newglossaryentry{DAPP}{
	name={DAPP},
	description={Decentralized application, indica una applicazione backend eseguita da una blockchain che supporta l'utilizzo di smart contract}
}
%FINE DAPP
%INIZIO APPLICAZIONEWEB
\newglossaryentry{Applicazioni Web}{
	name={Applicazioni Web},
	description={Applicazione usufruibile attraverso il browser, che non richiedono la necessità di installare programmi aggiuntivi sul proprio pc}
}
%FINE APPLICAZIONEWEB
%INIZIO WEBAPP
\newglossaryentry{WebAPP}{
	name={WebAPP},
	description={Vedere applicazioni Web}
}
%FINE WEBAPP
%INIZIO BROWSER
\newglossaryentry{Browser}{
	name={Browser},
	description={Software che permette l'accesso a risorse web di varia natura, consente l'accesso alle WebAPP e usualmente supporta l'aggiunta di plug-in}
}
%FINE BROWSER
%INIZIO FIREFOX
\newglossaryentry{Firefox}{
	name={Firefox},
	description={Chiamato anche Mozilla Firefox, browser open source sviluppato da Mozilla sulla base di Netscape, precedentemente conosciuto come Phoenix}
}
%FINE FIREFOX
%INIZIO CHROME
\newglossaryentry{Chrome}{
	name={Chrome},
	description={Chiamato anche Google Chrome, browser sviluppato da Google, disponibile in versione open source con caratteristiche simili sotto il nome di Chomium}
}
%FINE CHROME
%INIZIO PLUGIN
\newglossaryentry{PLug-in}{
	name={Plug-in},
	description={Componente aggiuntivo che può essere installato in un software per ampliarne le funzionalità}
}
%FINE PLUGIN
%INIZIO CAPITOLATODAPPALTO
\newglossaryentry{Capitolato d'appalto}{
	name={Capitolato d'appalto},
	description={\`{E} un documento tecnico per la definizione delle specifiche tecniche delle opere che andranno ad eseguirsi per effetto del relativo contratto, di cui è solitamente parte integrante}
}
%FINE CAPITOLATODAPPALTO
%INIZIO COMMITTENTE
\newglossaryentry{Committente}{
	name={Committente},
	description={Figura che commissiona un lavoro, può essere una persona fisica o una figura giuridica. Ha potere decisionale e di spesa relativo alla gestione dell'appalto}
}
%FINE COMMITTENTE
%INIZIO CSS
\newglossaryentry{CSS}{
	name={CSS},
	description={Cascading Style Sheets, cioè fogli di stile a cascata, è un linguaggio utilizzato per definire l'aspetto grafico delle pagine web e dei documenti XML}
}
%FINE CSS
%INIZIO SCSS
\newglossaryentry{SCSS}{
	name={SCSS},
	description={Sassy CSS, estensione di CSS che implementa nuove funzionalità. Viene compilato in CSS prima di essere utilizzato dai browser, questo garantisce la compatibilità con tutti i browser che supportano i fogli di stile}
}
%FINE SCSS
%INIZIO AIRBNB
\newglossaryentry{Airbnb}{
	name={Airbnb},
	description={Compagnia operante a San Francisco fondata nel 2008, autrice di una guida di stile per la codifica in linguaggio JavaScript molto diffusa, denominata Airbnb JavaScript Style Guide}
}
%FINE AIRBNB
%INIZIO ECMA
\newglossaryentry{ECMA}{
	name={ECMA},
	description={Associazione fondata nel 1961 che si occupa della standardizzazione nel settore informatico. Definisce il linguaggio ECMAScript, dal quale deriva il linguaggio JavaScript}
}
%FINE ECMA
%INIZIO HTML
\newglossaryentry{HTML}{
	name={HTML},
	description={HyperText Markup Language, linguaggio di markup per la formattazione e la impaginazione di pagine web ipertestuali}
}
%FINE HTML
%INIZIO REACT
\newglossaryentry{React}{
	name={React},
	description={Chiamata anche React.js, è una libreria JavaScript che favorisce la costruzione di interfacce grafice, dichiarativo e basato su l'utilizzo di componenti}
}
%FINE REACT
%INIZIO REDUX
\newglossaryentry{Redux}{
	name={Redux},
	description={Chiamata anche Redux.js, è una libreria JavaScript che favorisce la gestione dello stato delle applicazione web, spesso utilizzata assieme a React}
}
%FINE REDUX
%INIZIO JAVASCRIPT
\newglossaryentry{JavaScript}{
	name={JavaScript},
	description={Linguaggio di scripting orientato agli oggetti e agli eventi, interpretabile dai browser. \`{E} standardizzato dalla ECMA nel 1997 con il nome ECMAScript}
}
%FINE JAVASCRIPT
%INIZIO SOLIDITY
\newglossaryentry{Solidity}{
	name={Solidity},
	description={Linguaggio di scrittura per smart contract su rete Ethereum ispirato da JavaScript e C++}
}
%FINE SOLIDITY
%INIZIO FRONTEND
\newglossaryentry{Front-end}{
	name={Front-end},
	description={La parte del software visibile all'utente e con cui egli può interagire}
}
%FINE FRONTEND
%INIZIO BACKEND
\newglossaryentry{Back-end}{
	name={Back-end},
	description={La parte del software che permette l'effettivo funzionamento delle iterazioni compiute dall'utente}
}
%FINE BACKEND
%INIZIO PDCA
\newglossaryentry{PDCA}{
	name={PDCA},
	description={Vedere Ciclo di Deming}
}
%FINE PDCA
%INIZIO CICLODIDEMING
\newglossaryentry{Ciclo di Deming}{
	name={Ciclo di Deming},
	description={Modello per il miglioramento continuo della qualità in un'ottica a lungo raggio, anche detto PDCA (Plan, Do, Check, Act)}
}
%FINE CICLODIDEMING
%INIZIO TEXSTUDIO
\newglossaryentry{TexStudio}{
	name={TexStudio},
	description={Ambiente di scrittura integrato per favorire la creazione di documenti LaTeX, con numerose feature e personalizzazioni che rendono più efficace scrivere documenti}
}
%FINE TEXSTUDIO
%INIZIO WEBSTORM
\newglossaryentry{WebStorm}{
	name={WebStorm},
	description={IDE sviluppato da Jetbrains per lo sviluppo di applicazioni JavaScript, con numerose feature e personalizzazioni che rendono più efficace la scrittura del codice}
}
%FINE WEBSTORM
%INIZIO TRENDER
\newglossaryentry{Trender}{
	name={Trender},
	description={Sistema open source che permette il tracciamento di requisiti, casi d'uso, attori, verbali, package, classi e loro metodi.}
}
%FINE TRENDER
%INIZIO VALIDAZIONE
\newglossaryentry{Validazione}{
	name={Validazione},
	description={Controllo effettuato per accertarsi che tutti i requisiti richiesti siano stati soddisfatti ed attestare la conformità con quanto richiesto dal committente}
}
%FINE VALIDAZIONE
%INIZIO VERIFICA
\newglossaryentry{Verifica}{
	name={Verifica},
	description={Processo di supporto che accerta che le esecuzioni delle attività non abbiano introdotto errori}
}
%FINE VERIFICA
%INIZIO COMMIT
\newglossaryentry{Commit}{
	name={Commit},
	description={Azione che salva le modifiche effettuate ai file nel repository locale, richiede una breve descrizione su quanto aggiunto o modificato}
}
%FINE COMMIT
%INIZIO PUSH
\newglossaryentry{Push}{
	name={Push},
	description={Azione che invia tutti i commit presenti nel repository locale a quello condiviso con gli altri membri del team di sviluppo}
}
%FINE PUSH
%INIZIO PULL
\newglossaryentry{Pull}{
	name={Pull},
	description={Azione che scarica dal repository remoto tutti i commit effettuati dagli altri membri del team di sviluppo e li applica ai propri file}
}
%FINE PULL
%INIZIO PDF
\newglossaryentry{PDF}{
	name={PDF},
	description={Portable Document Format, formato sviluppato da Adobe per la rappresentazione di documenti in modo indipendente al dispositivo utilizzato}
}
%FINE PDF
%INIZIO TEX
\newglossaryentry{Tex}{
	name={Tex},
	description={Linguaggio per la stesura di documenti matematici e scientifici, permette il versionamento essendo puramente testuale}
}
%FINE TEX
%INIZIO TICKET
\newglossaryentry{Ticket}{
	name={Ticket},
	description={Notifica di segnalazione di una attività da svolgere per l'avanzamento del progetto oppure una segnalazione di errore all'interno del software da risolvere}
}
%FINE TICKET
%INIZIO CHIAVEPUBBLICA
\newglossaryentry{Chiave pubblica}{
	name={Chiave pubblica},
	description={Anche abbreviata in indirizzo utente o account, è una stringa esadecimale di 41 carattere utilizzata per l'identificazione. Viene utilizzata anche per gli trasferimenti di Ether}
}
%FINE CHIAVEPUBBLICA
%INIZIO INDIRIZZO
\newglossaryentry{Indirizzo}{
	name={Indirizzo},
	description={\`{E} una stringa esadecimale di 41 carattere utilizzata per l'identificazione di contratti o di account}
}
%FINE INDIRIZZO
%INIZIO SMARTCONTRACT
\newglossaryentry{Smart contract}{
	name={Smart contract},
	description={Protocolli per l'esecuzione di un contratto, il quale viene gestito in modo automatico senza bisogno dell'intervento umano, rimuovendo così la possibilità di violarlo o di manometterlo}
}
%FINE SMARTCONTRACT
%INIZIO ETHEREUM
\newglossaryentry{Ethereum}{
	name={Ethereum},
	description={Piattaforma decentralizzata per la gestione di smart contract e di cryptovaluta Ether}
}
%FINE ETHEREUM
%INIZIO ETHER
\newglossaryentry{Ether}{
	name={Ether},
	description={Moneta virtuale o cryptocurrency utilizzata nella rete Ethereum, convertibile in gas}
}
%FINE ETHER
%INIZIO EVM
\newglossaryentry{EVM}{
	name={Ethereum Virtual Machine},
	description={Macchina virtuale sulla quale vengono eseguiti gli smart contract. Attualmente la loro esecuzione è assegnata prevalentemente a processori grafici}
}
%FINE EVM
%INIZIO TRAVIS
\newglossaryentry{Travis}{
	name={Travis},
	description={Servizio di integrazione continua distribuito e ospitato utilizzato per costruire e testare progetti software ospitati su GitHub}
}
%FINE TRAVIS

%INIZIO LINTING
\newglossaryentry{Linting}{
	name={Linting},
	description={Analisi della qualità del codice statica individuando possibili errori prima della compilazione/esecuzione di linguaggi di programmazione.}
}
%FINE LINTING