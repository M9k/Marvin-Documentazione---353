\usepackage{glossaries}
\makeglossaries

%\newglossaryentry{nome}{
%	name={nome},
%	description={descrizione}
%}
%INIZIO SLACK
\newglossaryentry{Slack}{
	name={Slack},
	description={Applicazione di messaggistica istantanea pensata per la collaborazione tra i membri di uno o più gruppi di lavoro. Offre la possibilità di creare canali privati tematizzati, gruppi privati e chat dirette. Tutto ciò che è presente in Slack può essere cercato, che sia un file, una conversazione o le persone stesse. Inoltre, integra molte applicazioni di terze parti rendendo più performante il suo uso}
}
%FINE SLACK

%INIZIO BOT
\newglossaryentry{Bot}{
	name={Bot},
	description={Programma che accede alla rete attraverso gli stessi canali utilizzati dagli umani, e fornisce agli utenti servizi e informazioni velocemente e in maniera automatizzata}
}
%FINE BOT

%INIZIO SKYPE
\newglossaryentry{Skype}{
	name={Skype},
	description={Software di messaggistica istantanea e VoIP, cioè permette di effettuare chiamate e videochiamate utilizzando la rete Internet}
}
%FINE SKYPE

%INIZIO REPOSITORY
\newglossaryentry{repository}{
	name={Repository},
	description={Ambiente di un sistema informativo in cui vengono gestiti i metadati attraverso tabelle relazionali. L'insieme di tabelle, regole e motori di calcolo tramite cui si gestiscono i metadati prende il nome di metabase}
}
%FINE REPOSITORY

%INIZIO ASANA
\newglossaryentry{Asana}{
	name={Asana},
	description={Applicazione web e mobile ideata per tracciare e organizzare il lavoro di un gruppo. Permette la creazione di task e la loro assegnazione, la creazione di workspace che contengno i progetti e l'aggiunta e rimozione di commenti}
}
%FINE ASANA

%INIZIO INSTAGANTT
\newglossaryentry{Instagantt}{
	name={Instagantt},
	description={Applicazione integrabile con Asana che permette di creare, modificare e visualizzare diagrammi di Gantt sincronizzati con le attività programmate in Asana}
}
%FINE INSTAGANTT

%INIZIO GANTT
\newglossaryentry{Gantt}{
	name={Diagrammi di Gantt},
	description={Strumento di supporto alla gestione dei progetti. \'{E} costituito partendo da un asse orizzontale a rappresentazione dell'arco temporale totale del progetto, suddiviso in fasi incrementali, e da un'asse verticale a rappresentazione delle mansioni o attività che costituiscono il progetto}
}
%FINE GANTT

%INIZIO REQUISITI
\newglossaryentry{requisiti}{
	name={Requisiti},
	description={Funzionalità che il prodotto software deve offrire}
}
%FINE REQUISITI

%INIZIO STANDUPS
\newglossaryentry{standups}{
	name={Stand-up Meeting },
	description={Alcune metodologie di sviluppo del software utilizzano brevi riunioni tra i partecipanti al progetto per fare il punto della situazione. A turno, vengono esposti problemi e dubbi nati durante il lavoro di sviluppo e conseguentemente coordinati gli sforzi per risolvere questi problemi. }
}
%FINE STANDUPS

%INIZIO GITHUB
\newglossaryentry{GitHub}{
	name={GitHub},
	description={Servizio di web hosting per lo sviluppo di progetti software che usa il sistema di controllo di versione Git}
}
%FINE GITHUB

%INIZIO UNIWEB
\newglossaryentry{Uniweb}{
	name={Uniweb},
	description={Portale dell'Università degli Studi di Padova disponibile all'indirizzo web \nURI{https://uniweb.unipd.it/Home.do} che permette a studenti di gestire la propria carriera universitaria, come visuailizzare il libretto e iscriversi agli appelli d'esame e ai docenti di predisporre le liste e di inserire i risultati degli appelli d'esame}
}
%FINE UNIWEB

%INIZIO ANNO
\newglossaryentry{Anno}{
	name={Anno},
	description={Con il termine anno, o anno accademico, si definisce un periodo di tempo della durata approssimativamente di un anno nel quale l'università svolge le proprie attività}
}
%FINE ANNO

%INIZIO CORSO
\newglossaryentry{Corso}{
	name={Corso},
	description={Con il termine corso, o corso di laurea, si definisce un percorso di studi che porta ad una determinata laurea}
}
%FINE CORSO

%INIZIO ESAME
\newglossaryentry{Esame}{
	name={Esame},
	description={Con il termine esame, o esame universitario, si definisce un insieme di lezioni frontali ed eventualmente laboratori che veicolano delle informazioni agli studenti e si concludono con degli appelli}
}
%FINE ESAME

%INIZIO BLOCKCHAIN
\newglossaryentry{blockchain}{
	name={Blockchain},
	description={Registro pubblico e distribuito delle transizioni e dei contratti sulla quale si basa una cryptovaluta. Quanto utilizzata nei documenti allegati si fa sempre riferimento alla blockchain alla base di Ethereum}
}
%FINE BLOCKCHAIN

%INIZIO BLOCCO
\newglossaryentry{blocco}{
	name={Blocco},
	description={Parte di una blockchain che contiene le transizioni. Per essere aggiunto alla blockchain deve avere l'approvazione da parte dei miner}
}
%FINE BLOCCO

%INIZIO MINING
\newglossaryentry{mining}{
	name={Mining},
	description={Processo con il quale si va a confermare la validità dei blocchi e quindi ad inserirli nella blockchain, in caso di esito negativo si procede all'eliminazione del blocco non ancora confermato}
}
%FINE MINING

%INIZIO MININGFARM
\newglossaryentry{mining farm}{
	name={Mining},
	description={Insieme, solitamente di grosse dimensione, di hardware dedicato ad attività di mining}
}
%FINE MININGFARM

%INIZIO MINER
\newglossaryentry{miner}{
	name={Miner},
	description={Entità che svolge attività di mining, può essere utilizzato per indicare l'hardware oppure la persona che possiede una mining farm}
}
%FINE MINER

%INIZIO TRANSAZIONE
\newglossaryentry{transazione}{
	name={Transazione},
	description={Messaggio che richiede ai miner una operazione di alterazione della blockchain, come ad esempio l'interazione con un contratto che ne modifica lo stato oppure un trasferimento di Ether}
}
%FINE TRANSAZIONE

%INIZIO CHIAVEPRIVATA
\newglossaryentry{chiave privata}{
	name={Chiave privata},
	description={Anche abbreviata in chiave, è una stringa esadecimale di 65 caratteri utilizzata per firmare i messaggi, in modo da garantire l'autenticità}
}
%FINE CHIAVEPRIVATA

%INIZIO CHIAVEPUBBLICA
\newglossaryentry{chiave pubblica}{
	name={Chiave pubblica},
	description={Anche abbreviata in indirizzo utente o account, è una stringa esadecimale di 41 carattere utilizzata per l'identificazione. Viene utilizzata anche per gli trasferimenti di Ether}
}
%FINE CHIAVEPUBBLICA

%INIZIO INDIRIZZO
\newglossaryentry{Indirizzo}{
	name={Indirizzo},
	description={\`{E} una stringa esadecimale di 41 carattere utilizzata per l'identificazione di contratti o di account}
}
%FINE INDIRIZZO

%INIZIO SMARTCONTRACT
\newglossaryentry{smart contract}{
	name={Smart contract},
	description={Protocolli per l'esecuzione di un contratto, il quale viene gestito in modo automatico senza bisogno dell'intervento umano, rimuovendo così la possibilità di violarlo o di manometterlo}
}
%FINE SMARTCONTRACT


%INIZIO ETHEREUM
\newglossaryentry{Ethereum}{
	name={Ethereum},
	description={Piattaforma decentralizzata per la gestione di smart contract e di cryptovaluta Ether}
}
%FINE ETHEREUM

%INIZIO EVM
\newglossaryentry{EVM}{
	name={Ethereum Virtual Machine},
	description={Macchina virtuale sulla quale vengono eseguiti gli smart contract. Attualmente la loro esecuzione è assegnata prevalentemente a processori grafici}
}
%FINE EVM

