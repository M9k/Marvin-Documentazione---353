\documentclass[PianoDiProgetto.tex]{subfiles}

\begin{document}

\chapter{Modello di sviluppo}
Come modello di sviluppo per il ciclo di vita del software è stato deciso di adottare il \textbf{modello incrementale}, per garantire qualità, conformità e maturità del prodotto.

\section{Modello incrementale}
In un modello di sviluppo incrementale inizialmente viene pianificato il numero di incrementi da eseguire, sulla base dei \citGloss{requisiti} fondamentali e quelli desiderabili del prodotto software richiesti dal proponente. A ciascuno di questi viene poi associata una priorità in base alla loro utilità in quel determinato frangente temporale.\\
I \citGloss{requisiti} da sviluppare vengono poi suddivisi nei vari incrementi, privilegiando quelli con priorità maggiore. La consegna del prodotto quindi non viene eseguita tutta insieme, ma è anch'essa incrementale.\\
Durante la fase di sviluppo di un incremento non è possibile modificare i \citGloss{requisiti} decisi prima di iniziare lo sviluppo dell'incremento corrente, ma è invece accettabile aggiungere dei \citGloss{requisiti} da sviluppare negli incrementi successivi. Al termine dello sviluppo, l'incremento viene aggiunto al prodotto software, dimostrandone il grado di efficacia. Se il prodotto non è finito, si procederà poi con gli incrementi successivi. I \textbf{vantaggi} principali di questo modello sono principalmente due: sviluppando inizialmente i \citGloss{requisiti} di maggiore importanza e priorità essi saranno soggetti a maggiori verifiche e producendo rilasci continui è possibile avere anticipatamente un prototipo del prodotto.
\nImg{images/modIncrementale.png}{Modello incrementale}{8.2}

\end{document}