\documentclass[VER-2017-12-08.tex]{subfiles}
\begin{document}

\chapter{Informazioni sulla riunione}
\begin{itemize}
	\item \textbf{Motivo della riunione:} \`{E} stata indetta questa riunione a scopo di conoscenza reciproca dei membri del gruppo, per decidere il nome e il logo del gruppo, i principali strumenti organizzativi e il capitolato per il progetto;
	\item \textbf{Luogo e data:} Aula Luf1 in via Luzzati a Padova;
	\item \textbf{Ora di inizio:} 13:15
	\item \textbf{Ora di fine:} 15:00
	\item \textbf{Partecipanti del gruppo \gruppo:}
	\begin{itemize}
		\item \Davide;
		\item \Elena;
		\item \Gianluca;
		\item \Mirco;
		\item \Parwinder;
		\item \Riccardo;
		\item \Valentina.
	\end{itemize}
\end{itemize}
\chapter{Ordine del giorno}	
Di seguito sono riportati i punti dell'ordine del giorno che sono stati discussi durante la riunione:
\begin{enumerate}
	\item Nome e logo del gruppo;
	\item Strumenti organizzativi;
	\item Capitolato;
\end{enumerate}
\chapter{Resoconto}
\begin{enumerate}
	\item \textbf{Nome e logo del gruppo:} per identificare il gruppo durante la formazione il giorno \nData{3}{11}{2017} era stato dato il numero 11. Il nome scelto, vale a dire \gruppo\ (si pronuncia ``Tre-cinque-tre'') deriva proprio da questo. La somma delle tre cifre dà come risultato 11 e fa riferimento anche ad una citazione nascosta di una serie tv chiamata ``Stranger Things'' a cui tutti i membri del gruppo sono molto appassionati. Infatti, per il logo il gruppo si è ispirato a quello di questa serie tv riadattandolo al nome scelto.
	\item \textbf{Strumenti organizzativi:} sono stati scelti i seguenti strumenti organizzativi:
	\begin{itemize}
		\item \textbf{Account Google:} è stato creato un account Google con indirizzo \mailgroup\ e accessibile a tutti i membri del gruppo;
		\item \textbf{\citGloss{Asana}:} è stato scelta questa applicazione per predisporre una panoramica per il controllo dei task;
		\item \textbf{\citGloss{GitHub}:} scelto come strumento per il versionamento;
		\item \textbf{\citGloss{Slack}:} scelto per la comunicazione, perché permette la creazione di diversi canali e l'integrazione con GitHub e Asana.
	\end{itemize}
	\item \textbf{Capitolato:} tutti i membri del gruppo hanno compilato prima della riunione una tabella scrivendo per ogni capitolato degli aspetti positivi e negativi. Alla riunione, visionando la tabella, sono stati analizzati tutti i capitolati proposti e alla fine la scelta è ricaduta sul C6 con il progetto \progetto\ proposto dall'azienda \Proponente. Il gruppo ha ritenuto interessante l'utilizzo della \citGloss{blockchain} \citGloss{Ethereum} e del grande numero di tecnologie innovative proposte. 
\end{enumerate} 
\end{document}