\documentclass[main.tex]{subfiles}

\begin{document}
\chapter{Capitolato C1 - Ajarvis}
\section{Descrizione generale}
Il capitolato C1 si pone l'obbiettivo di sviluppare un'applicazione con Machine Learning in grado di ascoltare standup giornalieri, analizzandone i dialoghi e fornendo un'analisi e dei report associati.\\
Questo permetterebbe di evidenziare dinamiche comune ai progetti evitando perdita di dati.
%\section{Descrizione generale}

\section{Obiettivo finale}
L’obiettivo è la realizzazione di un'applicazione in grado di registrare l'audio dello standup giornaliero permettendo in tempo reale o successivamente di effettuare conversione speech-to-text framite Google Cloud Speech API Google per poi effettuare analisi del dialogo testuale basata su Google Natural Language API.\\
A questo punto l'utente potrà visualizzare tramite interfaccia web per i report delle analisi effettuata. 
%\section{Obiettivo finale}

\section{Tecnologie richieste}
\begin{itemize}
	\item \textbf{Google Cloud Platform}, insieme di API per store dati, store DB e conversione speech-to-text oltre che algoritmi di machine learning.
	\item \textbf{ Framework Node.js Express}, framework per linguaggio Node.js 
	\item \textbf{ HTML5 e CSS3 con Twitter Bootstrap}, framework sviluppo UI per siti web.
\end{itemize}
%\section{Tecnologie richieste}

\section{Valutazione finale}
Il gruppo 353 ha valutato positivamente la proposta poiché il progetto faceva uso di tecnologie innovative come l'uso dell'intelligenza artificiale tramite API google.
Inoltre il prodotto sarebbe stato interessante potendo essere usato in molti ambiti futuri.\\
Dopo aver analizzato maggiormente le specifiche è stato scelto di scartare il capitolato perché pur usando tecnologie di machine learning  è stato ritenuto complesso realizzare soluzioni migliori di quelle preesistenti.
%\section{Valutazione finale}

\end{document}