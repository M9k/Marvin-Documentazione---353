\documentclass[PianoDiQualifica.tex]{subfiles}

\begin{document}

\chapter{Qualità di prodotto}

\section{Scopo}
Per garantire una buona qualità di prodotto, il gruppo \gruppo ha individuato dallo standard ISO/IEC 9126 le qualità che ritiene più importanti nell’arco del ciclo di vita del prodotto e le ha istanziate individuando obiettivi e metriche coerenti con i livelli di qualità perseguiti.

\section{Qualità dei documenti}
I documenti prodotti dal gruppo \gruppo dovranno essere leggibili, comprensibili e corretti dal punto di vista ortografico, sintattico, logico e semantico.
\subsection{Comprensione}
\subsubsection{Obiettivi di qualità}
\begin{itemize}
	\item \textbf{Leggibilità:} i documenti prodotti dovranno essere leggibili e comprensibili a persone con licenza di istruzione media;
	\item \textbf{Correttezza ortografica:} i documenti prodotti non dovranno contenere errori ortografici.
\end{itemize}

\subsubsection{Metriche}
\begin{itemize}
	\item \textlink{MPDD001TAB}{MPDD001}{\textbf{MPDD001 Indice di Gulpease:}} è l'indice di leggibilità tarato sulla lingua italiana. Considera due variabili linguistiche: la lunghezza della parola e la lunghezza della frase rispetto al numero di lettere. La formulata per il suo calcolo è la seguente:
	\[IG=89+\dfrac{300*N_F-10*N_L}{N_P}\] Dove $ N_F $ è il numero delle frasi, $ N_L $ il numero delle lettere e $ N_P $ il numero delle parole. Il risultato $I$ è un numero compreso tra 0 e 100. In generale risulta che i testi con indice inferiore a:
	\begin{itemize}
		\item 80 sono difficili da leggere per chi ha una licenza elementare;
		\item 60 sono difficili da leggere per chi ha una licenza media;
		\item 40 sono difficili da leggere per chi ha un diploma superiore.
	\end{itemize}	
	\item \textlink{MPDD002TAB}{MPDD002}{\textbf{MPDD002 Formula di Flesch:}} è una formula che serve per misurare la leggibilità di un testo in inglese:
	\[F=206,835-(0,846*S)-(1,015*P)\] Dove $ S $ è il numero delle sillabe, calcolato su un campione di 100 parole e $ P $ è il numero medio di parole per frase.
	La leggibilità è alta se $F$ è superiore a 60, media se fra 50 e 60, bassa sotto a 50;
	\item \textlink{MPDD003TAB}{MPDD003}{\textbf{MPDD003 Errori ortografici:}} gli errori ortografici possono essere identificati tramite lo strumento \textquoteleft Controllo ortografico\textquoteright\ presente in \citGloss{TexStudio}. Sarà poi compito del Verificatore correggerli.  	
\end{itemize}
		
\section{Qualità del software}
\subsection{Funzionalità}
Rappresenta la capacità del prodotto di fornire tutte le funzionalità che sono state individuate attraverso l'\adr.	
\subsubsection{Obiettivi qualità}
Il gruppo \gruppo si impegnerà per perseguire:
\begin{itemize}
	\item \textbf{Adeguatezza}: le funzionalità fornite siano conformi rispetto le aspettative;
	\item \textbf{Accuratezza}: il prodotto fornisca i risultati attesi, con il livello di dettaglio richiesto. 
\end{itemize}	
\subsubsection{Metriche}
\begin{itemize}
	\item \textlink{MPDS001TAB}{MPDS001}{\textbf{MPDS001 Copertura requisiti obbligatori:}} indica la percentuale dei requisiti obbligatori coperti dall'implementazione. La sua formula di misurazione è la seguente: \[CRO=(\frac{N_{ROS}}{N_{RO}})*100\] Dove $ N_{ROS} $ è il numero di requisiti obbligatori soddisfatti e $ N_{RO} $ è il numero totale dei requisiti obbligatori;
	\item \textlink{MPDS002TAB}{MPDS002}{\textbf{MPDS002 Copertura requisiti accettati:}} indica la percentuale dei requisiti desiderabili e facoltativi coperti dall'implementazione. La sua formula di misurazione è la seguente: \[CRA=(\frac{N_{RAS}}{N_{RA}})*100\] Dove $ N_{RAS} $ è il numero di requisiti accettati soddisfatti e $ N_{RA } $ è il numero totale dei requisiti accettati;
	\item \textlink{MPDS003TAB}{MPDS003}{\textbf{MPDS003 Accuratezza rispetto alle attese:}} indica la percentuale di risultati concordi alle attese. La sua formula di misurazione è la seguente: \[ARA=(1-\frac{N_{TD}}{N_{TE}})*100\] Dove $ N_{TD} $ è il numero di test che producono risultati discordi alle attese e $ N_{TE} $ è il numero di test-case eseguiti.
\end{itemize}
		
\subsection{Affidabilità}
Rappresenta la capacità del prodotto software di svolgere correttamente le sue funzioni durante il suo utilizzo, anche in caso in cui si presentino situazioni anomale.
\subsubsection{Obiettivi di qualità}
L'esecuzione del prodotto dovrà presentare le seguenti caratteristiche:
\begin{itemize}%TODO failure e fault glossario(slide tullio)?
	\item \textbf{Maturità:} evitare che si verifichino malfunzionamenti, operazioni illegali e failure in seguito a fault;
	\item \textbf{Tolleranza agli errori:} nel caso in cui si presentino degli errori, dovuti a guasti o ad un uso scorretto dell'applicativo, questi devo essere gestiti in modo da mantenere alto il livello di prestazioni.
\end{itemize}
\subsubsection{Metriche}
\begin{itemize}
	\item \textlink{MPDS004TAB}{MPDS004}{\textbf{MPDS004 Densità di failure:}} indica la percentuale di testing che si sono concluse in failure. La sua formula di misurazione è la seguente: \[DF=(\frac{N_{FR}}{N_{TE}})*100\] Dove $ N_{FR} $ è il numero di failure rilevati durante l'attività di testing e $ N_{TE} $ è il numero di test-case eseguiti;
	\item \textlink{MPDS005TAB}{MPDS005}{\textbf{MPDS005 Blocco di operazioni non corrette:}} indica la percentuale di funzionalità in grado di gestire correttamente i fault che potrebbero verificarsi. La sua formula di misurazione è la seguente: \[BNC=(\frac{N_{FE}}{N_{ON}})*100\] Dove $ N_{FE} $ è il numero di failure evitati durante i test effettuati e $ N_{ON} $ è il numero di test-case eseguiti che prevedono l'esecuzione di operazioni non corrette, causa di possibili failure.
\end{itemize}
\subsection{Usabilità}
Rappresenta la capacità del prodotto di essere facilmente comprensibile ed attraente in ogni sua parte per qualsiasi utente che lo andrà ad utilizzare.
\subsubsection{Obiettivi di qualità}
Il prodotto dovrà puntare ai seguenti obiettivi di usabilità:
\begin{itemize}
	\item \textbf{Comprensibilità:} l'utente deve essere in grado di riconoscere le funzionalità offerte dal software e deve comprendere le modalità di utilizzo per raggiungere i risultati attesi;
	\item \textbf{Apprendibilità:} deve essere data la possibilità all'utente di imparare ad utilizzare l'applicazione senza troppo impegno;
	\item \textbf{Operabilità:} le funzioni presenti devono essere coerenti con le aspettative dell'utente;
	\item \textbf{Attrattiva:} il software deve essere piacevole per chi ne fa uso.
\end{itemize}
\subsubsection{Metriche}
\begin{itemize}
	\item \textlink{MPDS006TAB}{MPDS006}{\textbf{MPDS006 Comprensibilità delle funzioni offerte:}} indica la percentuale di operazioni comprese in modo immediato dall'utente, senza la consultazione del manuale. La sua formula di misurazione è la seguente: \[CFC=(\frac{N_{FC}}{N_{FO}})*100\] Dove $ N_{FC} $ è il numero di funzionalità comprese in modo immediato dall'utente durante l'attività di testing del prodotto e $ N_{FO} $ è il numero di funzionalità offerte dal sistema;
	\item \textlink{MPDS007TAB}{MPDS007}{\textbf{MPDS007 Facilità di apprendimento delle funzionalità:}} indica il tempo medio impiegato dall'utente nell'imparare ad usare correttamente una data funzionalità. Si misura tramite un indicatore numerico che indica i minuti impiegati da un utente per apprendere il funzionamento di una certa funzionalità;
	\item \textlink{MPDS008TAB}{MPDS008}{\textbf{MPDS008 Consistenza operazionale in uso:}} indica la percentuale di messaggi e funzionalità offerte all'utente che rispettano le sue aspettative riguardo al comportamento del software. La sua formula di misurazione è la seguente: \[COU=(\frac{N_{MFU}}{N_{MFO}})*100\] Dove $ N_{MFU} $ è il numero di messaggi e funzionalità che non rispettano le aspettative dell'utente e $ N_{MFO} $ è il numero di messaggi e funzionalità offerte dal sistema.	
\end{itemize}	
\subsubsection{Misurazione}
Queste metriche di usabilità misurate tramite sessioni di prova con utenti esterni per dar modo di ottenere feedback reali e misurazioni attendibili.\\ Ogni sessione sarò registrata per uso interno al gruppo tramite screen recorder e microfono, gli utenti sarannò istruiti sul cercare e/o utilizzare una serie di funzionalità tracciate da un Verificatore.
Combinando le misurazioni effettuate si determineranno i valori per le metriche sopra descritte.

\subsection{Efficienza}
Rappresenta la capacità di eseguire le funzionalità offerte dal software nel minor tempo possibile utilizzando al tempo stesso il minor numero di risorse disponibili.
\subsubsection{Obiettivi di qualità}
Il prodotto dovrà essere efficiente, in particolare:
\begin{itemize}
	\item \textbf{Comportamento rispetto al tempo:} per svolgere le sue funzioni il software deve fornire adeguati tempi di risposta ed elaborazione;
	\item \textbf{Utilizzo delle risorse:} il software quando esegue le sue funzionalità deve utilizzare un appropriato numero e tipo di risorse.
\end{itemize}
\subsubsection{Metriche}
\begin{itemize}
	\item \textlink{MPDS009TAB}{MPDS009}{\textbf{MPDS009 Tempo di risposta:}} indica il tempo medio che intercorre fra la richiesta software di una determinata funzionalità e la restituzione del risultato all'utente. La sua formula di misurazione è la seguente: \[TR=\frac{\sum_{i=1}^n T_i}{n}\] Dove $ T_i $ è il tempo intercorso fra la richiesta $ i $ di una funzionalità ed il comportamento delle operazioni necessarie a restituire un risultato a tale richiesta.	
\end{itemize}
\subsection{Manutenibilità}
Rappresenta la capacità del prodotto di essere modificato, tramite correzioni, miglioramenti o adattamenti del software a cambiamenti negli ambienti, nei \citGloss{requisiti} e nelle specifiche funzionali.
\subsubsection{Obiettivi di qualità}
Le operazioni di manutenzione andranno agevolate il più possibile adottando le seguenti caratteristiche:
\begin{itemize}
	\item \textbf{Analizzabilità:} il software deve consentire una rapida identificazione delle possibili cause di errori e malfunzionamenti;
	\item \textbf{Modificabilità:} il prodotto originale deve permettere eventuali cambiamenti in alcune sue parti;
	\item \textbf{Stabilità:} non devono insorgere effetti indesiderati in seguito a modifiche effettuate sul software;
	\item \textbf{Testabilità:} il software deve poter essere facilmente testato per valiare le modifiche effettuate.
\end{itemize}
\subsubsection{Metriche}
\begin{itemize}
	\item \textlink{MPDS010TAB}{MPDS010}{\textbf{MPDS010 Capacità di analisi di failure:}} indica la percentuale di modifiche effettuate in risposta a failure che hanno portato all'introduzione di nuove failure in altre componenti del sistema. La sua formula di misurazione è la seguente: \[CAF=(\frac{N_{FI}}{N_{FR}})*100\] Dove $ N_{FI} $ è il numero di failure delle quali sono state individuate le cause e $ N_{FR} $ è il numero di failure rilevate;
	\item \textlink{MPDS011TAB}{MPDS011}{\textbf{MPDS011 Impatto delle modifiche:}} indica la percentuale di modifiche effettuate in risposta a failure che hanno portato all'introduzione di nuove failure in altre componenti del sistema. La sua formula di misurazione è la seguente: \[IM=(\frac{N_{FRF}}{N_{FR}})*100\] Dove $ N_{FRF} $ è il numero di failure risolte con l'introduzione di nuove failure e $ N_{FR} $ è il numero di failure risolte.
\end{itemize}
\subsection{Portabilità}
Rappresenta la capacità del software di poter essere utilizzato su diversi ambienti.
\subsubsection{Obiettivi di qualità}
Sarò agevolata la portabilità del prodotto adottando i seguenti obiettivi:
\begin{itemize}
	\item \textbf{Adattabilità:} il prodotto deve adattarsi a tutti quelli ambienti di lavoro nei quali è stato previsto un suo utilizzo, senza dover apportare modifiche dello stesso;
	\item \textbf{Sostituibilità:} l'applicativo deve poter sostituire un altro software che ha lo stesso scopo e lavora nel medesimo ambiente.
\end{itemize}
\subsubsection{Metriche}
\begin{itemize}
	\item \textlink{MPDS012TAB}{MPDS012}{\textbf{MPDS012 Versioni dei browser supportate:}} indica la percentuale di versioni di browser attualmente supportate, fra quelle individuate dai requisiti. La sua formula di misurazione è la seguente: \[VB=(\frac{N_{VS}}{N_{VI}})*100\] Dove $ N_{VS} $ è il numero di versioni di browser supportate dal prodotto e $ N_{VI} $ è il numero di versioni di browser che devono essere supportate dal prodotto;
	\item \textlink{MPDS013TAB}{MPDS013}{\textbf{MPDS013 Inclusione di funzionalità da altri prodotti:}} indica la percentuale del software utilizzato in precedenza dall'utente che produce risultati simili a quelli ottenuti dal prodotto in oggetto. La sua formula di misurazione è la seguente: \[IFP=(\frac{N_{FPA}}{N_{FPP}})*100\] Dove $ N_{FPA} $ è il numero di funzionalità del software utilizzato in precedenza dall'utente che produce risultati simili a quelli ottenuti dal prodotto in oggetto e $ N_{FPP} $ è il numero di funzionalità offerte dal software utilizzato in precedenza dall'utente. 
\end{itemize}
\newpage
\section{Tabella delle metriche}
% ID METRICA
% M --> metrica
% PD --> prodotto
% D/S --> documenti/software (macrocategorie)
% 00n --> numero incrementale
Questa tabella indica i \textbf{range} di accettazione e di ottimalità per le metriche utilizzate per la qualità di prodotto:
\begin{table}[H]
	\begin{center}
		\begin{tabu} to \textwidth {
				>{\centering}m{0.15\linewidth}
				>{\centering}m{0.45\linewidth}
				>{\centering}m{0.18\linewidth} 
				>{\centering\arraybackslash}m{0.17\linewidth}
			}
			\tableHeaderStyle
			\textbf{ID} & \textbf{Nome} & \textbf{Range di accettazione} & \textbf{Range di ottimalità}\\
			\multicolumn{4}{c}{\textbf{Documenti}}\\
			\textlink{MPDD001}{MPDD001TAB}{MPDD001} & Indice di Gulpease & 50 - 100 & 60 - 100\\
			\textlink{MPDD002}{MPDD002TAB}{MPDD002} & Formula di Flesch & 40 - 60 & 50 - 60\\ 
			\textlink{MPDD003}{MPDD003TAB}{MPDD003} & Errori ortografici & 100\% corretti & 100\% corretti\\ 
		
			\hline
			\multicolumn{4}{c}{\textbf{Software}}\\
			\textlink{MPDS001}{MPDS001TAB}{MPDS001} & Copertura requisiti obbligatori & 100\% & 100\%\\
			\textlink{MPDS002}{MPDS002TAB}{MPDS002} & Copertura requisiti accettati & 60\% - 100\% & 80\% - 100\%\\
			\textlink{MPDS003}{MPDS003TAB}{MPDS003} & Accuratezza rispetto alle attese & 90\% - 100\% & 100\%\\
			\textlink{MPDS004}{MPDS004TAB}{MPDS004} & Densità di failure & 0\% - 10\% & 0\% \\
			\textlink{MPDS005}{MPDS005TAB}{MPDS005} & blocco di operazioni non corrette & 80\% - 100\% & 100\%\\
			\textlink{MPDS006}{MPDS006TAB}{MPDS006} & Comprensibilità delle funzioni offerte & 80\% - 100\% & 90\% - 100\%\\
			\textlink{MPDS007}{MPDS007TAB}{MPDS007} & Facilità di apprendimento delle funzionalità & 0 - 20 min & 0 - 10 min\\
			\textlink{MPDS008}{MPDS008TAB}{MPDS008} & Consistenza operazionale in uso & 80\% - 100\% & 90\% - 100\%\\  
			\textlink{MPDS009}{MPDS009TAB}{MPDS009} & Tempo di risposta & 0 - 8 sec & 0 - 3 sec \\
			\textlink{MPDS010}{MPDS010TAB}{MPDS010} & Capacità di analisi di failure & 60\% - 100\% & 80\% - 100\% \\
			\textlink{MPDS011}{MPDS011TAB}{MPDS011} & Impatto delle modifiche & 0\% - 20\% & 0\% - 10\% \\
			\textlink{MPDS012}{MPDS012TAB}{MPDS012} & Versioni di browser supportate & 70\% - 100\% & 100\%\\
			\textlink{MPDS013}{MPDS013TAB}{MPDS013} & Inclusione di funzionalità da altri prodotti & 80\% - 100\% & 90\% - 100\% \\
		\end{tabu}
		\caption{Tabella delle metriche della qualità di prodotto}
		\vspace{-1em}
	\end{center}
\end{table}

\section{The Twelve-Factor App}
Nel capitolato è stata citata la documentazione delle "The Twelve-Factor App", ed è stato chiarito con la Proponente, nell'incontro descritto nel \textbf{VER-2018-01-09}, che la documentazione sarà da ritenersi minimale e limitata a delle brevi descrizioni nella sezione seguente nelle prossime fasi.\\
Il lavoro su queste sarà concentrato sul rispetto di questi principi, analizzati e riassunti in appendice al documento.\\\\
Di questi non saranno applicabili totalmente o parzialmente i seguenti:
\begin{itemize}
	\item \textbf{5:} server su cui gira l'applicazione non dovrà eseguire nulla se non inviare solamente file \citGloss{HTML} e \citGloss{JavaScript};
	\item \textbf{6:} l'applicazione non possiede processi non avendo uno stato;
	\item \textbf{7:} non è possibile il binding delle porte usando Metamask come unico protocollo verso server esterno Infuria;
	\item \textbf{8:} concorrenza non realizzabile sviluppando un prodotto decentralizzato;
	\item \textbf{9:} è inapplicabile, non c'è nulla da fermare o far partire, non c'è nulla che crasha sul lato server;
	\item \textbf{11:} il loggin non sarà perseguibile in remoto nativamente all'architettura decentralizzata dovendo quindi appoggiarsi a servizi di log esterno in caso di necessità, sarà possibile localmente nel broswer.
\end{itemize}

\end{document}