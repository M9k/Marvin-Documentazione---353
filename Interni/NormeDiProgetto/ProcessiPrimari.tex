\documentclass[NormeDiProgetto.tex]{subfiles}

\begin{document}

\chapter{Processi primari}

\section{Fornitura}
\subsection{Studio di fattibilità}
\subsection[Rapporti di fornitura con la Proponente Red Babel]{Rapporti di fornitura con la Proponente \\ Red Babel}
\subsection{Documentazione fornita}

\section{Sviluppo}
\subsection{Analisi dei Requisiti}
\subsection{Codifica}
In questa sotto-sezione vengono elencate le norme alle quali i Programmatori devono attenersi durante l'attività di programmazione e implementazione.\\
Ogni norma è rappresentata da un paragrafo. Ciascuna ha un titolo, una breve descrizione e, se necessario, un esempio che illustra i modi accettati o meno. Alcune di esse includono inoltre una lista di possibili eccezioni d'uso. L'uso di norme e convenzioni è fondamentale per permettere la generazione di codice leggibile e uniforme, agevolare le fasi di manutenzione, verifica e validazione e migliorare la qualità del prodotto.
\subfile{Codifica/Javascript.tex}
\subfile{Codifica/React.tex}
\subfile{Codifica/Solidity.tex}
\subfile{Codifica/Scss.tex}

\subsection{Progettazione}

\end{document}