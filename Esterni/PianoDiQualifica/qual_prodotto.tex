\documentclass[PianoDiQualifica.tex]{subfiles}

\begin{document}

\chapter{Qualità di prodotto}

\section{Scopo}
Per garantire una buona qualità di prodotto, il gruppo \gruppo ha individuato dallo standard ISO/IEC 9126 le qualità che ritiene più importanti nell'arco del ciclo di vita del prodotto e le ha istanziate individuando obiettivi e metriche coerenti con i livelli di qualità perseguiti.

\section{Prodotti}

\subsection{Qualità dei documenti}
I documenti prodotti dal gruppo \gruppo dovranno essere leggibili, comprensibili e corretti dal punto di vista ortografico, sintattico, logico e semantico.
\subsubsection {Obiettivi di qualità: comprensione}
\begin{itemize}
	\item \textbf{Leggibilità:} i documenti prodotti dovranno essere leggibili e comprensibili a persone con licenza di istruzione media;
	\item \textbf{Correttezza ortografica:} i documenti prodotti non dovranno contenere errori ortografici.
\end{itemize}
Verranno utilizzate le seguenti metriche definite nelle \ndp alla \S{3.5}:
\begin{itemize}
	\item \textlink{MPDD001TAB}{MPDD001}{MPDD001} Indice di Gulpease;
	\item \textlink{MPDD002TAB}{MPDD002}{MPDD002} Formula di Flesch;
	\item \textlink{MPDD003TAB}{MPDD003}{MPDD003} Errori ortografici.
\end{itemize}

\subsection{Qualità del software}
\subsubsection{Funzionalità:} Rappresenta la capacità del prodotto di fornire tutte le funzionalità che sono state individuate attraverso l'\adr.	
\paragraph{Obiettivi qualità}
Il gruppo \gruppo si impegnerà per perseguire:
\begin{itemize}
	\item \textbf{Adeguatezza}: le funzionalità fornite siano conformi rispetto le aspettative;
	\item \textbf{Accuratezza}: il prodotto fornisca i risultati attesi, con il livello di dettaglio richiesto. 
\end{itemize}
Verranno utilizzate le seguenti metriche definite nelle \ndp alla \S{3.5}:
\begin{itemize}
	\item \textlink{MPDS001}{MPDS001TAB}{MPDS001} Copertura requisiti obbligatori;
	\item \textlink{MPDS002TAB}{MPDS002}{MPDS002} Copertura requisiti accettati;
	\item \textlink{MPDS003TAB}{MPDS003}{MPDS003} Accuratezza rispetto alle attese.
\end{itemize}

\subsubsection{Affidabilità}
Rappresenta la capacità del prodotto software di svolgere correttamente le sue funzioni durante il suo utilizzo, anche in caso in cui si presentino situazioni anomale.
\paragraph{Obiettivi di qualità}
L'esecuzione del prodotto dovrà presentare le seguenti caratteristiche:
\begin{itemize}
	\item \textbf{Maturità:} evitare che si verifichino malfunzionamenti, operazioni illegali e \citGloss{failure} in seguito a \citGloss{fault};
	\item \textbf{Tolleranza agli errori:} nel caso in cui si presentino degli \citGloss{errori}, dovuti a guasti o ad un uso scorretto dell'applicativo, questi devo essere gestiti in modo da mantenere alto il livello di prestazioni.
\end{itemize}
Verranno utilizzate le seguenti metriche definite nelle \ndp alla \S{3.5}:
\begin{itemize}
	\item \textlink{MPDS004TAB}{MPDS004}{MPDS004} Densità di failure;
	\item \textlink{MPDS005TAB}{MPDS005}{MPDS005} Blocco di operazioni non corrette.
\end{itemize}

\subsubsection{Usabilità}
Rappresenta la capacità del prodotto di essere facilmente comprensibile ed attraente in ogni sua parte per qualsiasi utente che lo andrà ad utilizzare.
\paragraph{Obiettivi di qualità}
Il prodotto dovrà puntare ai seguenti obiettivi di usabilità:
\begin{itemize}
	\item \textbf{Comprensibilità:} l'utente deve essere in grado di riconoscere le funzionalità offerte dal software e deve comprendere le modalità di utilizzo per raggiungere i risultati attesi;
	\item \textbf{Apprendibilità:} deve essere data la possibilità all'utente di imparare ad utilizzare l'applicazione senza troppo impegno;
	\item \textbf{Operabilità:} le funzioni presenti devono essere coerenti con le aspettative dell'utente;
	\item \textbf{Attrattiva:} il software deve essere piacevole per chi ne fa uso.
\end{itemize}
Verranno utilizzate le seguenti metriche definite nelle \ndp alla \S{3.5}:
\begin{itemize}
	\item \textlink{MPDS006TAB}{MPDS006}{MPDS006} Comprensibilità delle funzioni offerte;
	\item \textlink{MPDS007TAB}{MPDS007}{MPDS007} Facilità di apprendimento delle funzionalità;
	\item \textlink{MPDS008TAB}{MPDS008}{MPDS008} Consistenza operazionale in uso.
\end{itemize}

\subsubsection{Efficienza}
Rappresenta la capacità di eseguire le funzionalità offerte dal software nel minor tempo possibile utilizzando al tempo stesso il minor numero di risorse disponibili.
\paragraph{Obiettivi di qualità}
Il prodotto dovrà essere efficiente, in particolare:
\begin{itemize}
	\item \textbf{Comportamento rispetto al tempo:} per svolgere le sue funzioni il software deve fornire adeguati tempi di risposta ed elaborazione;
	\item \textbf{Utilizzo delle risorse:} il software quando esegue le sue funzionalità deve utilizzare un appropriato numero e tipo di risorse.
\end{itemize}
Verranno utilizzate le seguenti metriche definite nelle \ndp alla \S{3.5}:
\begin{itemize}
	\item \textlink{MPDS009TAB}{MPDS009}{MPDS009} Tempo di risposta.
\end{itemize}

\subsubsection{Manutenibilità}
Rappresenta la capacità del prodotto di essere modificato, tramite correzioni, miglioramenti o adattamenti del software a cambiamenti negli ambienti, nei \citGloss{requisiti} e nelle specifiche funzionali.
\paragraph{Obiettivi di qualità}
Le operazioni di manutenzione andranno agevolate il più possibile adottando le seguenti caratteristiche:
\begin{itemize}
	\item \textbf{Analizzabilità:} il software deve consentire una rapida identificazione delle possibili cause di errori e malfunzionamenti;
	\item \textbf{Modificabilità:} il prodotto originale deve permettere eventuali cambiamenti in alcune sue parti;
	\item \textbf{Stabilità:} non devono insorgere effetti indesiderati in seguito a modifiche effettuate sul software;
	\item \textbf{Testabilità:} il software deve poter essere facilmente testato per validare le modifiche effettuate.
\end{itemize}
Verranno utilizzate le seguenti metriche definite nelle \ndp alla \S{3.5}:
\begin{itemize}
	\item \textlink{MPDS010TAB}{MPDS010}{MPDS010} Capacità di analisi di failure;
	\item \textlink{MPDS011TAB}{MPDS011}{MPDS011} Impatto delle modifiche.
\end{itemize}

\subsubsection{Portabilità}
Rappresenta la capacità del software di poter essere utilizzato su diversi ambienti.
\paragraph{Obiettivi di qualità}
Sarò agevolata la portabilità del prodotto adottando i seguenti obiettivi:
\begin{itemize}
	\item \textbf{Adattabilità:} il prodotto deve adattarsi a tutti quelli ambienti di lavoro nei quali è stato previsto un suo utilizzo, senza dover apportare modifiche dello stesso;
	\item \textbf{Sostituibilità:} l'applicativo deve poter sostituire un altro software che ha lo stesso scopo e lavora nel medesimo ambiente.
\end{itemize}
Verranno utilizzate le seguenti metriche definite nelle \ndp alla \S{3.5}:
\begin{itemize}
	\item \textlink{MPDS012TAB}{MPDS012}{MPDS012} Versioni dei browser supportate;
	\item \textlink{MPDS013TAB}{MPDS013}{MPDS013} Inclusione di funzionalità da altri prodotti.
\end{itemize}

\section{Tabella delle metriche}
% ID METRICA
% M --> metrica
% PD --> prodotto
% D/S --> documenti/software (macrocategorie)
% 00n --> numero incrementale
Questa tabella indica i \textbf{range} di accettazione e di ottimalità per le metriche utilizzate per la qualità di prodotto:
\begin{table}[H]
	\begin{center}
		\begin{tabu} to \textwidth {
				>{\centering}m{0.15\linewidth}
				>{\centering}m{0.45\linewidth}
				>{\centering}m{0.18\linewidth} 
				>{\centering\arraybackslash}m{0.17\linewidth}
			}
			\tableHeaderStyle
			\textbf{ID} & \textbf{Nome} & \textbf{Range di accettazione} & \textbf{Range di ottimalità}\\
			\multicolumn{4}{c}{\textbf{Documenti}}\\
			\textlink{MPDD001}{MPDD001TAB}{MPDD001} & Indice di Gulpease & 50 - 100 & 60 - 100\\
			\textlink{MPDD002}{MPDD002TAB}{MPDD002} & Formula di Flesch & 40 - 60 & 50 - 60\\ 
			\textlink{MPDD003}{MPDD003TAB}{MPDD003} & Errori ortografici & 100\% corretti & 100\% corretti\\ 
		
			\hline
			\multicolumn{4}{c}{\textbf{Software}}\\
			\textlink{MPDS001}{MPDS001TAB}{MPDS001} & Copertura requisiti obbligatori & 100\% & 100\%\\
			\textlink{MPDS002}{MPDS002TAB}{MPDS002} & Copertura requisiti accettati & 60\% - 100\% & 80\% - 100\%\\
			\textlink{MPDS003}{MPDS003TAB}{MPDS003} & Accuratezza rispetto alle attese & 90\% - 100\% & 100\%\\
			\textlink{MPDS004}{MPDS004TAB}{MPDS004} & Densità di failure & 0\% - 10\% & 0\% \\
			\textlink{MPDS005}{MPDS005TAB}{MPDS005} & blocco di operazioni non corrette & 80\% - 100\% & 100\%\\
			\textlink{MPDS006}{MPDS006TAB}{MPDS006} & Comprensibilità delle funzioni offerte & 80\% - 100\% & 90\% - 100\%\\
			\textlink{MPDS007}{MPDS007TAB}{MPDS007} & Facilità di apprendimento delle funzionalità & 0 - 20 min & 0 - 10 min\\
			\textlink{MPDS008}{MPDS008TAB}{MPDS008} & Consistenza operazionale in uso & 80\% - 100\% & 90\% - 100\%\\  
			\textlink{MPDS009}{MPDS009TAB}{MPDS009} & Tempo di risposta & 0 - 8 sec & 0 - 3 sec \\
			\textlink{MPDS010}{MPDS010TAB}{MPDS010} & Capacità di analisi di failure & 60\% - 100\% & 80\% - 100\% \\
			\textlink{MPDS011}{MPDS011TAB}{MPDS011} & Impatto delle modifiche & 0\% - 20\% & 0\% - 10\% \\
			\textlink{MPDS012}{MPDS012TAB}{MPDS012} & Versioni di browser supportate & 70\% - 100\% & 100\%\\
			\textlink{MPDS013}{MPDS013TAB}{MPDS013} & Inclusione di funzionalità da altri prodotti & 80\% - 100\% & 90\% - 100\% \\
		\end{tabu}
		\caption{Tabella delle metriche della qualità di prodotto}
		\vspace{-1em}
	\end{center}
\end{table}

\section{The Twelve-Factor App}
Nel capitolato è stata citata la documentazione delle "The Twelve-Factor App", ed è stato chiarito con la Proponente, nell'incontro descritto nel \textbf{VER-2018-01-09}, che la documentazione sarà da ritenersi minimale e limitata a delle brevi descrizioni nella sezione seguente nelle prossime fasi.\\
Il lavoro su queste sarà concentrato sul rispetto di questi principi, analizzati e riassunti in appendice al documento.\\\\
Di questi non saranno applicabili totalmente o parzialmente i seguenti:
\begin{itemize}
	\item \textbf{5:} il server su cui gira l'applicazione non dovrà eseguire nulla se non inviare solamente file \citGloss{HTML} e \citGloss{JavaScript};
	\item \textbf{6:} l'applicazione non possiede processi non avendo uno stato;
	\item \textbf{7:} non è possibile il binding delle porte usando Metamask come unico protocollo verso il server esterno Infura;
	\item \textbf{8:} concorrenza non realizzabile sviluppando un prodotto decentralizzato;
	\item \textbf{9:} è inapplicabile, non c'è nulla da fermare o far partire, non c'è nulla che possa interrompere la sua esecuzione inavvertitamente sul lato server;
	\item \textbf{11:} il log non sarà perseguibile in remoto nativamente all'architettura decentralizzata dovendo quindi appoggiarsi a servizi di log esterno in caso di necessità, sarà possibile localmente nel browser.
\end{itemize}

\end{document}