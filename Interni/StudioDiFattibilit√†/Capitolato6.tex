\documentclass[StudioDiFattibilità.tex]{subfiles}
\begin{document}
\chapter{Capitolato C6 - Marvin}
\section{Descrizione generale}
Il capitolato C6 si pone l'obbiettivo di sviluppare una piattaforma web che simuli le funzionalità base offerte da Uniweb, utilizzando al posto del database la rete Ethereum attraverso l'uso di smart contract.\\
L'iterazione con il sistema avviene da parte dell'università stessa, dei professori e degli alunni e dovrà essere possibile l'aggiunta di corsi, l'assegnazione dei professori, la pubblicazione degli esiti degli esami e l'accettazione o meno dei voti.
\section{Obiettivo finale}
L'obbiettivo finale è la creazione di un sito web che possa accedere attraverso il plugin Metamask, disponibile per Google Chrome e Mozilla Firefox, alla rete Ethereum ed interagire con smart contract, al fine di simulare alcuni aspetti della piattaforma Uniweb riguardanti i corsi, gli esami e l'accettazione dei voti.\\
Tutta la business logic dovrà essere sulla rete Ethereum, il sito web servirà solamente per interfacciarsi ad essa, ne deriva la necessità di dover porre controlli e restrizioni direttamente sugli smart contract.\\
Il sistema dovrà funzionare su una simulazione locale della blockchain e sulla rete di test Ropsten, in quanto la sua implementazione sulla rete ufficiale Ethereum avrebbe un costo.
\section{Tecnologie richieste}
\begin{itemize}
	\item \textbf{Solidity}, linguaggio orientato agli oggetti per la scrittura di smart contract.
	\item \textbf{React}, libreria open source Javascript per il supporto dello strato di presentazione nell'ambito web.
	\item \textbf{Redux}, libreria open source Javascript per la gestione degli stati di React o AngularJS.
	\item \textbf{Sass}, estensione del linguaggio CSS che permette l'utilizzo di funzionalità aggiuntive.
	\item \textbf{Redux-minimal}, Boilerplate comprendente React, Redux, Sass e svariati altri componenti che facilitano lo sviluppo e lo svolgimento di test.
	\item \textbf{Truffle}, framework per lo sviluppo di smart contract su rete Ethereum.
	\item \textbf{Web3}, API Javascript per effettuare chiamate remote a un nodo Ethereum.
	\item \textbf{Metamask}, plugin per browser che permette un collegamento a un nodo di una rete Ethereum, compresi i nodi locali e la rete Ropsten.
\end{itemize}
\section{Valutazione finale}
Il capitolato ha avuto una accoglienza molto positiva da parte di tutti i membri del gruppo 353, soprattutto grazie alle numerose tecnologie innovative presenti, in particolare l'utilizzo di smart contract, la rete Ethereum e la libreria React.\\
Un altro aspetto accolto positivamente è il fatto di dover sviluppare una DApp a fine non unicamente finanziario, che risulta una tipologia di impiego delle blockchain ancora poco utilizzato, ma che potrebbe avere una enorme importanza nel futuro.\\
Sono comunque emersi alcune possibili problematiche, soprattutto riguardo il futuro della rete Ethereum, che attualmente risulta molto instabile come estensione e si teme che possa essere lasciata in secondo piano rispetto a blockchain che non implementano smart contract, come ad esempio Bitcoin e Litecoin.\\
Un ultimo aspetto negativo legato a questo capitolati è l'estrema innovazione di alcune tecnologie, soprattutto riguardanti la blockchain, che risultano ancora non ben definite quindi parzialmente incomplete di definizioni formali riguardanti l'utilizzo e senza best practice.
\end{document}