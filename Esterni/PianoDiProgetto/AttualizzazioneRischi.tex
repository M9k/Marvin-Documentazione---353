\documentclass[PianoDiProgetto.tex]{subfiles}
\begin{document}

\chapter{Attualizzazione dei rischi }
\taburowcolors[2] 2{tableLineOne .. tableLineTwo}
\tabulinesep = ^3mm_2mm
\begin{longtabu} to \linewidth {
		 >{\centering}m{.17\linewidth} 
		 >{\centering}m{.17\linewidth} 
		 >{\raggedright}m{.57\linewidth} 
	 }
	\caption[Attualizzazione dell'analisi dei rischi]{Attualizzazione dell'analisi dei rischi}
	\endlastfoot
	\rowfont{\bfseries\sffamily\leavevmode\color{white}}
	\rowcolor{tableHeader}
	\textbf{Codice} & \textbf{Periodo} & \textbf{Mitigazione} \\
	\endhead
	
	\rowcolor{tableLineOne} \textbf{RT001} & \textbf{Analisi} &  {
		Si è verificato un piccolo problema legato all'ambiente \citGloss{GitHub}. Alcuni membri non avendo esperienza con quest'ambiente, hanno avuto necessità di dedicare più tempo per imparare ad utilizzare questa tecnologia.
	}\\
	\hline

	\multicolumn{2}{c}{\textbf{Miglioramenti:}} & Problem solving in coppia tra due membri.
	 L'affiancamento anche remoto servirà a trasmettere conoscenze da membri più produttivi con altri in difficoltà dovendo essere chiari e concisi nella spiegazioni che altrimenti con video o guide potrebbero non rimuovere dubbi o incertezze.  

\\
\end{longtabu}

\end{document}