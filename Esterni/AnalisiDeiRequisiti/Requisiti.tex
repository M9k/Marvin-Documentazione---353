\documentclass[AnalisiDeiRequisiti.tex]{subfiles}

\begin{document}

\definecolor{CHeader}{HTML}{D22E2E}
\definecolor{CHeaderText}{HTML}{FFFFFF}
\definecolor{CRighePari}{HTML}{DFDFDF}
\definecolor{CRigheDispari}{HTML}{F4F4F4}
\definecolor{CCaption}{HTML}{FFFFFF}

\chapter{Requisiti}
\section{Classificazione dei requisiti}
I \citGloss{requisiti} vengono classificati e assegnati a un identificativo univoco secondo quanto definito nel documento \textit{Norme di progetto \vrquattro}.

\subsection{Requisiti funzionali}

\label{table:Tabella requisiti funzionali}
\rowcolors{2}{CRighePari}{CRigheDispari}
\renewcommand*{\arraystretch}{1.2}
\begin{longtable}[H]{p{2.6cm}p{2.5cm}p{5cm}p{2cm}}
	\rowcolor{CHeader} 
	\color{CHeaderText} \textbf{Identificatore} & \color{CHeaderText} \textbf{Importanza} & \color{CHeaderText} \textbf{Descrizione} & \color{CHeaderText} \textbf{Fonti} \\
	\endhead
	R0F1 & Obbligatorio & L'amministratore può gestire gli utenti & Interno \\
	R0F1.1 & Obbligatorio & L'amministratore può ottenere una lista di tutti gli utenti non abilitati & Interno \\
	R0F1.1.1 & Obbligatorio & L'amministratore può ottenere l'indirizzo di tutti gli utenti non abilitati & Interno \\
	R0F1.1.2 & Obbligatorio & L'amministratore può ottenere il nome di tutti gli utenti non abilitati & Interno \\
	R0F1.1.3 & Obbligatorio & L'amministratore può ottenere il cognome di tutti gli utenti non abilitati & Interno \\
	R0F1.1.4 & Obbligatorio & L'amministratore può ottenere il ruolo desiderato di tutti gli utenti non abilitati & Interno \\
	R0F1.1.5 & Obbligatorio & L'amministratore può ottenere il corso desiderato, ove possibile, di tutti gli utenti non abilitati & Interno \\
	R0F1.2 & Obbligatorio & L'amministratore può abilitare un utente & Interno \\
	R0F1.3 & Obbligatorio & L'amministratore può rimuovere un utente & Interno \\
	R0F1.4 & Obbligatorio & L'amministratore può ottenere una lista di tutti gli studenti & Interno \\
	R0F1.4.1 & Obbligatorio & L'amministratore può ottenere l'indirizzo di tutti studenti & Interno \\
	R0F1.4.2 & Obbligatorio & L'amministratore può ottenere il nome di tutti gli studenti & Interno \\
	R0F1.4.3 & Obbligatorio & L'amministratore può ottenere il cognome di tutti gli studenti & Interno \\
	R0F1.4.4 & Obbligatorio & L'amministratore può ottenere il corso di tutti gli studenti & Interno \\
	R0F2 & Obbligatorio & L'amministratore può gestire i professori & Interno \\
	R0F2.1 & Obbligatorio & L'amministratore può ottenere una lista di tutti i professori & Interno \\
	R0F2.1.1 & Obbligatorio & L'amministratore può ottenere l'indirizzo di tutti i professori & Interno \\
	R0F2.1.2 & Obbligatorio & L'amministratore può ottenere il nome di tutti i professori & Interno \\
	R0F2.1.3 & Obbligatorio & L'amministratore può ottenere il cognome di tutti i professori & Interno \\
	R0F2.2 & Obbligatorio & L'amministratore può assegnare un esame a un determinato professore & Capitolato \\
	R0F3 & Obbligatorio & L'amministratore può gestire i corsi di laurea relativi agli anni accademici & Capitolato \\
	R0F3.1 & Obbligatorio & L'amministratore può creare un nuovo \citGloss{corso di laurea} a partire da un \citGloss{anno accademico} & Capitolato \\ 
	R0F3.1.1 & Obbligatorio & L'amministratore deve indicare un codice identificativo per ogni \citGloss{corso di laurea} & Interno \\
	R0F3.2 & Obbligatorio & L'amministratore può ottenere una lista di tutti i corsi di laurea & Interno \\
	R0F3.2.1 & Obbligatorio & L'amministratore può ottenere una lista di tutti i corsi di laurea di un determinato \citGloss{anno accademico} & Interno \\
	R0F3.2.2 & Obbligatorio & Ogni corso di laurea deve essere visualizzato tramite la sua sigla  & Interno \\
	R0F3.2.3 & Obbligatorio & Ogni corso di laurea mostrato deve essere accompagnato dall'anno di riferimento ove non indicato dall'utente stesso & Interno \\
	R0F3.3 & Obbligatorio & L'amministratore può creare nuovi esami & Capitolato \\ 
	R0F3.3.1 & Obbligatorio & I nuovi esami devono aver indicato il nome & Interno \\
	R0F3.3.2 & Obbligatorio & I nuovi esami devono aver indicato il numero di crediti & Interno \\
	R0F3.3.3 & Obbligatorio & I nuovi esami devono aver indicato la obbligatorietà & Interno \\	
	R0F3.4 & Obbligatorio & L'amministratore può ottenere una lista di tutti gli esami dato il \citGloss{corso di laurea} & Capitolato \\
	R0F3.4.1 & Obbligatorio & Per ogni esame l'amministratore deve visualizzarne la sigla & Interno \\
	R0F3.4.2 & Obbligatorio & Per ogni esame l'amministratore deve visualizzarne il numero di crediti & Interno \\
	R0F3.4.3 & Obbligatorio & L'amministratore può ottenere il nome e il cognome di quale professore è assegnato o sta per essere assegnato a un determinato esame & Interno \\
	R0F3.4.4  & Obbligatorio & L'amministratore può associare a un esame il relativo professore & Interno \\
	R0F3.4.4.1  & Obbligatorio & L'amministratore deve poter visualizzare l'indirizzo di un professore prima di assegnarlo a un determinato esame per evitare problemi in caso di omonimi & Interno \\
	R0F3.4.5 & Obbligatorio & L'amministratore può ottenere la sigla del corso di laurea al quale appartiene un determinato esame ove non indicato dall'amministratore stesso oppure se richiesto & Interno \\
	R0F3.4.6 & Obbligatorio & L'amministratore può ottenere il numero degli studenti iscritti a un determinato corso & Interno \\
	R0F3.4.7 & Obbligatorio & L'amministratore può ottenere l'indirizzo del professore associato a un determinato corso
	& Interno \\
	\clearpage
	R0F4 & Obbligatorio & Un professore può gestire gli esami a lui assegnati & Capitolato \\
	R0F4.1 & Obbligatorio & Un professore può ottenere la lista di tutti gli esami a lui assegnati & Capitolato \\
	R0F4.1.1 & Obbligatorio & Un professore può ottenere il codice di tutti gli esami a lui assegnati & Capitolato \\
	R0F4.1.2 & Obbligatorio & Un professore può ottenere il relativo corso di laurea di tutti gli esami a lui assegnati & Capitolato \\
	R0F4.2 & Obbligatorio & Un professore può ottenere la lista di tutti gli studenti associati a un determinato esame & Capitolato \\
	R0F4.2.1 & Obbligatorio & Un professore può ottenere l'indirizzo di tutti gli studenti associati a un determinato esame & Interno \\
	R0F4.2.2 & Obbligatorio & Un professore può ottenere il nome di tutti gli studenti associati a un determinato esame & Interno \\
	R0F4.2.3 & Obbligatorio & Un professore può ottenere il cognome di tutti gli studenti associati a un determinato esame & Interno \\
	R0F4.3 & Obbligatorio & Un professore può registrare un esito a un dato esame a un dato studente registrato a quell'esame & Capitolato \\
	R0F5 & Obbligatorio & Uno studente può gestire l'iscrizione agli esami & Capitolato \\
	R0F5.1 & Obbligatorio & Uno studente può vedere l'elenco degli esami ai quali è iscritto & Capitolato \\
	R0F5.1.1 & Obbligatorio & Uno studente può ottenere il numero di crediti assegnati agli esame & Interno \\
	R0F5.1.2 & Obbligatorio & Uno studente può ottenere l'obbligatorietà degli esami & Interno \\
	R0F5.1.3 & Obbligatorio & Uno studente può ottenere lo stato di superamento degli esami & Interno \\
	R0F5.1.4 & Obbligatorio & Uno studente può ottenere il totale dei suoi crediti & Interno \\
	R0F5.1.5 & Obbligatorio & Uno studente può ottenere il numero di crediti da raggiungere per la laurea & Interno \\
	R0F5.2 & Obbligatorio & Uno studente può ottenere l'elenco degli esami opzionali & Capitolato \\
	R0F5.2.1 & Obbligatorio & Uno studente può ottenere il numero di crediti degli esami opzionali & Capitolato \\
	R0F5.2.2 & Obbligatorio & Uno studente può iscriversi a un esame opzionale & Capitolato \\
	R0F6 & Obbligatorio & L'utente può effettuare il login & Interno \\
	R0F6.1 & Obbligatorio & Il login deve avvenire tramite il controllo delle chiavi, senza ulteriori azioni da parte dell'utente & Interno \\
	R0F7 & Obbligatorio & L'utente può effettuare il logout & Interno \\
	R0F8 & Obbligatorio & L'utente non ancora registrato può registrarsi & Capitolato \\
	R0F8.1 & Obbligatorio & La registrazione necessita di nome, cognome, categoria (studente o professore) e selezione del corso di laurea se si tratta di uno studente & Capitolato \\
	R1F9 & Desiderabile & L'utente può leggere una breve guida sull'uso di MetaMask e sul pagamento delle operazioni & Interno \\	
	R0F10 & Obbligatorio & L'utente deve poter vedere preventivamente il costo in \citGloss{Gas}, \citGloss{Ether} e Euro dell'operazione che sta per eseguire & Capitolato, VER-2018-01-09 \\	
	R0F11 & Obbligatorio & L'università deve poter gestire gli amministratori & VER-2017-12-08.1 \\
	R0F11.1 & Obbligatorio & L'università deve poter aggiungere amministratori attraverso l'inserimento del loro indirizzo & VER-2017-12-08.1 \\
	R0F11.2 & Obbligatorio & L'università deve poter rimuovere amministratori & VER-2017-12-08.2 \\
	R0F11.3 & Obbligatorio & L'università deve poter ottenere la lista di tutti gli amministratori & Interno \\
	R0F11.3.1 & Obbligatorio & L'università deve poter visualizzare l'indirizzo di ogni amministratore & Interno \\
	R0F12 & Obbligatorio & L'università può gestire gli anni accademici & Capitolato \\
	R0F12.1 & Obbligatorio & L'università può aggiungere un \citGloss{anno accademico} & Capitolato \\
	R0F12.1.1 & Obbligatorio & I nuovi anni accademici devono aver indicato il relativo anno solare & Capitolato \\
	R0F12.2 & Obbligatorio & L'università può rimuovere un \citGloss{anno accademico} & Capitolato \\
	R0F12.3 & Obbligatorio & L'università e gli amministratori possono ottenere una lista degli anni accademici & Interno \\
	R0F12.3.1 & Obbligatorio & Gli anni accademici, all'interno delle liste devono indicare l'anno solare & Interno \\
	\hiderowcolors
	\caption{Tabella dei requisiti funzionali}
\end{longtable}

\subsection{Requisiti di qualità}

\label{table:Tabella requisiti di qualita'}
\rowcolors{2}{CRighePari}{CRigheDispari}
\renewcommand*{\arraystretch}{1.2}
\begin{longtable}[H]{p{2.5cm}p{2.5cm}p{5cm}p{2cm}}
	\rowcolor{CHeader} 
	\color{CHeaderText} \textbf{Identificatore} & \color{CHeaderText} \textbf{Importanza} & \color{CHeaderText} \textbf{Descrizione} & \color{CHeaderText} \textbf{Fonti} \\
	\endhead
	R0Q1 & Obbligatorio & La progettazione e il codice devono seguire le norme e le metriche riportate nel documento allegato \pdq \vrquattro & Interno \\
	R0Q2 & Obbligatorio & L'approccio di scrittura di \citGloss{JavaScript} deve essere promise Centric Approach & Capitolato \\
	R0Q2.1 & Obbligatorio & L'applicativo non deve fare uso di callback in presenza di alternative alle ultime & VER-2017-11-22.1 \\
	R0Q3 & Obbligatorio & Il codice \citGloss{JavaScript} deve attenersi al \citGloss{AirBNB} \citGloss{JavaScript} style guide & Capitolato \\
	R0Q4 & Obbligatorio & Lo sviluppo deve essere supportato dall'utilizzo del tool ESLint & Capitolato \\
	R0Q5 & Obbligatorio & Dovrà essere fornito un manuale utente in lingua inglese che tratterà l'uso da parte di studenti e professori & VER-2017-11-22.2 \\
	R0Q6 & Obbligatorio & Dovrà essere fornito un manuale di deploy e di utilizzo da parte degli amministratori in lingua inglese & VER-2017-11-22.3 \\
	R0Q7 & Obbligatorio & Il codice sorgente deve essere pubblicato sulla piattaforma \citGloss{GitHub}, BitBucket o GitLab & Capitolato \\
	R0Q8 & Obbligatorio & Il codice deve attenersi il più possibile alle guide linea de "App a 12 Fattori" & Capitolato \\ 
	\hiderowcolors
	\caption{Tabella dei requisiti di qualità}
\end{longtable}

\subsection{Requisiti di vincolo}

\label{table:Tabella requisiti di vincolo}
\rowcolors{2}{CRighePari}{CRigheDispari}
\renewcommand*{\arraystretch}{1.2}
\begin{longtable}[H]{p{2.5cm}p{2.5cm}p{5cm}p{2cm}}
	\rowcolor{CHeader} 
	\color{CHeaderText} \textbf{Identificatore} & \color{CHeaderText} \textbf{Importanza} & \color{CHeaderText} \textbf{Descrizione} & \color{CHeaderText} \textbf{Fonti} \\
	\endhead
	R0V1 & Obbligatorio & L'applicativo dovrà essere sviluppato attraverso l'uso di tecnologie web & Capitolato \\
	R0V1.1 & Obbligatorio & L'applicativo necessiterà della piattaforma Node.js per soddisfare le dipendenze delle librerie richieste citate nei \citGloss{requisiti} sottostanti. & Capitolato \\
	R0V1.2 & Obbligatorio & L'applicativo dovrà essere sviluppato con \citGloss{JavaScript} 8 (ES8) & Capitolato \\
	R0V1.3 & Obbligatorio & L'applicativo dovrà essere sviluppato con il boilerplate \citGloss{Redux} Minimal & Capitolato \\
	R0V1.4 & Obbligatorio & L'applicativo dovrà essere sviluppato utilizzando \citGloss{React} 15.x & Capitolato \\
	R0V1.5 & Obbligatorio & L'applicativo dovrà essere sviluppato utilizzando \citGloss{Redux} 3.x & Capitolato \\
	R0V1.6 & Obbligatorio & Il deploy del sito andrà eseguito utilizzando Surge.sh & Capitolato \\
	R0V1.7 & Desiderabile & È desiderabile l'utilizzo di \citGloss{SCSS} in sostituzione a \citGloss{CSS} & Capitolato \\
	R0V2 & Obbligatorio & Gli \citGloss{smart contract} dovranno essere scritti in linguaggio \citGloss{Solidity} & Capitolato \\
	R0V3 & Obbligatorio & La connessione alla rete \citGloss{Ethereum} deve avvenire tramite MetaMask & Capitolato \\
	R0V3.1 & Obbligatorio & I test riguardanti gli \citGloss{smart contract} dovranno essere eseguiti in una rete locale e almeno in una rete pubblica & Capitolato \\
	R0V3.2 & Obbligatorio & Il deploy degli \citGloss{smart contract} dovrà avvenire su rete locale testrpc e rete di test Ropsten & Capitolato \\
	R2V3.3 & Opzionale & È apprezzabile un deploy finale sulla rete principale di \citGloss{Ethereum} & Capitolato \\
	R0V4 & Obbligatorio & Lo sviluppo degli \citGloss{smart contract} dovrà avvenire utilizzando il framework Truffle & Capitolato \\
	R0V5 & Obbligatorio & L'applicativo deve essere accessibile e utilizzabile dal \citGloss{browser} Mozilla \citGloss{Firefox} a partire dalla versione 52 & Interno \\
	R0V6 & Obbligatorio & L'applicativo deve essere accessibile e utilizzabile dal \citGloss{browser} Google \citGloss{Chrome} a partire dalla versione 57 & Interno \\
	R1V7 & Desiderabile & L'applicativo deve essere accessibile e utilizzabile da un \citGloss{browser} mobile, per le versioni supportate fare riferimento alle controparti PC di \citGloss{Firefox} e \citGloss{Chrome} & Interno \\
	R0V8 & Obbligatorio & Un utente non deve poter compiere azioni sul sistema senza aver fatto l'accesso a esso & Capitolato \\ 
	R0V9 & Obbligatorio & L'applicazione dei principi de "App a 12 Fattori" deve essere documentata ove possibile & Capitolato, VER-2018-01-09 \\
	R0V10 & Obbligatorio & Il codice sorgente deve essere pubblicato con licenza MIT & Capitolato \\
	\hiderowcolors
	\caption{Tabella dei requisiti di vincolo}
\end{longtable}

\subsection{Requisiti prestazionali}

Non sono stati individuati \citGloss{requisiti} prestazionali in quanto la maggior parte delle attività, per essere concluse, necessitano di un' interazione con una rete \citGloss{Ethereum}, e quindi non risultano costanti o prevedibili con precisione.\\
Qualsiasi operazione effettuata in una rete \citGloss{Ethereum} reale ha un tempo di soddisfacimento casuale influenzato dal carico della rete nel momento della richiesta e, nel caso di operazioni che vanno a modificare lo stato di un contratto, del quantitativo di Ether pagati per ogni unità di \citGloss{Gas}.\\
 
 
\newpage
\section{Tracciamento}
\subsection{Tracciamento fonti-requisiti}

\label{table:Tabella di tracciamento fonti-requisiti}


\rowcolors{2}{CRighePari}{CRigheDispari}
\renewcommand*{\arraystretch}{1.2}
\begin{longtable}[H]{p{2cm}p{5cm}p{5cm}}
	\rowcolor{CHeader} 
	\color{CHeaderText} \textbf{Fonte} & \color{CHeaderText} \textbf{Nome fonte} & \color{CHeaderText} \textbf{Requisiti} \\
	\endhead
	Capitolato & & \makecell[tl]{ R0F2.2 \\
	 R0F3 \\
	 R0F3.1 \\
	 R0F3.3 \\
	 R0F3.4 \\
	 R0F4 \\
	 R0F4.1 \\
	 R0F4.1.1 \\
	 R0F4.1.2 \\
	 R0F4.2 \\
	 R0F4.3 \\
	 R0F5 \\
	 R0F5.1 \\
	 R0F5.2 \\
	 R0F5.2.1 \\
	 R0F5.2.2 \\
	 R0F7 \\
	 R0F8 \\
	 R0F8.1 \\
	 R0F10 \\
	 R0F12\\
	 R0F12.1 \\
	 R0F12.1.1 \\
	 R0Q2 \\
	 R0Q3 \\
	 R0Q4 \\
	 R0Q7 \\
	 R0Q8 \\
	 R0V1 \\
	 R0V1.1 \\
	 R0V1.2 \\
	 R0V1.3 \\
	 R0V1.4 } \\
\rowcolor{CRigheDispari}
& & \makecell[tl]{ %TODO aggiustare alla fine
	 R0V1.5 \\
	 R0V1.6 \\
	 R0V1.7 \\
	 R0V2 \\
	 R0V3 \\
	 R0V3.1 \\
	 R0V3.2 \\
	 R2V3.3 \\
	 R0V4 \\
	 R1V7 \\
	 R0V8 \\
	 R0V9 \\
	 R0V10 } \\
	
	\rowcolor{CRighePari}
	Interno & & \makecell[tl]{ R0F1 \\
	 R0F1.1 \\
	 R0F1.1.1 \\
	 R0F1.1.2 \\
	 R0F1.1.3 \\
	 R0F1.1.4 \\
	 R0F1.1.5 \\
	 R0F1.2 \\
	 R0F1.3 \\
	 R0F1.4 \\
	 R0F1.4.1 \\
	 R0F1.4.2 \\
	 R0F1.4.3 \\
	 R0F1.4.4 \\
	 R0F2 \\
	 R0F2.1 \\
	 R0F2.1.1 \\
	 R0F2.1.2 \\
	 R0F2.1.3 \\
	 R0F3 \\
	 R0F3.1.1 \\
	 R0F3.2 \\
	 R0F3.2.1 \\
	 R0F3.2.2 \\
	 R0F3.2.3 \\
	 R0F3.3.1
 } \\
& & \makecell[tl]{ %TODO aggiustare alla fine
	 R0F3.3.2 \\
	 R0F3.3.3 \\
	 R0F3.4 \\
	 R0F3.4.1 \\
	 R0F3.4.2 \\
	 R0F3.4.3 \\
	 R0F3.4.4 \\
	 R0F3.4.4.1 \\
	 R0F3.4.5 \\
	 R0F3.4.6 \\
	 R0F3.4.7 \\
	 R0F4.2.1 \\
	 R0F4.2.2 \\
	 R0F4.2.3 \\
	 R0F5.1.1 \\
	 R0F5.1.2 \\
	 R0F5.1.3 \\
	 R0F5.1.4 \\
	 R0F5.1.5 \\
	 R0F6 \\
	 R0F6.1 \\
	 R1F9 \\
	 R0F11.3 \\
	 R0F11.3.1 \\
	 R0F12.3 \\
	 R0F12.3.1 \\
	 R0Q1 \\
	 R0V5 \\
	 R0V6 } \\
	
	VER-2017-11-22 & Verbale & \makecell[tl]{ R0Q2.1 \\
	 R0Q5 \\
	 R0Q6 } \\
	
	VER-2017-12-08 & Verbale & \makecell[tl]{ R0F11 \\
	 R0F11.1 \\
	 R0F11.2 } \\
	VER-2018-01-09 & Verbale & \makecell[tl]{ R0F10 \\
	R0V9 } \\ 
	
	UC1 & Breve guida & R1F9 \\
	UC2 & Login & R0F6 \\
	UC2.1 & Login automatico & R0F6.1 \\
	UC2.2 & Visualizzazione messaggio di errore riguardo a chiave assente & R0F6.1 \\
	UC2.3 & Visualizzazione messaggio di errore riguardo a MetaMask assente & R0F6.1 \\
	UC2.4 & Visualizzazione messaggio di errore riguardo a chiave non registrata & R0F6.1 \\
	UC3 & Registrazione & R0F8 \\
	UC3.1 & Inserimento nome & R0F8.1 \\
	UC3.2 & Inserimento cognome & R0F8.1 \\
	UC3.3 & Selezione categoria & R0F8.1 \\
	UC3.4 & Selezione corso di laurea & R0F8.1 \\
	UC3.5 & Visualizzazione errore campo non compilato & R0F8.1 \\
	UC3.6 & Visualizzazione errore utente già registrato & R0F8.1 \\
	UC3.7 & Visualizzazione errore chiave non presente & R0F8.1 \\
	UC3.8 & Visualizzazione errore MetaMask non installato & R0F8.1 \\
	UC4 & Logout & R0F7 \\
	UC5 & Amministrazione & \makecell[tl]{ R0F1 \\
		R0F2 \\
		R0F12 } \\
	UC5 & Gestione Utenti & R0F1 \\
	UC5.1 & Visualizzazione lista degli studenti approvati & R0F1.4, R0F2.1 \\
	UC5.1.1 & Visualizzazione singolo utente approvato & R0F1.4, R0F2.1 \\
	UC5.1.1.1 & Visualizzazione indirizzi degli studenti approvati & R0F1.4.1, R0F2.1.1 \\
	UC5.1.1.2 & Visualizzazione nomi degli studenti approvati & R0F1.4.2, R0F2.1.2 \\
	UC5.1.1.3 & Visualizzazione cognomi degli studenti approvati & R0F1.4.3, R0F2.1.3 \\
	UC5.2 & Visualizzazione lista degli studenti approvati & R0F1.4 \\
	UC5.2.1 & Visualizzazione singolo studente approvato & R0F1.4 \\
	UC5.2.1.1 & Visualizzazione corsi degli studenti approvati & R0F1.4.4 \\
	UC5.3 & Visualizzazione lista dei professori approvati & R0F2.1 \\
	UC5.3.1 & Visualizzazione singolo professore approvato & R0F2.1 \\
	UC5.4 & Visualizzazione lista utenti in attesa di abilitazione & R0F1.1 \\
	UC5.4.1 & Visualizzazione singolo utente in attesa di abilizatione & R0F1.1 \\
	UC5.4.1.1 & Visualizzazione indirizzi degli utenti in attesa di abilitazione & R0F1.1.1 \\
	UC5.4.1.2 & Visualizzazione nomi degli utenti in attesa di abilitazione & R0F1.1.2 \\
	UC5.4.1.3 & Visualizzazione cognomi degli utenti in attesa di abilitazione & R0F1.1.3 \\
	UC5.4.1.4 & Visualizzazione ruolo degli utenti in attesa di abilitazione &R0F1.1.4  \\
	UC5.4.1.5 & Visualizzazione corsi degli utenti in attesa di abilitazione & R0F1.1.5 \\
	UC5.5 & Abilitazione utente & R0F1.2 \\
	UC5.6 & Rimozione utente & R0F1.3 \\
	UC6 & Gestione anni accademici & R0F12 \\
	UC6.1 & Aggiunta \citGloss{anno accademico} & R0F12.1 \\
	UC6.1.1 & Inserimento anno solare di riferimento & R0F12.1.1 \\
	UC6.1.2 & Visualizzazione messaggio di \citGloss{anno accademico} già presente & R0F12.1.1 \\
	UC6.1.3 & Visualizzazione messaggio anno non compilato & R0F12.1.1 \\
	UC6.1.4 & Visualizzazione messaggio anno malformato & R0F12.1.1 \\
	UC6.2 & Visualizzazione lista di tutti gli anni accademici & R0F12.3 \\
	UC6.2.1 & Visualizzazione singolo anno accademico & R0F12.3.1 \\
	UC6.2.1.1 & Visualizzazione anni solari relativo ad un anno accademico & R0F12.3.1 \\
	UC6.3 & Eliminazione anno accademico vuoto & R0F12.2 \\
	UC7 & Gestione corsi di laurea & R0F3 \\
	UC7.1 & Creazione \citGloss{corso di laurea} & \makecell[tl]{
		R0F3.1\\
		R0F3.1.1 } \\
	UC7.1.1 & Inserimento sigla corso di laurea & \makecell[tl]{
		R0F3.1\\
		R0F3.1.1 } \\
	UC7.1.2 & Visualizzazione errore sigla invalida & \makecell[tl]{
		R0F3.1\\
		R0F3.1.1 } \\
	UC7.2 & Visualizzazione lista completa dei corsi & R0F3.2 \\
	UC7.2.1 & Visualizzazione corso & R0F3.2 \\
	UC7.2.1.1 & Visualizzazione sigla del corso di laurea & R0F3.2.2 \\
	UC7.2.1.2 & Visualizzazione anno solare dell'anno accademico associato al corso di laurea & R0F3.2.3 \\
	UC7.3 & Visualizzazione lista corsi di laurea per \citGloss{anno accademico} & R0F3.2.1  \\
	UC7.3.1 & Visualizzazione corso & R0F3.2.1\\
	UC7.3.1.1 & Visualizzazione sigle dei corsi di laurea per \citGloss{anno accademico} & R0F3.2.2  \\
	UC7.4 & Creazione esame in un corso & R0F3.3\\
	UC7.4.1 & Inserimento nome esame & R0F3.3.1\\
	UC7.4.2 & Inserimento numero crediti esami & R0F3.3.2\\
	UC7.4.3 & Inserimento obbligatorietà esame & R0F3.3.3\\
	UC7.4.4 & Visualizzazione errore nome esame non valido & R0F3.3.1\\
	UC7.4.5 & Visualizzazione errore numero crediti esame non valido & R0F3.3.2\\
	UC7.5 & Visualizzazione lista esami per \citGloss{corso di laurea} & R0F3.4 \\
	UC7.5.1 & Visualizzazione esame appartenente a un determinato corso & R0F3.4 \\
	UC7.5.1.1 & Visualizzazione sigla esame & R0F3.4.1 \\
	UC7.5.1.2 & Visualizzazione crediti esame & R0F3.4.2 \\
	UC7.5.1.3 & Visualizzazione nome del professori associato se presente & R0F3.4.3 \\
	UC7.5.1.4 & Visualizzazione cognome del professore associato se presente & R0F3.4.3 \\
	UC7.6 & Visualizzazione lista esami & R0F3.4 \\
	UC7.6.1 & Visualizzazione esame & R0F3.4 \\
	UC7.6.1.1 & Visualizzazione sigla esame & R0F3.4.1 \\
	UC7.6.1.2 & Visualizzazione crediti esame & R0F3.4.2 \\
	UC7.6.1.3 & Visualizzazione sigla del corso di laurea associato & R0F3.4.5 \\
	UC7.6.1.4 & Visualizzazione nome del professori associato se presente & R0F3.4.3 \\
	UC7.6.1.5 & Visualizzazione cognome del professore associato se presente & R0F3.4.3 \\
	UC7.7 & Visualizzazione dettagli esame & R0F3.4 \\
	UC7.7.1 & Visualizzazione sigla esame & R0F3.4.1 \\
	UC7.7.2 & Visualizzazione crediti esame & R0F3.4.2 \\
	UC7.7.3 & Visualizzazione corso di laurea associato all'esame se presente & R0F3.4.5 \\
	UC7.7.4 & Visualizzazione nome del professore associato all'esame se presente & R0F3.4.3 \\
	UC7.7.5 & Visualizzazione cognome del professore associato all'esame se presente & R0F3.4.3 \\
	UC7.7.6 & Visualizzazione numero di studenti iscritti all'esame & R0F3.4.6 \\
	UC7.7.7 & Visualizzazione indirizzo del professore associato all'esame se presente & R0F3.4.7 \\
	UC7.8 & Visualizzazione di tutti gli anni accademici & R0F12.3 \\
	UC7.8.1 & Visualizzazione anni solari relativi agli anni accademici & R0F12.3.1 \\
	UC7.9 & Visualizzazione lista professori da associare ad un esame & R0F3.4.3 \\
	UC7.9.1 & Visualizzazione nome per ogni professore & R0F3.4.3 \\
	UC7.9.2 & Visualizzazione cognome per ogni professore & R0F3.4.3 \\
	UC7.9.3 & Visualizzazione indirizzo per ogni professore & R0F3.4.4.1 \\
	UC7.10 & Associazione professore all'esame & R0F3.4.4 \\
	UC8 & Gestione aspetti relativi agli esami & R0F4 \\
	UC8.1 & Visualizzazione lista degli esami & R0F4.1 \\
	UC8.1.1 & Visualizzazione codici degli esami & R0F4.1.1 \\
	UC8.1.2 & Visualizzazione corsi degli esami & R0F4.1.2 \\
	UC8.2 & Visualizzazione lista degli studenti & R0F4.2  \\
	UC8.2.1 & Visualizzazione indirizzi degli studenti dell'esame & R0F4.2.1 \\
	UC8.2.2 & Visualizzazione nomi degli studenti dell'esame & R0F4.2.2 \\
	UC8.2.3 & Visualizzazione cognomi degli studenti dell'esame & R0F4.2.3 \\
	UC8.3 & Registrazione valutazione di un esame & R0F4.3 \\
	UC8.3.1 & Inserimento voto numerico & R0F4.3 \\
	UC8.3.2 & Visualizzazione errore voto inferiore a 18 & R0F4.3 \\
	UC8.3.2 & Visualizzazione errore voto non numerico & R0F4.3 \\
	UC9 & Gestione aspetti relativi allo studente & R0F5 \\
	UC9.1 & Visualizzazione lista degli esami & R0F5.1 \\
	UC9.1.1 & Visualizzazione dei crediti degli esami ai quali è iscritto & R0F5.1.1 \\
	UC9.1.2 & Visualizzazione della obbligatorietà degli esami ai quali è iscritto & R0F5.1.2 \\
	UC9.1.3 & Visualizzazione delle valutazioni degli esami ai quali è iscritto	& R0F5.1.3 \\
	UC9.2 & Visualizzazione degli esami opzionali e dei loro crediti & R0F5.2 \\
	UC9.3 & Iscrizione a un esame opzionale & R0F5.2.2 \\
	UC9.4 & Visualizzazione delle informazioni relative ai crediti & R0F5.1.4 \\ 
	UC9.4.1 & Visualizzazione somma dei crediti & R0F5.1.4 \\ 
	UC9.4.2 & Visualizzazione crediti mancanti & R0F5.1.5 \\ 
	UC9.4.3 & Visualizzazione crediti ottenibili dagli esami a cui è iscritto &  \makecell[tl]{
		R0F5.1.4 \\ 
		R0F5.2.1 } \\
	UC10 & Gestione degli amministratori & R0F11 \\
	UC10.1 & Aggiunta di un amministratore & R0F11.1 \\
	UC10.1.1 & Inserimento \citGloss{chiave pubblica} & R0F11.1 \\
	UC10.1.2 & Visualizzazione messaggio di errore riguardo a chiave mal formata & R0F11.1 \\
	UC10.1.3 & Visualizzazione messaggio di errore riguardo a chiave già registrata & R0F11.1 \\
	UC10.2 & Visualizzazione lista degli amministratori & R0F11.3 \\
	UC10.2.1 & Visualizzazione singolo amministratore & R0F11.3 \\
	UC10.2.1.1 & Visualizzazione indirizzi degli amministratori & R0F11.3.1 \\
	UC10.3 & Rimozione amministratore & R0F11.2 \\	
	UC11 & Visualizzazione quantità di \citGloss{Gas}, \citGloss{Ether} e costo delle operazioni & R0F10 \\
	\hiderowcolors
	\caption{Tabella di tracciamento fonti-requisiti}
\end{longtable}

\newpage
\subsection{Tracciamento requisiti-fonti}

\label{table:Tabella di tracciamento requisiti-fonti}
\rowcolors{2}{CRighePari}{CRigheDispari}
\renewcommand*{\arraystretch}{1.2}
\begin{longtable}[H]{p{2cm}p{5.2cm}p{5cm}}
	\rowcolor{CHeader} 
	\color{CHeaderText} \textbf{Requisito} & \color{CHeaderText} \textbf{Descrizione requisito} & \color{CHeaderText} \textbf{Fonti} \\
	% REGEX for Notepad++
	% (.*)&(.*)&(.*)&(.*)      --->       \1& \3& \\\\ \ 
	\endhead
	R0F1 & L'amministratore può gestire gli utenti & \makecell[tl]{
		Interno \\
		UC5
	} \\
	R0F1.1 & L'amministratore può ottenere una lista degli utenti non abilitati & \makecell[tl]{
		Interno \\
		UC5.4
	} \\
	R0F1.1.1 & L'amministratore può ottenere l'indirizzo degli utenti non abilitati & \makecell[tl]{
		Interno \\
		UC5.4.1.1
	} \\
	R0F1.1.2 & L'amministratore può ottenere il nome degli utenti non abilitati & \makecell[tl]{
		Interno \\
		UC5.4.1.2
	} \\
	R0F1.1.3 & L'amministratore può ottenere il cognome degli utenti non abilitati & \makecell[tl]{
		Interno \\
		UC5.4.1.3
	} \\
	R0F1.1.4 & L'amministratore può ottenere il ruolo desiderato degli utenti non abilitati & \makecell[tl]{
		Interno \\
		UC5.4.1.4
	} \\
	R0F1.1.5 & L'amministratore può ottenere il corso desiderato, ove possibile, degli utenti non abilitati & \makecell[tl]{
		Interno \\
		UC5.4.1.5
	} \\

	R0F1.2 & L'amministratore può abilitare un utente & \makecell[tl]{
		Interno \\ 
		UC5 \\ 
		UC5.5
	} \\
	R0F1.3 & L'amministratore può rimuovere un utente & \makecell[tl]{
		Interno \\ 
		UC5.6
	} \\

	R0F1.4 & L'amministratore può ottenere una lista degli studenti & \makecell[tl]{
		Interno \\
		UC5.1.1 \\
		UC5.2
	} \\
	R0F1.4.1 & L'amministratore può ottenere l'indirizzo degli studenti & \makecell[tl]{
		Interno \\
		UC5.1.1.1
	} \\
	R0F1.4.2 & L'amministratore può ottenere il nome degli studenti & \makecell[tl]{
		Interno \\
		UC5.1.1.2
	} \\
	R0F1.4.3 & L'amministratore può ottenere il cognome degli studenti & \makecell[tl]{
		Interno \\
		UC5.1.1.3
	} \\
	R0F1.4.4  & L'amministratore può ottenere il corso degli studenti & \makecell[tl]{
		Interno \\
		UC5.2.1.1
	} \\

	R0F2 & L'amministratore può gestire i professori & \makecell[tl]{
		Interno
	} \\
	R0F2.1 & L'amministratore può ottenere una lista di tutti i professori & \makecell[tl]{
		Interno \\ 
		UC5.1 \\
		UC5.3 
	} \\
	R0F2.1.1 & L'amministratore può ottenere l'indirizzo dei professori & \makecell[tl]{
		Interno \\ 
		UC5.1.1.1
	} \\
	R0F2.1.2 & L'amministratore può ottenere il nome dei professori & \makecell[tl]{
		Interno \\ 
		UC5.1.1.2
	} \\
	R0F2.1.3 & L'amministratore può ottenere il cognome dei professori & \makecell[tl]{
		Interno \\ 
		UC5.1.1.3
	} \\
	R0F2.2 & L'amministratore può assegnare un esame a un determinato professore & \makecell[tl]{
		Capitolato
	} \\
	R0F12& L'università può gestire gli anni accademici & \makecell[tl]{
		Capitolato \\ 
		UC6
	} \\
	R0F12.1 & L'università può aggiungere un \citGloss{anno accademico} & \makecell[tl]{
		Capitolato \\ 
		UC6.1 \\ 
	} \\
	R0F12.1.1 & I nuovi anni accademici devono aver indicato il relativo anno solare  & \makecell[tl]{
		Capitolato \\ 
		UC6.1.1 \\
		UC6.1.2 \\ 
		UC6.1.3
		UC6.1.4
	} \\
	R0F12.2 & L'università può rimuovere un \citGloss{anno accademico} & \makecell[tl]{
	Capitolato \\ 
	UC6.3
	} \\
	R0F12.3 &  L'università e gli amministratori possono ottenere una lista degli anni accademici, con indicato l'anno solare & \makecell[tl]{
		Interno \\ 
		UC6.2 \\
		UC7.8
	} \\ 
	R0F12.3.1 &  L'università e gli amministratori possono ottenere una lista degli anni accademici, con indicato l'anno solare & \makecell[tl]{
	Interno \\ 
	UC6.2.1 \\
	UC7.8.1
	} \\ 
	R0F3 & L'amministratore può gestire i corsi di laurea relativi agli anni accademici & \makecell[tl]{
		Capitolato \\ 
		UC7 
	} \\
	R0F3.1 & L'amministratore può creare un nuovo \citGloss{corso di laurea} a partire da un \citGloss{anno accademico} & \makecell[tl]{
		Capitolato \\ 
		UC7.1 \\
		UC7.1.1 \\
		UC7.1.2
	} \\
	R0F3.1.1 & L'amministratore deve indicare un codice identificativo per ogni \citGloss{corso di laurea} & \makecell[tl]{
		Interno \\ 
		UC7.1 \\
		UC7.1.2
	} \\ 
	R0F3.2 & L'amministratore può ottenere una lista di tutti i corsi di laurea dato un \citGloss{anno accademico} & \makecell[tl]{
		Interno \\ 
		UC7.2
	} \\
	R0F3.2.1 & L'amministratore può ottenere una lista di tutti i corsi di laurea di un determinato \citGloss{anno accademico} & \makecell[tl]{
		Interno \\ 
		UC7.3
	} \\
	R0F3.2.2 & Ogni corso di laurea deve essere visualizzato tramite la sua sigla & \makecell[tl]{
		Interno \\ 
		UC7.2.1.1 \\
		UC7.3.1.1
	} \\
	R0F3.2.3 & Ogni corso di laurea mostrato deve essere accompagnato dall'anno di riferimento ove non indicato dall'utente stesso & \makecell[tl]{
		Interno \\ 
		UC7.2.1.2 \\
	} \\
	R0F3.3 & L'amministratore può creare nuovi esami & \makecell[tl]{
		Capitolato \\ 
		UC7.4
	} \\
	R0F3.3.1 & I nuovi esami devono aver indicato il nome & \makecell[tl]{
		Interno \\ 
		UC7.4.1 \\
		UC7.4.4
	} \\
	R0F3.3.2 & I nuovi esami devono aver indicato il numero di crediti & \makecell[tl]{
		Interno \\ 
		UC7.4.2 \\
		UC7.4.5
	} \\
	R0F3.3.3 & I nuovi esami devono aver indicato l'obbligatorietà & \makecell[tl]{
		Interno \\ 
		UC7.4.3
	} \\
	R0F3.4 & L'amministratore può ottenere una lista di tutti gli esami dato il \citGloss{corso di laurea} & \makecell[tl]{
		Interno \\ 
		UC7.5 \\
		UC7.6 \\
		UC7.7
	} \\
	R0F3.4.1 & Per ogni esame l'amministratore deve visualizzarne la sigla & \makecell[tl]{
		Interno \\
		UC7.5.1.1 \\
		UC7.6.1.1 \\
		UC7.7.1
	} \\
	R0F3.4.2 & Per ogni esame l'amministratore deve visualizzarne il numero di crediti & \makecell[tl]{
		Interno \\
		UC7.5.1.2 \\
		UC7.6.1.2 \\
		UC7.7.2
	} \\
	R0F3.4.3 & L'amministratore può ottenere il nome e il cognome di quale professore è assegnato o sta per essere assegnato a un determinato esame & \makecell[tl]{
		Interno \\ 
		UC7.5.1.3 \\
		UC7.5.1.4 \\
		UC7.7.4 \\
		UC7.6.1.5 \\
		UC7.6.1.4 \\
		UC7.7.5 \\
		UC7.9 \\
		UC7.9.1 \\
		UC7.9.2
	} \\
	R0F3.4.4  & L'amministratore può associare a un esame il relativo professore & \makecell[tl]{
		Interno	 \\
		UC7.10 	
	} \\
	R0F3.4.4.1 & L'amministratore deve poter visualizzare l'indirizzo di un professore prima di assegnarlo a un determinato esame per evitare problemi in caso di omonimi & \makecell[tl]{
		Interno	 \\ 
		UC7.8.3 \\
	} \\
	R0F3.4.5 & L'amministratore può ottenere la sigla del corso di laurea al quale appartiene un determinato esame ove non indicato dall'amministratore stesso oppure se richiesto & \makecell[tl]{
		Interno	 \\ 
		UC7.7.5
	} \\
	R0F3.4.6 & L'amministratore può ottenere il numero degli studenti iscritti a un determinato corso  & \makecell[tl]{
		Interno	 \\ 
		UC7.7.6
	} \\
	R0F3.4.7 & L'amministratore può ottenere l'indirizzo del professore associato a un determinato corso  & \makecell[tl]{
		Interno	 \\ 
		UC7.7.7
	} \\
	R0F4 & Un professore può gestire gli esami a lui assegnati & \makecell[tl]{
		Capitolato \\ 
		UC8
	} \\
	R0F4.1 & Un professore può ottenere la lista di tutti gli esami a lui assegnati & \makecell[tl]{
		Capitolato \\ 
		UC8.1
	} \\
	R0F4.1.1 & Un professore può ottenere il codice di tutti gli esami a lui assegnati & \makecell[tl]{
		Capitolato \\ 
		UC8.1.1
	} \\
	R0F4.1.2 & Un professore può ottenere il relativo corso di laurea di tutti gli esami a lui assegnati& \makecell[tl]{
		Capitolato \\ 
		UC8.1.2
	} \\
	R0F4.2 & Un professore può ottenere la lista di tutti gli studenti associati a un determinato esame & \makecell[tl]{
		Capitolato  \\ 
		UC8.2
	} \\
	R0F4.2.1 & Un professore può ottenere l'indirizzi di tutti studenti associati a un determinato esame & \makecell[tl]{
		Interno  \\ 
		UC8.2.1
	} \\
	R0F4.2.2 & Un professore può ottenere il nome di tutti gli studenti associati a un determinato esame & \makecell[tl]{
		Interno  \\ 
		UC8.2.2
	} \\
	R0F4.2.3 & Un professore può ottenere il cognome di tutti gli studenti associati a un determinato esame & \makecell[tl]{
		Interno  \\ 
		UC8.2.3
	} \\
	R0F4.3 & Un professore può registrare un esito a un dato esame a un dato studente registrato a quell'esame & \makecell[tl]{
		Capitolato \\ 
		UC8.3 \\
		UC8.3.1 \\
		UC8.3.2 \\
		UC8.3.3
	} \\
	R0F5 & Uno studente può gestire l'iscrizione agli esami & \makecell[tl]{
		Capitolato \\ 
		UC9
	} \\
	R0F5.1 & Uno studente può vedere l'elenco degli esami ai quali è iscritto & \makecell[tl]{
		Capitolato \\ 
		UC9.1
	} \\
	R0F5.1.1 & Uno studente può ottenere il numero di crediti assegnati agli esame & \makecell[tl]{
		Interno \\ 
		UC9.1.1
	} \\
	R0F5.1.2 & Uno studente può ottenere l'obbligatorietà degli esame & \makecell[tl]{
		Interno \\ 
		UC9.1.2
	} \\
	R0F5.1.3 & Uno studente può ottenere lo stato di superamento degli esame & \makecell[tl]{
		Interno \\ 
		UC9.1.3
	} \\
	R0F5.1.4 & Uno studente può ottenere il totale dei suoi crediti & \makecell[tl]{
		Interno \\ 
		UC9.4 \\
		UC9.4.1 \\
		UC9.4.3
	} \\
	R0F5.1.5 & Uno studente può ottenere il numero di crediti da raggiungere per la laurea & \makecell[tl]{
		Interno \\
		UC9.4.2
	} \\
	R0F5.2 & Uno studente può ottenere l'elenco degli esami opzionali & \makecell[tl]{
		Capitolato \\ 
		UC9.2
	} \\
	R0F5.2.1 & Uno studente può ottenere il numero di crediti degli esami opzionali & \makecell[tl]{
		Capitolato \\ 
		UC9.4.3
	} \\
	R0F5.2.2 & Uno studente può iscriversi a un esame opzionale & \makecell[tl]{
		Capitolato \\ 
		UC9.3
	} \\
	R0F6 & L'utente può effettuare il login & \makecell[tl]{
		Interno \\ 
		UC2
	} \\
	R0F6.1 & Il login deve avvenire tramite il controllo delle chiavi, senza ulteriori azioni da parte dell'utente & \makecell[tl]{
		Interno \\ 
		UC2.1
	} \\
	R0F7 & L'utente può effettuare il logout & \makecell[tl]{
		Capitolato \\ 
		UC2.1  \\
		UC2.2 \\
		UC2.3 \\
		UC2.4 \\ 
		UC4
	} \\
	R0F8 & L'utente non ancora registrato può registrarsi & \makecell[tl]{
		Capitolato \\ 
		UC3
	} \\
	R0F8.1 & La registrazione necessita di nome, cognome, categoria (studente o professore) e selezione del corso di laurea se si tratta di uno studente & \makecell[tl]{
		Capitolato \\
		UC3.1 \\
		UC3.2 \\
		UC3.3 \\
		UC3.4 \\
		UC3.5 \\
		UC3.6 \\
		UC3.7 \\
		UC3.8
	} \\
	R1F9 & L'utente può leggere una breve guida sull'uso di MetaMask e sul pagamento delle operazioni & \makecell[tl]{
		Interno \\ 
		UC1
	} \\
	R0F10 & L'utente deve poter vedere preventivamente il costo in \citGloss{Gas}, \citGloss{Ether} e Euro dell'operazione che sta per eseguire & \makecell[tl]{
		Capitolato \\
		UC11
	} \\
	R0F11 & L'università deve poter gestire gli amministratori & \makecell[tl]{
		VER-2017-12-08 \\
		UC10
	} \\
	R0F11.1 & L'università deve poter aggiungere amministratori attraverso l'inserimento del loro indirizzo & \makecell[tl]{
		VER-2017-12-08 \\
		UC10.1 \\
		UC10.1.1 \\ 
		UC10.1.2 \\
		UC10.1.3
	} \\
	R0F11.2 & L'università deve poter rimuovere amministratori & \makecell[tl]{
		VER-2017-12-08 \\
		UC10.3
	} \\
	R0F11.3 & L'università deve poter ottenere la lista di tutti gli amministratori & \makecell[tl]{
		Interno \\
		UC10.2.1
	} \\
	R0F11.3.1& L'università deve poter visualizzare l'indirizzo di ogni amministratore & \makecell[tl]{
		Interno \\
		UC10.2.1.1
	} \\
	R0Q1 & La progettazione e il codice devono seguire le norme e le metriche riportate nei documenti allegati X e Y & \makecell[tl]{
		Interno
	} \\
	R0Q2 & L'approccio di scrittura di \citGloss{JavaScript} deve essere promise Centric Approach & \makecell[tl]{
		Capitolato
	} \\
	R0Q2.1 & L'applicativo non deve fare uso di callback in presenza di alternative alle ultime & \makecell[tl]{
		VER-2017-11-22
	} \\
	R0Q3 & Il codice \citGloss{JavaScript} deve attenersi al \citGloss{AirBNB} \citGloss{JavaScript} style guide & \makecell[tl]{
		Capitolato
	} \\
	R0Q4 & Lo sviluppo deve essere supportato dall'utilizzo del tool ESLint & \makecell[tl]{
		Capitolato
	} \\
	R0Q5 & Dovrà essere fornito un manuale utente in lingua inglese che tratterà l'uso da parte di studenti e professori & \makecell[tl]{
		VER-2017-11-22
	} \\
	R0Q6 & Dovrà essere fornito un manuale di deploy e di utilizzo da parte degli amministratori in lingua inglese & \makecell[tl]{
		VER-2017-11-22
	} \\
	R0Q7 & Il codice sorgente deve essere pubblicato sulla piattaforma \citGloss{GitHub}, BitBucket o GitLab & \makecell[tl]{
		Capitolato
	} \\
	R0Q8 & Il codice deve attenersi il più possibile alle guide linea de "App a 12 Fattori" & \makecell[tl]{
		Capitolato
	} \\
	R0V1 & L'applicativo dovrà essere sviluppato attraverso l'uso di tecnologie web & \makecell[tl]{
		Capitolato
	} \\
	R0V1.1 & L'applicativo dovrà essere sviluppato con Node.js & \makecell[tl]{
		Capitolato
	} \\
	R0V1.2 & L'applicativo dovrà essere sviluppato con \citGloss{JavaScript} 8 (ES8) & \makecell[tl]{
		Capitolato
	} \\
	R0V1.3 & L'applicativo dovrà essere sviluppato con il boilerplate \citGloss{Redux} Minimal & \makecell[tl]{
		Capitolato
	} \\
	R0V1.4 & L'applicativo dovrà essere sviluppato utilizzando \citGloss{React} 15.x & \makecell[tl]{
		Capitolato
	} \\
	R0V1.5 & L'applicativo dovrà essere sviluppato utilizzando \citGloss{Redux} 3.x & \makecell[tl]{
		Capitolato
	} \\
	R0V1.6 & Il deploy del sito andrà eseguito utilizzando Surge.sh & \makecell[tl]{
		Capitolato
	} \\
	R0V1.7 & È desiderabile l'utilizzo di S\citGloss{CSS} in sostituzione a \citGloss{CSS} & \makecell[tl]{
		Capitolato
	} \\
	R0V2 & Gli \citGloss{smart contract} dovranno essere scritti in linguaggio \citGloss{Solidity} & \makecell[tl]{
		Capitolato
	} \\
	R0V3 & La connessione alla rete \citGloss{Ethereum} deve avvenire tramite MetaMask & \makecell[tl]{
		Capitolato
	} \\
	R0V3.1 & I test riguardanti gli \citGloss{smart contract} dovranno essere eseguiti in una rete locale e almeno in una rete pubblica & \makecell[tl]{
		Capitolato
	} \\
	R0V3.2 & Il deploy degli \citGloss{smart contract} dovrà avvenire su rete locale testrpc e rete di test Ropsten & \makecell[tl]{
		Capitolato
	} \\
	R2V3.3 & È apprezzabile un deploy finale sulla rete principale di \citGloss{Ethereum} & \makecell[tl]{
		Capitolato
	} \\
	R0V4 & Lo sviluppo degli \citGloss{smart contract} dovrà avvenire utilizzando il framework Truffle & \makecell[tl]{
		Capitolato
	} \\
	R0V5 & L'applicativo deve essere accessibile e utilizzabile dal \citGloss{browser} Mozilla \citGloss{Firefox} a partire dalla versione 52 & \makecell[tl]{
		Interno
	} \\
	R0V6 & L'applicativo deve essere accessibile e utilizzabile dal \citGloss{browser} Google \citGloss{Chrome} a partire dalla versione 57 & \makecell[tl]{
		Interno
	} \\
	R1V7 & L'applicativo deve essere accessibile e utilizzabile da un \citGloss{browser} mobile, per le versioni supportate fare riferimento alle controparti PC di \citGloss{Firefox} e \citGloss{Chrome} & \makecell[tl]{
		Capitolato
	} \\
	R0V8 & Un utente non deve poter compiere azioni sul sistema senza aver fatto l'accesso a esso & \makecell[tl]{
		Capitolato
	}\\
	R0V9 & L'applicazione dei principi de "App a 12 Fattori" deve essere documentata & \makecell[tl]{
		Capitolato 
		\\
		\\
		\\
		\\
	}\\
	R0V10 & Il codice sorgente deve essere pubblicato con licenza MIT & \makecell[tl]{
		Capitolato
	}\\
	\hiderowcolors
	\caption{Tabella di tracciamento requisiti-fonti}
\end{longtable}

\newpage
\section{Riepilogo}

\label{table:Riepilogo del numero dei requisiti individuati}
\rowcolors{2}{CRighePari}{CRigheDispari}
\renewcommand*{\arraystretch}{1.2}
\begin{longtable}[H]{p{2.8cm}p{2.9cm}p{2.9cm}p{2.9cm}p{1.5cm}}
	\rowcolor{CHeader}
	\color{CHeaderText} \textbf{Tipologia} & \color{CHeaderText} \textbf{0 Obbligatori} & \color{CHeaderText} \textbf{1 Desiderabili} & \color{CHeaderText} \textbf{2 Opzionali} & \color{CHeaderText} \textbf{Totale} \\
	Funzionali & 76 & 1 & 0 & 77 \\
	Di qualità & 9 & 0 & 0 & 9 \\
	Di vincolo & 17 & 2 & 1 & 20 \\
	Prestazionali & 0 & 0 & 0 & 0 \\
	\hiderowcolors
	\caption{Riepilogo del numero dei requisiti individuati}
\end{longtable}

\end{document}