\documentclass[PianoDiProgetto.tex]{subfiles}

\begin{document}
\chapter{Consuntivo di periodo e preventivo a finire}
Di seguito sono riportati i consuntivi e un breve resoconto della pianificazione per ogni periodo di progetto.\\\\
Nei consuntivi vengono riportate le ore e i costi preventivati e a fianco tra parentesi una valore: 
\begin{itemize}
	\item \textbf{Positivo:} se sono state necessarie meno ore di lavoro rispetto al preventivo o se il costo è diminuito;
	\item \textbf{Negativo:} se sono state necessarie più ore di lavoro rispetto al preventivo o se il costo è aumentato;
\end{itemize}
Se il valore tra parentesi non è presente vuol dire che il numero di ore preventivate o il costo è stato rispettato.\\\\
Per una maggiore lettura delle tabelle, i valori pari a 0 (zero) sono stati omessi.

\newpage
\section{Periodo di Analisi dei requisiti}
\subsection{Consuntivo}
La seguente tabella presenta le ore di lavoro utilizzate dai componenti del gruppo \gruppo\ nei vari ruoli di progetto rispetto al preventivo:
\begin{table}[H]
	\begin{center}
		\begin{tabu} to \textwidth {
				>{\centering}m{0.3\linewidth} 
				>{\centering}m{0.05\linewidth}  
				>{\centering}m{0.05\linewidth}  
				>{\centering}m{0.05\linewidth}
				>{\centering}m{0.05\linewidth}
				>{\centering}m{0.05\linewidth}
				>{\centering}m{0.05\linewidth}
				>{\centering\arraybackslash}m{0.15\linewidth}
			}
			\tableHeaderStyle			
			\textbf{Nome} & \textbf{Re} & \textbf{Ad} & \textbf{An} & \textbf{Pj} & \textbf{Pr} & \textbf{Ve} & \textbf{Totale} \\
			\Davide &  & 11 (+3) & 4 &  &  & 10 (-3) & 25 \\
			\Elena & 10 & 8 (-3) & 3 &  &  & 4 (+3) & 25 \\
			\Gianluca &  & 2 (-3) & 15 (-1) &  &  & 8 (+4) & 25 \\
			\Mirco &  & 4 (+4) & 16 (-1) &  &  & 5 (-3) & 25 \\
			\Parwinder & 8 & 8 (-7) & 3 (+3) &  &  & 6 (+4) & 25 \\
			\Riccardo &  & 11 (+3) & 4 (+2) &  &  & 10 (-5) & 25 \\
			\Valentina & 5 & 8 (-4) & 4 &  &  & 8 (+4)& 25 \\
		\end{tabu}
		\caption{Resoconto orario - Consuntivo Analisi dei requisiti}
		\vspace{-1em}
	\end{center}
\end{table}	
\newpage
La seguente tabella riassume i dati del consuntivo per il periodo di Analisi dei requisiti: 
\begin{table}[H]
	\begin{center}
		\begin{tabu} to \textwidth { 
				>{\centering}m{0.4\linewidth} 
				>{\centering}m{.2\linewidth}  
				>{\centering\arraybackslash}m{0.25\linewidth}  }
			\tableHeaderStyle
			\textbf{Ruolo} & \textbf{Ore} & \textbf{Costo in \euro} \\
			\resp & 23 & 690,00 \\
			\amme & 52 (-7) & 1.040,00 (-140,00)\\
			\alista & 49 (+3) & 1.225,00 (+75,00)\\
			\proga &  &  \\
			\progre &  &  \\
			\vere & 51 (+4) & 765,00 (+60) \\
			\textbf{Totale} & \textbf{175} & \textbf{3.725,00} \\
			\textbf{Differenza} & \textbf{0 ore} & \textbf{-5,00 \euro} \\
		\end{tabu}
		\caption{Resoconto economico - Consuntivo Analisi dei requisiti}
		\vspace{-1em}
	\end{center}
\end{table}

\subsection{Variazioni nella pianificazione}
Non sono stati commessi dei ritardi durante l'esecuzione delle varie attività previste.

\subsection{Conclusione}
Durante il periodo di Analisi dei requisiti è stato necessario usare più ore rispetto a quelle preventivate per il ruolo di \amme\ perché le attività comprendenti la stesura del \pdq, il \pdp\ e delle \ndp\ hanno necessitato più tempo rispetto a quello preventivato. Si è invece riusciti a recuperare qualche ora nel ruoli di \alista\ e \vere.\\
Il risultato complessivo del periodo è di nessuna ora lavorativa oltre il previsto, ma con una spesa aggiunta di 5,00 \euro.

\newpage
\section{Preventivo a finire}
La seguente tabella presenta l'attuale preventivo a finire. Il macro periodo iniziale di investimento, comprendente i periodi di Analisi dei requisiti e Consolidamento dei requisiti non è incluso nel totale delle ore rendicontate ed è riportato solo a scopo riassuntivo. Se il valore del consuntivo di un determinato periodo non è ancora presente, per il conteggio totale verrà utilizzato il valore presente nel preventivo.

\begin{table}[H]
	\begin{center}	
		\begin{tabu}to \textwidth  {
				>{\centering}m{0.45\linewidth} 
				>{\centering}m{0.2\linewidth}   
				>{\centering\arraybackslash}m{0.2\linewidth}}
			\tableHeaderStyle
			\textbf{Periodo} & \textbf{Preventivo \euro} & \textbf{Consuntivo \euro} \\
			
			Analisi dei requisiti & 3.720,00 & 3.725,00\\
			Consolidamento dei requisiti & 985,00 & - \\
			Progettazione della base tecnologica & 3.882,00 & - \\
			Progettazione di dettaglio e codifica & 6.713,00 & - \\
			Validazione e collaudo & 2.581,00 & - \\
			\textbf{Totale} & 17.911,00 & 17.916,00 \\
			\textbf{Rendicontato} & 13.176,00 & 13.181,00\\
			
		\end{tabu}
	\end{center}
	\caption{Consuntivo conclusivo}
\end{table}


\end{document}