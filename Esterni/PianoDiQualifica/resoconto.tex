\documentclass[PianoDiQualifica.tex]{subfiles}

\begin{document}

\chapter{Resoconto attività di verifica}
\section{Metriche sui documenti}
In questa fase del progetto le uniche metriche istanziate sono quelle riguardanti i processi e i documenti.
\subsection{Metriche processi}
\begin{table}[H]
	\begin{center}
		\begin{tabu} to \textwidth {
				>{\centering}m{0.34\linewidth}
				>{\centering}m{0.33\linewidth}
				>{\centering}m{0.33\linewidth} 
			}
			\tableHeaderStyle
			\textbf{ID} & \textbf{Valore ottenuto} & \textbf{Commento} \\%TODO metto valori
			\textbf{MPS001} &  & Ottimo risultato grazie ad Asana \\
			\textbf{MPS002} &  & Ottimo risultato grazie a \\
		\end{tabu}
		\caption{Resoconto delle misurazioni delle metriche di processo}
		\vspace{-1em}
	\end{center}
\end{table}

\subsection{Metriche prodotto}
In questa fase del progetto le uniche metriche di prodotto istanziate sono quelle riguardanti i documenti.
\subsection{Errori ortografici}
Tutti i documenti, dopo l'attento lavoro dei verificatori e il feedback positivo rilasciato dallo strumento di controllo ortografico dell'ambiente \citGloss{TexStudio}, risultano privi di errori ortografici, raggiungendo il valore accettabile e ottimale della metrica \textlink{MPDD003TAB}{MPDD003}{\textbf{MPDD003 Errori ortografici}}.

\subsection{Indice di Gulpease}
Grazie ad alcuni script automatici è stato possibile istanziare la metrica \textlink{MPDD001TAB}{MPDD001}{\textbf{MPDD001 Indice di Gulpease}} in modo coerente e ripetibile.\\
L'esecuzione degli script ha rivelato l'ottimo lavoro dei redattori, ottenendo valori nei range di accettazione per 2 documenti e valori nel range ottimale per 8 documenti, su un totale di 10 documenti.\\
Nella tabella sottostante è mostrato il dettaglio dei risultati ottenuti.
\begin{table}[H]
	\begin{center}
		\begin{tabu} to \textwidth {
				>{\centering}m{0.34\linewidth}
				>{\centering}m{0.33\linewidth}
				>{\centering}m{0.33\linewidth} 
			}
			\tableHeaderStyle
			\textbf{Nome documento} & \textbf{Indice di Gulpease} & \textbf{Esito} \\
			\textbf{Piano di Progetto} & 62.70 & Ottimo \\
			\textbf{Piano di Qualifica} & 55.16 & Accettato \\
			\textbf{Norme di Progetto} & 60.27 & Ottimo \\
			\textbf{Analisi dei Requisiti} & 90.89 & Ottimo \\
			\textbf{Studio di Fattibilità} & 58.61 & Accettato \\
			\textbf{Lettera di Presentazione} & 98.25 & Ottimo \\
			\textbf{VER-2017-11-13} & 67.49 & Ottimo \\
			\textbf{VER-2017-11-22} & 63.78 & Ottimo \\
			\textbf{VER-2017-12-08} & 62.99 & Ottimo \\
			\textbf{VER-2018-01-09} & 69.52 & Ottimo \\
			
		\end{tabu}
		\caption{Resoconto delle misurazioni sulla metrica MPDD001 - Indice di Gulpease}
		\vspace{-1em}
	\end{center}
\end{table}

\subsection{Formula di Flesch}
Non essendo ancora presenti documenti redatti in lingua inglese, non c'è stata ragione di istanziare questa metrica.
	
%\section{Revisione di Progettazione}	

%\section{Revisione di Qualifica}

%\section{Revisione di Accettazione}

\end{document}