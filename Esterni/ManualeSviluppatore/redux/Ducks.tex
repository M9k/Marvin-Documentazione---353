\documentclass[../redux]{subfiles}
\begin{document}
	\subsection{Ducks}
	To keep our app as simple as possible, we decided to use a modular approach to describe all the logical components.\\
	With Redux, we opted to use Ducks\footnote{\nURI{https://github.com/erikras/ducks-modular-redux}}.\\
	Ducks is a proposal for bundling reducers, action types and actions in a single file. To have the possibility to extend and reuse our ducks, we decided to use extensible-duck\footnote{\nURI{https://github.com/investtools/extensible-duck}} implementation.

	\subsubsection{Ducks list}
	In this section are reported all the ducks, with a little explanation of their function.
	\begin{itemize}
		\item Admin: used by an admin to manage teachers and students;
		\item AdminEmployer: used by the founder to manage the admins;
		\item Booking: used to log-in to Marvin;
		\item Course: used to insert and manage degree courses;
		\item Evaluator: used to assign votes;
		\item manageExam: used to manage the exams in a degree course;
		\item manageYears: used to manage all the academic years;
		\item Session: used to set up an user's session;
		\item Student: used by a student to manage their course of study;
		\item TeacherExam: used by a teacher to manage their exams.
	\end{itemize}

	\subsubsection{UML}
	\begin{figure}[H]
		\centering
		\includegraphics[width=1\linewidth]{"../diagrammi/redux/ducks"}
		\caption{Ducks components}
		\label{fig:Ducks components}
	\end{figure}
\end{document}