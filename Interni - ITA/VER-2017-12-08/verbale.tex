\documentclass[VER-2017-12-08.tex]{subfiles}
\begin{document}

\chapter{Informazioni sulla riunione}
\begin{itemize}
	\item \textbf{Motivo della riunione:} La riunione è stata effettuata utilizzando \citGloss{Skype} ed è servita a conoscere il Proponente Milo Ertola di persona. Sono stati chiariti alcuni dubbi sia da parte del gruppo 353 che dalla Proponente. Sono stati presentati e discussi i casi d'uso e i \citGloss{requisiti} fin'ora individuati.

	\item \textbf{Luogo e data:} Padova-Amsterdam 08-12-2107;
	\item \textbf{Ora di inizio:} 13:00;
	\item \textbf{Ora di fine:} 13:30;
	\item \textbf{Partecipanti:}
	\begin{itemize}
		\item Proponente:
		\begin{itemize}
			\item Alessandro Maccagnan;
			\item Milo Ertola;
		\end{itemize}
		\item Membri del gruppo 353:
		\begin{itemize}
			\item \Davide;
			\item \Elena;
			\item \Gianluca;
			\item \Mirco;
			\item \Parwinder;
			\item \Riccardo;
			\item \Valentina.
		\end{itemize}
	\end{itemize}
\end{itemize}
\chapter{Ordine del giorno}	
Di seguito sono riportati i punti dell'ordine del giorno che sono stati discussi insieme alla Proponente:
\begin{enumerate}
	\item Macro contratto; 
	\item Lingua dell'applicazione;
	\item Registrazione utente;
	\item Il personale amministrativo deve essere unico o devono esserci molteplici amministratori;
	\item La corrispondenza tra  "corso" e corso di laurea ed "esame" e corso singolo;
	\item Esami opzionali;
	\item Registrazione di un professore.
\end{enumerate}
\chapter{Resoconto}
\begin{enumerate}
	\item \textbf{Macro contratto:} L'incontro si è aperto con il gruppo \gruppo\ che espone l'idea di utilizzare un macro contratto, la quale è stata sconsigliata dalla Proponente perché più complessa da gestire e meno sicura. L'idea migliore è quella di implementare dei contratti modulari, in cui ogni contratto gestisce le azioni solo di quel modulo.\\
	\'{E} stato poi spiegato il funzionamento di React-Redux: React si occupa della visualizzazione mentre Redux dello stato del componente. Viene introdotto il concetto di immutabilità, cioè un oggetto non deve essere modificato, ma è necessario fare la copia di esso e poi modificarlo.\\
	Questo principio si ritrova anche nella blockchain, dove ogni blocco non viene mai modificato e lo stato finale è derivato dalla composizione di tutti i blocchi. Non viene mai persa un'informazione, ma viene sempre aggiunta.
	\item \textbf{Lingua:} E' stato chiarito che la lingua dell'interfaccia grafica deve essere sviluppata in inglese. Inoltre, anche i documenti che possono essere letti all'esterno del corso, le informazioni presenti nella \citGloss{repository} GitHub, i commenti al codice devono essere in inglese;
	\item \textbf{Registrazione:} La registrazione dell'utente non è stata specificata, la scelta è lasciata al gruppo;
	\item \textbf{Amministratori:} La Proponente ha consigliato l'implementazione di più amministratori, e non un unica entità di gestione dell'intero sistema. L'introduzione di appelli è facoltativa, ma è importante la data di registrazione del voto nel libretto;
	\item \textbf{Organizzazione:} Le scelte organizzative sono lasciate alla decisione del gruppo 353. Vengono reputate importanti le argomentazioni alla base delle scelte effettuate. Inoltre, fondamentale è utilizzare un approccio sicuro: più i contratti sviluppati sono piccoli, più è semplice gestire la sicurezza (ad esempio impedire che un utente semplice possa impersonificarsi professore);
	\item \textbf{Esami opzionali:} Gli esami opzionali richiesti nel capitolato corrispondono agli esami a scelta necessari per raggiungere i crediti sufficienti per la laurea;
	\item \textbf{Registrazione professori:} Come già spiegato prima, è lasciata al gruppo la scelta implementativa, tenendo sempre a mente sicurezza e modularità.
\end{enumerate}
\section{Argomenti discussi ma assenti dall'ordine del giorno}
\begin{itemize}
	\item Particolari raccomandazioni sono state fatte riguardo alla scrittura di codice in JavaScript: è stato richiesto l'utilizzo del linting fin da subito, per facilitare la navigazione nel codice;
	\item Un consiglio dato al gruppo riguarda la creazione di prototipi: se è in dubbio la validità di un'idea oppure se il dubbio riguarda la scelta tra più idee, è stato raccomandata la creazione di un prototipo, affermando che risolve velocemente i dubbi senza una perdita importante di tempo.
\end{itemize}
\end{document}