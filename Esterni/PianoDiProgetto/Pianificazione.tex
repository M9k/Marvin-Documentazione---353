\documentclass[PianoDiProgetto.tex]{subfiles}

\begin{document}

\chapter{Pianificazione}
La pianificazione del gruppo \gruppo è stata costruita sulla base delle scadenze riportate in \S \ref{scadenze} di questo documento. Seguendo quelle scadenze è stato deciso di suddividere lo sviluppo in cinque periodi, raggruppati in due macro periodi:
\begin{itemize}
	\item \textbf{Investimento:}
		\begin{itemize}
			\item \textbf{Analisi dei requisiti};
			\item \textbf{Consolidamento dei requisiti}.
		\end{itemize}
	\item \textbf{Preventivo:}
		\begin{itemize}
			\item \textbf{Progettazione della base tecnologica};
			\item \textbf{Progettazione di dettaglio e codifica};
			\item \textbf{Validazione e collaudo}.
		\end{itemize}
\end{itemize}
Il primo macro periodo è a carico del gruppo \gruppo, mentre il secondo è a carico del \citGloss{committente}.\\
Ognuna di queste cinque fasi è poi stata scomposta nelle attività che verranno svolte durante la fase stessa, come riportato nei corrispettivi diagrammi di \citGloss{Gantt}. \`{E} stata segnata, come scadenza di ogni fase, il giorno di consegna dei materiali che sono stati prodotti dalle attività svolte. Tali scadenze sono segnate come \citGloss{milestone}.
Sono state inoltre segnate come \citGloss{milestone} anche le verifiche che avverranno sotto forma di discussioni Agile per la \tb e \pb, in data ancora da definire.\\ Ogni attività è rappresentata tramite le sue sotto attività, mostrando una rappresentazione ad alto livello del lavoro che verrà svolto.
\newpage

\section{Analisi dei requisiti}
Il periodo di analisi comincia il \nData{23}{11}{2017} e si conclude il \nData{16}{01}{2018}: l'inizio coincide con la formazione del gruppo e l'avviamento del lavoro; la conclusione con la scadenza scelta per la consegna dei documenti per l'entrata nel progetto.

\subsection{Incrementi}
Durante questo periodo vengono effettuati 5 incrementi e le attività principali svolte sono:
\begin{itemize}
	\item \textbf{Norme di Progetto:}
	in questa attività l'Amministratore stabilisce tutte le norme a cui il gruppo \gruppo deve sottostare durante tutta la durata del progetto. Viene redatto il documento di supporto \ndp contenente tutte le norme stabilite. Questa attività è considerata critica dato che le \ndp sono essenziali, infatti esse stabiliscono anche le norme e gli strumenti che verranno usati per la stesura dei documenti;
	
	
	\item \textbf{Studio di Fattibilità:} 
	questa attività consiste nell'analisi dei vari capitolati proposti ed è essenziale per la scelta del capitolato che verrà svolto. Viene redatto il documento di supporto \sdf contenente l'analisi effettuata. L'attività è considerata critica ed è bloccate per l'inizio dell'\adr;
		
	\item \textbf{Analisi dei Requisiti}: 
		in questa attività gli Analisti ricavano tutti i requisiti del capitolato scelto e riportano nel documento \adr. Questa attività è molto importante e critica per il proseguimento del progetto;
	
	\item \textbf{Piano di Progetto:} in questa attività il Responsabile analizza le attività necessarie e le loro scadenze per la buona riuscita del progetto e l'\amme analizza i rischi nei quali il gruppo \gruppo può incombere durante il progetto.  Inoltre durante questa attività vengono suddivise le risorse disponibili per l'intera durata del progetto. Viene redatto il documento \pdp contenente la pianificazione ed analisi effettuata. L'attività è considerata critica e bloccante per la stesura della \lettera;
	\item \textbf{Piano di Qualifica}: in questa attività gli \ammi individuano i metodi per garantire la qualità del prodotto. In oltre viene redatto il documento \pdq contenente i metodi individualizzati;
	\item \textbf{Glossario:} vengono individualizzati tutti i termini considerati poco chiari ed ambigui e redatti nel documento \g.
		
\end{itemize}	 
\begin{landscape}
\subsection{Analisi - Diagramma di Gantt}
\nImg{images/gantt/analisi.png}{Diagramma di Gantt del periodo di Analisi}{20.5}			
\end{landscape}	
\section{Consolidamento dei requisiti}
Il periodo di consolidamento dei \citGloss{requisiti} inizia il \nData{16}{01}{2018} e termina il \nData{26}{01}{2018}: comincia il giorno della consegna dei documenti per la prima scadenza e termina il giorno della presentazione della Revisione dei \citGloss{requisiti}.

\subsection{Incrementi}
 In questo periodo verrà effettuato un incremento in cui l'attività principale è il miglioramento dei documenti e dell'\adr in vista dell'inizio del periodo di Progettazione della base tecnologica.
\subsection{Consolidamento dei requisiti - Diagramma di Gantt}
\nImg{images/gantt/consolidamento.png}{Diagramma di Gantt del periodo di Consolidamento dei requisiti}{14.5}

\newpage

\section{Progettazione della base tecnologica}
Il periodo di Progettazione della base tecnologica inizia il \nData{27}{01}{2018} e termina il \nData{12}{03}{2018}: comincia il giorno dopo la presentazione per la Revisione dei \citGloss{requisiti} e si conclude con la consegna dei documenti per la Revisione di Progettazione.

\subsection{Incrementi}
In questo periodo vengono effettuati 4 incrementi e le attività principali sono:

\begin{itemize}
	\item \textbf{Modifica e Verifica:} all'inizio del periodo vengono modificati e verificati i documenti (\ndp, \pdp, \pdq e \adr) seguendo le indicazioni risultanti dalla Revisione di Requisiti;
	%adr rimane?
	\item \textbf{Glossario:} questa attività comprende sia il miglioramento del documento \g che l'aggiunta di nuovi termini;
	\item \textbf{Technology Baseline:} 
	in questa attività vengono studiate, analizzate e scelte le tecnologie, i framework e le librerie per lo sviluppo del prodotto e il \citGloss{Proof of Concept}. Questa attività è considerata critica e bloccante per la prosecuzione del progetto. In una data ancora indefinita tra il \nData{27}{02}{2017} e il \nData{12}{03}{2017} avvererà una discussione in modalità Agile per la \citGloss{verifica} di questa analisi tramite presentazione o documento condiviso.
\end{itemize}

\begin{landscape}
	\subsection{Progettazione della base tecnologica - Diagramma di Gantt}
	\nImg{images/gantt/progArch.png}{Diagramma di Gantt del periodo di Progettazione della base tecnologica}{19}
\end{landscape}


\section{Progettazione di dettaglio e codifica}
Il periodo di Progettazione di dettaglio e codifica inizia il \nData{19}{03}{2018} e termina il \nData{16}{04}{2018}: comincia il giorno stesso della Revisione di Progettazione e si conclude con la consegna dei documenti per la Revisione di Qualifica.

\subsection{Incrementi}
Durante questo periodo vengono effettuati 3 incrementi e le attività principali svolte sono:
\begin{itemize}
	\item \textbf{Modifica e Verifica:} all'inizio del periodo vengono modificati e verificati i documenti(\ndp, \pdp e \pdq) seguendo le indicazioni risultanti dalla Revisione di Progettazione;
	\item \textbf{Glossario:} questa attività comprende sia il miglioramento del \g che l'aggiunta di nuovi termini;
%	\item \textbf{Lettera di presentazione:} questa attività prevede la stesura della lettera di presentazione per la Revisione di Qualifica;
	\item \textbf{Product Baseline:} questa attività presenta la baseline architetturale del prodotto tramite i diagrammi delle classi e di sequenza, mostrandone la coerenza con quanto mostrato durante l'attività di \tb;
	\item \textbf{Codifica:} questa attività consiste nella scrittura del codice e nella sua \citGloss{verifica};
	\item \textbf{Manuale Utente:} questa attività consiste nella redazione del \mut, contenente indicazioni sull'utilizzo dell'applicazione che sta venendo prodotta.

\end{itemize}
\begin{landscape}
		\subsection{Progettazione di dettaglio e codifica - Diagramma di Gantt}
	\nImg{images/gantt/qualifica.png}{Diagramma di Gantt del periodo Progettazione di dettaglio e codifica}{20.5}	
\end{landscape}


\section{Validazione e collaudo}
Il periodo di \citGloss{validazione} e collaudo inizia con il \nData{23}{04}{2018} e termina il \nData{07}{05}{2018}: comincia il giorno stesso della Revisione di Qualifica e si conclude con la consegna dei documenti per la Revisione di Accettazione.
\subsection{Incrementi}
Durante questo periodo vengono effettuati 2 incrementi, le attività principali svolte sono:
\begin{itemize}
\item \textbf{Modifica e Verifica:} all'inizio del periodo vengono modificati e verificati  i documenti (\ndp, \pdp, \pdq e \pb) seguendo le indicazioni risultanti dalla Revisione di Qualifica;
\item \textbf{Glossario:} questa attività comprende sia il miglioramento del \g che l'aggiunta dei nuovi termini;
\item \textbf{Validazione e Collaudo:} questa attività consiste nell'esecuzione di ulteriori test e miglioramenti dell'applicazione prodotto per assicurare che soddisfi tutti i vincoli qualitativi;
\item \textbf{Manuale Utente:} questa attività consiste nel miglioramento e completamento del \mut, contenente indicazioni sull'utilizzo dell'applicazione.

\end{itemize}
\begin{landscape}
\subsection{Validazione e collaudo - Diagramma di Gantt}
	\nImg{images/gantt/collaudo.png}{Diagramma di Gantt del periodo Validazione e collaudo}{21}
\end{landscape}	
\end{document}
