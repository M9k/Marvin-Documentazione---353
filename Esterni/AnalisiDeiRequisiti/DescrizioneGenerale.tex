\documentclass[AnalisiDeiRequisiti.tex]{subfiles}

\begin{document}

\chapter{Descrizione generale}
\section{Obbiettivi del prodotto}
L'obiettivo finale del prodotto è la realizzazione di un simulatore di \citGloss{Uniweb}: portale che consente a studenti, professori e università la gestione dei propri dati e dei processi burocratici in forma telematica. \\
La peculiarità del sistema consiste nel suo dominio tecnologico: la quasi totalità della logica applicativa dovrà essere consegnata nella \citGloss{blockchain} pubblica "\citGloss{Ethereum}" ed esposta agli utenti tramite un'interfaccia web. \\
Lo scopo fondamentale è dimostrare la concreta possibilità di sviluppare applicazioni decentralizzate che non riguardino esclusivamente il mondo delle criptovalute.

\section{Funzionalità del prodotto}
Le funzioni del prodotto si avvicinano a quelle basilari offerte dal sistema \citGloss{Uniweb} dell'Università di Padova:
\begin{itemize}
	\item La possibilità per gli studenti di iscriversi ad un \citGloss{corso di laurea} e gestire il proprio percorso accademico;
	\item La possibilità per i professori di gestire i propri corsi di laurea e gli esami erogati agli studenti;
	\item La possibilità per il gestore dell'università e per gli amministratori di gestire gli account degli studenti e dei professori e di inserire anni accademici, corsi di laurea ed esami.
\end{itemize}
Data la particolarità della rete \citGloss{Ethereum}, in cui ogni operazione di scrittura ha un costo economico, l'applicativo fornirà all'utente, in maniera preventiva, la possibilità di visualizzare la stima del prezzo di ogni operazione da effettuare.

\subsection{Funzionalità offerte agli studenti}
Gli studenti avranno la possibilità di consultare gli esiti dei propri esami e, ove il corso lo prevede, l'iscrizione ad esami opzionali.\\
Oltre a ciò sarà presente una schermata riassuntiva che permetta allo studente di conoscere il numero di crediti che ha accumulato fino a quel momento e quanti ne servono per raggiungere la laurea.

\subsection{Funzionalità offerte ai professori}
I professori avranno la possibilità di consultare l'elenco degli esami di loro competenza, visualizzando per ognuno gli studenti iscritti ed assegnare le valutazioni ad essi.

\subsection{Funzionalità offerte agli amministratori}
Gli amministratori avranno il compito di gestire l'inserimento dei corsi di laurea all'interno degli anni accademici e la creazione di esami all'interno dei corsi.\\
Questo compito è stato delegato a loro e non viene gestito direttamente dal gestore dell'università soprattutto per motivi pratici e di sicurezza, in quanto l'aggiunta degli esami sarebbe un lavoro oneroso da svolgere singolarmente e la condivisione della chiave privata relativa alla gestione dell'università avrebbe portato a problemi di sicurezza.\\
Ulteriormente avranno il compito di approvare o declinare le richieste di sottoscrizione all'università da parte di professori e studenti ed avranno la possibilità di rimuoverli successivamente.


\subsection{Funzionalità offerte al gestore dell'università}
L'università avrà il compito di gestire gli amministratori, inserendoli e rimuovendoli dal sistema.\\
Un secondo compito di sua competenza è la creazione degli anni accademici, ed in caso di errato inserimento può anche eliminarne, ma solamente se non sono stati inseriti corsi associati a quel determinato anno.\\
Come precedentemente accennato non sarà il gestore dell'università a doversi occupare direttamente dell'inserimento dei corsi di laurea e degli esami, in quanto sarebbe un compito davvero oneroso per un singolo individuo, quindi deve delegare altre persone, cioè gli amministratori, per l'esecuzione e la suddivisione di questo impegno.\\
Ovviamente nulla vieta al gestore dell'università la creazione di un secondo account da registrare come amministratore per svolgere, od aiutare nello svolgimento, il ruolo di amministratore.

\section{Caratteristiche degli utenti}
Il prodotto si rivolge ad una classe d'utenza collusa con il mondo accademico.\\
L'utente, per accedere al sistema, dovrà possedere alcune nozioni base sul funzionamento della rete \citGloss{Ethereum} e del plugin \citGloss{Metamask}, conoscenze che potranno essere acquisite tramite una breve guida fornita dall'applicazione.

\section{Piattaforma d'esecuzione}
Il prodotto finale può essere suddiviso in due macro parti con due diverse piattaforme d'esecuzione: il backend verrà ospitato dalla rete \citGloss{Ethereum}, l'interfacciamento ad esso (frontend) sarà fruibile tramite un qualsiasi \citGloss{browser} web predisposto all'installazione del plugin \citGloss{Metamask}.

\section{Vincoli generali}
L'utente, per usufruire del servizio, dovrà possedere un \citGloss{browser} con installato il plugin \citGloss{Metamask}, una connessione a internet e una coppia di chiavi (pubblica-privata) compatibile con la rete \citGloss{Ethereum}.	

\end{document}