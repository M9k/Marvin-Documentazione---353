\documentclass[AnalisiDeiRequisiti.tex]{subfiles}

\begin{document}

\chapter{Requisiti}
\section{Classificazione dei requisiti}
I requisiti vengono classificati ed assegnati loro un identificativo univoco secondo quanto definito nel documento \textit{Norme di progetto v1.0.0}.

% Please add the following required packages to your document preamble:
% \usepackage[table,xcdraw]{xcolor}
% If you use beamer only pass "xcolor=table" option, i.e. \documentclass[xcolor=table]{beamer}
% \usepackage{longtable}
% Note: It may be necessary to compile the document several times to get a multi-page table to line up properly
% Please add the following required packages to your document preamble:
% \usepackage[table,xcdraw]{xcolor}
% If you use beamer only pass "xcolor=table" option, i.e. \documentclass[xcolor=table]{beamer}
% \usepackage{longtable}
% Note: It may be necessary to compile the document several times to get a multi-page table to line up properly

\subsection{Requisiti funzionali}

\label{table:Tabella requisiti funzionali}
\begin{longtable}[H]{|p{2.5cm}|p{2.5cm}|p{5cm}|p{2cm}|}
	\hline
	\rowcolor[HTML]{38FFF8} 
	\textbf{Identificatore} & \textbf{Importanza} & \textbf{Descrizione} & \textbf{Fonti} \\ \hline
	\endhead
	R0F1 & Obbligatorio & L'amministratore può gestire gli utenti & Interno \\ \hline
	R0F1.1 & Obbligatorio & L'amministratore può ottenere una lista di tutti gli utenti non abilitati & Interno \\ \hline
	R0F1.2 & Obbligatorio & L'amministratore può abilitare un utente & Interno \\ \hline
	R0F1.3 & Obbligatorio & L'amministratore può rimuovere un utente & Interno \\ \hline
	R0F1.4 & Obbligatorio & L'amministratore può ottenere una lista di tutti gli studenti & Interno \\ \hline
	R0F2 & Obbligatorio & L'amministratore può gestire i professori & Interno \\ \hline
	R0F2.1 & Obbligatorio & L'amministratore può ottenere una lista di tutti i professori & Interno \\ \hline
	R0F2.2 & Obbligatorio & L'amministratore può assegnare un esame ad un determinato professore & Capitolato \\ \hline
	R0F3 & Obbligatorio & L'amministratore può gestire gli anni accademici & Capitolato \\ \hline
	R0F3.1 & Obbligatorio & L'amministratore può aggiungere un anno accademico & Capitolato \\ \hline
	R0F3.2 & Obbligatorio & L'amministratore può ottenere una lista di gli anni accademici & Interno \\ \hline
	R0F3.3 & Obbligatorio & L'amministratore può aggiungere dei corsi di laurea ad ogni anno accademico & Capitolato \\ \hline
	R2F3.4 & Opzionale & L'amministratore può rimuovere corsi di laurea da ogni anno accademico & Interni \\ \hline
	R0F3.5 & Obbligatorio & L'amministratore può gestire i corsi di laurea relativi agli anni accademici & Capitolato \\ \hline
	R0F3.5.1 & Obbligatorio & L'amministratore può creare un nuovo corso di laurea a partire da un anno accademico & Capitolato \\ \hline
	R0F3.5.2 & Obbligatorio & L'amministratore può ottenere una lista di tutti i corsi di laurea dato un anno accademico & Interno \\ \hline
	R0F3.5.3 & Obbligatorio & L'amministratore può creare nuovi esami & Capitolato \\ \hline
	R0F3.5.4 & Obbligatorio & L'amministratore può aggiungere degli esami ad ogni corso di laurea & Capitolato \\ \hline
	R0F3.5.5 & Opzionale & L'amministratore può rimuovere degli esami da un corso di laurea & Interno \\ \hline
	R0F3.5.6 & Obbligatorio & L'amministratore può ottenere una lista di tutti gli esami dato il corso di laurea & Interno \\ \hline
	R0F3.5.6.1 & Obbligatorio & L'amministratore può ottenere l'informazione di quale professore è assegnato a un determinato esame & Interno \\ \hline
	R0F3.5.6.2  & Obbligatorio & L'amministratore può associare ad un esame il relativo professore & Interno \\ \hline
	R0F4 & Obbligatorio & Un professore può gestire gli esami a lui assegnati & Capitolato \\ \hline
	R0F4.1 & Obbligatorio & Un professore può ottenere la lista di tutti gli esami a lui assegnati & Capitolato \\ \hline
	R0F4.2 & Obbligatorio & Un professore può ottenere la lista di tutti gli studenti associati a un determinato esame & Capitolato \\ \hline
	R0F4.3 & Obbligatorio & Un professore può registrare un esito ad un dato esame ad un dato studente registrato a quell'esame & Capitolato \\ \hline
	R0F5 & Obbligatorio & Uno studente può gestire l'iscrizione agli esami & Capitolato \\ \hline
	R0F5.1 & Obbligatorio & Uno studente può vedere l'elenco degli esami ai quali è iscritto & Capitolato \\ \hline
	R0F5.1.1 & Obbligatorio & Uno studente può ottenere il numero di crediti assegnati agli esame & Interno \\ \hline
	R0F5.1.2 & Obbligatorio & Uno studente può ottenere l'obbligatorietà degli esame & Interno \\ \hline
	R0F5.1.3 & Obbligatorio & Uno studente può ottenere lo stato di superamento degli esame & Interno \\ \hline
	R0F5.1.4 & Obbligatorio & Uno studente può ottenere il totale dei suoi crediti & Interno \\ \hline
	R0F5.1.5 & Obbligatorio & Uno studente può ottenere il numero di crediti da raggiungere per la laurea & Interno \\ \hline
	R0F5.2 & Obbligatorio & Uno studente può ottenere l'elenco degli esami opzionali & Capitolato \\ \hline
	R0F5.2.1 & Obbligatorio & Uno studente può ottenere il numero di crediti degli esami opzionali & Capitolato \\ \hline
	R0F5.2.2 & Obbligatorio & Uno studente può iscriversi ad un esame opzionale & Capitolato \\ \hline
	R0F6 & Obbligatorio & L'utente può effettuare il login & Interno \\ \hline
	R0F6.1 & Obbligatorio & Il login deve avvenire tramite il controllo delle chiavi, senza ulteriori azioni da parte dell'utente & Interno \\ \hline
	R0F7 & Obbligatorio & L'utente può effettuare il logout & Interno \\ \hline
	R0F8 & Obbligatorio & L'utente non ancora registrato può registrarsi & Capitolato \\ \hline
	R0F8.1 & Obbligatorio & La registrazione necessita di nome, cognome, categoria (studente o professore) e selezione del corso di laura se si tratta di uno studente & Capitolato \\ \hline
	R1F9 & Desiderabile & L'utente può leggere una breve guida sull'uso di MetaMask e sul pagamento delle operazioni & Interno \\ \hline	
	R0F10 & Obbligatorio & L'utente deve poter vedere preventivamente il costo in Gas, Ether e Euro dell'operazione che sta per eseguire & Capitolato \\ \hline	
	R0F11 & Obbligatorio & L'università deve poter gestire gli amministratori & VE171209 \\ \hline %TODO: controllare se il nome del verbale coindice
	R0F11.1 & Obbligatorio & L'università deve poter aggiungere amministratori & VE171209 \\ \hline %TODO: controllare se il nome del verbale coindice
	R0F11.2 & Obbligatorio & L'università deve poter rimuovere amministratori & VE171209 \\ \hline %TODO: controllare se il nome del verbale coindice
	R0F11.3 & Obbligatorio & L'università deve poter ottenere la lista di tutti gli amministratori & Interno \\ \hline 
	\caption{Tabella dei requisiti funzionali}
\end{longtable}

\subsection{Requisiti di qualità}

\label{table:Tabella requisiti di qualita'} %TODO: cercare un modo di inserire à nei label
%TODO: concordare notazione dei verbali
\begin{longtable}[H]{|p{2.5cm}|p{2.5cm}|p{5cm}|p{2cm}|}
	\hline
	\rowcolor[HTML]{38FFF8} 
	\textbf{Identificatore} & \textbf{Importanza} & \textbf{Descrizione} & \textbf{Fonti} \\ \hline
	\endhead
	R0Q1 & Obbligatorio & La progettazione e il codice devono seguire le norme e le metriche riportate nei documenti allegati X e Y & Interno \\ \hline %TODO: indicare allegati
	R0Q2 & Obbligatorio & L'approccio di scrittura di JavaScript deve essere promise Centric Approach & Capitolato \\ \hline
	R0Q2.1 & Obbligatorio & L'applicativo non deve fare uso di callback in presenza di alternative alle ultime & VE171122 \\ \hline
	R0Q3 & Obbligatorio & Il codice Javascript deve attenersi al Airbnb Javascript style guide & Capitolato \\ \hline
	R0Q4 & Obbligatorio & Lo sviluppo deve essere supportato dall'utilizzo del tool ESLint & Capitolato \\ \hline
	R0Q5 & Obbligatorio & Dovrà essere fornito un manuale utente in lingua inglese che tratterà l'uso da parte di studenti e professori & VE171122 \\ \hline
	R0Q6 & Obbligatorio & Dovrà essere fornito un manuale di deploy e di utilizzo da parte degli amministratori in lingua inglese & VE171122 \\ \hline %TODO controllare sigla del verbale
	R0Q7 & Obbligatorio & Il codice sorgente deve essere pubblicato sulla piattaforma GitHub, BitBucket o GitLab & Capitolato \\ \hline
	R0Q8 & Obbligatorio & Il codice deve attenersi il più possibile alle guide linea de "App a 12 Fattori" & Capitolato \\ \hline
	\caption{Tabella dei requisiti di qualità}
\end{longtable}

\subsection{Requisiti di vincolo}

\label{table:Tabella requisiti di vincolo}

\begin{longtable}[H]{|p{2.5cm}|p{2.5cm}|p{5cm}|p{2cm}|}
	\hline
	\rowcolor[HTML]{38FFF8} 
	\textbf{Identificatore} & \textbf{Importanza} & \textbf{Descrizione} & \textbf{Fonti} \\ \hline
	\endhead
	R0V1 & Obbligatorio & L'applicativo dovrà essere sviluppato attraverso l'uso di tecnologie web & Capitolato \\ \hline
	R0V1.1 & Obbligatorio & L'applicativo dovrà essere sviluppato con Node.js & Capitolato \\ \hline
	R0V1.2 & Obbligatorio & L'applicativo dovrà essere sviluppato con JavaScript 8 (ES8) & Capitolato \\ \hline
	R0V1.3 & Obbligatorio & L'applicativo dovrà essere sviluppato con il boilerplate Redux Minimal & Capitolato \\ \hline
	R0V1.4 & Obbligatorio & L'applicativo dovrà essere sviluppato utilizzando React & Capitolato \\ \hline
	R0V1.5 & Obbligatorio & L'applicativo dovrà essere sviluppato utilizzando Redux & Capitolato \\ \hline
	R0V1.6 & Obbligatorio & Il deploy del sito andrà eseguito utilizzando Surge.sh & Capitolato \\ \hline
	R0V1.7 & Desiderabile & È desiderabile l'utilizzo di SCSS in sostituzione a CSS & Capitolato \\ \hline
	R0V2 & Obbligatorio & Gli smart contract dovranno essere scritti in linguaggio Solidity & Capitolato \\ \hline
	R0V3 & Obbligatorio & La connessione alla rete Ethereum deve avvernire tramite MetaMask & Capitolato \\ \hline
	R0V3.1 & Obbligatorio & I test riguartandi gli smart contract dovranno essere eseguiti in una rete locale ed almeno in una rete pubblica & Capitolato \\ \hline
	R0V3.2 & Obbligatorio & Il deploy degli smart contract dovrà avvenire su rete locale testrpc e rete di test Ropsten & Capitolato \\ \hline
	R2V3.3 & Opzionale & È apprezzabile un deploy finale sulla rete principale di Ethereum & Capitolato \\ \hline
	R0V4 & Obbligatorio & Lo sviluppo degli smart contract dovrà avvenire utilizzando il framework Truffle & Capitolato \\ \hline
	R0V5 & Obbligatorio & L'applicativo deve essere accessibile ed utilizzabile dal browser Mozilla Firefox a partire dalla versione X & Interno \\ \hline %TODO: definire versione
	R0V6 & Obbligatorio & L'applicativo deve essere accessibile ed utilizzabile dal browser Google Chrome a partire dalla versione X & Interno \\ \hline %TODO: definire versione
	R1V7 & Desiderabile & L'applicativo deve essere accessibile ed utilizzabile da un browser mobile, per le versioni supportate fare riferimento alle controparti PC di Firefox e Chrome & Interno \\ \hline
	R0V8 & Obbligatorio & Un utente non deve poter compiere azioni sul sistema senza aver fatto l'accesso ad esso & Capitolato \\ \hline
	R0V9 & Obbligatorio & L'applicazione dei principi de "App a 12 Fattori" deve essere documentata & Capitolato \\ \hline
	R0V10 & Obbligatorio & Il codice sorgente deve essere pubblicato con licenza MIT & Capitolato \\ \hline
	\caption{Tabella dei requisiti di vincolo}
\end{longtable}

\subsection{Requisiti prestazionali}

Non sono stati individuati requisiti prestazionali, in quanto la maggior parte delle attività, per essere concluse, necessitano di una interazione con una rete Ethereum, e quindi non risultano costanti o prevedibili con precisione.\\
Qualsiasi operazione effettuata in una rete Ethereum reale ha un tempo di soddisfacimento casuale influenzato dal carico della rete nel momento della richiesta e, nel caso di operazioni che vanno a modificare lo stato di un contratto, del quantitativo di Ether pagati per ogni unità di Gas.\\
 
\begin{comment}
\label{table:Tabella requisiti prestazionali}
\begin{longtable}[H]{|p{2.5cm}|p{2.5cm}|p{5cm}|p{2cm}|}
	\hline
	\rowcolor[HTML]{38FFF8} 
	\textbf{Identificatore} & \textbf{Importanza} & \textbf{Descrizione} & \textbf{Fonti} \\ \hline
	\endhead
	&  &  &  \\ \hline
	&  &  &  \\ \hline
	&  &  &  \\ \hline
	&  &  &  \\ \hline
	&  &  &  \\ \hline
	\caption{Tabella dei requisiti prestazionali}
\end{longtable}
\end{comment}
\section{Tracciamento}
\subsection{Tracciamento fonti-requisiti}

\label{table:Tabella di tracciamento fonti-requisiti}
\begin{longtable}[H]{|p{2cm}|p{5cm}|p{5cm}|}
	\hline
	\rowcolor[HTML]{38FFF8} 
	\textbf{Fonte} & \textbf{Nome fonte} & \textbf{Requisiti} \\ \hline
	\endhead
	Capitolato & & R0F2.2 \\
	& & R0F3 \\
	& & R0F3.1 \\
	& & R0F3.3 \\
	& & R0F3.5 \\
	& & R0F3.5.1 \\
	& & R0F3.5.3 \\
	& & R0F3.5.4 \\
	& & R0F4 \\
	& & R0F4.1 \\
	& & R0F4.2 \\
	& & R0F4.3 \\
	& & R0F5 \\
	& & R0F5.1 \\
	& & R0F5.2 \\
	& & R0F5.2.1 \\
	& & R0F5.2.2 \\
	& & R0F7 \\
	& & R0F8 \\
	& & R0F8.1 \\
	& & R0F10 \\
	& & R0Q2 \\
	& & R0Q3 \\
	& & R0Q4 \\
	& & R0Q7 \\
	& & R0Q8 \\
	& & R0V1 \\
	& & R0V1.1 \\
	& & R0V1.2 \\
	& & R0V1.3 \\
	& & R0V1.4 \\
	& & R0V1.5 \\
	& & R0V1.6 \\
	& & R0V1.7 \\
	& & R0V2 \\
	& & R0V3 \\
	& & R0V3.1 \\
	& & R0V3.2 \\
	& & R2V3.3 \\
	& & R0V4 \\
	& & R1V7 \\
	& & R0V8 \\
	& & R0V9 \\
	& & R0V10 \\ \hline
	
	Interno & & R0F1 \\
	& & R0F1.1 \\
	& & R0F1.2 \\
	& & R0F1.3 \\
	& & R0F1.4 \\
	& & R0F2 \\
	& & R0F2.1 \\
	& & R0F3.2 \\
	& & R0F3.4 \\
	& & R0F3.5.2 \\
	& & R0F3.5.5 \\
	& & R0F3.5.6 \\
	& & R0F3.5.6.1 \\
	& & R0F3.5.6.2 \\
	& & R0F5.1.1 \\
	& & R0F5.1.2 \\
	& & R0F5.1.3 \\
	& & R0F5.1.4 \\
	& & R0F5.1.5 \\
	& & R0F6 \\
	& & R0F6.1 \\
	& & R1F9 \\
	& & R0F11.3 \\
	& & R0Q1 \\
	& & R0V5 \\
	& & R0V6 \\ \hline
	
	VE171122 & Verbale & R0Q2.1 \\
	& & R0Q5 \\
	& & R0Q6 \\ \hline
	
	VE171209 & Verbale & R0F11 \\
	& & R0F11.1 \\
	& & R0F11.2 \\ \hline
	
	UC1 & Breve guida & R1F9 \\ \hline
	UC2 & Login & R0F6 \\ \hline
	UC2.1 & Login automatico & R0F6.1 \\ \hline
	UC2.2 & Visualizzazione messaggio di errore riguardo a chiave assente & R0F6.1 \\ \hline
	UC2.3 & Visualizzazione messaggio di errore riguardo a chiave malformata & R0F6.1 \\ \hline
	UC2.4 & Visualizzazione messaggio di errore riguardo a chiave non registrata & R0F6.1 \\ \hline
	UC3 & Registrazione & R0F8 \\ \hline %TODO
	UC3.1 & Inserimento nome & R0F8.1 \\ \hline
	UC3.2 & Inserimento cognome & R0F8.1 \\ \hline
	UC3.3 & Selezione categoria & R0F8.1 \\ \hline
	UC3.4 & Selezione corso & R0F8.1 \\ \hline
	UC3.5 & Visualizzazione errore campo non compilato & R0F8.1 \\ \hline
	UC3.6 & Visualizzazione errore utente già registrato & R0F8.1 \\ \hline
	UC3.7 & Visualizzazione errore chiave non presente & R0F8.1 \\ \hline
	UC3.8 & Visualizzazione errore chiave malformata & R0F8.1 \\ \hline
	UC4 & Logout & R0F7 \\ \hline
	UC5 & Amministrazione &  R0F1 \\ 
	& & R0F2 \\
	& & R0F3 \\ \hline
	UC5.1 & Gestione Utenti &  R0F1 \\ \hline
	UC5.1.1 & Visualizzazione lista di studenti & R0F1.4 \\ \hline
	UC5.1.2 & Visualizzazione lista di professori &  R0F2.1\\ \hline
	UC5.1.3 & Visualizzazione di utenti non abilitati & R0F1.1 \\ \hline
	UC5.1.4 & Abilitazione utente & R0F1.2 \\  \hline 
	UC5.1.5 & Rimozione utente & R0F1.3 \\ \hline
	UC5.2 & Gestione anni accademici &  R0F3 \\ \hline
	UC5.2.1 & Aggiunta anno accademico &  R0F3.1 \\ \hline
	UC5.2.2 & Visualizzazione lista di tutti gli anni accademici & R0F3.2 \\ \hline
	UC5.2.3 & Aggiunta di un corso di laurea ad un anno accademico & R0F3.3 \\ \hline
	UC5.2.4 & Visualizzazione messaggio anno malformato & R0F3.1 \\ \hline
	UC5.2.5 & Visualizzazione messaggio di anno accademico già presente & R0F3.1 \\ \hline
	UC5.2.6 & Visualizzazione messaggio anno non compilato &  R0F3.1 \\ \hline
	UC5.3 & Gestione corsi di laurea &  R0F3.5 \\  \hline
	UC5.3.1 & Creazione corso di laurea &  R0F3.5.1 \\ \hline
	UC5.3.2 & Visualizzazione errore mancata compilazione campi & R0F3.5.1 \\ \hline
	UC5.3.3 & Visualizzazione lista completa dei corsi &  R0F3.5.2 \\ \hline
	UC5.3.4 & Visualizzazione lista corsi di laurea per anno accademico & R0F3.5.2  \\ \hline
	UC5.3.5 & Crea un esame nel corso &  R0F3.5.3\\ \hline
	UC5.3.6 & Visualizzazione errore campi non compilati nella creazione dell'esame & R0F3.5.3  \\ \hline
	UC5.3.7 & Visualizzazione lista esami per corso di laurea & R0F3.5.6 \\ \hline
	UC5.3.8 & Visualizzazione lista esami &  R0F3.5.6 \\ \hline
	UC5.3.9 & Aggiunta esame a corso &  R0F3.5.4 \\ \hline
	UC5.3.10 & Rimozione esame da un corso &  R0F3.5.5 \\ \hline
	UC5.3.11 & Associazione professore a esame & R0F3.5.6.2 \\ \hline
	UC5.3.12 & Rimozione corso di laurea da anno accademico & R2F34 \\ \hline
	UC5.3.13 & Visualizzazione dettagli esame & R0F3.5.6.1 \\ \hline
	%-----------------------------------------------------
	UC6 & Gestione aspetti relativi agli esami & R0F4 \\ \hline
	UC6.1 & Visualizzazione lista degli esami & R0F4.1 \\ \hline
	UC6.2 & Visualizzazione lista degli studenti & R0F4.2  \\ \hline
	UC6.3 & Registrazione valutazione di un esame & R0F4.3 \\ \hline
	UC7 & Gestione aspetti relativi allo studente & R0F5 \\ \hline
	UC7.1 & Visualizzazione lista degli esami & R0F5.1 \\ \hline
	UC7.1.1 & Visualizzazione dei crediti degli esami ai quali è iscritto & R0F5.1.1 \\ \hline
	UC7.1.2 & Visualizzazione della obbligatorietà degli esami ai quali è iscritto & R0F5.1.2 \\ \hline
	UC7.1.3 & Visualizzazione delle valutazioni degli esami ai quali è iscritto	& R0F5.1.3 \\ \hline
	UC7.2 & Visualizzazione degli esami opzionali e dei loro crediti & R0F5.2 \\ \hline
	UC7.3 & Registrazione ad un esame opzionale & R0F5.2.2 \\ \hline
	UC7.4 & Visualizzazione delle informazioni relative ai crediti & R0F5.1.4 \\ 
	& & R0F5.2.1 \\ \hline
	UC8 & Gestione degli amministratori & R0F11 \\ \hline
	UC8.1 & Aggiunta di un amministratore & R0F11.1 \\ \hline
	UC8.1.1 & Inserimento chiave pubblica & R0F11.1 \\ \hline
	UC8.1.2 & Visualizzazione messaggio di errore riguardo a chiave mal formata & R0F11.1 \\ \hline
	UC8.1.3 & Visualizzazione messaggio di errore riguardo a chiave già registrata & R0F11.1 \\ \hline
	UC8.2 & Visualizzazione lista degli amministratori & R0F11.3 \\ \hline
	UC8.3 & Rimozione amministratore & R0F11.2 \\ \hline
	UC9 & Visualizzazione quantità di Gas, Ether e costo delle operazioni & R0F10 \\ \hline

	\caption{Tabella di tracciamento fonti-requisiti}
\end{longtable}

\subsection{Tracciamento requisiti-fonti}

\label{table:Tabella di tracciamento requisiti-fonti}
\begin{longtable}[H]{|p{2cm}|p{5cm}|p{5cm}|}
	\hline
	\rowcolor[HTML]{38FFF8} 
	\textbf{Requisito} & \textbf{Descrizione requisito} & \textbf{Fonti} \\ \hline
	% REGEX for Notepad++
	% (.*)&(.*)&(.*)&(.*)      --->       \1& \3&  \\\\ \\hline
	\endhead
	R0F1 &  L'amministratore può gestire gli utenti & Interno \\ 
	& & UC5 \\  
	& & UC5.1 \\ \hline
	R0F1.1 &  L'amministratore può ottenere una lista di tutti gli utenti non abilitati & Interno \\ 
	& & UC5 \\  
	& & UC5.1.3 \\ \hline
	R0F1.2 &  L'amministratore può abilitare un utente & Interno \\ 
	& & UC5 \\ 
	& & UC5.1.4 \\   \hline
	R0F1.3 &  L'amministratore può rimuovere un utente & Interno \\ 
	& & UC5.1.5 \\ \hline
	R0F1.4 &  L'amministratore può ottenere una lista di tutti gli studenti & Interno \\  
	& & UC5.1.1 \\ \hline
	R0F2 &  L'amministratore può gestire i professori & Interno \\ \hline
	R0F2.1 &  L'amministratore può ottenere una lista di tutti i professori & Interno \\ 
	& & UC5.1.2 \\  \hline
	R0F2.2 &  L'amministratore può assegnare un esame ad un determinato professore & Capitolato \\ \hline
	R0F3 &  L'amministratore può gestire gli anni accademici & Capitolato \\ 
	& & UC5.2 \\ \hline
	R0F3.1 &  L'amministratore può aggiungere un anno accademico & Capitolato \\ 
	& & UC5.2.1 \\ 
	& & UC5.2.4 \\ 
	& & UC5.2.5 \\ 
	& & UC5.2.6 \\ \hline
	R0F3.2 &  L'amministratore può ottenere una lista di gli anni accademici & Interno \\ 
	& & UC5.2.2 \\ \hline
	R0F3.3 &  L'amministratore può aggiungere dei corsi di laurea ad ogni anno accademico & Capitolato \\ 
	& & UC5.2.3 \\ \hline
	R2F3.4 &  L'amministratore può rimuovere corsi di laurea da ogni anno accademico & Interno \\ \hline
	R0F3.5 &  L'amministratore può gestire i corsi di laurea relativi agli anni accademici & Capitolato \\ 
	& & UC5.3 \\  \hline
	R0F3.5.1 &  L'amministratore può creare un nuovo corso di laurea a partire da un anno accademico & Capitolato \\ 
	& & UC5.3.1 \\  
	& & UC5.3.2 \\ \hline
	R0F3.5.2 &  L'amministratore può ottenere una lista di tutti i corsi di laurea dato un anno accademico & Interno \\ 
	& & UC5.3.3 \\  
	& & UC5.3.4 \\ \hline
	R0F3.5.3 &  L'amministratore può creare nuovi esami & Capitolato \\ 
	& & UC5.3.5 \\  
	& & UC5.3.6 \\ \hline
	R0F3.5.4 &  L'amministratore può aggiungere degli esami ad ogni corso di laurea & Capitolato \\ 
	& & UC5.3.9 \\ \hline
	R0F3.5.5 &  L'amministratore può rimuovere degli esami da un corso di laurea & Interno \\ 
	& & UC5.3.10 \\ \hline
	R0F3.5.6 &  L'amministratore può ottenere una lista di tutti gli esami dato il corso di laurea & Interno \\ 
	& & UC5.3.7 \\ 
	& & UC5.3.8 \\ \hline
	R0F3.5.6.1 &  L'amministratore può ottenere l'informazione di quale professore è assegnato a un determinato esame & Interno \\ \hline
	R0F3.5.6.2  &  L'amministratore può associare ad un esame il relativo professore & Interno \\ 
	& & UC5.3.11 \\ \hline
	R0F4 &  Un professore può gestire gli esami a lui assegnati & Capitolato \\ 
	& & UC6 \\ \hline
	R0F4.1 &  Un professore può ottenere la lista di tutti gli esami a lui assegnati & Capitolato \\ 
	& & UC6.1 \\ \hline
	R0F4.2 &  Un professore può ottenere la lista di tutti gli studenti associati a un determinato esame & Capitolato  \\ 
	& & UC6.2 \\ \hline
	R0F4.3 &  Un professore può registrare un esito ad un dato esame ad un dato studente registrato a quell'esame & Capitolato \\ 
	& & UC6.3 \\ \hline
	R0F5 &  Uno studente può gestire l'iscrizione agli esami & Capitolato \\ 
	& & UC7 \\ \hline
	R0F5.1 &  Uno studente può vedere l'elenco degli esami ai quali è iscritto & Capitolato \\ 
	& & UC7.1 \\ \hline
	R0F5.1.1 &  Uno studente può ottenere il numero di crediti assegnati agli esame & Interno \\ 
	& & UC7.1.1 \\ \hline
	R0F5.1.2 &  Uno studente può ottenere l'obbligatorietà degli esame & Interno \\ 
	& & UC7.1.2 \\ \hline
	R0F5.1.3 &  Uno studente può ottenere lo stato di superamento degli esame & Interno \\ 
	& & UC7.1.3 \\ \hline
	R0F5.1.4 &  Uno studente può ottenere il totale dei suoi crediti & Interno \\ 
	& & UC7.4 \\ \hline
	R0F5.1.5 &  Uno studente può ottenere il numero di crediti da raggiungere per la laurea & Interno \\ \hline
	R0F5.2 &  Uno studente può ottenere l'elenco degli esami opzionali & Capitolato \\ 
	& & UC7.2 \\ \hline
	R0F5.2.1 &  Uno studente può ottenere il numero di crediti degli esami opzionali & Capitolato \\ 
	& & UC7.4 \\ \hline
	R0F5.2.2 &  Uno studente può iscriversi ad un esame opzionale & Capitolato \\ 
	& & UC7.3 \\ \hline
	R0F6 &  L'utente può effettuare il login & Interno \\ 
	& & UC2 \\ \hline
	R0F6.1 &  Il login deve avvenire tramite il controllo delle chiavi, senza ulteriori azioni da parte dell'utente & Interno \\ 
	& & UC2.1 \\ \hline
	R0F7 &  L'utente può effettuare il logout & Capitolato \\ 
	& & UC2.1  \\
	& & UC2.2 \\
	& & UC2.3 \\
	& & UC2.4 \\ 
	& & UC4 \\ \hline
	R0F8 &  L'utente non ancora registrato può registrarsi & Capitolato \\ 
	& & UC3 \\ \hline
	R0F8.1 &  La registrazione necessita di nome, cognome, categoria (studente o professore) e selezione del corso di laura se si tratta di uno studente & Capitolato \\
	& & UC3.1 \\
	& & UC3.2 \\
	& & UC3.3 \\
	& & UC3.4 \\
	& & UC3.5 \\
	& & UC3.6 \\
	& & UC3.7 \\
	& & UC3.8 \\ \hline
	R1F9 &  L'utente può leggere una breve guida sull'uso di MetaMask e sul pagamento delle operazioni & Interno \\ 
	& & UC1 \\ \hline
	R0F10 &  L'utente deve poter vedere preventivamente il costo in Gas, Ether e Euro dell'operazione che sta per eseguire & Capitolato \\
	& & UC9 \\ \hline
	R0F11 &  L'università deve poter gestire gli amministratori & VE171209 \\
	& & UC8 \\ \hline
	R0F11.1 &  L'università deve poter aggiungere amministratori & VE171209 \\
	& & UC8.1 \\
	& & UC8.1.1 \\ 
	& & UC8.1.2 \\
	& & UC8.1.3 \\ \hline
	R0F11.2 &  L'università deve poter rimuovere amministratori & VE171209 \\
	& & UC8.3 \\ \hline
	R0F11.3 &  L'università deve poter ottenere la lista di tutti gli amministratori & Interno \\
	& & UC8.2 \\ \hline
	R0Q1 &  La progettazione e il codice devono seguire le norme e le metriche riportate nei documenti allegati X e Y & Interno \\ \hline
	R0Q2 &  L'approccio di scrittura di JavaScript deve essere promise Centric Approach & Capitolato \\ \hline
	R0Q2.1 &  L'applicativo non deve fare uso di callback in presenza di alternative alle ultime & VE171122 \\ \hline
	R0Q3 &  Il codice Javascript deve attenersi al Airbnb Javascript style guide & Capitolato \\ \hline
	R0Q4 &  Lo sviluppo deve essere supportato dall'utilizzo del tool ESLint & Capitolato \\ \hline
	R0Q5 &  Dovrà essere fornito un manuale utente in lingua inglese che tratterà l'uso da parte di studenti e professori & VE171122 \\ \hline
	R0Q6 &  Dovrà essere fornito un manuale di deploy e di utilizzo da parte degli amministratori in lingua inglese & VE171122 \\ \hline
	R0Q7 &  Il codice sorgente deve essere pubblicato sulla piattaforma GitHub, BitBucket o GitLab & Capitolato \\ \hline
	R0Q8 &  Il codice deve attenersi il più possibile alle guide linea de "App a 12 Fattori" & Capitolato \\ \hline
	R0V1 &  L'applicativo dovrà essere sviluppato attraverso l'uso di tecnologie web & Capitolato \\ \hline
	R0V1.1 &  L'applicativo dovrà essere sviluppato con Node.js & Capitolato \\ \hline
	R0V1.2 &  L'applicativo dovrà essere sviluppato con JavaScript 8 (ES8) & Capitolato \\ \hline
	R0V1.3 &  L'applicativo dovrà essere sviluppato con il boilerplate Redux Minimal & Capitolato \\ \hline
	R0V1.4 &  L'applicativo dovrà essere sviluppato utilizzando React & Capitolato \\ \hline
	R0V1.5 &  L'applicativo dovrà essere sviluppato utilizzando Redux & Capitolato \\ \hline
	R0V1.6 &  Il deploy del sito andrà eseguito utilizzando Surge.sh & Capitolato \\ \hline
	R0V1.7 &  È desiderabile l'utilizzo di SCSS in sostituzione a CSS & Capitolato \\ \hline
	R0V2 &  Gli smart contract dovranno essere scritti in linguaggio Solidity & Capitolato \\ \hline
	R0V3 &  La connessione alla rete Ethereum deve avvernire tramite MetaMask & Capitolato \\ \hline
	R0V3.1 &  I test riguartandi gli smart contract dovranno essere eseguiti in una rete locale ed almeno in una rete pubblica & Capitolato \\ \hline
	R0V3.2 &  Il deploy degli smart contract dovrà avvenire su rete locale testrpc e rete di test Ropsten & Capitolato \\ \hline
	R2V3.3 &  È apprezzabile un deploy finale sulla rete principale di Ethereum & Capitolato \\ \hline
	R0V4 &  Lo sviluppo degli smart contract dovrà avvenire utilizzando il framework Truffle & Capitolato \\ \hline
	R0V5 &  L'applicativo deve essere accessibile ed utilizzabile dal browser Mozilla Firefox a partire dalla versione X & Interno \\ \hline
	R0V6 &  L'applicativo deve essere accessibile ed utilizzabile dal browser Google Chrome a partire dalla versione X & Interno \\ \hline
	R1V7 &  L'applicativo deve essere accessibile ed utilizzabile da un browser mobile, per le versioni supportate fare riferimento alle controparti PC di Firefox e Chrome & Capitolato \\ \hline
	R0V8 &  Un utente non deve poter compiere azioni sul sistema senza aver fatto l'accesso ad esso & Capitolato \\ \hline
	R0V9 &  L'applicazione dei principi de "App a 12 Fattori" deve essere documentata & Capitolato \\ \hline
	R0V10 &  Il codice sorgente deve essere pubblicato con licenza MIT & Capitolato \\ \hline
	\caption{Tabella di tracciamento requisiti-fonti}
\end{longtable}

\section{Riepilogo}

\label{table:Riepilogo del numero dei requisiti individuati}
\begin{longtable}[H]{|p{2.8cm}|p{2.9cm}|p{2.9cm}|p{2.9cm}|p{1.5cm}|}
	\hline
	\rowcolor[HTML]{38FFF8} 
	\textbf{Tipologia} & \textbf{0 Obbligatori} & \textbf{1 Desiderabili} & \textbf{2 Opzionali} & \textbf{Totale} \\ \hline
	Funzionali & 46 & 1 & 1 & 47 \\ \hline
	Di qualità & 0 & 0 & 0 & 9 \\ \hline
	Di vincolo & 18 & 1 & 1 & 20 \\ \hline
	Prestazionali & 0 & 0 & 0 & 0 \\ \hline
	
	\caption{Riepilogo del numero dei requisiti individuati}
\end{longtable}


\end{document}