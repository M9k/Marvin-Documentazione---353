\documentclass[AnalisiDeiRequisiti.tex]{subfiles}

\begin{document}

\chapter{Casi d'uso}
\section{Attori dei casi d'uso}
\subsection{Attori primari}
\begin{enumerate}
	\begin{figure}[h]
		\centering
		\includegraphics[width=0.8\linewidth]{attoriPrincipali.jpg}
		\caption{Gerarchia attori primari}
		\label{fig:Gerarchia attori primari}
	\end{figure}
	
	\item \textbf{Utente generico}\\
	Si riferisce ad un utente generico che accede al sito.\\
	
	\item \textbf{Utente non autenticato}\\
	Ci si riferisce ad un utente generico che non ha ancora effettuato il login.\\
	
	\item \textbf{Utente autenticato}\\
	 Ci si riferisce ad un utente generico con \citGloss{indirizzo} valido ed autenticato nel sistema tramite la procedura di login.\\
	
	\item \textbf{Amministratore}\\
	Ci si riferisce ad un utente autenticato nel sistema nel ruolo di amministratore, ha i permessi per gestire i professori.\\
	
	\item \textbf{Professore}\\
	Ci si riferisce ad un utente autenticato nel sistema nel ruolo di professore.\\
	
	\item \textbf{Studente}\\
	Ci si riferisce ad un utente autenticato nel sistema nel ruolo di studente.\\
		
	\item \textbf{Università}\\
	Ci si riferisce ad un utente autenticato nel sistema nel ruolo di fondatore e rappresentante dell'università, ha i permessi per gestire gli amministratori.\\	
\end{enumerate}

\subsection{Attori secondari}
\begin{enumerate}
	\item \textbf{\citGloss{MetaMask}}\\
	Plugin per \citGloss{browser} che permette di interfacciarsi ad una rete \citGloss{Ethereum}.\\
	
	\item \textbf{Ufficio Universitario}\\
	Entità fisica che consente l'immatricolazione degli studenti e la registrazione dei professori.\\
\end{enumerate}

\section{Elenco dei casi d'uso}

\begin{figure}[H]
	\centering
	\includegraphics[width=0.8\linewidth]{UC.jpg}
	\caption{Casi d'uso basilari}
	\label{fig:Casi d'uso basilari}
\end{figure}

%   -------------------------------------------------------------------------------------
%   ----------                 MODELLO PER GLI USER CASE              -------------------
%   -------------------------------------------------------------------------------------
\begin{comment}
\subsection{UCX - Nome}
\begin{itemize}
	\item \textbf{Attori primari:} ;\\
	\item \textbf{Attori secondari:} ;\\
	\item \textbf{Scopo e descrizione:} ;\\
	\item \textbf{Scenario principale:} ;\\
	\item \textbf{Scenario alternativo:} ;\\
	\item \textbf{Flusso principale degli eventi:};\\
	\begin{enumerate}
		\item L'utente... ;
		\item L'utente... ;
	\end{enumerate}
	\item \textbf{Estensioni:}\\
		\begin{enumerate}
		\item Se l'utente... ;[UCX.X.X]
	\end{enumerate}
	\item \textbf{Precondizione:} ;\\
	\item \textbf{Postcondizione:} .\\
\end{itemize}
\end{comment}

\subsection{UC1 - Breve guida}
\begin{itemize}
	\item \textbf{Attori primari:} utente generico;
	\item \textbf{Scopo e descrizione:} l'utente visualizza una breve guida di introduzione su come installare il plugin \citGloss{MetaMask} e su come gestire le chiavi in modo da istruirlo sulle modalità di accesso al sistema;
	\item \textbf{Scenario principale:} l'utente accede alla guida;
	\item \textbf{Precondizione:} il sistema è raggiungibile e funzionante e l'utente desidera aprire la guida;
	\item \textbf{Postcondizione:} l'utente ha avuto delle nozioni riguardanti l'accesso al sistema.
\end{itemize}
\subsection{UC2 - Login}
\begin{itemize}
	\item \textbf{Attori primari:} utente non autenticato;
	\item \textbf{Attori secondari:} \citGloss{MetaMask};
	\item \textbf{Scopo e descrizione:} l'utente richiede il login al sistema attraverso il plugin \citGloss{MetaMask};
	\item \textbf{Scenario principale:} l'utente non ancora riconosciuto dal sistema effettua il login;
	\item \textbf{Precondizione:} l'utente non è stato riconosciuto dal sistema;
	\item \textbf{Postcondizione:} l'utente viene riconosciuto da parte del sistema.\\
\end{itemize}

\begin{figure}[H]
	\centering
	\includegraphics[width=1.0\linewidth]{UC2.jpg}
	\caption{UC2 - Login}
	\label{fig:UC2 - Login}
\end{figure}

\subsection{UC2.1 - Login automatico}
\begin{itemize}
\item \textbf{Attori primari:} utente non autenticato;
\item \textbf{Attori secondari:} \citGloss{MetaMask};
\item \textbf{Scopo e descrizione:} l'utente attende il login da parte del sistema senza effettuare nessuna operazione aggiuntiva;
\item \textbf{Scenario principale:} l'utente non ancora riconosciuto dal sistema richiede il login;
\item \textbf{Estensioni:}
\begin{enumerate}
	\item Se l'utente non ha a disposizione una \citGloss{chiave} viene avvisato con un errore a riguardo [UC2.2];
	\item Se l'utente non ha il plugin \citGloss{MetaMask} installato nel \citGloss{browser} viene avvisato con un errore a riguardo [UC2.3];
	\item Se l'utente ha a disposizione una \citGloss{chiave} non registrata viene avvisato con un errore a riguardo [UC2.4];
	\item Se l'utente tenta di fare il login con un account in attesa di approvazione viene avvisato con una schermata informativa a riguardo [UC2.5].
\end{enumerate}
\item \textbf{Precondizione:} l'utente ha richiesto al sistema di venire riconosciuto;
\item \textbf{Postcondizione:} l'utente, tramite una interazione con \citGloss{MetaMask}, viene riconosciuto da parte del sistema.
\end{itemize}
\subsection{UC2.2 - Visualizzazione messaggio di errore riguardo a \citGloss{chiave} assente}
\begin{itemize}
	\item \textbf{Attori primari:} utente non autenticato;
	\item \textbf{Scopo e descrizione:} l'utente viene avvisato del fatto che non è stato possibile da parte del sistema il recupero della \citGloss{chiave};
	\item \textbf{Scenario principale:} l'utente visualizza l'errore relativo all'assenza di una \citGloss{chiave} per accedere al sistema;
	\item \textbf{Precondizione:} l'utente richiede il login senza possedere una \citGloss{chiave};
	\item \textbf{Postcondizione:} l'utente è consapevole di dover interagire con \citGloss{MetaMask} per poter fornire una \citGloss{chiave}.
\end{itemize}
\subsection{UC2.3 - Visualizzazione messaggio di errore riguardo a \citGloss{MetaMask} assente}
\begin{itemize}
	\item \textbf{Attori primari:} utente non autenticato;
	\item \textbf{Scopo e descrizione:} l'utente viene avvisato del fatto sta cercando di accedere al sistema senza il plugin \citGloss{MetaMask};
	\item \textbf{Scenario principale:} l'utente visualizza il messaggio d'errore relativo alla mancanza del plugin \citGloss{MetaMask};
	\item \textbf{Precondizione:} l'utente richiede il login senza avere a disposizione il plugin \citGloss{MetaMask};
	\item \textbf{Postcondizione:} l'utente è consapevole della necessità di possedere \citGloss{MetaMask} per poter accedere al sistema.
\end{itemize}
\subsection{UC2.4 - Visualizzazione messaggio di errore riguardo a chiave non registrata}
\begin{itemize}
	\item \textbf{Attori primari:} utente non autenticato;
	\item \textbf{Scopo e descrizione:} l'utente viene avvisato del fatto che ha fornito una \citGloss{chiave} non registrata nel sistema;
	\item \textbf{Scenario principale:} l'utente viene informato che la sua \citGloss{chiave} non risulta registrata, in quanto il relativo \citGloss{indirizzo} non è presente nel sistema, suggerendone la registrazione;
	\item \textbf{Precondizione:} l'utente richiede il login utilizzando una \citGloss{chiave} non registrata;
	\item \textbf{Postcondizione:} l'utente è consapevole di dover effettuare la registrazione per accedere al sistema.
\end{itemize}
\subsection{UC2.5 - Visualizzazione schermata informativa account non abilitato}
\begin{itemize}
	\item \textbf{Attori primari:} utente non autenticato;
	\item \textbf{Scopo e descrizione:} l'utente viene informato che il suo \citGloss{indirizzo} non è ancora stato abilitato all'uso dei servizi;
	\item \textbf{Scenario principale:} il sistema mostra una schermata informativa per avvisare l'utente che il suo account è in attesa di essere abilitato;
	\item \textbf{Precondizione:} l'utente tenta di accedere al sistema senza che il suo account sia stato abilitato da un amministratore;
	\item \textbf{Postcondizione:} l'utente è consapevole di dover attendere l'abilitazione del suo account da parte di un amministratore per accedere al sistema.
\end{itemize}
\subsection{UC3 - Registrazione}
\begin{itemize}
	\item \textbf{Attori primari:} utente non autenticato;
	\item \textbf{Attori secondari:} \citGloss{MetaMask}, Ufficio Universitario;
	\item \textbf{Scopo e descrizione:} l'utente non autenticato compila i campi richiesti per registrarsi nel sistema, successivamente si dovrà presentare all'ufficio universitario con i documenti per l'identificazione;
	\item \textbf{Scenario principale:}
	\begin{enumerate}
		\item L'utente inserisce il suo nome [UC3.1];
		\item L'utente inserisce il suo cognome [UC3.2];
		\item L'utente seleziona la sua categoria [UC3.3];
		\item L'utente seleziona il corso desiderato [UC3.4].
	\end{enumerate}
	\item \textbf{Precondizione:} l'utente è entrato nel sistema e ha scelto di registrarsi;
	\item \textbf{Postcondizione:} la registrazione è andata a buon fine e la \citGloss{chiave} dell'utente è stata registrata nel sistema, salvandone l'\citGloss{indirizzo}.
\end{itemize}

\begin{figure}[H]
	\centering
	\includegraphics[width=1.0\linewidth]{UC3.jpg}
	\caption{UC3 - Registrazione}
	\label{fig:UC3 - Registrazione}
\end{figure}


\subsection{UC3.1 - Inserimento nome}
\begin{itemize}
	\item \textbf{Attori Primari:} utente non autenticato;
	\item \textbf{Scopo e Descrizione:} il sistema \citGloss{verifica} la validità della \citGloss{chiave} ed offre l'interfaccia per inserire il nome dell'utente. Il nome unito al cognome non necessita sia univoco, essendo solamente una indicazione per facilitare l'identificazione, in quanto a definire univocamente l'account ci pensa la chiave pubblica, che andrà a sostituire la matricola di UniWeb;
	\item \textbf{Scenario principale:} l'utente fornisce tramite un dispositivo di input il suo nome;
	\item \textbf{Estensioni:}
		\begin{enumerate}
			\item Se l'utente non ha un \citGloss{chiave} viene avvisato tramite un messaggio d'errore [UC3.7];
			\item Se l'utente non possiede \citGloss{MetaMask} viene avvisato tramite messaggio d'errore [UC3.8];
			\item Se la \citGloss{chiave} fornita dall'utente è già presente nel sistema l'utente viene avvisato tramite un messaggio d'errore [UC3.6];
			\item Se l'utente non compila il campo viene avvisato con un messaggio d'errore [UC3.5].
		\end{enumerate}
	\item \textbf{Precondizione:} l'utente vuole inserire il proprio nome per poter proseguire con la procedura di registrazione;
	\item \textbf{Postcondizione:} l'utente ha inserito il proprio nome nel campo fornito dal sistema e la validità della sua \citGloss{chiave} è stata controllata.
\end{itemize}	
\subsection{UC3.2 - Inserimento cognome}
\begin{itemize}
	\item \textbf{Attori Primari:} utente non autenticato;
	\item \textbf{Scopo e Descrizione:} il sistema offre l'interfaccia per inserire il cognome dell'utente. Il cognome unito al nome non necessita sia univoco, essendo solamente una indicazione per facilitare l'identificazione, in quanto a definire univocamente l'account ci pensa la chiave pubblica, che andrà a sostituire la matricola di UniWeb;
	\item \textbf{Scenario principale:} l'utente fornisce tramite un dispositivo di input il suo cognome;
	\item \textbf{Estensioni:}
		\begin{enumerate}
				\item Se l'utente non compila il campo viene avvisato dal sistema [UC3.5].
		\end{enumerate}		
	\item \textbf{Precondizione:} l'utente vuole inserire il proprio cognome;
	\item \textbf{Postcondizione:} l'utente ha inserito il suo cognome nel campo fornito dal sistema.
\end{itemize}
\subsection{UC3.3 - Selezione categoria}
\begin{itemize}
	\item \textbf{Attori Primari:} utente non autenticato;
	\item \textbf{Scopo e Descrizione:} il sistema offre l'interfaccia per selezionare la categoria di registrazione (studente o professore);
	\item \textbf{Scenario principale:} l'utente seleziona tramite un dispositivo di input la sua categoria;
	\item \textbf{Precondizione:} l'utente vuole inserire la sua categoria di registrazione;
	\item \textbf{Postcondizione:} l'utente ha selezionato la sua categoria nel campo fornito dal sistema, se la categoria scelta è quella del professore l'utente termina la procedura di registrazione.
\end{itemize}
\subsection{UC3.4 - Selezione corso di laurea}
\begin{itemize}
	\item \textbf{Attori Primari:} utente non autenticato;
	\item \textbf{Scopo e Descrizione} il sistema offre l'interfaccia per selezionare il \citGloss{corso di laurea} desiderato da un elenco, specificando per ognuno la sigla di riferimento;
	\item \textbf{Scenario principale:} l'utente seleziona tramite un dispositivo di input il \citGloss{corso di laurea} scelto;
	\item \textbf{Precondizione:} l'utente ha selezionato la registrazione come studente, il sistema offre un campo per la selezione del \citGloss{corso di laurea} e rimane in attesa dell'input;
	\item \textbf{Postcondizione:} l'utente ha selezionato il \citGloss{corso di laurea} desiderato e termina la procedura di registrazione.
\end{itemize}

\subsection{UC3.5 - Visualizzazione errore campo non compilato}
\begin{itemize}
	\item \textbf{Attori Primari:} utente non autenticato;
	\item \textbf{Scopo e Descrizione} il sistema avvisa l'utente che non ha compilato il campo selezionato;
	\item \textbf{Scenario principale:} il sistema mostra un messaggio d'errore all'utente che indica la mancata compilazione del campo selezionato;
	\item \textbf{Precondizione:} l'utente non compila il campo selezionato;
	\item \textbf{Postcondizione:} l'utente è consapevole di non aver compilato il campo selezionato.
\end{itemize}
\subsection{UC3.6 - Visualizzazione errore utente già registrato}
\begin{itemize}
	\item \textbf{Attori Primari:} utente non autenticato;
	\item \textbf{Scopo e Descrizione} il sistema avvisa l'utente che la sua \citGloss{chiave} è già presente nel sistema, facendo collidere il suo \citGloss{indirizzo} con uno precedentemente inserito, e che quindi non può procedere alla registrazione;
	\item \textbf{Scenario principale:} il sistema mostra un messaggio d'errore all'utente che indica che sta tentando di effettuare una registrazione utilizzando una \citGloss{chiave} già utilizzata nel sistema;
	\item \textbf{Precondizione:} l'utente tenta di registrarsi utilizzando una \citGloss{chiave} precedentemente registrata;
	\item \textbf{Postcondizione:} l'utente è consapevole di star utilizzando una \citGloss{chiave} che corrisponde ad un account precedentemente registrato nel sistema.
\end{itemize}
\subsection{UC3.7 - Visualizzazione errore chiave non presente}
\begin{itemize}
	\item \textbf{Attori Primari:} utente non autenticato;
	\item \textbf{Scopo e Descrizione} il sistema avvisa l'utente che non ha fornito una \citGloss{chiave} e quindi non può procedere alla registrazione;
	\item \textbf{Scenario principale:} il sistema mostra un messaggio d'errore all'utente indicandogli l'assenza della \citGloss{chiave};
	\item \textbf{Precondizione:} l'utente tenta di registrarsi senza fornire una \citGloss{chiave};
	\item \textbf{Postcondizione:} l'utente è consapevole di non possedere una \citGloss{chiave}.
\end{itemize}
\subsection{UC3.8 - Visualizzazione errore \citGloss{MetaMask} non installato}
\begin{itemize}
	\item \textbf{Attori Primari:} utente non autenticato;
	\item \textbf{Scopo e Descrizione} il sistema avvisa l'utente del fatto che non possiede il plugin \citGloss{MetaMask};
	\item \textbf{Scenario principale:} il sistema mostra un messaggio d'errore all'utente indicandogli che non ha il plugin \citGloss{MetaMask} installato;
	\item \textbf{Precondizione:} l'utente inizia la procedura di registrazione senza possedere \citGloss{MetaMask};
	\item \textbf{Postcondizione:} l'utente è consapevole di dover installare il plugin \citGloss{MetaMask} per potersi registrare.
\end{itemize}
\subsection{UC4 - Logout}
\begin{itemize}
	\item \textbf{Attori primari:} utente Autenticato;
	\item \textbf{Scopo e descrizione:} l'utente desidera effettuare il logout dal sistema e gli vengono fornite le istruzioni per effettuarlo, in quanto per completare l'operazione l'utente deve interagire direttamente con il plugin \citGloss{MetaMask};
	\item \textbf{Scenario principale:} l'utente autenticato richiede il logout dal sistema;
	\item \textbf{Precondizione:} l'utente è autenticato e vuole effettuare il logout;
	\item \textbf{Postcondizione:} l'utente è informato riguardo alla modalità di logout dal sistema.
\end{itemize}

\subsection{UC5 - Gestione Utenti}
\begin{itemize}
	\item \textbf{Attori primari:} amministratore;
	\item \textbf{Scopo e descrizione:} l'amministratore è in grado di eseguire operazioni sugli studenti all'interno dell'ateneo;
	\item \textbf{Scenario principale:} l'amministratore visualizza una lista di operazioni eseguibili sugli studenti e sceglie quale fare;
	\item \textbf{Precondizione:} l'amministratore ha scelto di gestire gli studenti; 
	\item \textbf{Postcondizione:}l'amministratore è a conoscenza delle operazioni che può eseguire sugli studenti e decide di effettuarne una.
\end{itemize}

\begin{figure}[H]
	\centering
	\includegraphics[width=0.8\linewidth]{UC5.jpg}
	\caption{UC5 - Gestione Utenti}
	\label{fig:UC5 - Gestione Utenti}
\end{figure}

\subsection{UC5.1 - Visualizzazione lista degli studenti approvati}
\begin{itemize}
	\item \textbf{Attori primari:} amministratore;
	\item \textbf{Scopo e descrizione:} l'amministratore visualizza una lista di tutti gli studenti registrati nel sistema, visualizzando per ognuno di essi le informazioni basilari;
	\item \textbf{Scenario principale:} l'amministratore richiede al sistema una lista di studenti per poterli visualizzare ed eventualmente compiere altre operazioni su di essi;
	\item \textbf{Precondizione:} l'amministratore è nell'area di gestione utenti e richiede una lista di tutti gli studenti; 
	\item \textbf{Postcondizione:} l'amministratore ottiene la lista.
\end{itemize}
%TODO: immagine
\subsection{UC5.1.1 - Visualizzazione indirizzi degli studenti approvati}
\begin{itemize}
	\item \textbf{Attori primari:} amministratore;
	\item \textbf{Scopo e descrizione:} l'amministratore visualizza una lista di tutti gli \citGloss{indirizzi} della rete \citGloss{Ethereum} degli studenti registrati nel sistema;
	\item \textbf{Scenario principale:} l'amministratore richiede al sistema una lista di studenti per poterne visualizzare l'\citGloss{indirizzo} ed eventualmente compiere altre operazioni su di essi;
	\item \textbf{Precondizione:} l'amministratore è nell'area di gestione utenti e richiede una lista di tutti gli studenti per visualizzarne i relativi \citGloss{indirizzi}.
	\item \textbf{Postcondizione:} l'amministratore ottiene la lista con indicati gli \citGloss{indirizzi}.
\end{itemize}
\subsection{UC5.1.2 - Visualizzazione nomi degli studenti approvati}
\begin{itemize}
	\item \textbf{Attori primari:} amministratore;
	\item \textbf{Scopo e descrizione:} l'amministratore visualizza una lista di tutti i nomi degli studenti registrati nel sistema;
	\item \textbf{Scenario principale:} l'amministratore richiede al sistema una lista di studenti per poterne visualizzare il nome ed eventualmente compiere altre operazioni su di essi;
	\item \textbf{Precondizione:} l'amministratore è nell'area di gestione utenti e richiede una lista di tutti gli studenti per visualizzarne i relativi nomi.
	\item \textbf{Postcondizione:} l'amministratore ottiene la lista con indicati i nomi.
\end{itemize}
\subsection{UC5.1.3 - Visualizzazione cognomi degli studenti approvati}
\begin{itemize}
	\item \textbf{Attori primari:} amministratore;
	\item \textbf{Scopo e descrizione:} l'amministratore visualizza una lista di tutti i cognomi degli studenti registrati nel sistema;
	\item \textbf{Scenario principale:} l'amministratore richiede al sistema una lista di studenti per poterne visualizzare il cognome ed eventualmente compiere altre operazioni su di essi;
	\item \textbf{Precondizione:} l'amministratore è nell'area di gestione utenti e richiede una lista di tutti gli studenti per visualizzarne i relativi cognomi.
	\item \textbf{Postcondizione:} l'amministratore ottiene la lista con indicati i cognomi.
\end{itemize}
\subsection{UC5.1.4 - Visualizzazione corsi degli studenti approvati}
\begin{itemize}
	\item \textbf{Attori primari:} amministratore;
	\item \textbf{Scopo e descrizione:} l'amministratore visualizza una lista di tutti gli studenti registrati nel sistema con indicato il relativo corso universitario;
	\item \textbf{Scenario principale:} l'amministratore richiede al sistema una lista di studenti per poterne visualizzare il corso universitario ed eventualmente compiere altre operazioni su di essi;
	\item \textbf{Precondizione:} l'amministratore è nell'area di gestione utenti e richiede una lista di tutti gli studenti per visualizzarne il corso universitario.
	\item \textbf{Postcondizione:} l'amministratore ottiene la lista con indicati i corsi universitari.
\end{itemize}
\subsection{UC5.2 - Visualizzazione lista dei professori approvati}
\begin{itemize}
	\item \textbf{Attori primari:} amministratore;
	\item \textbf{Scopo e descrizione:} l'amministratore visualizza una lista di tutti i professori registrati nel sistema, visualizzando per ognuno di essi le informazioni basilari;
	\item \textbf{Scenario principale:} l'amministratore richiede al sistema una lista di professori per poterli visualizzare ed eventualmente compiere altre operazioni su di essi;
	\item \textbf{Precondizione:} l'amministratore è nell'area di gestione utenti e richiede una lista di tutti i professori; 
	\item \textbf{Postcondizione:} l'amministratore ottiene la lista.
\end{itemize}
%TODO: immagine
\subsection{UC5.2.1 - Visualizzazione indirizzi dei professori approvati}
\begin{itemize}
	\item \textbf{Attori primari:} amministratore;
	\item \textbf{Scopo e descrizione:} l'amministratore visualizza una lista di tutti gli \citGloss{indirizzi} della rete \citGloss{Ethereum} dei professori registrati nel sistema;
	\item \textbf{Scenario principale:} l'amministratore richiede al sistema una lista dei professori per poterne visualizzare l'\citGloss{indirizzo} ed eventualmente compiere altre operazioni su di essi;
	\item \textbf{Precondizione:} l'amministratore è nell'area di gestione utenti e richiede una lista di tutti i professori per visualizzarne i relativi \citGloss{indirizzi}.
	\item \textbf{Postcondizione:} l'amministratore ottiene la lista con indicati gli \citGloss{indirizzi}.
\end{itemize}
\subsection{UC5.2.2 - Visualizzazione nomi dei professori approvati}
\begin{itemize}
	\item \textbf{Attori primari:} amministratore;
	\item \textbf{Scopo e descrizione:} l'amministratore visualizza una lista di tutti i nomi dei professori registrati nel sistema;
	\item \textbf{Scenario principale:} l'amministratore richiede al sistema una lista dei professori per poterne visualizzare il nome ed eventualmente compiere altre operazioni su di essi;
	\item \textbf{Precondizione:} l'amministratore è nell'area di gestione utenti e richiede una lista di tutti i professori per visualizzarne i relativi nomi.
	\item \textbf{Postcondizione:} l'amministratore ottiene la lista con indicati i nomi.
\end{itemize}
\subsection{UC5.2.3 - Visualizzazione cognomi dei professori approvati}
\begin{itemize}
	\item \textbf{Attori primari:} amministratore;
	\item \textbf{Scopo e descrizione:} l'amministratore visualizza una lista di tutti i cognomi dei professori registrati nel sistema;
	\item \textbf{Scenario principale:} l'amministratore richiede al sistema una lista dei professori per poterne visualizzare il cognome ed eventualmente compiere altre operazioni su di essi;
	\item \textbf{Precondizione:} l'amministratore è nell'area di gestione utenti e richiede una lista di tutti i professori per visualizzarne i relativi cognomi.
	\item \textbf{Postcondizione:} l'amministratore ottiene la lista con indicati i cognomi.
\end{itemize}
\subsection{UC5.3 - Visualizzazione lista utenti in attesa di abilitazione}
\begin{itemize}
	\item \textbf{Attori primari:} amministratore;
	\item \textbf{Scopo e descrizione:} l'amministratore visualizza una lista di tutti gli utenti ancora non abilitati, ottenendo per ognuno tutte le informazioni utili;
	\item \textbf{Scenario principale:} l'amministratore richiede al sistema una lista di utenti in attesa di abilitazione per poterli visualizzare ed eventualmente compiere altre operazioni su di essi;
	\item \textbf{Precondizione:} l'amministratore è nell'area di gestione studenti e richiede una lista di studenti in attesa di abilitazione; 
	\item \textbf{Postcondizione:} l'amministratore ottiene la lista.
\end{itemize}
%TODO: immagine
\subsection{UC5.3.1 - Visualizzazione indirizzi degli utenti in attesa di abilitazione}
\begin{itemize}
	\item \textbf{Attori primari:} amministratore;
	\item \textbf{Scopo e descrizione:} l'amministratore visualizza una lista di tutti gli \citGloss{indirizzi} della rete \citGloss{Ethereum} degli utenti in attesa di abilitazione;
	\item \textbf{Scenario principale:} l'amministratore richiede al sistema una lista degli utenti in attesa di abilitazione per poterne visualizzare l'\citGloss{indirizzo} ed eventualmente compiere altre operazioni su di essi;
	\item \textbf{Precondizione:} l'amministratore è nell'area di gestione utenti e richiede una lista di tutti gli utenti in attesa di abilitazione per visualizzarne i relativi \citGloss{indirizzi}.
	\item \textbf{Postcondizione:} l'amministratore ottiene la lista con indicati gli \citGloss{indirizzi}.
\end{itemize}
\subsection{UC5.3.2 - Visualizzazione nomi degli utenti in attesa di abilitazione}
\begin{itemize}
	\item \textbf{Attori primari:} amministratore;
	\item \textbf{Scopo e descrizione:} l'amministratore visualizza una lista di tutti i nomi degli utenti in attesa di abilitazione;
	\item \textbf{Scenario principale:} l'amministratore richiede al sistema una lista degli utenti in attesa di abilitazione per poterne visualizzare il nome ed eventualmente compiere altre operazioni su di essi;
	\item \textbf{Precondizione:} l'amministratore è nell'area di gestione utenti e richiede una lista di tutti gli utenti in attesa di abilitazione per visualizzarne i relativi nomi.
	\item \textbf{Postcondizione:} l'amministratore ottiene la lista con indicati i nomi.
\end{itemize}
\subsection{UC5.3.3 - Visualizzazione cognomi degli utenti in attesa di abilitazione}
\begin{itemize}
	\item \textbf{Attori primari:} amministratore;
	\item \textbf{Scopo e descrizione:} l'amministratore visualizza una lista di tutti i cognomi degli utenti in attesa di abilitazione;
	\item \textbf{Scenario principale:} l'amministratore richiede al sistema una lista degli utenti in attesa di abilitazione per poterne visualizzare il cognome ed eventualmente compiere altre operazioni su di essi;
	\item \textbf{Precondizione:} l'amministratore è nell'area di gestione utenti e richiede una lista di tutti gli utenti in attesa di abilitazione per visualizzarne i relativi cognomi.
	\item \textbf{Postcondizione:} l'amministratore ottiene la lista con indicati i cognomi.
\end{itemize}
\subsection{UC5.3.4 - Visualizzazione ruolo degli utenti in attesa di abilitazione}
\begin{itemize}
	\item \textbf{Attori primari:} amministratore;
	\item \textbf{Scopo e descrizione:} l'amministratore visualizza una lista degli utenti in attesa di abilitazione ed i relativi ruoli che desiderano ricoprire una volta approvati, cioè professori o studenti.
	\item \textbf{Scenario principale:} l'amministratore richiede al sistema una lista degli utenti in attesa di abilitazione per poterne visualizzare il ruolo desiderato ed eventualmente compiere altre operazioni su di essi;
	\item \textbf{Precondizione:} l'amministratore è nell'area di gestione utenti e richiede una lista di tutti gli utenti in attesa di abilitazione per visualizzarne i relativi ruoli.
	\item \textbf{Postcondizione:} l'amministratore ottiene la lista con indicati i ruoli.
\end{itemize}
\subsection{UC5.3.5 - Visualizzazione corsi degli utenti in attesa di abilitazione}
\begin{itemize}
	\item \textbf{Attori primari:} amministratore;
	\item \textbf{Scopo e descrizione:} l'amministratore visualizza una lista di tutti gli utenti in attesa di abilitazione con indicato il relativo corso universitario nel caso l'utente abbia richiesto di essere uno studente;
	\item \textbf{Scenario principale:} l'amministratore richiede al sistema una lista degli utenti in attesa di abilitazione per poterne visualizzare il corso universitario ove possibile ed eventualmente compiere altre operazioni su di essi;
	\item \textbf{Precondizione:} l'amministratore è nell'area di gestione utenti e richiede una lista di tutti gli utenti in attesa di abilitazione per visualizzarne il corso universitario ove possibile.
	\item \textbf{Postcondizione:} l'amministratore ottiene la lista con indicati i corsi universitari ove possibile.
\end{itemize}
\subsection{UC5.4 - Abilitazione utente}
\begin{itemize}
	\item \textbf{Attori primari:} amministratore;
	\item \textbf{Scopo e descrizione:} l'amministratore abilita la registrazione di un utente;
	\item \textbf{Scenario principale:}
	\begin{enumerate}
		\item L'amministratore visualizza la lista degli utenti in attesa di abilitazione [UC7];
		\item L'amministratore lo abilita [UC5.4].
	\end{enumerate}
	\item \textbf{Precondizione:} l'amministratore ha richiesto una lista di utenti non abilitati ed ha individuato l'account che desidera abilitare; 
	\item \textbf{Postcondizione:} l'utente indicato viene abilitato e il suo stato viene aggiornato nel sistema.
\end{itemize}
\subsection{UC5.5 - Rimozione utente}
\begin{itemize}
	\item \textbf{Attori primari:} amministratore;
	\item \textbf{Scopo e descrizione:} l'amministratore rimuove dal sistema un utente non ancora abilitato o già abilitato;
	\item \textbf{Scenario principale:}
	\begin{enumerate}
		\item L'amministratore visualizza gli utenti abilitati o in attesa di abilitazione e ne seleziona uno [UC5.1 o UC6 o UC7];
		\item L'amministratore rimuove l'utente selezionato [UC5.5].
	\end{enumerate}
	\item \textbf{Precondizione:} l'amministratore ha richiesto una lista di studenti o professori ed ha individuato l'account che desidera rimuovere;
	\item \textbf{Postcondizione:} l'utente indicato viene rimosso dal sistema.
\end{itemize}
\subsection{UC6 - Gestione anni accademici}
\begin{itemize}
	\item \textbf{Attori primari:} università;
	\item \textbf{Scopo e descrizione:} l'utente è già riconosciuto dal sistema come rappresentante dell'università ed è in grado di eseguire operazioni di amministrazione riguardanti gli anni accademici dell'ateneo;
	\item \textbf{Scenario principale:} l'utente che rappresenta l'università visualizza una lista di operazioni eseguibili sugli anni accademici e ne sceglie una;
	\item \textbf{Precondizione:} l'utente che rappresenta l'università ha scelto di effettuare una operazione riguardante gli anni accademici; 
	\item \textbf{Postcondizione:} l'utente che rappresenta l'università è a conoscenza delle possibili operazioni riguardanti gli anni accademici e può eseguire quella di suo interesse.
\end{itemize}

\begin{figure}[H]
	\centering
	\includegraphics[width=1.1\linewidth]{UC6.jpg}
	\caption{UC6 - Gestione Anni Accademici}
	\label{fig:UC6 - Gestione Anni Accademici}
\end{figure}

\subsection{UC6.1 - Aggiunta anno accademico}
\begin{itemize}
	\item \textbf{Attori primari:} università;
	\item \textbf{Scopo e descrizione:} l'utente è già riconosciuto dal sistema come rappresentante dell'università ed aggiunge un \citGloss{anno accademico} nel sistema fornendo l'anno solare di riferimento;
	\item \textbf{Scenario principale:} l'utente che rappresenta l'università inserisce il numero dell'anno solare da aggiungere in un campo fornito dal sistema;
	\item \textbf{Estensioni:}
	\begin{enumerate}
		\item Se l'utente che rappresenta l'università inserisce un anno non numerico viene mostrato un messaggio d'errore relativo alla invalidità del dato fornito [UC6.4];
		\item Se l'utente che rappresenta l'università inserisce un'anno già presente nel sistema viene mostrato un messaggio d'errore relativo alla non univocità dell'anno [UC6.5];
		\item Se l'utente che rappresenta l'università lascia il campo vuoto l'utente viene avvisato con un relativo messaggio d'errore [UC6.6].
	\end{enumerate}
	\item \textbf{Precondizione:} l'utente che rappresenta l'università ha scelto di aggiungere un \citGloss{anno accademico}; 
	\item \textbf{Postcondizione:} l'\citGloss{anno accademico} creato viene inserito nel sistema.
\end{itemize}
%TODO: immagine 6.2
\subsection{UC6.2 - Visualizzazione lista di tutti gli anni accademici}
\begin{itemize}
	\item \textbf{Attori primari:} università;
	\item \textbf{Scopo e descrizione:} l'utente è già riconosciuto dal sistema come rappresentante dell'università e visualizza una lista di tutti gli anni accademici presenti nel sistema, visualizzandone l'anno solare di riferimento;
	\item \textbf{Scenario principale:} l'utente che rappresenta l'università richiede la lista di tutti gli anni del sistema;
	\item \textbf{Precondizione:} l'utente che rappresenta l'università è nell'area di gestione Anni Accademici e vuole ottenere la lista di tutti gli anni presenti nel sistema; 
	\item \textbf{Postcondizione:} l'utente che rappresenta l'università ottiene la lista.
\end{itemize}
\subsection{UC6.2.1 - Visualizzazione anni solari relativi agli anni accademici}
\begin{itemize}
	\item \textbf{Attori primari:} università;
	\item \textbf{Scopo e descrizione:} l'utente già riconosciuto dal sistema come rappresentante dell'università visualizza una lista di tutti gli anni solari relativi agli anni accademici inseriti nella rete \citGloss{Ethereum};
	\item \textbf{Scenario principale:} L'utente che rappresenta l'università richiede al sistema la lista di anni accademici inseriti nel sistema per visualizzarli ed eventualmente compiere operazioni su di esse;
	\item \textbf{Precondizione:} l'utente che rappresenta l'università è nell'area di gestione Anni Accademici richiede una lista degli ultimi per visualizzarne gli anni solari ad essi associati; 
	\item \textbf{Postcondizione:} l'utente che rappresenta l'università ottiene la lista con indicati gli anni solari.
\end{itemize}
\subsection{UC6.3 - Visualizzazione messaggio anno malformato}
\begin{itemize}
	\item \textbf{Attori primari:} università;
	\item \textbf{Scopo e descrizione:} l'utente che rappresenta l'università visualizza un messaggio che lo informa che l'anno inserito non è numerico;
	\item \textbf{Scenario principale:} il sistema mostra all'utente che rappresenta l'università un messaggio d'errore relativo alla malformazione dell'anno inserito;
	\item \textbf{Precondizione:} l'utente che rappresenta l'università prova a creare un nuovo \citGloss{anno accademico} inserendo un anno non numerico; 
	\item \textbf{Postcondizione:} l'utente che rappresenta l'università è consapevole di aver provato ad inserire un anno non numerico.
\end{itemize}
\subsection{UC6.4 - Visualizzazione messaggio di anno accademico già presente}
\begin{itemize}
	\item \textbf{Attori primari:} università;
	\item \textbf{Scopo e descrizione:} l'utente che rappresenta l'università visualizza un messaggio che indica che un \citGloss{anno accademico} come quello inserito è già presente nel sistema;
	\item \textbf{Scenario principale:} il sistema mostra all'utente che rappresenta l'università un messaggio d'errore relativo alla non univocità dell'\citGloss{anno accademico} inserito;
	\item \textbf{Precondizione:} l'utente che rappresenta l'università tenta di inserire un \citGloss{anno accademico} già presente nel sistema; 
	\item \textbf{Postcondizione:} l'utente che rappresenta l'università è consapevole di aver provato ad inserire un \citGloss{anno accademico} già presente nel sistema.
\end{itemize}
\subsection{UC6.5 - Visualizzazione messaggio anno non compilato}
\begin{itemize}
	\item \textbf{Attori primari:} università;
	\item \textbf{Scopo e descrizione:} l'utente che rappresenta l'università visualizza un messaggio che lo avvisa di aver lasciato vuoto il campo dell'\citGloss{anno accademico};
	\item \textbf{Scenario principale:} il sistema mostra un messaggio d'errore relativo alla non compilazione dell'\citGloss{anno accademico};
	\item \textbf{Precondizione:} l'utente che rappresenta l'università tenta di inserire un \citGloss{anno accademico} vuoto; 
	\item \textbf{Postcondizione:} l'utente che rappresenta l'università è consapevole di aver provato ad inserire un \citGloss{anno accademico} vuoto.
\end{itemize}
\subsection{UC6.6 - Eliminazione anno accademico vuoto}
\begin{itemize}
\item \textbf{Attori primari:} università;
\item \textbf{Scopo e descrizione:} l'utente che rappresenta l'università desidera la rimozione di un anno accademico che non presenta ancora dei corsi di laurea al suo interno, ad esempio per rimediare ad un inserimento con informazioni errate;
\item \textbf{Scenario principale:} l'utente che rappresenta l'università desidera rimuovere un anno accademico che non presenta corsi collegati dal sistema;
\item \textbf{Precondizione:} l'utente che rappresenta l'università individua nella lista degli \citGloss{anni accademici} [UC6.2] un anno che non presenta corsi associati che desidera rimuovere; 
\item \textbf{Postcondizione:} L'\citGloss{anno accademico} indicato è stato rimosso dal sistema.
\end{itemize}
\subsection{UC7 - Gestione corsi di laurea}
\begin{itemize}
	\item \textbf{Attori primari:} amministratore;
	\item \textbf{Scopo e descrizione:} l'amministratore è in grado di eseguire operazioni sui corsi di laurea. La gestione non è stata assegnata al gestore dell'università in quanto molto onerosa in termini di tempo e quindi difficilmente adempibile da una singola persona;
	\item \textbf{Scenario principale:} l'amministratore visualizza una lista di operazioni eseguibili sui corsi di laurea e ne sceglie una;
	\item \textbf{Precondizione:} l'utente ha scelto di gestire i corsi di laurea; 
	\item \textbf{Postcondizione:} l'utente è a conoscenza delle possibili operazioni da eseguire sui corsi di laurea e ne ha scelta una.
\end{itemize}

\begin{figure}[H]
	\centering
	\includegraphics[width=1.1\linewidth]{UC7.jpg}
	\caption{UC7 - Gestione Corsi di laurea}
	\label{fig:UC7 - Gestione Corsi di laurea}
\end{figure}

\subsection{UC7.1 - Creazione corso di laurea}
\begin{itemize}
	\item \textbf{Attori primari:} amministratore;
	\item \textbf{Scopo e descrizione:} l'amministratore crea un nuovo corso di laurea associato ad un \citGloss{anno accademico}, indicando la sigla da utilizzare per riferirsi al corso;
	\item \textbf{Estensioni:}
		\begin{enumerate}
			\item Se l'amministratore inserisce una sigla invalida viene avvisato con un messaggio [UC7.2].
		\end{enumerate}
	\item \textbf{Precondizione:} l'amministratore, dopo aver ottenuto una lista di \citGloss{anni accademici} [UC7.11] ed indicatone uno desidera inserire un nuovo corso di laurea; 
	\item \textbf{Postcondizione:} il \citGloss{corso di laurea} è stato creato ed inserito nel sistema.
\end{itemize}
\subsection{UC7.2 - Visualizzazione errore sigla invalida}
\begin{itemize}
	\item \textbf{Attori primari:} amministratore;
	\item \textbf{Scopo e descrizione:} viene mostrato un messaggio d'errore per avvisare l'amministratore che non è possibile utilizzare la sigla inserita, in quanto non rispetta il pattern indicato dal sistema oppure risulta già in uso;
	\item \textbf{Scenario principale:} il sistema avvisa l'amministratore che non è possibile utilizzare la sigla indicata;
	\item \textbf{Precondizione:} l'amministratore ha tentato di creare un \citGloss{corso di laurea} utilizzando una sigla che non rispetta i \citGloss{requisiti} indicati dal sistema; 
	\item \textbf{Postcondizione:} l'amministratore è consapevole di non poter utilizzare quella determinata sigla per la creazione di un nuovo corso.
\end{itemize}
%TODO: immagine 7.3
\subsection{UC7.3 - Visualizzazione lista completa dei corsi}
\begin{itemize}
	\item \textbf{Attori primari:} amministratore;
	\item \textbf{Scopo e descrizione:} vengono mostrati tutti i corsi di laurea presenti nel sistema, visualizzandone la sigla e l'anno solare dell'anno accademico associato;
	\item \textbf{Scenario principale:} l'amministratore visualizza tutti i corsi di laurea presenti nel sistema;
	\item \textbf{Precondizione:} l'amministratore ha richiesto la visualizzazione di tutti i corsi di laurea; 
	\item \textbf{Postcondizione:} l'amministratore ottiene la lista.
\end{itemize}	
\subsection{UC7.3.1 - Visualizzazione sigle dei corsi di laurea}
\begin{itemize}
	\item \textbf{Attori primari:} amministratore;
	\item \textbf{Scopo e descrizione:} l'amministratore visualizza la lista delle sigle relative ai corsi di laurea presenti nel sistema;
	\item \textbf{Scenario principale:} l'amministratore richiede una lista di corsi di laurea presenti nel sistema;
	\item \textbf{Precondizione:} l'amministratore ha richiesto la lista dei corsi di laurea presenti nel sistema e vuole visualizzarne le sigle ad essi associate; 
	\item \textbf{Postcondizione:} l'amministratore ottiene la lista con le sigle associate.
\end{itemize}
\subsection{UC7.3.2 - Visualizzazione anni accademici associati ai corsi di laurea}
\begin{itemize}
	\item \textbf{Attori primari:} amministratore;
	\item \textbf{Scopo e descrizione:} l'amministratore visualizza la lista degli anni accademici relativi ai corsi di laurea presenti nel sistema;
	\item \textbf{Scenario principale:} l'amministratore richiede una lista di corsi di laurea presenti nel sistema;
	\item \textbf{Precondizione:} l'amministratore ha richiesto la lista dei corsi di laurea presenti nel sistema e vuole visualizzarne gli anni accademici ad essi associati; 
	\item \textbf{Postcondizione:} l'amministratore ottiene la lista con gli anni accademici associati.
\end{itemize}
%TODO: immagine 7.4
\subsection{UC7.4 - Visualizzazione lista corsi di laurea per anno accademico}
\begin{itemize}
	\item \textbf{Attori primari:} amministratore;
	\item \textbf{Scopo e descrizione:} vengono visualizzati i corsi di laurea relativi ad un determinato \citGloss{anno accademico}, ottenendone le sigle;
	\item \textbf{Precondizione:} l'amministratore ha visualizzato una lista di anni accademici [UC7.11] e vuole visualizzare i corsi di laurea all'interno di uno di essi; 
	\item \textbf{Postcondizione:} l'amministratore ottiene la lista desiderata.
\end{itemize}
\subsection{UC7.4.1 - Visualizzazione sigle dei corsi di laurea per anno accademico}
\begin{itemize}
	\item \textbf{Attori primari:} amministratore;
	\item \textbf{Scopo e descrizione:} l'amministratore visualizza la lista delle sigle relative ai corsi di laurea presenti nel sistema associati ad un determinato anno accademico;
	\item \textbf{Scenario principale:} l'amministratore richiede una lista di corsi di laurea associati ad un determinato anno accademico presenti nel sistema;
	\item \textbf{Precondizione:} l'amministratore ha richiesto la lista dei corsi di laurea presenti nel sistema associati ad un anno accademico e vuole visualizzarne le sigle ad essi associate; 
	\item \textbf{Postcondizione:} l'amministratore ottiene la lista con le sigle associate.
\end{itemize}
\subsection{UC7.5 - Creazione esame in un corso}
\begin{itemize}
	\item \textbf{Attori primari:} amministratore;
	\item \textbf{Scopo e descrizione:} l'amministratore crea ed assegna un esame all'interno di un \citGloss{corso di laurea} indicandone il nome, la quantità di crediti e l'obbligatorietà;
	\item \textbf{Scenario principale:} l'amministratore inserisce un nuovo esame all'interno di un corso;
	\item \textbf{Estensioni:}
	\begin{enumerate}
		\item Se l'amministratore inserisce un nome invalido ottiene un messaggio d'errore [UC7.6].
	\end{enumerate}
	\item \textbf{Precondizione:} l'amministratore ha ottenuto una lista di corsi di laurea per \citGloss{anno accademico} [UC7.3] e ne ha selezionato uno, volendo creare un esame al suo interno; 
	\item \textbf{Postcondizione:} l'esame viene inserito nel sistema e associato al \citGloss{corso di laurea} selezionato.
\end{itemize}
\subsection{UC7.6 - Visualizzazione errore nome esame non valido}
\begin{itemize}
	\item \textbf{Attori primari:} amministratore;
	\item \textbf{Scopo e descrizione:} viene mostrato un messaggio d'errore relativo all'invalidità del nome del corso, in quanto non rispetta un pattern indicato dal sistema oppure già utilizzato;
	\item \textbf{Scenario principale:} l'amministratore visualizza il messaggio d'errore;
	\item \textbf{Precondizione:} l'amministratore tenta di creare un esame utilizzando un nome che non rispetta i \citGloss{requisiti} indicati dal sistema; 
	\item \textbf{Postcondizione:} l'amministratore è a conoscenza di aver inserito un nome invalido.
\end{itemize}
%TODO: immagine 7.7
\subsection{UC7.7 - Visualizzazione lista esami per corso}
\begin{itemize}
	\item \textbf{Attori primari:} amministratore;
	\item \textbf{Scopo e descrizione:} vengono visualizzati gli esami relativi ad un determinato \citGloss{corso di laurea}, con indicato per ognuno di essi il nome, il numero di crediti e, se presente, il nome ed il cognome del professore assegnatogli;
	\item \textbf{Scenario principale:}
	\begin{enumerate}
		\item L'amministratore ottiene una lista dei corsi di laurea e seleziona quello di suo interesse [UC7.3 o UC7.4];
		\item Il sistema mostra la lista di esami associati al corso selezionato.
	\end{enumerate}
	\item \textbf{Precondizione:} l'amministratore ha visualizzato una lista di corsi di laurea [UC7.3 o UC7.4] e vuole visualizzare gli esami all'interno di uno di essi; 
	\item \textbf{Postcondizione:} l'amministratore ottiene la lista desiderata.
\end{itemize}
\subsection{UC7.7.1 - Visualizzazione sigle esami per corso di laurea}
\begin{itemize}
	\item \textbf{Attori primari:} amministratore;
	\item \textbf{Scopo e descrizione:} l'amministratore visualizza la lista dei nomi degli esami presenti nel sistema per visualizzarli ed eventualmente effettuare operazioni su di essi;
	\item \textbf{Scenario principale:} il sistema mostra la lista di nomi degli esami associati ad un determinato corso di laurea presenti nel sistema;
	\item \textbf{Precondizione:} l'amministratore ha richiesto una lista di esami associati ad un corso di laurea; 
	\item \textbf{Postcondizione:} l'amministratore visualizza la lista di nomi degli esami associati ad un determinato corso di laurea.
\end{itemize}
\subsection{UC7.7.2 - Visualizzazione crediti esami per corso di laurea}
\begin{itemize}
	\item \textbf{Attori primari:} amministratore;
	\item \textbf{Scopo e descrizione:} l'amministratore visualizza la lista dei crediti degli esami presenti nel sistema per visualizzarli ed eventualmente effettuare operazioni su di essi;
	\item \textbf{Scenario principale:} il sistema mostra la lista di crediti degli esami associati ad un determinato corso di laurea presenti nel sistema;
	\item \textbf{Precondizione:} l'amministratore ha richiesto una lista di esami associati ad un corso di laurea; 
	\item \textbf{Postcondizione:} l'amministratore visualizza la lista di crediti degli esami associati ad un determinato corso di laurea.
\end{itemize}
\subsection{UC7.7.3 - Visualizzazione nomi dei professori associati agli esami per corso di laurea se presenti}
\begin{itemize}
	\item \textbf{Attori primari:} amministratore;
	\item \textbf{Scopo e descrizione:} l'amministratore visualizza la lista dei nomi dei professori degli esami presenti nel sistema per visualizzarli ed eventualmente effettuare operazioni su di essi;
	\item \textbf{Scenario principale:} il sistema mostra la lista di nomi dei professori degli esami associati ad un determinato corso di laurea presenti nel sistema;
	\item \textbf{Precondizione:} l'amministratore ha richiesto una lista di esami associati ad un corso di laurea e l'esame di riferimento ha un professore associato; 
	\item \textbf{Postcondizione:} l'amministratore visualizza la lista di nomi dei professori degli esami associati ad un determinato corso di laurea.
\end{itemize}
\subsection{UC7.7.4 - Visualizzazione cognomi dei professori associati agli esami per corso di laurea se presenti}
\begin{itemize}
	\item \textbf{Attori primari:} amministratore;
	\item \textbf{Scopo e descrizione:} l'amministratore visualizza la lista dei cognomi dei professori degli esami presenti nel sistema per visualizzarli ed eventualmente effettuare operazioni su di essi;
	\item \textbf{Scenario principale:} il sistema mostra la lista di cognomi dei professori degli esami associati ad un determinato corso di laurea presenti nel sistema;
	\item \textbf{Precondizione:} l'amministratore ha richiesto una lista di esami associati ad un corso di laurea e l'esame di riferimento ha un professore associato; 
	\item \textbf{Postcondizione:} l'amministratore visualizza la lista di cognomi dei professori degli esami associati ad un determinato corso di laurea.
\end{itemize}
%TODO: immagine 7.8
\subsection{UC7.8 - Visualizzazione lista esami}
\begin{itemize}
	\item \textbf{Attori primari:} amministratore;
	\item \textbf{Scopo e descrizione:} vengono visualizzati gli esami all'interno del sistema, con indicato per ognuno di essi il nome, il numero di crediti, la sigla del corso ad esso associato e, se presente, il nome ed il cognome del professore assegnatogli;
	\item \textbf{Scenario principale:} l'amministratore visualizza la lista di tutti gli esami presenti nel sistema;
	\item \textbf{Precondizione:} l'amministratore vuole vedere tutti gli esami presenti nel sistema; 
	\item \textbf{Postcondizione:} l'amministratore ottiene la lista desiderata.
\end{itemize}
\subsection{UC7.8.1 - Visualizzazione sigla esami}
\begin{itemize}
	\item \textbf{Attori primari:} amministratore;
	\item \textbf{Scopo e descrizione:} l'amministratore visualizza la lista dei nomi degli esami presenti nel sistema per visualizzarli ed eventualmente effettuare operazioni su di essi;
	\item \textbf{Scenario principale:} il sistema mostra la lista di nomi degli esami presenti nel sistema;
	\item \textbf{Precondizione:} l'amministratore ha richiesto una lista di esami; 
	\item \textbf{Postcondizione:} l'amministratore visualizza la lista di nomi degli esami.
\end{itemize}
\subsection{UC7.8.2 - Visualizzazione crediti esami}
\begin{itemize}
	\item \textbf{Attori primari:} amministratore;
	\item \textbf{Scopo e descrizione:} l'amministratore visualizza la lista dei crediti degli esami presenti nel sistema per visualizzarli ed eventualmente effettuare operazioni su di essi;
	\item \textbf{Scenario principale:} il sistema mostra la lista di crediti degli esami presenti nel sistema;
	\item \textbf{Precondizione:} l'amministratore ha richiesto una lista di esami; 
	\item \textbf{Postcondizione:} l'amministratore visualizza la lista di crediti degli esami.
\end{itemize}
\subsection{UC7.8.3 - Visualizzazione sigla corsi di laurea associati ad esami se presenti}
\begin{itemize}
	\item \textbf{Attori primari:} amministratore;
	\item \textbf{Scopo e descrizione:} l'amministratore visualizza la lista delle sigle dei corsi di laurea associati agli esami presenti nel sistema per visualizzarli ed eventualmente effettuare operazioni su di essi;
	\item \textbf{Scenario principale:} il sistema mostra la lista delle sigle dei corsi di laurea associati agli esami presenti nel sistema;
	\item \textbf{Precondizione:} l'amministratore ha richiesto una lista di esami e l'esame di riferimento ha un corso di laurea associato; 
	\item \textbf{Postcondizione:} l'amministratore visualizza la lista delle sigle dei corsi di laurea associati agli esami.
\end{itemize}
\subsection{UC7.8.4 - Visualizzazione nomi dei professori associati agli esami se presenti}
\begin{itemize}
	\item \textbf{Attori primari:} amministratore;
	\item \textbf{Scopo e descrizione:} l'amministratore visualizza la lista dei nomi dei professori degli esami presenti nel sistema per visualizzarli ed eventualmente effettuare operazioni su di essi;
	\item \textbf{Scenario principale:} il sistema mostra la lista di nomi dei professori degli esami presenti nel sistema;
	\item \textbf{Precondizione:} l'amministratore ha richiesto una lista di esami e l'esame di riferimento ha un professore associato; 
	\item \textbf{Postcondizione:} l'amministratore visualizza la lista di nomi dei professori degli esami.
\end{itemize}
\subsection{UC7.8.5 - Visualizzazione cognomi dei professori associati agli esami se presenti}
\begin{itemize}
	\item \textbf{Attori primari:} amministratore;
	\item \textbf{Scopo e descrizione:} l'amministratore visualizza la lista dei cognomi dei professori degli esami presenti nel sistema per visualizzarli ed eventualmente effettuare operazioni su di essi;
	\item \textbf{Scenario principale:} il sistema mostra la lista di cognomi dei professori degli esami presenti nel sistema;
	\item \textbf{Precondizione:} l'amministratore ha richiesto una lista di esami e l'esame di riferimento ha un professore associato; 
	\item \textbf{Postcondizione:} l'amministratore visualizza la lista di cognomi dei professori degli esami.
\end{itemize}
\subsection{UC7.9 - Visualizzazione dettagli esame}
\begin{itemize}
	\item \textbf{Attori primari:} amministratore;
	\item \textbf{Scopo e descrizione:} l'amministratore visualizza il nome dell'esame, la relativa quantità di crediti, la sigla del corso di laurea associato, il nome, cognome e \citGloss{indirizzo} del professore associato ed il numero di studenti iscritti.
	\item \textbf{Scenario principale:}
	\begin{enumerate}
		\item L'amministratore ottiene una lista degli esami e seleziona quello di suo interesse [UC7.7 o UC7.8];
		\item Il sistema mostra una schermata informativa contenente i dati relativi all'esame [UC7.9].
	\end{enumerate}
	\item \textbf{Precondizione:} l'amministratore vuole visualizzare i dettagli di un esame dopo averlo individuato in una elenco di esami [UC7.7 o UC7.8]; 
	\item \textbf{Postcondizione:} l'amministratore ha ottenuto le informazioni ricercate.
\end{itemize}
\subsection{UC7.9.1 - Visualizzazione sigla esame}
\begin{itemize}
	\item \textbf{Attori primari:} amministratore;
	\item \textbf{Scopo e descrizione:} l'amministratore vuole visualizzare il nome dell'esame per effettuare, eventualmente, delle operazioni su di esso;
	\item \textbf{Scenario principale:} l'amministratore visualizza il nome dell'esame;
	\item \textbf{Precondizione:} l'amministratore ha selezionato un esame e vuole visualizzarne il nome; 
	\item \textbf{Postcondizione:} l'amministratore ha visualizzato il nome dell'esame.
\end{itemize}
\subsection{UC7.9.2 - Visualizzazione crediti esame}
\begin{itemize}
	\item \textbf{Attori primari:} amministratore;
	\item \textbf{Scopo e descrizione:} l'amministratore vuole visualizzare i crediti dell'esame per effettuare, eventualmente, delle operazioni su di esso;
	\item \textbf{Scenario principale:} l'amministratore visualizza i crediti dell'esame;
	\item \textbf{Precondizione:} l'amministratore ha selezionato un esame e vuole visualizzarne i crediti; 
	\item \textbf{Postcondizione:} l'amministratore ha visualizzato i crediti relativi all'esame.
\end{itemize}
\subsection{UC7.9.3 - Visualizzazione corso di laurea associato all'esame se presente}
\begin{itemize}
	\item \textbf{Attori primari:} amministratore;
	\item \textbf{Scopo e descrizione:} l'amministratore vuole visualizzare il nome del corso di laurea associato all'esame per effettuare, eventualmente, delle operazioni su di esso;
	\item \textbf{Scenario principale:} l'amministratore visualizza il nome del corso di laurea associato all'esame;
	\item \textbf{Precondizione:} l'amministratore ha selezionato un esame e questo ha un corso di laurea associato; 
	\item \textbf{Postcondizione:} l'amministratore ha visualizzato il nome del corso di laurea associato all'esame.
\end{itemize}
\subsection{UC7.9.4 - Visualizzazione nome del professore associato all'esame se presente}
\begin{itemize}
	\item \textbf{Attori primari:} amministratore;
	\item \textbf{Scopo e descrizione:} l'amministratore vuole visualizzare il nome del professore associato all'esame per effettuare, eventualmente, delle operazioni su di esso;
	\item \textbf{Scenario principale:} l'amministratore visualizza il nome del professore associato all'esame;
	\item \textbf{Precondizione:} l'amministratore ha selezionato un esame e questo ha un professore associato; 
	\item \textbf{Postcondizione:} l'amministratore ha visualizzato il nome del professore associato all'esame.
\end{itemize}
\subsection{UC7.9.5 - Visualizzazione cognome del professore associato all'esame se presente}
\begin{itemize}
	\item \textbf{Attori primari:} amministratore;
	\item \textbf{Scopo e descrizione:} l'amministratore vuole visualizzare il cognome del professore associato all'esame per effettuare, eventualmente, delle operazioni su di esso;
	\item \textbf{Scenario principale:} l'amministratore visualizza il cognome del professore associato all'esame;
	\item \textbf{Precondizione:} l'amministratore ha selezionato un esame e questo ha un professore associato; 
	\item \textbf{Postcondizione:} l'amministratore ha visualizzato il cognome del professore associato all'esame.
\end{itemize}
\subsection{UC7.9.6 - Visualizzazione numero di studenti iscritti all'esame}
\begin{itemize}
	\item \textbf{Attori primari:} amministratore;
	\item \textbf{Scopo e descrizione:} l'amministratore vuole visualizzare il numero di studenti iscritti all'esame per effettuare, eventualmente, delle operazioni su di esso;
	\item \textbf{Scenario principale:} l'amministratore visualizza il numero di studenti iscritti all'esame;
	\item \textbf{Precondizione:} l'amministratore ha selezionato un esame e vuole visualizzarne il numero di studenti iscritti; 
	\item \textbf{Postcondizione:} l'amministratore ha visualizzato il numero di studenti iscritti.
\end{itemize}
\subsection{UC7.10 - Associa professore a esame}
\begin{itemize}
	\item \textbf{Attori primari:} amministratore;
	\item \textbf{Scopo e descrizione:} l'amministratore associa ad un esame un professore;
	\item \textbf{Scenario principale:};
	\begin{enumerate}
		\item L'amministratore seleziona un'esame [UC7.10.1];
		\item L'amministratore ottiene una lista dei professori non associati all'esame selezionato [UC7.10.2];
		\item L'amministratore seleziona il professore e lo associa all'esame [UC7.10.3].
	\end{enumerate}
	\item \textbf{Precondizione:} l'amministratore desidera associare un professore ad un esame; 
	\item \textbf{Postcondizione:} il professore selezionato viene associato all'esame e il loro stato aggiornato nel sistema.
\end{itemize}
%TODO: immagine
\subsection{UC7.10.1 - Visualizzazione lista di esami non ancora assegnati}
\begin{itemize}
\item \textbf{Attori primari:} amministratore;
\item \textbf{Scopo e descrizione:} l'amministratore ottiene una lista dei codici e i relativi corsi degli esami che non presentano un professore associato;
\item \textbf{Scenario principale:} l'amministratore ottiene una lista di esami ai quali può associare un nuovo professore, essendone sprovvisti;
\item \textbf{Precondizione:} l'amministratore desidera ottenere una lista di esami sprovvisti di professore associato; 
\item \textbf{Postcondizione:} l'amministratore ottiene la lista richiesta e può procedere a identificare l'esame di suo interesse.
\end{itemize}
\subsection{UC7.10.2 - Visualizzazione lista professori selezionato un esame}
\begin{itemize}
	\item \textbf{Attori primari:} amministratore;
	\item \textbf{Scopo e descrizione:} l'amministratore visualizza una lista dei nomi, cognomi e indirizzi dei professori per identificare quale assegnare all'esame precedentemente indicato;
	\item \textbf{Scenario principale:} il sistema mostra all'amministratore una lista di professori in modo da consentire la scelta di quale associare all'esame precedentemente indicato;
	\item \textbf{Precondizione:} l'amministratore ha selezionato l'esame desiderato [UC7.10.1] e desidera ottenere la lista dei professori per poterne associare uno;  
	\item \textbf{Postcondizione:} l'amministratore è a conoscenza dell'elenco dei professori.
\end{itemize}
\subsection{UC7.10.3 - Associazione professore all'esame}
\begin{itemize}
	\item \textbf{Attori primari:} amministratore;
	\item \textbf{Scopo e descrizione:} l'amministratore associa un professore dopo averlo individuato in una lista contenente tutti i professori ad un esame che ne è ancora sprovvisto;
	\item \textbf{Scenario principale:} l'amministratore associa ad un esame un determinato professore;
	\item \textbf{Precondizione:} l'amministratore ha selezionato un esame sprovvisto di assegnazione a un professore [UC7.10.1] e ha ottenuto una lista di professori [UC7.1.2], volendo associarne uno; 
	\item \textbf{Postcondizione:} il professore selezionato viene associato all'esame desiderato.
\end{itemize}
%TODO: immagine 7.11
\subsection{UC7.11 - Visualizzazione di tutti gli anni accademici}
\begin{itemize}
	\item \textbf{Attori primari:} amministratore;
	\item \textbf{Scopo e descrizione:} l'utente è già riconosciuto dal sistema come amministratore e visualizza una lista di tutti gli anni accademici presenti nel sistema, visualizzandone l'anno solare di riferimento;
	\item \textbf{Scenario principale:} l'amministratore richiede la lista di tutti gli anni del sistema;
	\item \textbf{Precondizione:} l'amministratore desidera ottenere la lista di tutti gli anni presenti nel sistema; 
	\item \textbf{Postcondizione:} l'amministratore che rappresenta l'università ottiene la lista.
\end{itemize}
\subsection{UC7.11.1 - Visualizzazione anni solari relativi agli anni accademici}
\begin{itemize}
	\item \textbf{Attori primari:} amministratore;
	\item \textbf{Scopo e descrizione:} l'utente già riconosciuto dal sistema come amministratore visualizza una lista di tutti gli anni solari relativi agli anni accademici inseriti nella rete \citGloss{Ethereum};
	\item \textbf{Scenario principale:} l'amministratore richiede al sistema la lista di anni accademici inseriti nel sistema per visualizzarli ed eventualmente compiere operazioni su di esse;
	\item \textbf{Precondizione:} l'amministratore richiede una lista di anni accademici per visualizzarne gli anni solari ad essi associati; 
	\item \textbf{Postcondizione:} l'amministratore ottiene la lista con indicati gli anni solari.
\end{itemize}

\subsection{UC8 - Gestione aspetti relativi agli esami}
\begin{itemize}
	\item \textbf{Attori primari:} professore;
	\item \textbf{Scopo e descrizione:} l'utente è già riconosciuto dal sistema come professore e sceglie di utilizzare una funzionalità messa a sua disposizione da parte del sistema;
	\item \textbf{Precondizione:} il professore desidera effettuare delle operazioni relative agli esami a lui assegnati;
	\item \textbf{Postcondizione:} il professore ha visualizzato le operazioni messe a disposizione dal sistema e sceglie quale effettuare.
\end{itemize}

\begin{figure}[H]
	\centering
	\includegraphics[width=0.8\linewidth]{UC8.jpg}
	\caption{UC8 - Gestione aspetti relativi agli esami}
	\label{fig:UC8 - Gestione aspetti relativi agli esami}
\end{figure}

\subsection{UC8.1 - Visualizzazione lista degli esami}
\begin{itemize}
	\item \textbf{Attori primari:} professore;
	\item \textbf{Scopo e descrizione:} il professore visualizza una lista di tutti gli esami ai quali è assegnato;
	\item \textbf{Scenario principale:} il professore richiede al sistema la lista degli esami di sua competenza per poterla consultare;
	\item \textbf{Precondizione:} l'utente è già riconosciuto dal sistema come professore e richiede la visualizzazione della lista degli esami di sua competenza;
	\item \textbf{Postcondizione:} il professore ottiene la lista degli esami ai quali è assegnato per poterla consultare.
\end{itemize}
\subsection{UC8.1.1 - Visualizzazione codici degli esami}
\begin{itemize}
	\item \textbf{Attori primari:} professore;
	\item \textbf{Scopo e descrizione:} il professore visualizza per ogni esame il relativo codici;
	\item \textbf{Scenario principale:} il professore richiede al sistema la lista dei codici degli esami di sua competenza per poterla consultare;
	\item \textbf{Precondizione:}l'utente è già riconosciuto dal sistema come professore e richiede la visualizzazione dei codici degli esami di sua competenza.
	\item \textbf{Postcondizione:} il professore ottiene la lista con indicati i codici.
\end{itemize}
\subsection{UC8.1.2 - Visualizzazione corsi degli esami}
\begin{itemize}
	\item \textbf{Attori primari:} professore;
	\item \textbf{Scopo e descrizione:} il professore visualizza per ogni esame il relativo corso di laurea;
	\item \textbf{Scenario principale:} il professore richiede al sistema la lista dei corsi di laurea degli esami di sua competenza per poterla consultare;
	\item \textbf{Precondizione:}l'utente è già riconosciuto dal sistema come professore e richiede la visualizzazione dei corsi di laurea degli esami di sua competenza.
	\item \textbf{Postcondizione:} il professore ottiene la lista con indicati i corsi di laurea.
\end{itemize}

\subsection{UC8.2 - Visualizzazione lista degli studenti}
\begin{itemize}
	\item \textbf{Attori primari:} professore;
	\item \textbf{Scopo e descrizione:} il professore visualizza una lista di tutti gli studenti iscritti ad un esame a lui assegnato;
	\item \textbf{Scenario principale:} il professore richiede al sistema la lista degli studenti iscritti ad un esami di sua competenza per poterla consultare;
	\item \textbf{Flusso principale degli eventi:}
	\begin{enumerate}
		\item Il professore visualizza la lista degli esami di sua competenza [UC8.1];
		\item Il professore, una volta individuato l'esame al quale è interessato, ne richiede la lista degli studenti registrati [UC8.2].
	\end{enumerate}
	\item \textbf{Precondizione:} l'utente è già riconosciuto dal sistema come professore e richiede la visualizzazione della lista degli studenti iscritti a un determinato esame di sua competenza;
	\item \textbf{Postcondizione:} il professore ottiene la lista degli studenti iscritti all'esame richiesto per poterla consultare.
\end{itemize}

\subsection{UC8.2.1 - Visualizzazione indirizzi degli studenti dell'esame}
\begin{itemize}
	\item \textbf{Attori primari:} professore;
	\item \textbf{Scopo e descrizione:} il professore visualizza una lista di tutti gli \citGloss{indirizzi} della rete \citGloss{Ethereum} degli studenti registrati all'esame;
	\item \textbf{Scenario principale:} il professore richiede al sistema una lista di studenti per poterne visualizzare l'\citGloss{indirizzo} ed eventualmente compiere altre operazioni su di essi;
	\item \textbf{Precondizione:}l'utente è già riconosciuto dal sistema come professore e richiede la visualizzazione della lista degli studenti iscritti a un determinato esame di sua competenza per visualizzarne i relativi \citGloss{indirizzi}.
	\item \textbf{Postcondizione:} il professore ottiene la lista con indicati gli \citGloss{indirizzi}.
\end{itemize}
\subsection{UC8.2.2 - Visualizzazione nomi degli studenti dell'esame}
\begin{itemize}
	\item \textbf{Attori primari:} professore;
	\item \textbf{Scopo e descrizione:} il professore visualizza una lista di tutti i nomi degli studenti registrati all'esame;
	\item \textbf{Scenario principale:} il professore richiede al sistema una lista di studenti per poterne visualizzare il nome ed eventualmente compiere altre operazioni su di essi;
	\item \textbf{Precondizione:}l'utente è già riconosciuto dal sistema come professore e richiede la visualizzazione della lista degli studenti iscritti a un determinato esame di sua competenza per visualizzarne i relativi nomi.
	\item \textbf{Postcondizione:} il professore ottiene la lista con indicati i relativi nomi.
\end{itemize}
\subsection{UC8.2.3 - Visualizzazione cognomi degli studenti dell'esame}
\begin{itemize}
	\item \textbf{Attori primari:} professore;
	\item \textbf{Scopo e descrizione:} il professore visualizza una lista di tutti i cognomi degli studenti registrati all'esame;
	\item \textbf{Scenario principale:} il professore richiede al sistema una lista di studenti per poterne visualizzare il cognomi ed eventualmente compiere altre operazioni su di essi;
	\item \textbf{Precondizione:}l'utente è già riconosciuto dal sistema come professore e richiede la visualizzazione della lista degli studenti iscritti a un determinato esame di sua competenza per visualizzarne i relativi cognomi.
	\item \textbf{Postcondizione:} il professore ottiene la lista con indicati i relativi cognomi.
\end{itemize}

\subsection{UC8.3 - Registrazione valutazione di un esame}
\begin{itemize}
	\item \textbf{Attori primari:} professore;
	\item \textbf{Scopo e descrizione:} il professore inserisce nel sistema una valutazione;
	\item \textbf{Flusso principale degli eventi:}
	\begin{enumerate}
		\item Il professore visualizza la lista degli esami di sua competenza [UC8.1];
		\item Il professore, una volta individuato l'esame al quale è interessato, ne richiede la lista degli studenti registrati [UC8.2];
		\item Il professore consulta la lista degli studenti ed individua la persona alla quale deve inserire la valutazione [UC8.2];
		\item Il professore inserisce la valutazione allo studente [UC8.3].
	\end{enumerate}
	\item \textbf{Precondizione:} il professore possiede una valutazione da assegnare a uno studente riguardante un determinato esame e desidera registrarlo;
	\item \textbf{Postcondizione:} il voto è stato inserito nella \citGloss{blockchain} universitaria.
\end{itemize}


\subsection{UC9 - Gestione aspetti relativi allo studente}
\begin{itemize}
	\item \textbf{Attori primari:} studente;
	\item \textbf{Attori secondari:} \citGloss{MetaMask};
	\item \textbf{Scopo e descrizione:} l'utente è già riconosciuto dal sistema come studente e sceglie di utilizzare una funzionalità messa a sua disposizione da parte del sistema;
	\item \textbf{Precondizione:} lo studente desidera effettuare delle operazioni relative alla scelta di esami opzionali o alla visione delle valutazioni;
	\item \textbf{Postcondizione:} lo studente ha visualizzato le operazioni messe a disposizione dal sistema e sceglie quale effettuare.
\end{itemize}

\begin{figure}[H]
	\centering
	\includegraphics[width=0.8\linewidth]{UC9.jpg}
	\caption{UC9 - Gestione aspetti relativi allo studente}
	\label{fig:UC9 - Gestione aspetti relativi allo studente} %TODO da rifare con "7.3 Iscrizione"
\end{figure}


\subsection{UC9.1 - Visualizzazione delle informazione degli esami ai quali è iscritto}
\begin{itemize}
	\item \textbf{Attori primari:} studente;
	\item \textbf{Scopo e descrizione:} lo studente visualizza una lista di tutti gli esami ai quali è iscritto indicati dal nome;
	\item \textbf{Scenario principale:} lo studente richiede al sistema la lista degli esami ai quali è iscritto per poterla consultare;
	\item \textbf{Precondizione:} l'utente è già riconosciuto dal sistema come studente e richiede la visualizzazione della lista degli esami ai quali è iscritto ed le relative informazioni;
	\item \textbf{Postcondizione:} lo studente ottiene la lista per poterla consultare.
\end{itemize}

\begin{figure}[H]
	\centering
	\includegraphics[width=1.0\linewidth]{UC9_1.jpg}
	\caption{UC9.1 - Visualizzazione delle informazione degli esami ai quali è iscritto}
	\label{fig:UC9.1 - Visualizzazione delle informazione degli esami ai quali 'e iscritto}
\end{figure}

\subsection{UC9.1.1 - Visualizzazione dei crediti degli esami ai quali è iscritto}
\begin{itemize}
\item \textbf{Attori primari:} studente;
\item \textbf{Scopo e descrizione:} lo studente visualizza il numero di crediti per ogni esame al quale è iscritto;
\item \textbf{Scenario principale:} lo studente richiede al sistema il numero di crediti degli esami ai quali è iscritto per consultarli;
\item \textbf{Precondizione:} l'utente è già riconosciuto dal sistema come studente e richiede il numero di crediti per ogni esame ai quale è iscritto;
\item \textbf{Postcondizione:} lo studente ottiene le informazioni richieste.
\end{itemize}

\subsection{UC9.1.2 - Visualizzazione della obbligatorietà degli esami ai quali è iscritto}
\begin{itemize}
	\item \textbf{Attori primari:} studente;
	\item \textbf{Scopo e descrizione:} lo studente visualizza il numero di crediti di per ogni esame al quale è iscritto;
	\item \textbf{Scenario principale:} lo studente richiede al sistema il numero di crediti degli esami ai quali è iscritto per consultarli;
	\item \textbf{Precondizione:} l'utente è già riconosciuto dal sistema come studente e richiede il numero di crediti per ogni esame ai quale è iscritto;
	\item \textbf{Postcondizione:} lo studente ottiene le informazioni richieste.
\end{itemize}

\subsection{UC9.1.3 - Visualizzazione delle valutazioni degli esami ai quali è iscritto}
\begin{itemize}
	\item \textbf{Attori primari:} studente;
	\item \textbf{Scopo e descrizione:} lo studente visualizza la valutazione, in formato numerico con l'indicazione di averlo superato se il voto è superiore a 18, per ogni esame al quale è iscritto;
	\item \textbf{Scenario principale:} lo studente richiede al sistema le informazioni riguardanti le valutazioni degli esami ai quali è iscritto per consultarle;
	\item \textbf{Precondizione:} l'utente è già riconosciuto dal sistema come studente e richiede le informazioni riguardanti le valutazioni degli esami ai quali è iscritto;
	\item \textbf{Postcondizione:} lo studente ottiene le informazioni richieste.
\end{itemize}

\subsection{UC9.2 - Visualizzazione degli esami opzionali e dei loro crediti}
\begin{itemize}
	\item \textbf{Attori primari:} studente;
	\item \textbf{Scopo e descrizione:} lo studente visualizza una lista di tutti gli esami opzionali del suo corso, con indicati nomi e crediti, ai quali ha possibilità di iscriversi;
	\item \textbf{Scenario principale:} lo studente richiede al sistema la lista di tutti gli esami opzionali ai quali ha possibilità di iscriversi per poterla consultare;
	\item \textbf{Precondizione:} l'utente è già riconosciuto dal sistema come studente e richiede la visualizzazione della lista di tutti gli esami opzionali ai quali ha possibilità di iscriversi;
	\item \textbf{Postcondizione:} lo studente ottiene la lista per poterla consultare.
\end{itemize}

\subsection{UC9.3 - Iscrizione ad un esame opzionale}
\begin{itemize}
	\item \textbf{Attori primari:} studente;
	\item \textbf{Scopo e descrizione:} lo studente si iscrive a un determinato esame opzionale;
	\item \textbf{Flusso principale degli eventi:}
	\begin{enumerate}
		\item Lo studente visualizza la lista degli esami opzionali [UC9.2];
		\item Lo studente individua l'esame al quale è interessato iscriversi;
		\item Lo studente compie l'operazione di iscrizione.
	\end{enumerate}
	\item \textbf{Precondizione:} lo studente ha individuato l'esame opzionale al quale desidera iscriversi;
	\item \textbf{Postcondizione:} lo studente viene registrato ad un esame opzionale.
\end{itemize}

\subsection{UC9.4 - Visualizzazione delle informazioni relative ai crediti}
\begin{itemize}
	\item \textbf{Attori primari:} studente;
	\item \textbf{Scopo e descrizione:} lo studente visualizza un riepilogo del numero dei crediti che possiede e dell'obiettivo per poter conseguire la laurea;
	\item \textbf{Scenario principale:} lo studente richiede al sistema le informazioni riguardanti i suoi crediti;
	\item \textbf{Precondizione:} l'utente è già riconosciuto dal sistema come studente e richiede la visualizzazione delle informazioni relative ai suoi crediti;
	\item \textbf{Postcondizione:} lo studente ottiene le informazioni richieste per poterle consultare.
\end{itemize}
\subsection{UC10 - Gestione degli amministratori}
\begin{itemize}
	\item \textbf{Attori primari:} università;
	\item \textbf{Attori secondari:} \citGloss{MetaMask};
	\item \textbf{Scopo e descrizione:} l'utente è già riconosciuto dal sistema come rappresentante dell'università e sceglie di utilizzare una funzionalità messa a sua disposizione da parte del sistema;
	\item \textbf{Precondizione:} l'utente che rappresenta l'università desidera effettuare delle operazioni relative alla aggiunta o alla rimozione di amministratori;
	\item \textbf{Postcondizione:} l'utente che rappresenta l'università ha visualizzato le operazioni messe a disposizione dal sistema e sceglie quale effettuare.
\end{itemize}

\begin{figure}[H]
	\centering
	\includegraphics[width=1.0\linewidth]{UC10.jpg}
	\caption{UC10 - Gestione degli amministratori}
	\label{fig:UC10 - Gestione degli amministratori}
\end{figure}

\subsection{UC10.1 - Aggiunta di un amministratore}
\begin{itemize}
	\item \textbf{Attori primari:} università;
	\item \textbf{Scopo e descrizione:} l'utente che rappresenta l'università aggiunge un nuovo amministratore nel sistema, abilitandolo ad operare come definito dal sistema;
	\item \textbf{Precondizione:} l'utente che rappresenta l'università desidera l'aggiunta di un nuovo amministratore e possiede già l'indirizzo dell'amministratore, comunicatogli precedentemente. L'indirizzo si suppone essere precedentemente generato dall'amministratore utilizzando \citGloss{MetaMask} o un qualsiasi altro generatore; 
	\item \textbf{Postcondizione:} il nuovo amministratore è stato inserito nel sistema e può operare come tale.
\end{itemize}

\begin{figure}[H]
	\centering
	\includegraphics[width=1.0\linewidth]{UC10_1.jpg}
	\caption{UC10.1 - Aggiunta di un amministratore}
	\label{fig:UC10.1 - Aggiunta di un amministratore}
\end{figure}

\subsection{UC10.1.1 - Inserimento chiave pubblica}
\begin{itemize}
	\item \textbf{Attori primari:} università;
	\item \textbf{Scopo e Descrizione:} il sistema offre l'interfaccia per inserire la \citGloss{chiave pubblica} del nuovo amministratore;
	\item \textbf{Scenario principale:} l'utente che rappresenta l'università fornisce tramite un dispositivo di input il suo nome;	\begin{enumerate}
		\item Se la \citGloss{chiave pubblica} è mal formata l'utente viene avvisato con un relativo messaggio d'errore [UC10.1.2];
		\item Se la \citGloss{chiave pubblica} è già registrata nel sistema in un altro ruolo l'utente viene avvisato un relativo messaggio d'errore [UC10.1.3].
	\end{enumerate}
	\item \textbf{Precondizione:} il sistema offre un campo per la compilazione della \citGloss{chiave pubblica} del nuovo amministratore e rimane in attesa dell'input;
	\item \textbf{Postcondizione:} l'utente che rappresenta l'università ha inserito la \citGloss{chiave pubblica} del nuovo amministratore nel campo fornito dal sistema.
\end{itemize}
\subsection{UC10.1.2 - Visualizzazione messaggio di errore riguardo a chiave mal formata}
\begin{itemize}
	\item \textbf{Attori primari:} università;
	\item \textbf{Scopo e descrizione:} l'utente viene avvisato del fatto che ha fornito una \citGloss{chiave} ma formata, ad esempio di lunghezza invalida oppure non in caratteri esadecimali;
	\item \textbf{Scenario principale:} l'utente viene informato che la sua \citGloss{chiave} risulta mal formata, invitandolo a ricontrollarla;
	\item \textbf{Precondizione:} l'utente cerca di registrare nel sistema una \citGloss{chiave} mal formata;
	\item \textbf{Postcondizione:} l'utente è consapevole di aver tentato l'inserimento di una \citGloss{chiave} mal formata.
\end{itemize}
\subsection{UC10.1.3 - Visualizzazione messaggio di errore riguardo a chiave già registrata}
\begin{itemize}
	\item \textbf{Attori primari:} università;
	\item \textbf{Scopo e descrizione:} l'utente viene avvisato del fatto che ha fornito una \citGloss{chiave} già registrata in un altro ruolo all'interno dell'università, in quanto il suo \citGloss{indirizzo} collide con un secondo precedentemente inserito;
	\item \textbf{Scenario principale:} l'utente viene informato che la sua \citGloss{chiave} risulta già registrata;
	\item \textbf{Precondizione:} l'utente cerca di registrare nel sistema una \citGloss{chiave} precedentemente registrata in un altro ruolo;
	\item \textbf{Postcondizione:} l'utente è consapevole di aver tentato l'inserimento di una \citGloss{chiave} già registrata nell'università in un altro ruolo.
\end{itemize}


\subsection{UC10.2 - Visualizzazione lista degli amministratori}
\begin{itemize}
	\item \textbf{Attori primari:} università;
	\item \textbf{Scopo e descrizione:} l'utente che rappresenta l'università visualizza una lista di tutti gli amministratori;
	\item \textbf{Scenario principale:} l'utente che rappresenta l'università richiede al sistema la lista degli amministratori per poterla consultare ed eventualmente compiere altre operazioni su di essi;
	\item \textbf{Precondizione:} l'utente che rappresenta l'università è già riconosciuto dal sistema come tale e richiede la visualizzazione della lista degli amministratori;
	\item \textbf{Postcondizione:} l'utente che rappresenta l'università ottiene la lista degli amministratori per poterla consultare.
\end{itemize}
\subsection{UC10.2.1 - Visualizzazione indirizzi degli amministratori}
\begin{itemize}
	\item \textbf{Attori primari:} università;
	\item \textbf{Scopo e descrizione:} l'utente che rappresenta l'università visualizza per ogni amministratore il suo indirizzo;
	\item \textbf{Scenario principale:} l'utente che rappresenta l'università richiede al sistema la lista dell'indirizzo di ogni amministratore presente;
	\item \textbf{Precondizione:} l'utente che rappresenta l'università è già riconosciuto dal sistema come tale e richiede la visualizzazione degli indirizzi degli amministratori;
	\item \textbf{Postcondizione:} l'utente che rappresenta l'università ottiene la lista degli indirizzi degli amministratori.
\end{itemize}
\subsection{UC10.3 - Rimozione amministratore}
\begin{itemize}
	\item \textbf{Attori primari:} università;
	\item \textbf{Scopo e descrizione:} l'utente che rappresenta l'università rimuove dal sistema un amministratore;
	\item \textbf{Flusso principale degli eventi:}
	\begin{enumerate}
		\item L'utente che rappresenta l'università  visualizza la lista degli amministratori [UC10.2];
		\item L'utente che rappresenta l'università, una volta ottenuta la lista degli amministratori, individua quale desidera rimuovere [UC10.2];
		\item L'utente che rappresenta l'università richiede al sistema l'eliminazione dell'amministratore [UC10.3].
	\end{enumerate}
	\item \textbf{Precondizione:} l'utente che rappresenta l'università desidera eliminare un amministratore dal sistema;
	\item \textbf{Postcondizione:} l'amministratore indicato viene rimosso dal ruolo nell'università e non può più autenticarsi come tale.
\end{itemize}

\subsection{UC11 - Visualizzazione quantità di Gas, Ether e costo delle operazioni}
\begin{itemize}
	\item \textbf{Attori primari:} utente generico;
	\item \textbf{Attori secondari:} \citGloss{MetaMask};
	\item \textbf{Scopo e descrizione:} l'utente generico può vedere la quantità di \citGloss{Gas}, \citGloss{Ether} ed euro che andrà ad utilizzare per svolgere determinate azioni;
	\item \textbf{Precondizione:} l'utente generico desidera e richiede al sistema i costi delle operazioni;
	\item \textbf{Postcondizione:} il sistema fornisce all'utente generico le informazioni da lui richieste e l'utente generico acquisisce l'informazione.
\end{itemize}

\end{document}