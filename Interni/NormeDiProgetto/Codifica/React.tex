\documentclass[../ProcessiPrimari.tex]{subfiles}

\begin{document}	
	\subsubsection{JavaScript Syntax eXtension}
	La scrittura di codice in JavaScript Syntax eXtension, comunemente abbreviato in JSX, deve seguire tutte le norme descritte precedentemente per il linguaggio di programmazione JavaScript.\\
	In aggiunta sono elencate ulteriori norme riguardanti esclusivamente JSX, tratte da \textbf{Airbnb React/JSX Style Guide}\footnote{\href{https://github.com/airbnb/javascript/tree/master/react}{https://github.com/airbnb/javascript/tree/master/react}}. 

	\paragraph*{Indentazione 1: }
	va inserito sempre uno (1) spazio nei tag di chiusura.	
	\begin{center}{	
			\begin{minipage}{3cm}
				{\begin{center}SI\end{center}}
				\begin{Verbatim}[frame=single]
<Foo />
				\end{Verbatim}
			\end{minipage}
			\hfil
			\begin{minipage}{3cm}
				{\begin{center}NO\end{center}}
				\begin{Verbatim}[frame=single]
<Foo/>
				\end{Verbatim}
			\end{minipage}
		}
	\end{center}	
	\paragraph*{Proprietà 1: }
utilizzare sempre lo stile \texttt{mixedCase} per i nomi delle proprietà di un tag. 
	\begin{center}{
			\begin{minipage}{6cm}
				{\begin{center}SI\end{center}}
				\begin{Verbatim}[frame=single]
<Foo
  userName="hello"
  phoneNumber={12345678}
/>
				\end{Verbatim}
			\end{minipage}
			\hfil
			\begin{minipage}{6cm}
				{\begin{center}NO\end{center}}
				\begin{Verbatim}[frame=single]
<Foo
  UserName="hello"
  phone_number={12345678}
/>
				\end{Verbatim}
			\end{minipage}
		}
	\end{center}
	\paragraph*{Proprietà 2: }
il valore di una proprietà va omesso quando è implicitamente vero (\texttt{true}).  

\begin{center}{
		\begin{minipage}{5cm}
			{\begin{center}SI\end{center}}
			\begin{Verbatim}[frame=single]
<Foo
  hidden
/>
			\end{Verbatim}
		\end{minipage}
		\hfil
		\begin{minipage}{5cm}
			{\begin{center}NO\end{center}}
			\begin{Verbatim}[frame=single]
<Foo
  hidden={true}
/>
			\end{Verbatim}
		\end{minipage}
	}
	\end{center}	
	\paragraph*{I tag 1: }
si deve chiudere tutti i tag che non hanno i tag figli o altro contenuto senza utilizzare il relativo tag di chiusura.  

\begin{center}{\begin{minipage}{8cm}
			{\begin{center}SI\end{center}}
			\begin{Verbatim}[frame=single]
<Foo variant="stuff" />
			\end{Verbatim}
		\end{minipage}
		\vskip 1em
		\begin{minipage}{8cm}
			{\begin{center}NO\end{center}}
			\begin{Verbatim}[frame=single]
<Foo variant="stuff"></Foo>			
			\end{Verbatim}
		\end{minipage}
	}
\end{center}	
	\paragraph*{I tag 2: }
è vietato chiudere un tag con più di una proprietà sulla stessa riga. 

\begin{center}{
		\begin{minipage}{6cm}
			{\begin{center}SI\end{center}}
			\begin{Verbatim}[frame=single]
<Foo
  bar="bar"
  baz="baz"
/>
			\end{Verbatim}
		\end{minipage}
		\hfil
		\begin{minipage}{5cm}
			{\begin{center}NO\end{center}}
			\begin{Verbatim}[frame=single]
<Foo
  bar="bar"
  baz="baz" />
			\end{Verbatim}
		\end{minipage}
	}
\end{center}
\end{document}