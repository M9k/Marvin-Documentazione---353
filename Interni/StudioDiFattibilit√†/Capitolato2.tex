\documentclass[StudioDiFattibilità.tex]{subfiles}
\begin{document}
\chapter{Capitolato C2 - BlockCV}
\section{Descrizione generale}
Il capitolato C2 si pone l'obbiettivo di sviluppare un sistema di curriculum vitae distribuito su una blockchain privata.\\
I vantaggi di tale scelta sarebbero l'impossibilità di un utente di mentire riguardo alle esperienze passate e impedire la creazione di CV multipli mirati a determinate aziende.
\section{Obiettivo finale}
L'obiettivo finale è la creazione di una web application che consenta ad utenti ed enti certificati l'accesso alle informazioni, la modifica dei dati di propria competenza e la ricerca di lavori o di curriculum con requisiti specificati.\\
Gli utenti devono anche avere la possibilità di esportare il proprio CV in formato standard e di importarlo da altre sorgenti.\\
Va posta una particolare attenzione sul rispetto della privacy secondo il regolamento europeo GDPR 2016/679.
\section{Tecnologie richieste}
\begin{itemize}
	\item \textbf{Hyperledger Fabric}, framework per la creazione di blockchain con supporto a smart contract.
	\item \textbf{java EE}, ove possibile.
	\item \textbf{Play o Vaadin Elements}, frameworks per lo sviluppo di interfaccia grafica.
	\item \textbf{MongoDB o Cassandra}, database non relazionali, da utilizzare come supporto alla blockchain se si rivelasse necessario.
\end{itemize}
\section{Valutazione finale}
Il gruppo 353 ha valutato positivamente la proposta perché il progetto riguarda l'uso di blockchain, una tecnologia innovativa che è stata considerata come una probabile svolta tecnologica e finanziaria da parte di tutti i membri del gruppo.\\
Si è giudicato particolarmente entusiasmante il porre la concentrazione sulla governance delle blockchain e sulla gestione dei dati mantenendo il rispetto alla privacy.\\
Il capitolato non è stato scelto in quanto gli stessi aspetti positivi erano presenti anche su Marvin, che aggiungeva anche l'utilizzo di React, tecnologia che si sta ampliamente diffondendo in ambito web development.\\
Un aspetto che è stato giudicato negativo da parte del gruppo di questo capitolato è l'utilizzo di una blockchain privata, dettaglio che ci ha lasciato perplessi, in quanto la maggior parte delle catene di blocchi più diffuse sono pubbliche, ad esempio Bitcoin, Litecoin ed Ethereum.
\end{document}