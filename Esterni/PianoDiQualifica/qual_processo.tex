\documentclass[PianoDiQualifica.tex]{subfiles}

\begin{document}

\chapter{Qualità di processo}

\section{Scopo} 
Per garantire la qualità del prodotto è necessario perseguire la qualità dei processi che lo definiscono.
Per raggiungere questo obiettivo, si è deciso di seguire il principio di miglioramento continuo (\citGloss{PDCA}) e di adottare lo standard ISO/IEC 15504 denominato SPICE (Software Process Improvement and Capability Determination).

\section{Processi}
I processi a cui sono applicati i controlli di qualità sono descritti nelle \ndp.
\subsection{Tabella dei processi}
% ID Processo
% PROC --> metrica
% 00n --> numero incrementale
Questa tabella indica i processi presenti.
\begin{table}[H]
	\begin{center}
		\begin{tabu} to \textwidth {
				>{\centering}m{0.2\linewidth}
				>{\centering}m{0.7\linewidth}
			}
			\tableHeaderStyle
			\textbf{ID} & \textbf{Nome} \\
			\textbf{PROC001} & {Pianificazione di progetto, impostazione e controllo di processi}\\ 
			\textbf{PROC002} & {Verifica Software}\\ 
			\textbf{PROC003} & {Gestione rischi}\\ 
			
		\end{tabu}
		\caption{Tabella dei processi}
		\vspace{-1em}
	\end{center}
\end{table}

\subsection{Tabella delle metriche}
% ID METRICA
% M --> metrica
% PS--> processo
% (lettera per macro categoria, se esistono)
% 00n --> numero incrementale
Questa tabella indica i \textbf{range} di accettazione e di ottimalità per le metriche utilizzate per la qualità di processo:
\begin{table}[H]
	\begin{center}
		\begin{tabu} to \textwidth {
				>{\centering}m{0.1\linewidth}
				>{\centering}m{0.4\linewidth}
				>{\centering}m{0.2\linewidth} 
				>{\centering\arraybackslash}m{0.2\linewidth}
			}
			\tableHeaderStyle
			
			\textbf{ID} & \textbf{Nome} & \textbf{Range di accettazione} & \textbf{Range di ottimalità}\\
			
			\multicolumn{4}{c}{\textbf{PROC001}}\\
			\textlink{MPS001}{MPS001TAB}{MPS001} & Schedule Variance & $ \geq -5 \space giorni $ & $ \geq 0 giorni $ \\
			\textlink{MPS002}{MPS002TAB}{MPS002} & Budget Variance & $ \geq -10\% $ & $ \geq 0 $ \\
			
			\hline
			\multicolumn{4}{c}{\textbf{PROC002}}\\
			\textlink{MPS003}{MPS003TAB}{MPS003} & Function coverage & $ \geq 98\% $ & $ 100\% $\\
			\textlink{MPS004}{MPS004TAB}{MPS004} & Statement coverage &  $ \geq 97\% $& $ 100\% $\\
			\textlink{MPS005}{MPS005TAB}{MPS005} & \citGloss{branch} coverage & $ \geq 95\% $ & $ 100\% $\\
			\textlink{MPS006}{MPS006TAB}{MPS006} & Condition coverage & $ \geq 99\% $ & $ 100\% $\\

			\hline
			\multicolumn{4}{c}{\textbf{PROC003}}\\
			\textlink{MPS007}{MPS007TAB}{MPS007} & Indisponibilità servizi esterni & $ -3 $ & $ 0 $\\
			\textlink{MPS008}{MPS008TAB}{MPS008} & Rischi non previsti & $ -5 $ & $ 0 $\\ 


			
		\end{tabu}
		\caption{Tabella delle metriche della qualità di processo}
		\vspace{-1em}
	\end{center}
\end{table}

\section{Tabella delle metriche sui test PROC0004}
Questa tabella indica i \textbf{range} di accettazione e ottimalità per le misure effettuate sui test, per le misure effettuate su tutti i tipi di test saranno presenti due valori per range, TM, per i test di modulo e TH, per i test ad alto livello.\\
Una piccola legenda per facilitare la lettura della tabella: \\
\begin{itemize}
	\item \textbf{PRO:} percentuale requisiti obbligatori;
	\item \textbf{PRF:} percentuale requisiti facoltativi;
	\item \textbf{xh:} dove x è un numero, ore.
\end{itemize}

\begin{table}[H]
	\begin{center}
		\begin{tabu} to \textwidth {
				>{\centering}m{0.15\linewidth}
				>{\centering}m{0.45\linewidth}
				>{\centering}m{0.18\linewidth} 
				>{\centering\arraybackslash}m{0.17\linewidth}
			}
			\tableHeaderStyle
			\textbf{ID} & \textbf{Nome} & \textbf{Range di accettazione} & \textbf{Range di ottimalità}\\
			\multicolumn{4}{c}{\textbf{Per tutti i test}}\\
			\textlink{MTSA001}{MTSA001TAB}{MTSA001} & Percentuale di test case passati & 100\%-100\%(TM) PRC-100\%(TH) & 100\%-100\%(TM) 100\%-100\%(TH) \\ 
			\textlink{MTSA002}{MTSA002TAB}{MTSA002} & Percentuale di test case falliti & 0\%-0\%(TM) 0\%-PRF(TH) & 0\%-0\%(TM) 0\%-0\%(TH) \\
			\textlink{MTSA003}{MTSA003TAB}{MTSA003} & Tempo medio del team di sviluppo per la risoluzione di errori& 0h-4h(TM) 0h-4h(TH) & 0h-2h(TM) 0h-2h(TH) \\
			\textlink{MTSA004}{MTSA004TAB}{MTSA004} & Efficienza della progettazione dei test & 0.5h-3h & 1h-2h \\
			%\textlink{MTSA005}{MTSA005TAB}{MTSA005} & Tempo medio per il testing dei bug fix & ? & ? \\
			\textlink{MTSA005}{MTSA005TAB}{MTSA005} & Contenimento dei difetti & 60\%-100\% & 80\%-100\% \\
			\multicolumn{4}{c}{\textbf{Per i test ad alto livello}}\\
			\textlink{MTSH001}{MTSH001TAB}{MTSH001} & Percentuale di difetti sistemati & 90\%-100\% & 98\%-100\% \\
			\textlink{MTSH002}{MTSH002TAB}{MTSH002} & Copertura dei test eseguiti & 80\%-100\% & 95\%-100\%  \\
			\textlink{MTSH003}{MTSH003TAB}{MTSH003} & Copertura dei requisiti & PRO-100\% & 100\%-100\% \\
			\textlink{MTSH004}{MTSH004TAB}{MTSH004} & Difetti per requisito & Variabile in base alla gravità del difetto & 0  \\
			%\textlink{MTSH005}{MTSH005TAB}{MTSH005} & Tasso di iniezione dei difetti & ? & ? \\			
		\end{tabu}
		\caption{Tabella delle metriche per i test}
		\vspace{-1em}
	\end{center}
\end{table}

\end{document}