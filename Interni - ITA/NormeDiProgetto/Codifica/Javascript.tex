\documentclass[../processi_primari.tex]{subfiles}

\begin{document}	
	\subsubsection{JavaScript}
	In questa sotto-sezione vengono elencate le norme tratte dal \textbf{Airbnb JavaScript Style Guide}\footnote{\href{https://github.com/airbnb/javascript}{https://github.com/airbnb/javascript}}.
	
	\paragraph*{Identazione 1: }
	Vanno utilizzati due (2) spazi per ogni livello di indentazione.
	
	\begin{minipage}{6cm}
		{\begin{center}SI\end{center}}
		\begin{Verbatim}[frame=single]
function() {
..let name;
}
		\end{Verbatim}
	\end{minipage}
	\hfill
	\begin{minipage}{6cm}
		{\begin{center}NO\end{center}}
		\begin{Verbatim}[frame=single]
function() {
....let name;
}
		\end{Verbatim}
	\end{minipage}
	
	\paragraph*{Identazione 2: }
	Bisogna mettere uno (1) spazio prima della graffa principale.
		
	\begin{minipage}{6cm}
		{\begin{center}SI\end{center}}
		\begin{Verbatim}[frame=single]
function() {
..let name;
}
		\end{Verbatim}
	\end{minipage}
	\hfill
	\begin{minipage}{6cm}
		{\begin{center}NO\end{center}}
		\begin{Verbatim}[frame=single]
function(){
..let name;
}
		\end{Verbatim}
	\end{minipage}	\paragraph*{Identazione 3: }
Va inserito uno (1) spazio prima della parentesi di apertura degli statement di controllo (`if`, `while` etc.). \\
Non va inserito alcun spazio prima della lista degli argomenti nelle chiamate di funzioni e nelle dichiarazioni.\\


\begin{minipage}{6cm}
{\begin{center}SI\end{center}}
\begin{Verbatim}[frame=single]
function fight() {
  console.log('353');
}
\end{Verbatim}
\end{minipage}
\hfill
\begin{minipage}{6cm}
{\begin{center}NO\end{center}}
\begin{Verbatim}[frame=single]
function fight () {
  console.log ('353');
}
\end{Verbatim}
\end{minipage}


\paragraph*{Spazi: }
E' vietato l'utilizzo di tabulazioni, che devono essere necessariamente sostituite da spazi. Al fine di assicurare il rispetto di questa regola si consiglia di configurare adeguatamente il proprio editor o IDE.
\paragraph*{Linee vuote: }
bisogona lasciare una riga vuota dopo blocchi e prima di un nuovo statement.

\begin{minipage}{6cm}
	{\begin{center}SI\end{center}}
	\begin{Verbatim}[frame=single]
if (foo) {
  return bar;
}

return baz;
	
//oppure 
	
var obj = {
  foo: function() {
  },
	
  bar: function() {
  }
};
\end{Verbatim}
\end{minipage}
\hfill
\begin{minipage}{6cm}
	{\begin{center}NO\end{center}}
	\begin{Verbatim}[frame=single]
if (foo) {
  return bar;
 }
return baz;
	
//oppure
	
var obj = {
  foo: function() {
  },
  bar: function() {
  }
};
return obj;
	
	\end{Verbatim}
\end{minipage}
\paragraph{Parentesizzazione 1:} I blocchi di codice multi-riga vanno racchiusi tra parentesi
graffe. I blocchi di codice con una sola riga possono essere scritti senza parentesi, però vanno scritti sulla stessa riga.\\
\\
\begin{minipage}{6cm}
{\begin{center}SI\end{center}}
\begin{Verbatim}[frame=single]
if (test) return false;

if (test) {
  return false;
}

function() {
  return false;
}
\end{Verbatim}
\end{minipage}
\hfill
\begin{minipage}{7cm}
	{\begin{center}NO\end{center}}
	\begin{Verbatim}[frame=single]
if (test)
  return false;
  
function() { return false; }




  
	\end{Verbatim}
\end{minipage}
\paragraph{Parentesizzazione 2:} Le parantesi graffe iniziano nella stessa riga del codice, non in quella sottostante.\\

\begin{minipage}{6cm}
	{\begin{center}SI\end{center}}
	\begin{Verbatim}[frame=single]
if (test) {
  \\..
  if(test2) {
  \\..
    if(test3) {
	  \\..
    }
  }
}

	\end{Verbatim}
\end{minipage}
\hfill
\begin{minipage}{6cm}
	{\begin{center}NO\end{center}}
	\begin{Verbatim}[frame=single]
if (test) 
{
  if(test2)
  {
    if(test3)
    {
      ...
    }
  }
}
	\end{Verbatim}
\end{minipage}	
\paragraph{Parentesizzazione 3:} In caso di blocchi multi-riga con if e else, bisogna mettere else nella stessa riga della parentesi graffa che chiude il blocco if.\\

\begin{minipage}{6cm}
	{\begin{center}SI\end{center}}
	\begin{Verbatim}[frame=single]
if (test) {
  thing1();
  thing2();
} else {
  thing3();
}
	
	\end{Verbatim}
\end{minipage}
\hfill
\begin{minipage}{6cm}
	{\begin{center}NO\end{center}}
	\begin{Verbatim}[frame=single]
if (test) {
  thing1();
  thing2();
}
  else {
  thing3();
}
	\end{Verbatim}
\end{minipage}

\paragraph{Commenti:}Utilizzare // per commenti da una riga. Inserire i commenti di questo tipo su una nuova riga sopra il soggetto del commento. Inoltre bisogna lasciare una riga vuota prima del commento.


\begin{minipage}{6cm}
	{\begin{center}SI\end{center}}
	\begin{Verbatim}[frame=single]
// comment
var active = true;

function getType() {
  console.log('353');

  // comment
  var type = this._type;

  return type;
}
	\end{Verbatim}
\end{minipage}
\hfill
\begin{minipage}{6.5cm}
	{\begin{center}NO\end{center}}
	\begin{Verbatim}[frame=single]
var active = true; // comment

function getType() {
  console.log('353');
  // comment
  var type = this._type;

  return type;
}


	\end{Verbatim}
\end{minipage}

\paragraph{Variabili 1:}Una variabile deve sempre essere dichiarata usando var o let. Altrimenti tutte le variabili dichiarate saranno variabili globali. Bisogna sempre evitare di inquinare lo spazio di nomi globale. 
\begin{center}{\begin{minipage}{9.5cm}
{\begin{center}SI\end{center}}
\begin{Verbatim}[frame=single]
superPower = new SuperPower();
\end{Verbatim}
\end{minipage}
\paragraph{}
\begin{minipage}{9.5cm}
{\begin{center}NO\end{center}}
\begin{Verbatim}[frame=single]
var superPower = new SuperPower();
\end{Verbatim}
\end{minipage}
}
\end{center}


\paragraph{Variabili 2:}Utilizzare una sola dichiarazione var o let per ogni variabile. È più facile aggiungere dichiarazioni di variabili in questa maniera, e non c'è il rischio di scambiare una ";" con una ",". Inoltre, le variabili non assegnate vanno sempre dichiarate per ultimo. \\


\begin{minipage}{6cm}
	{\begin{center}SI\end{center}}
	\begin{Verbatim}[frame=single]
var objects = getItems();
var cond = true;
var stranger = 'things';
var notAssign;
	\end{Verbatim}
\end{minipage}
\hfill
\begin{minipage}{6.5cm}
	{\begin{center}NO\end{center}}
	\begin{Verbatim}[frame=single]
var objects = getItems(),
	cond = true,
	notAssign,
	stranger = 'things';
	\end{Verbatim}
\end{minipage}
\end{document}


\paragraph{Variabili 3:}Le variabili vanno assegnate o dichiarate solo quando si ha la necessità di usarle. In modo da evitare operazioni inutili. \\


\begin{minipage}{6cm}
	{\begin{center}SI\end{center}}
	\begin{Verbatim}[frame=single]
function cName(hasName) {
  if (hasName === 'test') {
    return false;
  }

  const name = getName();

  if (name === 'test') {
    this.setName('');
    return false;
  }

  return name;
}
	\end{Verbatim}
\end{minipage}
\hfill
\begin{minipage}{6cm}
	{\begin{center}NO\end{center}}
	\begin{Verbatim}[frame=single]
function cName(hasName) {
  const name = getName();

  if (hasName === 'test') {
    return false;
  }

  if (name === 'test') {
    this.setName('');
    return false;
  }

  return name;
}	
\end{Verbatim}
\end{minipage}
\end{document}