\documentclass[GlossarioEST.tex]{subfiles}
\begin{document}
\chapter{Introduzione}
\section{Scopo del documento}
\glossExpl
\section{Scopo del prodotto}
Lo scopo del prodotto è quello di realizzare una piattaforma web chiamata \progetto che simuli le funzionalità di base per studenti, docenti e università di Uniweb. L'applicativo al posto del database dovrà utilizzare la rete Ethereum interagendo con degli smart contract.
\section{Riferimenti}
\subsection{Riferimenti Normativi}
\begin{itemize}
	\item \textbf{\ndp \vruno}
\end{itemize}

%TODO cercare di sostituire wikipedia coprendo tutti gli argomenti
\subsection{Riferimenti Informativi}
\begin{itemize}
\item \textbf {Sito web Ethereum Project (in lingua inglese)\\
	\nURI{https://www.ethereum.org}}
\item \textbf {Risorse riguardanti Bitcoin\\
	\nURI{https://bitcoin.org/it/risorse}}
\item \textbf {Capitolato d'appalto\\
	\nURI{http://www.math.unipd.it/~tullio/IS-1/2017/Progetto/C6.pdf}}
\item \textbf {Sito web Uniweb di Padova\\
	\nURI{https://uniweb.unipd.it}}
\item \textbf {Guida GitHub (in lingua inglese)\\
	\nURI{https://help.github.com}}

\end{itemize}
\end{document}