\documentclass[StudioDiFattibilità.tex]{subfiles}

\begin{document}
	\chapter{Capitolato C1 - Ajarvis}
	\section{Descrizione generale}
	Il capitolato C1 si pone l'obbiettivo di sviluppare un'applicazione con Machine Learning in grado di ascoltare standup giornalieri, analizzandone i dialoghi e fornendo un'analisi tramite dei report associati.\\
	L'applicativo permetterebbe di evidenziare dinamiche comuni ai progetti ed evitare la perdita di informazioni verbali.
	
	\section{Obiettivo finale}
	L'obiettivo è la realizzazione di un'applicazione in grado di registrare l'audio dello stand-up giornaliero permettendo, in tempo reale o in un secondo momento, di effettuare conversione speech-to-text (da voce a testo) tramite Google Cloud Speech API e analizzare il contenuto del dialogo testuale tramite Google Natural Language API.\\
	L'utente potrà quindi visualizzare attraverso interfaccia web i report dell'analisi effettuata. 
	
	\section{Tecnologie richieste}
	\begin{itemize}
		\item \textbf{Google Cloud Platform}, insieme di API per store dati e conversione speech-to-text oltre che algoritmi di machine learning;
		\item \textbf{Express.js}, framework per lo sviluppo di applicazione web in Node.js;
		\item \textbf{HTML5 e CSS3 con Twitter Bootstrap}, framework di sviluppo UI per siti web.
	\end{itemize}
	
	\section{Valutazione finale}
	L'utilizzo di tecnologie innovative, come la suite di intelligenza artificiale Google, ha portato ad una valutazione positiva del capitolato da parte del gruppo 353.\\
	La vastità di ambiti utilizzativi del prodotto finale ha inoltre generato molto interesse in tutti i componenti del gruppo, tuttavia, dopo aver analizzato maggiormente le specifiche si è scelto di scartare il capitolato perché la sua realizzazione è stata ritenuta troppo complessa e difficilmente l'esibito finale avrebbe rappresentato una soluzione migliore a quelle già esistenti.
	
\end{document}