\documentclass[AnalisiDeiRequisiti.tex]{subfiles}

\begin{document}

\chapter{Requisiti}
\section{Classificazione dei requisiti}
I requisiti vengono classificati ed assegnati loro un identificativo univoco secondo quanto definito nel documento \textit{Norme di progetto v1.0.0}.

% Please add the following required packages to your document preamble:
% \usepackage[table,xcdraw]{xcolor}
% If you use beamer only pass "xcolor=table" option, i.e. \documentclass[xcolor=table]{beamer}
% \usepackage{longtable}
% Note: It may be necessary to compile the document several times to get a multi-page table to line up properly
% Please add the following required packages to your document preamble:
% \usepackage[table,xcdraw]{xcolor}
% If you use beamer only pass "xcolor=table" option, i.e. \documentclass[xcolor=table]{beamer}
% \usepackage{longtable}
% Note: It may be necessary to compile the document several times to get a multi-page table to line up properly

\subsection{Requisiti funzionali}

\label{table:Tabella requisiti funzionali}
\begin{longtable}[H]{|l|l|l|l|}
	\hline
	\rowcolor[HTML]{38FFF8} 
	\textbf{Identificatore} & \textbf{Importanza} & \textbf{Descrizione} & \textbf{Fonti} \\ \hline
	\endhead
	 &  &  &  \\ \hline
	 &  &  &  \\ \hline
	 &  &  &  \\ \hline
	 &  &  &  \\ \hline
	 &  &  &  \\ \hline
	\caption{Tabella dei requisiti funzionali}
\end{longtable}

\subsection{Requisiti di qualità}

\label{table:Tabella requisiti di qualita'} %TODO: cercare un modo di inserire à nei label
\begin{longtable}[H]{|l|l|l|l|}
	\hline
	\rowcolor[HTML]{38FFF8} 
	\textbf{Identificatore} & \textbf{Importanza} & \textbf{Descrizione} & \textbf{Fonti} \\ \hline
	\endhead
	&  &  &  \\ \hline
	&  &  &  \\ \hline
	&  &  &  \\ \hline
	&  &  &  \\ \hline
	&  &  &  \\ \hline
	\caption{Tabella dei requisiti di qualità}
\end{longtable}

\subsection{Requisiti di vincolo}

\label{table:Tabella requisiti di vincolo}
\begin{longtable}[H]{|l|l|l|l|}
	\hline
	\rowcolor[HTML]{38FFF8} 
	\textbf{Identificatore} & \textbf{Importanza} & \textbf{Descrizione} & \textbf{Fonti} \\ \hline
	\endhead
	&  &  &  \\ \hline
	&  &  &  \\ \hline
	&  &  &  \\ \hline
	&  &  &  \\ \hline
	&  &  &  \\ \hline
	\caption{Tabella dei requisiti di vincolo}
\end{longtable}

\subsection{Requisiti prestazionali}

\label{table:Tabella requisiti prestazionali}
\begin{longtable}[H]{|l|l|l|l|}
	\hline
	\rowcolor[HTML]{38FFF8} 
	\textbf{Identificatore} & \textbf{Importanza} & \textbf{Descrizione} & \textbf{Fonti} \\ \hline
	\endhead
	&  &  &  \\ \hline
	&  &  &  \\ \hline
	&  &  &  \\ \hline
	&  &  &  \\ \hline
	&  &  &  \\ \hline
	\caption{Tabella dei requisiti prestazionali}
\end{longtable}

\section{Tracciamento}
\subsection{Tracciamento fonti-requisiti}

\label{table:Tabella di tracciamento fonti-requisiti}
\begin{longtable}[H]{|l|l|l|l|}
	\hline
	\rowcolor[HTML]{38FFF8} 
	\textbf{Fonte} & \textbf{Requisiti} \\ \hline
	\endhead
	&  \\ \hline
	&  \\ \hline
	&  \\ \hline
	&  \\ \hline
	&  \\ \hline
	\caption{Tabella di tracciamento fonti-requisiti}
\end{longtable}

\subsection{Tracciamento requisiti-fonti}

\label{table:Tabella di tracciamento requisiti-fonti}
\begin{longtable}[H]{|l|l|l|l|}
	\hline
	\rowcolor[HTML]{38FFF8} 
	\textbf{Requisito} & \textbf{Fonti} \\ \hline
	\endhead
	&  \\ \hline
	&  \\ \hline
	&  \\ \hline
	&  \\ \hline
	&  \\ \hline
	\caption{Tabella di tracciamento requisiti-fonti}
\end{longtable}

\section{Riepilogo}
%TODO: statistiche

\end{document}