\documentclass[ManualeSviluppatore.tex]{subfiles}

\begin{document}

\chapter{Redux}
\section{Overview}
Redux is a javaScripy library that allow us to manage an application's state. \\ 
Combined with React, it store the state in a single object that can't be modify directly. When the state changes, a new object is created. \\
Some benefits to this approach are:
\begin{itemize}
	\item Predictability of outcome;
	\item Maintainability;
	\item Ease of testing.
\end{itemize}

\section{Marvin's Redux organization} %TODO titolo molto provvisorio
	\subsection{Ducks}
	To keep our app as simple as possible, we decided to use a modular approach to describe all the logical components.\\
	With Redux, we opted to use Ducks.\\
	Ducks is a proposal for bundling reducers, action types and actions in a single file.
		\subsubsection{UML}
			\begin{figure}[h]
			\centering
			\includegraphics[width=1\linewidth]{"diagrammi/ducks"}
			\caption{Ducks components}
			\label{fig:Ducks components}
		\end{figure}

\end{document}