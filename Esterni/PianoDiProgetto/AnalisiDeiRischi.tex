\documentclass[PianoDiProgetto.tex]{subfiles}

\begin{document}

\chapter{Analisi dei rischi}

\section{Tabella dei rischi}
Di seguito è stata effettuata un’analisi approfondita dei rischi atta ad ottimizzare l’avanzamento del progetto.
Ogni rischio è classificato secondo la seguente convenzione:\\
\textbf{\centerline{R[Tipologia][Identificativo]}}
\begin{itemize}
	\item La prima lettera (R) è l'abbreviazione di rischio;
	\item Il secondo valore indica la \textbf{tipologia} di rischio. E può assumere il valore:
	\begin{itemize}
		\item \textbf{M:} indica rischi legati ai membri del gruppo \gruppo;
		\item \textbf{T:} indica rischi legati ai mezzi tecnologici;
		\item \textbf{O:} indica rischi legati all’organizzazione del lavoro;
		\item \textbf{R:} indica rischi legati ai requisiti.
	\end{itemize}
	\item L'\textbf{identificativo} indica invece un numero progressivo.
\end{itemize}

% per tabelle, alterna i colori delle righe		
\taburowcolors[2] 2{tableLineOne .. tableLineTwo}
\tabulinesep = ^3mm_2mm

\begin{longtabu} to \linewidth { 
		>{\centering}m{.17\linewidth} 
		>{\raggedright}m{.33\linewidth} 
		>{\raggedright}m{0.25\linewidth} 
		>{\centering}m{0.17\linewidth} 
	}
	\caption[Tabella descrittiva dell'analisi dei rischi]{Tabella descrittiva dell'analisi dei rischi}
	\endlastfoot
	\rowfont{\bfseries\sffamily\leavevmode\color{white}}
	\rowcolor{tableHeader}
	\textbf{Nome Codice} & \textbf{Descrizione} & \textbf{Rilevamento} & \textbf{Grado di rischio} \\
	\endhead
	
	\taburowcolors{tableLineOne..tableLineTwo}
	
	% RISCHI MEMBRI GRUPPO 353
	\rowcolor{tableLineOne} \textbf{Scarsa esperienza RM001}
	&
	{\small Nessun membro del gruppo ha mai lavorato su un progetto così impegnativo, ciò potrebbe produrre dei ritardi.} 
	& 
	{\small Ogni membro comunicherà al Responsabile eventuali difficoltà.}
	 & \shortstack{Occorrenza: \\ \textbf{alta} \vspace{0.6em}\\ Pericolosità: \\ \textbf{alta} }\\
	\rowcolor{tableLineTwo} Piano di contingenza: 
	&
	 \multicolumn{3}{m{0.8075\linewidth}}{\small  I compiti ad elevata difficoltà verranno affidati a membri con maggiore esperienza. }\\
	\hline	
		
	\rowcolor{tableLineOne} \textbf{Contrasti nel gruppo RM002}
	&
	 {\small Quest'anno i gruppi sono stati formati in modo casuale. Ogni membro si trova a fare il progetto con un gruppo di persone che non conosceva. Questo potrebbe portare a tensioni e conflitti.}
	 &
	 {\small Sarà il compito del Responsabile monitorare la collaborazione tra i membri.}
	&
	 \shortstack{Occorrenza: \\ \textbf{bassa} \vspace{0.6em}\\ Pericolosità: \\ 
		\textbf{media} } \\
		\rowcolor{tableLineTwo} Piano di contingenza: 
	&
	\multicolumn{3}{m{0.8075\linewidth}}{\small  I ruoli saranno ruotati per minimizzare i contatti tra i membri in 
		conflitto.  }\\
	\hline
	
	\rowcolor{tableLineOne} \textbf{Disponibilità dei membri RM003} 
	&
	{\small Alcuni membri del gruppo \gruppo sono anche lavoratori. Quindi il tempo dedicato al progetto da questi membri potrebbe essere limitato.}
	&
	{\small Ogni membro comunicherà in anticipo gli impegni che possono causare dei ritardi.}
	&
	 \shortstack{Occorrenza: \\ \textbf{media} \vspace{0.6em}\\ Pericolosità: 
		\\ \textbf{alta} }\\
		\rowcolor{tableLineTwo} Piano di contingenza:
	&
	\multicolumn{3}{m{0.8075\linewidth}}
	{\small Il carico di lavoro verrà ridistribuito tra i membri con maggiore 
		disponibilità.}\\
	\hhline{====}	
	
	% RISCHI MEZZI TECNOLOGICI
	\rowcolor{tableLineOne} \textbf{Tecnologie da utilizzare RT001}
	&
	{\small Le tecnologie da studiare sono molto recenti e la documentazione fornita è spesso molto limitata e/o poco approfondita. Il tempo di apprendimento per queste tecnologie potrebbe causare dei ritardi nello svolgimento dei lavori.}
	&
	{\small Il Responsabile dovrà monitorare la preparazione dei vari membri rispetto al compito che devono svolgere.}
	&
	 \shortstack{Occorrenza: \\ \textbf{media} \vspace{0.6em}\\  
		Pericolosità: \\ \textbf{alta} } \\
	\rowcolor{tableLineTwo} Piano di contingenza:
	&
	\multicolumn{3}{m{0.8075\linewidth}}{\small Saranno individuati dai Responsabili dei video tutorial e progetti da analizzare per comprendere le fondamenta di quella tecnologia. In casi gravi i membri con più esperienza e familiarità con quella tecnologia dovranno aiutare il membro in difficoltà, ridistribuendo il carico di lavoro.}\\
	\hline
	
	\rowcolor{tableLineOne} \textbf{Strumenti software RT002}
	&
	{\small Il gruppo si affida su software di terze parti e piattaforme online, eventuali disfunzioni potrebbero causare errori o perdite di dati.}
	&
	{\small Difficilmente sarà possibile rilevare il 
		problema poiché dipende da fattori esterni.}
	&
	 \shortstack{Occorrenza: \\ \textbf{bassa} 
		\vspace{0.6em}\\ Pericolosità: \\ \textbf{media} }\\
	\rowcolor{tableLineTwo} Piano di contingenza:
	&
	\multicolumn{3}{m{0.8075\linewidth}}{\small Si effettueranno backup bi-settimanali dei 
		dati \citGloss{Asana} e \citGloss{GitHub}, basandosi sull'affidabilità degli strumenti scelti. Potrà essere utilizzato anche Turbo\footnote{\nURI{https://turbo.net/}} in caso di malfunzionamenti per avere un'immediata disponibilità dei software da utilizzare.}\\
	\hline
	
	\rowcolor{tableLineOne} \textbf{Problemi hardware RT003}
	&
	{\small Ogni membro utilizza il computer personale per lavorare al progetto, guasti hardware potrebbero causare perdite di dati e/o di tempo.}
	&
	{\small Ogni membro dovrà avvisare il gruppo in caso di comportamenti anomali del proprio computer.}
	&
	\shortstack{Occorrenza: \\ \textbf{bassa} 
		\vspace{0.6em}\\ Pericolosità: \\ \textbf{medio-bassa} }\\
	\rowcolor{tableLineTwo} Piano di contingenza:
	&
	\multicolumn{3}{m{0.8075\linewidth}}{\small Ogni componente del gruppo dovrà effettuare il backup di tutti i file relativi al progetto, compresi i file di configurazione degli ambienti di sviluppo ed eventuali appunti personali.}\\
	\hline
	
	\rowcolor{tableLineOne} \textbf{Problemi di connessione RT004}
	&
	{\small Essendo la maggior parte dei membri del gruppo pendolari, spesso le riunioni di gruppo potranno essere effettuate tramite Skype. Il malfunzionamento della connessione Internet di uno dei membri potrebbe impedire la sua partecipazione diretta.}
	&
	{\small Sarà compito di ogni membro avvisare gli altri in caso di guasti o malfunzionamenti alla connessione Internet.}
	&
	\shortstack{Occorrenza: \\ \textbf{bassa} 
		\vspace{0.6em}\\ Pericolosità: \\ \textbf{media} }\\
	\rowcolor{tableLineTwo} Piano di contingenza:
	&
	\multicolumn{3}{m{0.8075\linewidth}}{\small Chi non riuscirà a partecipare alla conferenza Skype potrà interagire con gli altri membri utilizzando Slack. Nel caso non avesse disponibilità di connessione ad internet immediata potrà ascoltare una registrazione dell'incontro Skype fornito dagli altri membri.}\\
	\hline
	
	\rowcolor{tableLineOne} \textbf{Configurazione software RT005}
	&
	{\small Siccome il progetto richiede l'utilizzo di molte tecnologie, sarà compito degli amministratori di  predisporre e configurare correttamente l'ambite di sviluppo nei propri computer.}
	&
	{\small Sarà compito sempre degli amministratori di verificare che nel proprio pc sia eseguendo Linux che Windows l'ambiente sia correttamente configurato, in quanto, in caso contrario, si potrebbero anche produrre dei ritardi o codice non corretto.}
	&
	\shortstack{Occorrenza: \\ \textbf{bassa} 
		\vspace{0.6em}\\ Pericolosità: \\ \textbf{alta} }\\
	\rowcolor{tableLineTwo} Piano di contingenza:
	&
	\multicolumn{3}{m{0.8075\linewidth}}{\small Si dovrà utilizzare versioni sempre aggiornate di Windows e dei software.}\\
	\hhline{====}
	
	% RISCHI ORGANIZZAZIONE DEL LAVORO
	\rowcolor{tableLineOne} \textbf{Costi delle attività RO001}
	 &
	{\small La pianificazione prevede un costo per le attività. Essendo tutti membri del gruppo inesperti, è possibile che i tempi vengano calcolati in modo errato.}
	&
	{\small Il Responsabile verificherà periodicamente lo stato delle attività, in modo da evitare eventuali ritardi nello sviluppo delle attività.}
	&
	 \shortstack{Occorrenza: \\ \textbf{media} \vspace{0.6em}\\  
		Pericolosità: \\ \textbf{alta} }\\
	\rowcolor{tableLineTwo} Piano di contingenza:
	&
	\multicolumn{3}{m{0.8075\linewidth}}{\small Se il ritardo è ingente, il Responsabile dovrà ridistribuire il carico di lavoro fra gli altri membri avendo come obiettivo primario quello di non far slittare le milestone.}\\
	\hline
	
	\rowcolor{tableLineOne} \textbf{Rotazione dei ruoli RO002}
	&
	{\small La pianificazione prevede una corretta rotazione dei ruoli di progetto tra i componenti del gruppo durante tutta la durata del progetto.}
	&
	{\small Sarà compito del responsabile verificare periodicamente che la rotazione stia avvedendo correttamente e che nessuno abbia problemi a ricoprire ruoli mai ricoperti in precedenza.}
	&
	\shortstack{Occorrenza: \\ \textbf{media-bassa} \vspace{0.6em}\\  
		Pericolosità: \\ \textbf{molto alta} }\\
	\rowcolor{tableLineTwo} Piano di contingenza:
	&
	\multicolumn{3}{m{0.8075\linewidth}}{\small In caso che qualche componente del gruppo abbia delle difficoltà nel nuovo ruolo affidato, sarà suo compito chiedere al suo predecessore maggiori informazioni in merito.}\\
	\hhline{====}
	
	
	% RISCHI REQUISITI
	\rowcolor{tableLineOne} \textbf{Comprensione dei requisiti RR001}
	&
	{\small È possibile che alcuni requisiti siano male interpretati dal gruppo \gruppo, portando così alla realizzazione di un prodotto software non soddisfacente per la Proponente \Proponente.}
	&
	{\small È necessario lavorare a stretto contatto con la Proponente, e in caso di dubbio, contattarli il prima possibile tramite il canale Slack.}
	&
	\shortstack{ Occorrenza: \\ \textbf{media} \vspace{0.6em}\\ 
		Pericolosità: \\ \textbf{molto alta} } \\
	\rowcolor{tableLineTwo} Piano di contingenza:
	&
	\multicolumn{3}{m{0.8075\linewidth}}{\small I requisiti mal interpretati dovranno essere il prima possibile corretti, così da poter realizzare correttamente il prodotto software richiesto.}\\
	\hline
	
	\rowcolor{tableLineOne} \textbf{Modifica dei requisiti RR002}
	&
	{\small È possibile che la Proponente \Proponente decida di voler apportare delle modifiche. Questo porterebbe a una parziale/totale riscrittura dell'\adr con conseguente rivalutazione della pianificazione delle attività.}
	&
	{\small È necessario lavorare a stretto contatto con la Proponente in modo da rilevare subito eventuali modifiche ai \citGloss{requisiti}.}
	&
	 \shortstack{ Occorrenza: \\ \textbf{bassa} \vspace{0.6em}\\ 
		Pericolosità: \\ \textbf{molto alta} } \\
	\rowcolor{tableLineTwo} Piano di contingenza:
	&
	\multicolumn{3}{m{0.8075\linewidth}}{\small In caso di un cambiamento sostanzioso dei requisiti, il gruppo  discuterà tali cambiamenti con la Proponente in modo da trovare un punto di accordo.}\\
	\hhline{====}
		
\end{longtabu}

\end{document}