\documentclass[PianoDiProgetto.tex]{subfiles}

\begin{document}

\chapter{Specifica dei test}

\section{Scopo}

\section{Tipi di test}
Sono stati individuati quattro livelli di testing e sono rispettivamente:
\begin{itemize}
	
	\item Test di unità [TU]: con questa tipologia di test si cerca di verificare la più piccola parte di lavoro prodotta da un programmatore. Questo si traduce tendenzialmente a verificare i metodi e le funzioni scritte;
	
	\item Test di integrazione [TI]: con questa tipologia di test si cerca di verificare le componenti
	di sistema;
	
	\item Test di sistema [TS]: con questa tipologia di test si cerca di verificare che il comportamento e il funzionamento dell’architettura siano corretti;
	
	\item Test di validazione [TV]: con questa tipologia di test si vuole verificare che il lavoro prodotto soddisfi quanto richiesto dal proponente.

\end{itemize}

\section{Test di Validazione}
	\subsection{caratteristiche e organizzazione}
	%es. TV[tiporequisiti][id][importanza]
	\subsection{tabella test}
	%es. tabella di tutti i test TV con formato 3 colonne come parametri di un TV 
	
\section{Test di Sistema}
	
\section{Test di integrazione}

\section{Test di unità}	
	
\end{document}