\documentclass[PianoDiQualifica.tex]{subfiles}

\begin{document}

\chapter{Test di sistema}
A causa della componente Metamask, non accessibile mediante script o macro a causa di motivi di sicurezza, non è possibile effettuare test di sistema automatizzati riguardanti le funzionalità del prodotto. \\

\rowcolors{2}{CRighePari}{CRigheDispari}
\begin{longtable}[H]{P{2cm}P{8.5cm}P{3cm}}
	\rowcolor{CHeader} 
	\color{CHeaderText}\textbf{Codice} & 
	\color{CHeaderText}\textbf{Descrizione} & 
	\color{CHeaderText}\textbf{Stato}\\
	\endhead
	TS1 & Viene verificato che il bundle dei file avvenga correttamente sul sistema Ubuntu 17.04 & Superato \\ 
	TS2 & Viene verificato che il bundle dei file avvenga correttamente sul sistema Microsoft Windows 7 e 10 & Superato \\ 
	TS3 & Viene verificato che il bundle dei file avvenga correttamente sul sistema FreeBSD 11 & Superato \\ 
	TS4 & Viene verificato che i file JavaScript prodotti siano correttamente caricati ed inizializzati dal browser web Mozilla Firefox 52 & Superato \\ 
	TS5 & Viene verificato che i file JavaScript prodotti siano correttamente caricati ed inizializzati dal browser web Mozilla Firefox 61.0a1 & Superato \\ 
	TS6 & Viene verificato che i file JavaScript prodotti siano correttamente caricati ed inizializzati dal browser web Google Chrome 57 & Superato \\ 
	TS7 & Viene verificato che i file JavaScript prodotti siano correttamente caricati ed inizializzati dal browser web Google Chrome 68 & Superato \\ 
	TS8 & Viene verificato che i file JavaScript prodotti siano correttamente caricati sulla piattaforma Surge.sh & Superato \\ 
	\hiderowcolors
	\caption{Test di sistema}
\end{longtable}

\rowcolors{2}{CRighePari}{CRigheDispari}
\begin{longtable}[H]{P{6.5cm}P{7cm}}
	\rowcolor{CHeader} 
	\color{CHeaderText}\textbf{Codice} & 
	\color{CHeaderText}\textbf{Requisito} \\
	\endhead
	TS1 & R0V1.1 \\ 
	TS2 & R0V1.1 \\ 
	TS3 & R0V1.1 \\ 
	TS4 & R0V5 \\ 
	TS5 & R0V5 \\ 
	TS6 & R0V6 \\ 
	TS7 & R0V6 \\ 
	TS8 & R0V1.6 \\  
	\hiderowcolors
	\caption{Tracciamento test di sistema - requisito}
\end{longtable}

\chapter{Test di validazione}

Per i test di validazione, ogni scenario di setup presuppone che il server node di test sia avviato senza errori sulla porta 8080 e che il tester possegga diversi account metamask con crediti disponibili sulla blockchain fittizia.
Il tester inoltre possiede l'account che ha effettuato il deploy dei contratti sulla rete, avendo la possibilità di effettuare il login come utente Universitario.
La specifica degli Attori (utenti) e dei casi d'uso è descritta nel documento Analisi Dei Requisiti V4.0.0.

\rowcolors{2}{CRighePari}{CRigheDispari}
\begin{longtable}[H]{P{2cm}P{8.5cm}P{3cm}}
	\rowcolor{CHeader} 
	\color{CHeaderText}\textbf{Codice} & 
	\color{CHeaderText}\textbf{Descrizione} & 
	\color{CHeaderText}\textbf{Stato}\\
	\endhead
	TV0 & Visualizzare la home page & Superato \\
	TV1 & Visualizzazione pagina contenente la guida per usare Marvin & Superato \\
	TV2.1A & Redirect alla pagina home relativa al gestore dell'università & Superato \\
	TV2.1B & Redirect alla pagina home relativa all'amministratore & Superato \\
	TV2.1C & Redirect alla pagina home relativa al professore & Superato \\
	TV2.1D & Redirect alla pagina home relativa allo studente & Superato \\
	TV2.2 & Visualizzazione messaggio di errore, l'utente non possiede una chiave & Superato \\ 
	TV2.3 & Visualizzazione messaggio di errore, l'utente non possiede Metamask & Superato \\ 
	TV2.4 & Visualizzazione messaggio di errore, la chiave dell'utente non è registrata & Superato \\ 
	TV2.5 & Visualizzazione messaggio di errore, l'utente non è ancora confermato & Superato \\ 
	TV3A & L'utente viene inserito tra gli studenti in attesa di approvazione con i dati da lui indicati & Superato \\ 
	TV3B & L'utente viene inserito tra i professori in attesa di approvazione con i dati da lui indicati & Superato \\ 
	TV3.1 & Nel campo “name” è presente la scritta “Mario” & Superato \\ 
	TV3.2 & Nel campo “surname” è presente la scritta “Rossi” & Superato \\ 
	TV3.3A & Nel campo “role” è presente la scritta “teacher” & Superato \\ 
	TV3.3B & Nel campo “role” è presente la scritta “student” & Superato \\ 
	TV3.4 & Nel campo “course” è presente il nome e l'indirizzo del corso aggiunto in fase di setup & Superato \\ 
	TV3.5A & Visualizzazione dell'avviso di errore relativo a campo non compilato & Superato \\ 
	TV3.5B & Visualizzazione dell'avviso di errore relativo a campo non compilato & Superato \\ 
	TV3.6 & Visualizzazione nella finestra di Metamask dell'errore relativo ad una operazione invalida & Superato \\ 
	TV3.7A & Operazione di registrazione non portata a termine & Superato \\ 
	TV3.7B & Operazione di registrazione non portata a termine & Superato \\ 
	TV3.8 & Operazione di registrazione non portata a termine & Superato \\ 
	TV4 & Visualizzazione pagina che descrive come eseguire il logout da Marvin grazie a UI metamask & Superato \\ 
	TV5 & Visualizzazione della home page di un amministratore & Superato \\ 
	TV5.1 & Visualizzazione lista degli utenti approvati & Superato \\ 
	TV5.1.1 & Visualizzazione di un singolo utente approvato con indirizzo, nome, cognome & Superato \\ 
	TV5.1.1.1 & Visualizzazione dell'indirizzo di un singolo utente approvato & Superato \\ 
	TV5.1.1.2 & Visualizzazione del nome di un singolo utente approvato & Superato \\ 
	TV5.1.1.3 & Visualizzazione del cognome di un singolo utente approvato & Superato \\ 
	TV5.2 & Visualizzazione lista degli studenti approvati & Superato \\ 
	TV5.2.1 & Visualizzazione di un singolo studente approvato & Superato \\ 
	TV5.2.1.1 & Visualizzazione nome del corso di un singolo studente approvato & Superato \\ 
	TV5.3 & Visualizzazione lista degli professori approvati & Superato \\ 
	TV5.3.1 & Visualizzazione di un singolo professore approvato & Superato \\ 
	TV5.4 & Visualizzazione lista degli utenti in attesa di abilitazione & Superato \\ 
	TV5.4.1 & Visualizzazione di un singolo utente in attesa di abilitazione & Superato \\ 
	TV5.4.1.1 & Visualizzazione dell'indirizzo di un utente in attesa di abilitazione & Superato \\ 
	TV5.4.1.2 & Visualizzazione del nome di un utente in attesa di abilitazione & Superato \\ 
	TV5.4.1.3 & Visualizzazione del cognome di un utente in attesa di abilitazione & Superato \\ 
	TV5.4.1.4 & Visualizzazione del ruolo di un utente in attesa di abilitazione & Superato \\ 
	TV5.4.1.5 & Visualizzazione nome del corso di un utente in attesa di abilitazione & Superato \\ 
	TV5.5 & Un utente viene abilitato & Superato \\ 
	TV6.1 & Aggiunta di un anno accademico & Superato \\ 
	TV6.1.1 & Inserimento anno solare di riferimento & Superato \\ 
	TV6.1.2 & Visualizzazione errore anno accademico già presente & Superato \\ 
	TV6.1.3 & Visualizzazione errore anno accademico non compilato & Superato \\ 
	TV6.1.4 & Visualizzazione errore anno mal formato & Superato \\ 
	TV6.2 & Visualizzazione lista di tutti gli anni accademici presenti nel sistema & Superato \\ 
	TV6.2.1 & Visualizzazione lista di tutti gli anni solari relativi agli accademici presenti nel sistema & Superato \\ 
	TV6.3 & Eliminazione di un anno accademico, solamente se esso è privo di corsi & Superato \\ 
	TV7 & Visualizzazione della lista di corsi presenti nel sistema per permettere la loro gestione & Superato \\ 
	TV7.1 & Aggiunta di un corso di laurea relativo ad un anno accademico & Superato \\ 
	TV7.1.1 & Inserimento sigla anno accademico & Superato \\ 
	TV7.1.2 & Visualizzazione errore sigla mal formata & Superato \\ 
	TV7.2 & Visualizzazione della lista di corsi presenti nel sistema & Superato \\ 
	TV7.2.1 & Visualizzazione singolo corso & Superato \\ 
	TV7.2.1.1 & Nel campo “name” è presente il nome del corso & Superato \\ 
	TV7.2.1.2 & Nel campo “year” è presente il l'anno solare del corso & Superato \\ 
	TV7.3 & Visualizzazione lista dei corsi di laurea per anno accademico & Superato \\ 
	TV7.3.1 & Visualizzazione di un singolo corso della lista dei corsi filtrati per anno accademico & Superato \\ 
	TV7.3.1.1 & Nel campo “name” della lista è presente il nome del corso di interesse & Superato \\ 
	TV7.4 & Aggiunta di un esame ad un corso di laurea & Superato \\ 
	TV7.4.1 & Aggiunta di un esame ad un corso di laurea con nome “Analisi” & Superato \\ 
	TV7.4.2 & Aggiunta di un esame da 9 CFU ad un corso di laurea & Superato \\ 
	TV7.4.3 & Aggiunta di un esame obbligatorio ad un corso di laurea & Superato \\ 
	TV7.4.4 & Visualizzazione errore nome non valido & Superato \\ 
	TV7.4.5 & Visualizzazione errore numero crediti non valido & Superato \\ 
	TV7.5 & Visualizzazione lista esami in un corso & Superato \\ 
	TV7.5.1 & Visualizzazione singolo esame della lista esami in un corso & Superato \\ 
	TV7.5.1.1 & il campo “name” contiene il nome dell'esame di interesse & Superato \\ 
	TV7.5.1.2 & il campo “credits” contiene il numero di crediti dell'esame di interesse & Superato \\ 
	TV7.5.1.3 & il campo “teacherName” contiene il nome del docente (se presente) relativo all'esame di interesse & Superato \\ 
	TV7.5.1.4 & il campo “teacherSurname” contiene il cognome del docente (se presente) relativo all'esame di interesse & Superato \\ 
	TV7.6 & visualizzazione della lista di tutti gli esami presenti nel sistema & Superato \\ 
	TV7.6.1 & visualizzazione singolo esame della lista di tutti gli esami presenti nel sistema & Superato \\ 
	TV7.6.1.1 & il campo “name” della lista contiene il nome dell'esame di interesse & Superato \\ 
	TV7.6.1.2 & il campo “name” della lista contiene il nome dell'esame di interesse & Superato \\ 
	% TODO & Superato \\ 
	% TV 8 & Superato \\ 
	% TV 9 & Superato \\ 
	TV10 & Visualizzazione  della pagina per la gestione degli amministratori & Superato \\ 
	TV10.1 & Un amministratore viene aggiunto & Superato \\ 
	TV10.1.1 & Inserimento della chiave pubblica di un nuovo amministratore & Superato \\ 
	TV10.1.2 & Visualizzazione del messaggio di errore riguardo la chiave malformata & Superato \\ 
	TV10.1.3 & Visualizzazione del messaggio di errore riguardo la chiave già registrata & Superato \\ 
	TV10.2 & Visualizzazione della lista degli amministratori & Superato \\ 
	TV10.2.1 & Visualizzazione di un singolo amministratore & Superato \\ 
	TV10.2.1.1 & Visualizzazione dell'indirizzo di un amministratore & Superato \\ 
	TV10.3 & Rimozione di un amministratore & Superato \\ 
	TV11 & Visualizzazione lista delle azioni eseguibili su Marvin e il costo associato & Superato \\ 

	\hiderowcolors
	\caption{Test di validazione}
\end{longtable}

\rowcolors{2}{CRighePari}{CRigheDispari}
\begin{longtable}[H]{P{6.5cm}P{7cm}}
	\rowcolor{CHeader} 
	\color{CHeaderText}\textbf{Codice} & 
	\color{CHeaderText}\textbf{Requisito} \\
	\endhead
	TVnumerorequisito & Ciao \\ 
	\hiderowcolors
	\caption{Tracciamento test di validazione - requisito}
\end{longtable}

\end{document}