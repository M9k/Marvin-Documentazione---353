\documentclass[NormeDiProgetto.tex]{subfiles}

\begin{document}
	
	\chapter{Processi di supporto}
	
	\section{Processi di documentazione}
	\subsection{Descrizione}
	Questo capitolo descrive le scelte e i principi che sono stati scelti per la
	stesura, verifica e approvazione riguardante la documentazione ufficiale.
	Tali norme sono tassative per tutti i documenti formali.
	Tali documenti sono elencati successivamente nella sotto-sezione "Documenti correnti".
	
	\subsection{Ciclo di vita documentazione}
	Ogni documento formale deve passare gli stadi di "Sviluppo", "Verifica" e "Approvato".
	\begin{itemize}
		\item \textbf{Sviluppo:} inizia con la creazione del documento e termina concludendo scrittura. In questa fase i Redattori aggiungo le parti assegnate tramite task/ticket? %:TODO quale dei due?
		Il passaggio alla fase di Verifica è automatizzato col termine della checklist di azioni pre\textunderscore verifica;
		%:TODO oppure con approvazione del Responsabile?
		
		\item \textbf{Verifica:} il documento entra nella fase di verifica i Verificatori effettueranno le procedure di controllo dello stesso elencate sezione %:TODO 
		Al termine del controllo in caso positivo il documento entra automaticamente in fase di Approvato altrimenti consegnano i loro riscontri al Responsabile di Progetto che provvederà ad assegnare nuovamente il documento in fase di Sviluppo;
		
		\item \textbf{Approvato:} l'approvazione di un documento coincide con il superamento positivo da parte di un Verificatore dello stadio di Verifica e diventa una versione ufficiale.
	\end{itemize}
	
	\subsection{Separazione documenti interni ed esterni}
	Ogni documento formale dovrà essere classificato come documento Interno
	oppure Esterno con le seguenti differenze:
	\begin{itemize}
		\item \textbf{Interno:} ha utilizzo interno al gruppo 353, redatto in lingua Italiana;
		\item \textbf{Esterno:} verrà condiviso con la Proponente e i Committenti in lingua Inglese.
	\end{itemize}
	Un documento formale farà sempre parte di una delle due categorie elencate.
	
	\subsection{Nomenclatura documenti}
	Tutti i documenti formali tranne “...” seguiranno questo sistema di nomenclatura:
	Esempio: NomeDocumento\textunderscore vX.Y.Z
	
	\begin{itemize}
		\item\textbf{ NomeDocumento:} indica il nome del documento senza spazi;
		\item \textbf{vX.Y.Z:} indica il numero di versionamento con X,Y e Z numeri interi e non negativi.	
	\end{itemize}
	Di seguito la spiegazione che assumono le versione del documento:
	\begin{itemize}
		\item \textbf{X:} rappresenta il numero di pubblicazioni formali del documento, ogni qual volta il documento verrà pubblicato il valore di Y e Z verrà azzerato;
		\item \textbf{Y:} rappresenta il numero di verifiche superate positivamente, se negativamente il valore non va incrementato, qualora sia superata viene incrementato di uno e si azzera il valore della Z;
		\item \textbf{Z:} rappresenta il numero di modifiche effettuate effettuate al documento durante il suo sviluppo.	
	\end{itemize}
	
	Formato dei file: ogni documento si trova nel formato .text durante il suo ciclo di vita.
	Dopo lo stato di “Approvato” per il documento viene quindi creato un PDF contenente la versione approvata dal Responsabile.
	
	\subsection{Documenti correnti}
	Qui di seguito si presentano i documenti formali in ordine alfabetico e la loro classificazione tra Interno od Esterno:
	\begin{itemize}
		\item \textbf{Analisi dei Requisiti:} utilizzo Esterno, sigla(AR)
		Lo scopo del documento è di analizzare ed esporre i requisiti del progetto. Il documento conterrà l'analisi di tutti i casi d'uso relativi al prodotto in sviluppo e i diagrammi di interazione con l'utente previsti dal prodottofinale;
		
		\item \textbf{Glossario:}
		utilizzo Esterno ;
		
		\item \textbf{Norme di Progetto:}
		utilizzo Interno ;
		
		\item \textbf{Piano di Progetto:}
		utilizzo Esterno ;
		
		\item \textbf{Piano di Qualifica:}	utilizzo Esterno ;
		
		\item \textbf{Studio di Fattibilità:}	utilizzo Interno .
		
	\end{itemize}
	
	\subsection{Norme tipografiche}
	
	\subsection{Struttura documentazione}
	E' stato creato un template di documento riutilizzabile con tutti i documenti ufficiali per garantire facilità di sviluppo maggiore.
	
	\subsection{Glossario}
	Il glossario è un documento unico per tutti i documenti, esso conterrà tutte le definizioni, in ordine lessicografico
	crescente dei termini inerenti al tema del progetto o che possono essere fraintesi. I termini che dovranno essere
	inseriti nel glossario saranno contrassegnati da una \textit{G} pedice all’interno dei documenti. Prima di inserire un nuovo termine bisognerà assicurarsi che non sia già presente.
	Il comando LATEX da utilizzare per contrassegnare un termine da glossario all’interno dei documenti è gloss, mentre l’inserimento di una nuova parola all’interno del glossario è glossDef.
	La scelta di creare un comando apposito per un operazione “elementare” è scaturita dall’agevolazione che porta
	alla stesura della documentazione: avendo un modo univoco di riconoscere i termini all’interno del glossario, è
	possibile automatizzare il controllo delle parole da glossario all’interno dei documenti.
	
	
	\subsection{Strumenti a supporto della documentazione} %TODO sono da lasciare qui o da mettere su strumenti usati per la verifica nei PROCESSI DI VERIFICA?
	La stesura dei documenti deve essere effettuata utilizzando il linguaggio di markup LATEX utilizzando l'ambiente TeXstudio con dizionario italiano ed inglese installati.
	
	
	\section{Processi di versionamento}
	
	\subsection{Descrizione}
	Per le parti versionabili del progetto e per i documenti ufficiali si è scelto l'utilizzo della tecnologia Git, usando il servizio di hosting di repository di GitHub.
	La condivisione dei documenti informali e delle parti non versionabili è invece effettuata tramite l'uso di una cartella Google Drive condivisa.
	
	\subsection{Struttura delle repository}
	E' stata realizzata solo una repository durante la fase di RR ossia quella relativa alla documentazione come "Documentazione353", si prevede inoltre di creare ulteriori repository per suddividere lo sviluppo e la codifica dell'applicazione del progetto. \\\\
	I file interni al repository "Documentazione353" sono organizzati secondo questa struttura:
	\begin{itemize}
		\item \textbf{Esterni - ENG}
				\begin{itemize}
				\item \textbf{AnalisiDeiRequisiti}
				\item \textbf{Glossario}
				\item \textbf{PianoDiProgetto}
				\item \textbf{PianoDiQualifica}
			\end{itemize}
		\item \textbf{Esterni - ENG}
				\begin{itemize}
					\item \textbf{NormeDiProgetto}
					\item \textbf{Glossario}
					\item \textbf{StudioDiFattibilita}
					\item \textbf{Verbali}
				\end{itemize}		
	\end{itemize}	
	All'interno di ogni cartella è stato definito sempre un main.tex e il file diariomodifiche.tex per poi suddividere le sottosezioni del documento descritte nel file main.tex in ulteriori file .tex.\\
	Nella root della repository è stato messo invece un file .bat che controlla errori in tutti i main e se non ci sono errori compila tutti i documenti in formato pdf.
	
	\subsection{Ciclo di vita branch}
	Per sfruttare il parallelismo nello sviluppo di uno stesso documento sono stati creati appositamente dei branch denominati con il nome dei membri del gruppo, i documenti baseline invece saranno contenuti solamente nel master branch.\\
	Il merge col master avviene quindi solamente quando un documento si trova in stato di "Approvato".
	
	\subsection{Aggiornamento della repository}
	Per l’aggiornamento del repository è previsto il seguente sottoprocesso motivato dalla sezione precedente "ciclo di vita":
	\begin{itemize}
		\item verificare di trovarsi sul branch personale con "git branch"(quella selezionata ha presenta asterisco *) e in caso cambiare branch con "git checkout"
		\item dare il comando git pull. Nel caso in cui si verifichino dei conflitti:
		\begin{itemize}
			\item dare il comando "git stash" per accantonare momentaneamente	le modifiche apportate;
			\item dare il comando "git pull";
			\item dare il comando "git stash apply" per ripristinare le modifiche.
		\end{itemize}	In questo modo il repository locale risulta aggiornato rispetto il repository remoto;
	
		\item dare il comando "git add \textasteriskcentered" , che aggiungerà i file modificati e quelli nuovi;
		\item dare il comando "git commit", e successivamente riassumere le modifiche effettuate, in caso aggiungere messaggio esteso oltre al titolo;
		\item dare il comando "git push".
	\end{itemize}
	
	\section{Processo di verifica}
	\subsection{Analisi statica}
	\subsection{Analisi dinamica}
	\subsection{Metriche}
	
	\subsection{Verifica Documentazione}
	\subsection{Verifica Diagrammi Casi d'uso}
	\subsection{Strumenti usati per la verifica}
	
	
	
	
	Il comando IL (Indice di Leggibilita) ha tre argomenti, nell'ordine:
	1) numero lettere
	2) numero frasi
	3) numero parole
	\makeatletter
	\newcommand{\IL}[3]{{%
			\dimendef\lettere 256 \lettere=#1\p@ 
			\dimendef\frasi 258   \frasi  =#2\p@ 
			\dimendef\parole 260  \parole =#3\p@ 
			\dimendef\lp 262
			\dimendef\fr 264
			\fr=\dimexpr \p@*\frasi/\parole\relax 
			\lp=\dimexpr \lettere*\p@/\parole\relax 
			\strip@pt\dimexpr 89\p@ -10\lp +300\fr  \relax
	}}
	\section{Indice di leggibilit\`a}
	\IL{676}{287}{36}\\
	\IL{676}{287}{367}\\
	\IL{144}{1}{26}
	
	
	
\end{document}