\documentclass[ManualeSviluppatore]{subfiles}

\begin{document}

% riferimento: https://www.eclipse.org/sirius/doc/developer/Sirius%20Developer%20Manual.html

\chapter{Solidity}
\section{Architecture}
\subsection{Architecture Overview}
The main contract, University, consist in a hierarchy of small contracts, and each one manages a part of the overall features that must be offered altogether.\\
University contract uses two more contracts to define users: Teacher and Student. These two contracts share some functionality, because they are made from specialization of another contract, User, which contains the basic functionality for a user.\\
Administrators don't need a contract because they don't have to keep track of a list or complex information, the University only has to keep their addresses.\\
University keeps a list of academic years, represented by the contract "Year", who has a list of courses, represented by the contract "Course", which has a list of the exams, represented by "Exam".\\

The University contract is responsible to stabilize the rules, preventing unauthorized user to do actions not within their competence. This control is also indirectly done from the subsequent instantiating of the contracts, providing more performance and reducing the costs.\\

\subsection{Chebotko Diagram}
\subsubsection{Overview of data structure}
The organization of data in the Blockchain cannot be comparable to a relational database, because it's impossible to recover in a quick way the data and/or make a query on it.\\
The structure of the data can be compared, which opportune consideration, to some non-relational database, also called NoSQL, "Not Only SQL".\\

A database identified similar to our case is Cassandra, which is distributed with no single point of failure, with high availability and guarantees a fast recovery of data only if they are recoverable without join between tables, an operation that is not even supported.\\

\subsubsection{Overview of Chebotko Diagram}
To get a better understanding of the architecture and to get a consistent design a Chetbotko diagram was created.\\
This diagram was designed for CassandraDB, but it can be used, with some shrewdness, to the data saved using maps in the Ethereum contracts.\\
The goal of this diagram is to demonstrate which data can be accessed using certain keys.\\
A second goal is demonstrating that all needed data are accessible quickly starting from an information that leads to it and visually check that no data is inaccessible.\\
Every map has two arrows, one that enters from above and one that comes out from under: the first one indicates the key that can be used in those maps to get other data, the second one indicates the information that can be recovered from it and used on another map as keys.\\

In this diagram, the divisions of responsibilities are still ignored, as it will be the task of the next phase to detect the best subdivision.\\
The current focus is only the understanding of data provision and their connection.\\

A difference compared to the standard Chebotko diagram used on CassandraDB is the presence of containers, highlighted by different colors, which represent multiple instances of the same data schema.\\
An arrow entering a map belonging to a different container must therefore know, in addition to the information to be used a key for the map itself, the address to which that container is positioned, obtainable through other maps of the previous container.\\
The only exception of access is the university container, as the address must be available to access the system itself.\\

\subsubsection{Chebotko Diagram for registered addresses and admin}
\begin{figure}[H]
	\centering
	\includegraphics[width=0.7\linewidth]{"diagrammi/chebotko/registered_and_admin"}
	\caption{Chebotko diagram for registered addresses and admin}
	\label{fig:Chebotko diagram for registered addresses and admin}
\end{figure}
You can check if a public address is already used within the application using the map "registered",	contrariwise  it is not possible to obtain the list of registered addresses because it isn't a useful information for the application. \\
On the contrary, the administrators, besides being verifiable through their address, can be listed using a progressive index until their term, determined by a value contained in the contract itself. \\

\subsubsection{Chebotko Diagram for teachers}
\begin{figure}[H]
	\centering
	\includegraphics[width=1\linewidth]{"diagrammi/chebotko/teacher"}
	\caption{Chebotko diagram for teachers}
	\label{fig:Chebotko diagram for teachers}
\end{figure}
The teachers, in addition to being verifiable by their public address, also need to obtain the address of the contract that represent them in order to perform actions on the system.
To allow this there is a map that contains the addresses of the contracts associated with the public addresses. \\
As for the enumeration, it is useful to be able to retrieve the list of professors' contracts, and not their public addresses. \\
The entire structure is duplicated, as one must keep the teachers confirmed and the other one who makes the registration request. \\

\subsubsection{Chebotko Diagram for students}
\begin{figure}[H]
	\centering
	\includegraphics[width=1\linewidth]{"diagrammi/chebotko/student"}
	\caption{Chebotko diagram for students}
	\label{fig:Chebotko diagram for students}
\end{figure}
The structure of the students is exactly mirrored to that of the professors, for further information refer to the Chebotko Diagram for teachers.

\subsubsection{Chebotko Diagram for years and courses}
\begin{figure}[H]
	\centering
	\includegraphics[width=0.9\linewidth]{"diagrammi/chebotko/year_course"}
	\caption{Chebotko diagram for years and courses}
	\label{fig:Chebotko diagram for years and courses}
\end{figure}
The management of academic years within the university is mirrored to that of professors and students. \\
The contract for the year offers the list of courses it contains in the form of a course contract, which can be used to obtain a list of exam contracts. \\

\subsubsection{Chebotko Diagram for exams}
\begin{figure}[H]
	\centering
	\includegraphics[width=1\linewidth]{"diagrammi/chebotko/exam"}
	\caption{Chebotko diagram for exams}
	\label{fig:Chebotko diagram for exams}
\end{figure}
The exam contract, accessible from student, teacher and course contracts, represent a university exam and keeps the list of students registered for that exam. \\
Obviously the exam will also contain a reference to the responsible teacher and the course of affiliation, but this information can not be represented in this diagram. \\

\subsection{UML Diagram}

\subsubsection{University contract}
\begin{figure}[H]
	\centering
	\includegraphics[width=0.5\linewidth]{"diagrammi/solidity/university"}
	\caption{University contract}
	\label{fig:University contract}
\end{figure}
The University contract is decomposed into a hierarchy to guarantee a better manageability, but when it will be placed in the blockchain it will be a single one containing all the aspects. \\
The base contract, University, has the role of maintaining information about the founder and checking if an address already has a role within the contract, thus blocking double accounts. \\
The second one, UniversityAdmin, maintains the list of the administrator and allows its management by the founder. \\
UniversityStudent and UniversityTeacher are responsible for maintaining the list of users and professors and providing access to their contracts. \\
UniversityYear contains the list of years and links with related contracts and UniversityExam offers the method to associate exams with professors. \\
The hierarchy order is not significant, as some contracts can be reversed without causing problems and without compromising the logic. \\

\subsubsection{User contracts}
\begin{figure}[H]
	\centering
	\includegraphics[width=0.9\linewidth]{"diagrammi/solidity/user"}
	\caption{User contracts}
	\label{fig:User contracts}
\end{figure}
Unlike the university, in this case the hierarchy is used similarly to the object-oriented programming, and not only to compose a larger contract. \\
The User contract offers the basic characteristics of a user, containing his public key to verify his identity, his name and his surname. \\
The Student contract extends the basic contract by adding information about the course and the exams, also providing methods for their manipulation. \\
The Teacher contract, in the same way as the Student contract, extends the basic contract by adding the exam list pertaining to the professor. \\

\subsubsection{Year contract}
\begin{figure}[H]
	\centering
	\includegraphics[width=0.7\linewidth]{"diagrammi/solidity/year"}
	\caption{Year contract}
	\label{fig:Year contract}
\end{figure}
The Year contract keeps the information of the solar year and contains the related courses. \\

\subsubsection{Course contract}
\begin{figure}[H]
\centering
\includegraphics[width=0.7\linewidth]{"diagrammi/solidity/course"}
\caption{Course contract}
\label{fig:Course contract}
\end{figure}
The Course contract keeps the information about his details and contains the related exams. \\

\subsubsection{Exam contract}
\begin{figure}[H]
\centering
\includegraphics[width=0.7\linewidth]{"diagrammi/solidity/exam"}
\caption{Exam contract}
\label{fig:Exam contract}
\end{figure}
The Exam contract keeps the information about his details, the contract of the assigned professor and the list of student enrolled to him. \\

\newpage
\section{Extend the contracts}

\subsection{Provide custom user type}
New user types can be added simply extending the User contract, or it can be directly used if there is no need for extra information. \\

\subsection{Extend the university contract}
To extend the University contract is sufficient to create a new contract that inherits from the last one of the current hierarchy. \\
Once the new contract is added, open \texttt{migration} folder in the Marvin root, then open the migration for the university and change the artifacts.require parameter, inserting the name of the new contract. \\
Care must be taken about the size of the overall contract, because it may exceed the gas limit for the Ethereum network. In this case, the contract has to be separated into two, deployed together and then linked to each other. \\
The actual gas limit can be checked on \nURI{https://ethstats.net}, at the moment in which this document is being written it is about 8'000'000 gas and the University contract uses just over half of this limit. \\


\end{document}