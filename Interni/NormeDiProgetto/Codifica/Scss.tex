\documentclass[../ProcessiPrimari.tex]{subfiles}

\begin{document}
	
\subsubsection{SCSS}
In questa sotto-sezione vengono elencate le norme tratte dalla \textbf{Airbnb\\ CSS/Scss Styleguide}\footnote{\nURI{https://github.com/airbnb/css}}, il cui utilizzo viene supportato tramite plugin Webstorm scss-lint\footnote{\nURI{https://plugins.jetbrains.com/plugin/7530-scss-lint}} utilizzando la corretta configurazione\footnote{\nURI{https://github.com/airbnb/css/blob/master/.scss-lint.yml}} ed il tool scss-lint\footnote{\nURI{https://github.com/brigade/scss-lint}}.
\paragraph*{Formattazione:}
\begin{itemize}
	\item Utilizzare due (2) spazi per ogni livello di indentazione;
	\item Per i nomi delle classi utilizzare la \texttt{lower-case-with-dashes};
	\item Non usare selettori ID;
	\item In caso di selettori multipli per una singola regola, inserire ogni selettore in una linea singola;
	\item Lasciare uno spazio prima dell'apertura della parentesi graffa di una regola;
	\item Nelle proprietà lasciare uno spazio dopo e non prima i due punti;
	\item Chiudere la parentesi graffa di una regola in una nuova linea;
	\item Lasciare una linea vuota tra una dichiarazione e l'altra;
	\item Dichiarare le classi raggruppate per tipologia e in ordine alfabetico.
\end{itemize}
\begin{center}{
\begin{minipage}{6cm}
{\begin{center}SI\end{center}}
\begin{Verbatim}[frame=single]
.avatar {
  border-radius: 50%;
  border: 2px solid white;
}

.one,
.per-line,
.selector
 {
  // ...
}
\end{Verbatim}
\end{minipage}
\hfil
\begin{minipage}{7cm}
{\begin{center}NO\end{center}}
\begin{Verbatim}[frame=single]
.wrong {
  // ...}
.avatar{
  border-radius:50%;
  border:2px solid white; }
.no, .nope, .not_good {
  // ...
}
#lol-no {
  // ...
}

\end{Verbatim}
\end{minipage}
}
\end{center}
\paragraph*{BEM:}
utilizzare la notazione BEM (Block-Element-Modifier) come convenzione per i nomi delle classi.
\begin{center}{
\begin{minipage}{13.5cm}
\begin{Verbatim}[frame=single]
// ListingCard.jsx
function ListingCard() {
  return (
    <article class="listingcard listingcard-featured">

      <h1 class="listingcard-title">Adorable Mission</h1>

      <div class="listingcard-content">
        <p>Vestibulum id ligula porta euismod semper.</p>
      </div>

    </article>
  );
}
\end{Verbatim}
\end{minipage}
}
\end{center}
\begin{center}{
\begin{minipage}{13.5cm}
\begin{Verbatim}[frame=single]
/* listingcard.\citGloss{CSS} */
.listingcard { }
.listingcard-featured { }
.listingcard-title { }
.listingcard-content { }
\end{Verbatim}
\end{minipage}
}
\end{center}
In questo esempio:
\begin{itemize}
	\item \texttt{.listingcard} è il `\citGloss{blocco}` e rappresenta il livello più alto del componente;
	\item \texttt{.listingcard-title} è un `elemento` e rappresenta un discendente di \texttt{.listingcard};
	\item \texttt{.listingcard-featured} è un `modificatore` e rappresenta un diverso stato o una variante del blocco \texttt{.listingcard};
\end{itemize}
\paragraph*{Commenti:}
\begin{itemize}
	\item Non usare commenti di \citGloss{blocco};
	\item Fare commenti in linee a se stanti, non in linee contenenti anche codice.
\end{itemize}
\paragraph*{Ordine di dichiarazione delle proprietà:}
\begin{enumerate}
	\item \textbf{Proprietà standard:} vanno dichiarate per prime e in ordine alfabetico;
\begin{center}{
\begin{minipage}{5cm}
\begin{Verbatim}[frame=single]
.btn-green {
  background: green;
  font-weight: bold;
  // ...
}
\end{Verbatim}
\end{minipage}
}
\end{center}
	\item \textbf{Dichiarazioni di \texttt{@include}:} vanno raggruppate sotto alla dichiarazione delle proprietà standard
\begin{center}{
\begin{minipage}{10cm}
\begin{Verbatim}[frame=single]
.btn-green {
  background: green;
  font-weight: bold;
  @include transition(background 0.5s ease);
  // ...
}
\end{Verbatim}
\end{minipage}
}
\end{center}
\item \textbf{Selettori nidificati:} vanno dichiarati dopo tutte le altre dichiarazioni e deve essere lasciata una linea vuota prima della loro dichiarazione
\begin{center}{
\begin{minipage}{10cm}
\begin{Verbatim}[frame=single]
.btn {
  background: green;
  font-weight: bold;
  @include transition(background 0.5s ease);

  .icon {
    margin-right: 10px;
  }
}
\end{Verbatim}
\end{minipage}
}
\end{center}
\end{enumerate}
\paragraph*{Extend:}
la direttiva \texttt{@extend} non va utilizzata perché ha un comportamento poco intuitivo e pericoloso, specialmente quando usata in selettori nidificati.
\paragraph*{Selettori annidati: }
non utilizzare più di tre livelli di annidamento
\begin{center}{
\begin{minipage}{5cm}
\begin{Verbatim}[frame=single]
.page-container {
  .content {
    .profile {
      // STOP!
    }
  }
}
\end{Verbatim}
\end{minipage}
}
\end{center}

	
\end{document}