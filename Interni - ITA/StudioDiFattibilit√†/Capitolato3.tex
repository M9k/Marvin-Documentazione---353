\documentclass[main.tex]{subfiles}
\begin{document}
\chapter{Capitolato C3 - Despeect: \\interfaccia grafica per Speect}
\section{Descrizione generale}
Il capitolato C3 si concentra sulla realizzazione di un'interfaccia grafica per Speect, una libreria per la creazione di sistemi di sintesi vocale, con lo scopo di ispezionarne il funzionamento e facilitare la risoluzione di problemi attraverso lo sviluppo di un tool di debug. 
\section{Obiettivo finale}
L'obiettivo finale consiste nella creazione di un’ applicazione grafica che consenta il caricamento da file di grafi HRG (Heterogeneous Relation Graph), la creazione dello stesso a partire da un file voice.json e la creazione di un tool di debug che permetta di analizzare le varie fasi dell'elaborazione del file vocale a partire da un file di testo.
\section{Tecnologie richieste}
\begin{itemize}
	\item \textbf{Speect}, libreria per sistemi di sintesi vocale, nella versione modificata dal proponente.
	\item \textbf{Gtk+ o Qt} librerie per la creazione di interfacce grafiche.
	\item \textbf{Glade o QtCreator} frameworks per la realizzazione di interfacce grafiche.
\end{itemize}
\section{Valutazione finale}
Il gruppo 353 ha espresso pareri positivi riguardo l'ambito della sintetizzazione vocale, un campo ritenuto da tutti i membri del gruppo di notevole importanza nel campo medico, dei trasporti e della comunicazione in generale.
\'{E} stato ritenuto particolarmente stimolante lo sviluppo di un debugger.
Il capitolato non è stato scelto in quanto non presenta tecnologie realmente innovative che possano fornire nuove capacità e conoscenze.
\end{document}