\documentclass[NormeDiProgetto.tex]{subfiles}

\begin{document}

\chapter{Processi primari}

\section{Fornitura}
In questa sezione vengono tratta le norme che i membri del gruppo 353 sono tenuti a rispettare al fine di proporsi e diventare fornitori nei confronti della Proponente Red Babbel e dei Committenti Prof. Tullio Vardanega e Prof. Riccardo Cardin nell'ambito della progettazione, sviluppo e consegna del prodotto Marvin.
\subsection{Studio di fattibilità}
In seguito alla presentazione ufficiale dei \textbf{Capitolati d'appalto} avvenuta Venerdì 10 novembre 2017 alle ore 10.30 presso l'aula 1C150 di Torre Archimede è stata convocata una riunione interna al gruppo per discutere in merito alla varie proposte presentate. Una volta stabilita la scelta del capitolato per il quale proporsi come fornitori, gli analisti hanno condotto un'ulteriore e approfondita attività di analisi dei rischi e delle opportunità culminata con la stesura del documento \textit{Studio di Fattibilità v 1.0.0}. Tale documento include le motivazioni che hanno portato il gruppo 353 a proporsi come fornitore per il prodotto indicato e riporta per ciascun capitolato:
\begin{itemize}
	\item \textbf{descrizione generale:} una sintesi del prodotto da sviluppare secondo quanto stabilito dal capitolato d'appalto;
	\item \textbf{obbiettivo finale:} rappresenta il Dominio Applicativo, cioè l'ambito di utilizzo del prodotto da sviluppare;
	\item \textbf{tecnologie richieste:} rappresenta il Dominio Tecnologico richiesto dal capitolato, raggruppando le tecnologie da impiegare nello sviluppo del progetto;
	\item \textbf{valutazione finale:} racchiude le motivazioni, i rischi, le criticità evidenziate per le quali il capitolato in questione è stato respinto o accettato.
\end{itemize}
\subsection[Rapporti di fornitura con la Proponente Red Babel]{Rapporti di fornitura con la Proponente \\ Red Babel}
Durante l'intero progetto si intende instaurare con la Proponente Reb Babbel e con i referenti Alessandro Maccagnan e Milo Ertola un profondo e quanto più costante rapporto al fine di:
\begin{itemize}
	\item determinare bisogni;
	\item stabilire scelte volte alla realizzazione e realizzazione del prodotto (vincoli sui requisiti);
	\item stabilire scelte volte alla definizione ed esecuzione di processi (vincoli di progetto);
	\item stimare i costi;
	\item accordarsi circa la qualifica di prodotto.
\end{itemize}
\subsection{Documentazione fornita}
Di seguito vengono elencati i documenti forniti alla Proponente Red Babbel e ai Committenti Prof. Tullio Vardanega e Prof. Riccardo Cardin al fine di assicurare la massima trasparenza circa le attività di:
\begin{itemize}
	\item \textbf{pianifica, consegna e completamento:} descritte all'interno del \textit{Piano di Progetto v 1.0.0};
	\item \textbf{analisi:} analisi dei requisiti e dei casi d'uso del gruppo vengono descritte all'interno del documento \textit{Analisi dei Requisiti v 1.0.0};
	\item \textbf{verifica e validazione:} descritte all'interno del \textit{Piano di Qualifica v	1.0.0};
	\item \textbf{garanzia della qualità dei processi e di prodotto:} descritte e definite nel sopra citato \textit{Piano di Qualifica v 1.0.0}.
\end{itemize}
\section{Sviluppo}
\subsection{Analisi dei Requisiti}
\subsection{Codifica}
In questa sotto-sezione vengono elencate le norme alle quali i Programmatori devono attenersi durante l'attività di programmazione e implementazione.\\
All'inizio verrano elencate delle norme generali a cui i Programmatori devono attenersi con qualsiasi linguaggio di programmazione utilizzato, mentre di seguito verranno elencate delle norme specifiche per i linguaggi Javascript, React, Solidity e Scss.\\
Ogni norma è rappresentata da un paragrafo. Ciascuna ha un titolo, una breve descrizione e, se necessario, un esempio che illustra i modi accettati o meno. Alcune di esse includono inoltre una lista di possibili eccezioni d'uso. L'uso di norme e convenzioni è fondamentale per permettere la generazione di codice leggibile e uniforme, agevolare le fasi di manutenzione, verifica e validazione e migliorare la qualità del prodotto.

\paragraph*{Convenzioni per i nomi: }
\begin{itemize}
	\item nomi da evitare perché facilmente confondibili con i numeri \texttt{1} e \texttt{0}:
	\begin{itemize}
		\item \texttt{l} (lettera minuscola elle);
		\item \texttt{O} (lettera maiuscola o);
		\item \texttt{I} (lettera maiuscola i).
	\end{itemize}
	\item tutti i nomi devono essere \textbf{unici} ed \textbf{esplicativi} al fine di evitare al più possibile ambiguità e comprensione.
\end{itemize}
Per i vari linguaggi verranno successivamente descritte altre norme per nomi di variabili, funzioni e altro facendo riferimento ai seguenti stili:
\begin{itemize}
	\item \texttt{lowercase};
	\item \texttt{lower\_case\_with\_underscores}
	\item \texttt{UPPERCASE}
	\item \texttt{UPPER\_CASE\_WITH\_UNDERSCORES}
	\item \texttt{CapitalizedWords} (o \texttt{CapWords})
	\item \texttt{mixedCase}
	\item \texttt{Capitalized\_Words\_With\_Underscores}
	\item \texttt{lower-case-with-dashes}	
\end{itemize}

\paragraph*{Convenzioni per la documentazione: }
\begin{itemize}
	\item tutti i nomi e i commenti al codice per la documentazione vanno scritti in \textbf{inglese};
	\item è possibile utilizzare in un commento la keyword \textbf{TODO} per indicare codice temporaneo e soluzioni a breve termine o evidentemente migliorabili;
	\item ogni file contenente codice deve avere la seguente \textbf{intestazione} contenuta in un commento e posta all'inizio del file stesso:
\begin{center}{
\begin{minipage}{12cm}
\begin{Verbatim}[frame=single]
File : nome file
Version : versione file
Type : tipo file
Date : data di creazione
Author : nome autore /i
E- mail : email autore /i

License : tipo licenza

Advice : lista avvertenze e limitazioni

Changelog :
Autore || Data || breve descrizione delle modifiche
\end{Verbatim}
\end{minipage}
}
\end{center}
	\item La \textbf{versione} del codice viene inserita all’interno dell’intestazione del file e rispetta il
	seguente formalismo:
	\begin{center}{\textbf{X.Y}}\end{center}	
	\begin{itemize}
		\item \textbf{X}: è l’indice di versione principale, un incremento di tale indice rappresenta un avanzamento della versione stabile, che porta il valore dell’indice Y ad essere azzerato;
		\item \textbf{Y}: è l’indice di modifica parziale, un incremento di tale indice rappresenta una verifica o una modifica rilevante, come per esempio la rimozione o l’aggiunta di una istruzione.
	\end{itemize}
	La versione \textit{1.0} deve rappresentare la prima versione del file completo e stabile, cioè quando le sue funzionalità obbligatorie sono state definite e si considerano funzionanti. Solo dalla versione \textit{1.0} è possibile testare il file, con degli appositi test definiti, per	verificarne l’effettivo funzionamento.
\end{itemize}


\subfile{Codifica/Javascript.tex}
\subfile{Codifica/React.tex}
\subfile{Codifica/Solidity.tex}
\subfile{Codifica/Scss.tex}

\subsection{Progettazione}

\end{document}