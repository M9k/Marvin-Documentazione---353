\documentclass[ManualeUtente]{subfiles}

\begin{document}

\chapter{Start}

\section{Software Requirements}
\begin{itemize}
	\item \href{https://www.mozilla.org/en-US/firefox/new/}{Mozilla Firefox} (v.52 or more) or	
	\href{https://www.google.com/intl/en/chrome/}{Google Chrome} (v.57 or more);
	\item A MetaMask digital wallet.
\end{itemize}

\textbf{WARNING:} If you use a mobile device, like a smartphone or a tablet, you need to use Mozilla Firefox.

\section{Requirements for Windows}
\begin{itemize}
	\item {Operative systems:} Windows 7, 8 or 10, 32-bit or 64-bit;
	\item {Processor:} Pentium 4 or newer processor that supports SSE2;
	\item {RAM:} 512MB of RAM / 2GB of RAM for the 64-bit version;
	\item {Hard drive:} 200MB of hard drive space;
	\item {Internet connection:} required.
\end{itemize}

\section{Requirements for Mac}
\begin{itemize}
	\item {Operative systems:} macOS 10.9 or higher;
	\item {Processor:} Intel x86 processor;
	\item {RAM:} 512MB of RAM;
	\item {Hard drive:} 200MB of hard drive space;
	\item {Internet connection:} required.
\end{itemize}

\section{Requirements for Linux and BSD}
\begin{itemize}
	\item {GTK+:} 3.4 or higher;
	\item {GLib:} 2.22 or higher;
	\item {Pango:} 1.14 or higher;
	\item {X.Org:} 1.0 or higher (1.7 or higher is recommended);
	\item {libstdc++:} 4.6.1 or higher;
	\item {Processor:} Pentium 4 or newer processor that supports SSE2;
	\item {RAM:} 512MB of RAM;
	\item {Hard drive:} 200MB of hard drive space;
	\item {Internet connection:} required.
\end{itemize}


\section{Install MetaMask}
\begin{enumerate}
	\item To install the plugin to the Mozilla Firefox browser, go to the following 
	\href{https://addons.mozilla.org/en-US/firefox/addon/ether-metamask/}{link}
	 and click on the \textquotedblleft Add to Firefox" button. If you are Google Chrome browser user, you can add the plugin by visiting the  \href{https://chrome.google.com/webstore/detail/metamask/nkbihfbeogaeaoehlefnkodbefgpgknn}{link} and then clicking the  \textquotedblleft Add to Chrome" button;
	\begin{figure}[H]
		\centering
		\fcolorbox{black}{white}{\includegraphics[width=0.8\linewidth]{"image/MetamaskAddToFirefox1"}}
		\caption{Add to Firefox}
		\label{fig:Add to Firefox}
	\end{figure}
	\item It will open a pop-up window. Click on \textquotedblleft Add" button to confirm the installation.
	\begin{figure}[H]
		\centering
		\fcolorbox{black}{white}{\includegraphics[width=0.6\linewidth]{"image/MetamaskAddToFirefox2"}}
		\caption{Confirm the adding}
		\label{fig:Confirm the adding}
	\end{figure}
\end{enumerate}

\section{Create a MetaMask digital wallet}
After the installation of Metamask accept the Privacy Policy and agree to the license agreements. Now let's see how to create a digital wallet in Metamask:
\begin{enumerate}	
	\item After you accept the license agreements, it will open the following plugin-in page. You have set up a new password for security reasons. This password will be used to unlock your account in future;
	\begin{figure}[H]
		\centering
		\fcolorbox{black}{white}{\includegraphics[width=0.5\linewidth]{"image/MetamaskCreate"}}
		\caption{Set up a new password}
		\label{fig:Set up a new password}
	\end{figure}
	\textbf{NOTE:} for security purposes, MetaMask will periodically request that you log back into your account using the password;
	\item Copy the 12 seed words and file them away somewhere safe (this helps to secure and restore your account).
	\begin{figure}[H]
		\centering
		\fcolorbox{black}{white}{\includegraphics[width=0.6\linewidth]{"image/MetamaskCreate2"}}
		\caption{Copy the 12 seed words somewhere safe}
		\label{fig:Copy the 12 seed words somewhere safe}
	\end{figure}
\end{enumerate}
\newpage
\section{Log-in}
\begin{enumerate}
	\item Unlock the MetaMask digital wallet as described on \S 2.9;
	\item Click "Login" in the homepage;
	\begin{figure}[H]
		\centering
		\fcolorbox{black}{white}{\includegraphics[width=1\linewidth]{"image/login"}}
		\caption{Click "login"}
		\label{fig:Click "login"}
	\end{figure}
\end{enumerate}

\section{Log-out}
To log-out from the Marvin you have to lock Metamask or switch to another account.
\begin{enumerate}
	\item Click the MetaMask browser extension icon;
	\item Click the menu icon;
	\begin{figure}[H]
		\centering
		\fcolorbox{black}{white}{\includegraphics[width=0.6\linewidth]{"image/Logout1"}}
		\caption{Click the menu icon}
		\label{fig:Click the menu icon}
	\end{figure}
	\item Click \textquotedblleft Log Out".
	\begin{figure}[H]
		\centering
		\fcolorbox{black}{white}{\includegraphics[width=0.6\linewidth]{"image/Logout2"}}
		\caption{Click "Log Out"}
		\label{fig:Click "Log Out"}
	\end{figure}
\end{enumerate}

\section{Unlock the MetaMask digital wallet}
Every time the browser is closed, Metamask will lock, so there will be a need to unlock it before using Marvin.
\begin{enumerate}
	\item Click the MetaMask browser extension icon;
	\item Insert the password used during the creation of the wallet;
	\item Click \textquotedblleft UNLOCK".
	\begin{figure}[H]
		\centering
		\fcolorbox{black}{white}{\includegraphics[width=0.6\linewidth]{"image/MetamaskUnlock"}}
		\caption{Unlock the wallet}
		\label{fig:Unlock the wallet}
	\end{figure}
\end{enumerate}

\section{Lock the MetaMask digital wallet}
For security reasons, it is recommended to lock Metamask before leaving the computer. \\
To do this, perform the following steps:
\begin{enumerate}
	\item Click the MetaMask browser extension icon;
	\item Click on the menu at the top right;
	\item Click \textquotedblleft Log Out".
	\begin{figure}[H]
		\centering
		\fcolorbox{black}{white}{\includegraphics[width=0.6\linewidth]{"image/MetamaskLock"}}
		\caption{Lock the wallet}
		\label{fig:Lock the wallet}
	\end{figure}
\end{enumerate}

\section{Restore a MetaMask digital wallet}
If you want to use your wallet on another computer or you simply want to restore your account in case you lose your password, you can follow these steps:
\begin{enumerate}
	\item Click on the MetaMask icon in your browser and then click \textquotedblleft Restore from seed phrase";
	\item Copy and paste the 12 seed words in the box and then set a new password;
	\item And then click \textquotedblleft Ok" button.
\end{enumerate}

\section{Connect to the application}
%TODO settare Ropsten Test Network?
Visit the page \nURI{http://marvin.surge.sh/} to use the application.

\section{Transactions}
All the reading operations on the blockchain are free but when we are writing something on the blockchain we have to pay with cryptocurrency, these operations are called transitions.  
Every transition must interact with metamask, by performing the following procedure:
\begin{enumerate}
	\item Wait for the Metamask window to open, if it does not open automatically, click on the Metamask logo on the browser bar to open it;
	\item It will show you all the information(costs etc) and ask you to confirm or reject the transition by clicking the correct button.
	\begin{figure}[H]
		\centering
		\fcolorbox{black}{white}{\includegraphics[width=0.6\linewidth]{"image/MetamaskPay"}}
		\caption{Pay a transaction clicking on "SUBMIT"}
		\label{fig:Pay a transaction clicking on "SUBMIT"}
	\end{figure}
\end{enumerate}

\newpage
\section{The prices page}
 In the price page, you can find a list of all operations of Marvin, and the relative cost in gas, ether or us dollar of each operation. We must recommend you to visit this page before doing any operation. You can open this page by clicking on the link \textquotedblleft Price" on the menu bar or the button in the homepage;
	\begin{figure}[H]
		\centering
		\fcolorbox{black}{white}{\includegraphics[width=0.8\linewidth]{"image/pricePage"}}
		\caption{Click "Price" to view the prices page}
		\label{fig:Click "Price" to view the prices page}
	\end{figure}


\section{The help page}
 This page has all the information about how to use Metamask for new users. You can open this page by clicking on the link \textquotedblleft Help" on the menu bar;
	\begin{figure}[H]
		\centering
		\fcolorbox{black}{white}{\includegraphics[width=0.8\linewidth]{"image/helpPage"}}
		\caption{Click "Help" to view the help page}
		\label{fig:Click "Help" to view the help page}
	\end{figure}

\newpage
\section{The license page}
This page contains the license. Click on the \textquotedblleft MIT License" link present on the footer of the application;
	\begin{figure}[H]
		\centering
		\fcolorbox{black}{white}{\includegraphics[width=0.8\linewidth]{"image/licensePage"}}
		\caption{Click "MIT License" to view the license}
		\label{fig:Click "MIT License" to view the license}
	\end{figure}


\end{document}