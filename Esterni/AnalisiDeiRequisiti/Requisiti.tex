\documentclass[AnalisiDeiRequisiti.tex]{subfiles}

\begin{document}

\definecolor{CHeader}{HTML}{D22E2E}
\definecolor{CHeaderText}{HTML}{FFFFFF}
\definecolor{CRighePari}{HTML}{DFDFDF}
\definecolor{CRigheDispari}{HTML}{F4F4F4}
\definecolor{CCaption}{HTML}{FFFFFF}

\chapter{Requisiti}
\section{Classificazione dei requisiti}
I \citGloss{requisiti} vengono classificati ed assegnati a un identificativo univoco secondo quanto definito nel documento \textit{Norme di progetto \vrdue}.

\subsection{Requisiti funzionali}

\label{table:Tabella requisiti funzionali}
\rowcolors{2}{CRighePari}{CRigheDispari}
\renewcommand*{\arraystretch}{1.2}
\begin{longtable}[H]{p{2.6cm}p{2.5cm}p{5cm}p{2cm}}
	\rowcolor{CHeader} 
	\color{CHeaderText} \textbf{Identificatore} & \color{CHeaderText} \textbf{Importanza} & \color{CHeaderText} \textbf{Descrizione} & \color{CHeaderText} \textbf{Fonti} \\
	\endhead
	R0F1 & Obbligatorio & L'amministratore può gestire gli utenti & Interno \\
	R0F1.1 & Obbligatorio & L'amministratore può ottenere una lista di tutti gli utenti non abilitati & Interno \\
	R0F1.1.1 & Obbligatorio & L'amministratore può ottenere l'indirizzo di tutti gli utenti non abilitati & Interno \\
	R0F1.1.2 & Obbligatorio & L'amministratore può ottenere il nome di tutti gli utenti non abilitati & Interno \\
	R0F1.1.3 & Obbligatorio & L'amministratore può ottenere il cognome di tutti gli utenti non abilitati & Interno \\
	R0F1.1.4 & Obbligatorio & L'amministratore può ottenere il ruolo desiderato di tutti gli utenti non abilitati & Interno \\
	R0F1.1.5 & Obbligatorio & L'amministratore può ottenere il corso desiderato, ove possibile, di tutti gli utenti non abilitati & Interno \\
	R0F1.2 & Obbligatorio & L'amministratore può abilitare un utente & Interno \\
	R0F1.3 & Obbligatorio & L'amministratore può rimuovere un utente & Interno \\
	R0F1.4 & Obbligatorio & L'amministratore può ottenere una lista di tutti gli studenti & Interno \\
	R0F1.4.1 & Obbligatorio & L'amministratore può ottenere l'indirizzo di tutti studenti & Interno \\
	R0F1.4.2 & Obbligatorio & L'amministratore può ottenere il nome di tutti gli studenti & Interno \\
	R0F1.4.3 & Obbligatorio & L'amministratore può ottenere il cognome di tutti gli studenti & Interno \\
	R0F1.4.4 & Obbligatorio & L'amministratore può ottenere il corso di tutti gli studenti & Interno \\
	R0F2 & Obbligatorio & L'amministratore può gestire i professori & Interno \\
	R0F2.1 & Obbligatorio & L'amministratore può ottenere una lista di tutti i professori & Interno \\
	R0F2.1.1 & Obbligatorio & L'amministratore può ottenere l'indirizzo di tutti i professori & Interno \\
	R0F2.1.2 & Obbligatorio & L'amministratore può ottenere il nome di tutti i professori & Interno \\
	R0F2.1.3 & Obbligatorio & L'amministratore può ottenere il cognome di tutti i professori & Interno \\
	R0F2.2 & Obbligatorio & L'amministratore può assegnare un esame ad un determinato professore & Capitolato \\
	R0F3 & Obbligatorio & L'amministratore può gestire i corsi di laurea relativi agli anni accademici & Capitolato \\
	R0F3.1 & Obbligatorio & L'amministratore può creare un nuovo \citGloss{corso di laurea} a partire da un \citGloss{anno accademico} & Capitolato \\ 
	R0F3.1.1 & Obbligatorio & L'amministratore deve indicare un codice identificativo per ogni \citGloss{corso di laurea} & Interno \\
	R0F3.2 & Obbligatorio & L'amministratore può ottenere una lista di tutti i corsi di laurea & Interno \\
	R0F3.2.1 & Obbligatorio & L'amministratore può ottenere una lista di tutti i corsi di laurea di un determinato \citGloss{anno accademico} & Interno \\
	R0F3.2.2 & Obbligatorio & Ogni corso di laurea deve essere visualizzato tramite la sua sigla  & Interno \\
	R0F3.2.3 & Obbligatorio & Ogni corso di laurea mostrato deve essere accompagnato dall'anno di riferimento ove non indicato dall'utente stesso & Interno \\
	R0F3.3 & Obbligatorio & L'amministratore può creare nuovi esami & Capitolato \\ 
	R0F3.4 & Obbligatorio & L'amministratore può ottenere una lista di tutti gli esami dato il \citGloss{corso di laurea} & Capitolato \\
	R0F3.4.1 & Obbligatorio & Per ogni esame l'amministratore deve visualizzarne la sigla & Interno \\
	R0F3.4.2 & Obbligatorio & Per ogni esame l'amministratore deve visualizzarne il numero di crediti & Interno \\
	R0F3.4.3 & Obbligatorio & L'amministratore può ottenere il nome ed il cognome di quale professore è assegnato o sta per essere assegnato ad un determinato esame & Interno \\
	R0F3.4.4  & Obbligatorio & L'amministratore può associare ad un esame il relativo professore & Interno \\
	R0F3.4.4.1  & Obbligatorio & L'amministratore deve poter visualizzare l'indirizzo di un professore prima di assegnarlo ad un determinato esame per evitare problemi in caso di omonimi & Interno \\
	R0F3.4.5 & Obbligatorio & L'amministratore può ottenere la sigla del corso di laurea al quale appartiene un determinato esame ove non indicato dall'amministratore stesso oppure se richiesto & Interno \\
	R0F3.4.6 & Obbligatorio & L'amministratore può ottenere il numero degli studenti iscritti a un determinato corso & Interno \\
	R0F4 & Obbligatorio & Un professore può gestire gli esami a lui assegnati & Capitolato \\
	R0F4.1 & Obbligatorio & Un professore può ottenere la lista di tutti gli esami a lui assegnati & Capitolato \\
	R0F4.1.1 & Obbligatorio & Un professore può ottenere il codice di tutti gli esami a lui assegnati & Capitolato \\
	R0F4.1.2 & Obbligatorio & Un professore può ottenere il relativo corso di laurea di tutti gli esami a lui assegnati & Capitolato \\
	R0F4.2 & Obbligatorio & Un professore può ottenere la lista di tutti gli studenti associati a un determinato esame & Capitolato \\
	R0F4.2.1 & Obbligatorio & Un professore può ottenere l'indirizzi di tutti studenti associati a un determinato esame & Interno \\
	R0F4.2.2 & Obbligatorio & Un professore può ottenere il nome di tutti gli studenti associati a un determinato esame & Interno \\
	R0F4.2.3 & Obbligatorio & Un professore può ottenere il cognome di tutti gli studenti associati a un determinato esame & Interno \\
	R0F4.3 & Obbligatorio & Un professore può registrare un esito ad un dato esame ad un dato studente registrato a quell'esame & Capitolato \\
	R0F5 & Obbligatorio & Uno studente può gestire l'iscrizione agli esami & Capitolato \\
	R0F5.1 & Obbligatorio & Uno studente può vedere l'elenco degli esami ai quali è iscritto & Capitolato \\
	R0F5.1.1 & Obbligatorio & Uno studente può ottenere il numero di crediti assegnati agli esame & Interno \\
	R0F5.1.2 & Obbligatorio & Uno studente può ottenere l'obbligatorietà degli esame & Interno \\
	R0F5.1.3 & Obbligatorio & Uno studente può ottenere lo stato di superamento degli esame & Interno \\
	R0F5.1.4 & Obbligatorio & Uno studente può ottenere il totale dei suoi crediti & Interno \\
	R0F5.1.5 & Obbligatorio & Uno studente può ottenere il numero di crediti da raggiungere per la laurea & Interno \\
	R0F5.2 & Obbligatorio & Uno studente può ottenere l'elenco degli esami opzionali & Capitolato \\
	R0F5.2.1 & Obbligatorio & Uno studente può ottenere il numero di crediti degli esami opzionali & Capitolato \\
	R0F5.2.2 & Obbligatorio & Uno studente può iscriversi ad un esame opzionale & Capitolato \\
	R0F6 & Obbligatorio & L'utente può effettuare il login & Interno \\
	R0F6.1 & Obbligatorio & Il login deve avvenire tramite il controllo delle chiavi, senza ulteriori azioni da parte dell'utente & Interno \\
	R0F7 & Obbligatorio & L'utente può effettuare il logout & Interno \\
	R0F8 & Obbligatorio & L'utente non ancora registrato può registrarsi & Capitolato \\
	R0F8.1 & Obbligatorio & La registrazione necessita di nome, cognome, categoria (studente o professore) e selezione del corso di laura se si tratta di uno studente & Capitolato \\
	R1F9 & Desiderabile & L'utente può leggere una breve guida sull'uso di MetaMask e sul pagamento delle operazioni & Interno \\	
	R0F10 & Obbligatorio & L'utente deve poter vedere preventivamente il costo in \citGloss{Gas}, \citGloss{Ether} e Euro dell'operazione che sta per eseguire & Capitolato, VER-2018-01-09 \\	
	R0F11 & Obbligatorio & L'università deve poter gestire gli amministratori & VER-2017-12-08.1 \\
	R0F11.1 & Obbligatorio & L'università deve poter aggiungere amministratori attraverso l'inserimento del loro indirizzo & VER-2017-12-08.1 \\
	R0F11.2 & Obbligatorio & L'università deve poter rimuovere amministratori & VER-2017-12-08.2 \\
	R0F11.3 & Obbligatorio & L'università deve poter ottenere la lista di tutti gli amministratori & Interno \\
	R0F11.3.1 & Obbligatorio & L'università deve poter visualizzare l'indirizzo di ogni amministratore & Interno \\
	R0F12 & Obbligatorio & L'università può gestire gli anni accademici & Capitolato \\
	R0F12.1 & Obbligatorio & L'università può aggiungere un \citGloss{anno accademico} & Capitolato \\
	R0F12.2 & Obbligatorio & L'università può rimuovere un \citGloss{anno accademico} & Capitolato \\
	R0F12.3 & Obbligatorio & L'università e gli amministratori possono ottenere una lista degli anni accademici, con indicato l'anno solare & Interno \\
	\hiderowcolors
	\caption{Tabella dei requisiti funzionali}
\end{longtable}

\subsection{Requisiti di qualità}

\label{table:Tabella requisiti di qualita'}
\rowcolors{2}{CRighePari}{CRigheDispari}
\renewcommand*{\arraystretch}{1.2}
\begin{longtable}[H]{p{2.5cm}p{2.5cm}p{5cm}p{2cm}}
	\rowcolor{CHeader} 
	\color{CHeaderText} \textbf{Identificatore} & \color{CHeaderText} \textbf{Importanza} & \color{CHeaderText} \textbf{Descrizione} & \color{CHeaderText} \textbf{Fonti} \\
	\endhead
	R0Q1 & Obbligatorio & La progettazione e il codice devono seguire le norme e le metriche riportate nel documento allegato \pdq \vruno & Interno \\
	R0Q2 & Obbligatorio & L'approccio di scrittura di \citGloss{JavaScript} deve essere promise Centric Approach & Capitolato \\
	R0Q2.1 & Obbligatorio & L'applicativo non deve fare uso di callback in presenza di alternative alle ultime & VER-2017-11-22.1 \\
	R0Q3 & Obbligatorio & Il codice \citGloss{JavaScript} deve attenersi al \citGloss{airbnb} \citGloss{JavaScript} style guide & Capitolato \\
	R0Q4 & Obbligatorio & Lo sviluppo deve essere supportato dall'utilizzo del tool ESLint & Capitolato \\
	R0Q5 & Obbligatorio & Dovrà essere fornito un manuale utente in lingua inglese che tratterà l'uso da parte di studenti e professori & VER-2017-11-22.2 \\
	R0Q6 & Obbligatorio & Dovrà essere fornito un manuale di deploy e di utilizzo da parte degli amministratori in lingua inglese & VER-2017-11-22.3 \\
	R0Q7 & Obbligatorio & Il codice sorgente deve essere pubblicato sulla piattaforma \citGloss{GitHub}, BitBucket o GitLab & Capitolato \\
	R0Q8 & Obbligatorio & Il codice deve attenersi il più possibile alle guide linea de "App a 12 Fattori" & Capitolato \\ 
	\hiderowcolors
	\caption{Tabella dei requisiti di qualità}
\end{longtable}

\subsection{Requisiti di vincolo}

\label{table:Tabella requisiti di vincolo}
\rowcolors{2}{CRighePari}{CRigheDispari}
\renewcommand*{\arraystretch}{1.2}
\begin{longtable}[H]{p{2.5cm}p{2.5cm}p{5cm}p{2cm}}
	\rowcolor{CHeader} 
	\color{CHeaderText} \textbf{Identificatore} & \color{CHeaderText} \textbf{Importanza} & \color{CHeaderText} \textbf{Descrizione} & \color{CHeaderText} \textbf{Fonti} \\
	\endhead
	R0V1 & Obbligatorio & L'applicativo dovrà essere sviluppato attraverso l'uso di tecnologie web & Capitolato \\
	R0V1.1 & Obbligatorio & L'applicativo necessiterà della piattaforma Node.js per soddisfare le dipendenze delle librerie richieste citate nei \citGloss{requisiti} sottostanti. & Capitolato \\
	R0V1.2 & Obbligatorio & L'applicativo dovrà essere sviluppato con \citGloss{JavaScript} 8 (ES8) & Capitolato \\
	R0V1.3 & Obbligatorio & L'applicativo dovrà essere sviluppato con il boilerplate \citGloss{Redux} Minimal & Capitolato \\
	R0V1.4 & Obbligatorio & L'applicativo dovrà essere sviluppato utilizzando \citGloss{React} 15.x & Capitolato \\
	R0V1.5 & Obbligatorio & L'applicativo dovrà essere sviluppato utilizzando \citGloss{Redux} 3.x & Capitolato \\
	R0V1.6 & Obbligatorio & Il deploy del sito andrà eseguito utilizzando Surge.sh & Capitolato \\
	R0V1.7 & Desiderabile & È desiderabile l'utilizzo di \citGloss{SCSS} in sostituzione a \citGloss{CSS} & Capitolato \\
	R0V2 & Obbligatorio & Gli \citGloss{smart contract} dovranno essere scritti in linguaggio \citGloss{Solidity} & Capitolato \\
	R0V3 & Obbligatorio & La connessione alla rete \citGloss{Ethereum} deve avvernire tramite MetaMask & Capitolato \\
	R0V3.1 & Obbligatorio & I test riguartandi gli \citGloss{smart contract} dovranno essere eseguiti in una rete locale ed almeno in una rete pubblica & Capitolato \\
	R0V3.2 & Obbligatorio & Il deploy degli \citGloss{smart contract} dovrà avvenire su rete locale testrpc e rete di test Ropsten & Capitolato \\
	R2V3.3 & Opzionale & È apprezzabile un deploy finale sulla rete principale di \citGloss{Ethereum} & Capitolato \\
	R0V4 & Obbligatorio & Lo sviluppo degli \citGloss{smart contract} dovrà avvenire utilizzando il framework Truffle & Capitolato \\
	R0V5 & Obbligatorio & L'applicativo deve essere accessibile ed utilizzabile dal \citGloss{browser} Mozilla \citGloss{Firefox} a partire dalla versione 52 & Interno \\
	R0V6 & Obbligatorio & L'applicativo deve essere accessibile ed utilizzabile dal \citGloss{browser} Google \citGloss{Chrome} a partire dalla versione 57 & Interno \\
	R1V7 & Desiderabile & L'applicativo deve essere accessibile ed utilizzabile da un \citGloss{browser} mobile, per le versioni supportate fare riferimento alle controparti PC di \citGloss{Firefox} e \citGloss{Chrome} & Interno \\
	R0V8 & Obbligatorio & Un utente non deve poter compiere azioni sul sistema senza aver fatto l'accesso ad esso & Capitolato \\ 
	R0V9 & Obbligatorio & 'applicazione dei principi de "App a 12 Fattori" deve essere documentato ove possibile l'utilizzo & Capitolato, VER-2018-01-09 \\
	R0V10 & Obbligatorio & Il codice sorgente deve essere pubblicato con licenza MIT & Capitolato \\
	\hiderowcolors
	\caption{Tabella dei requisiti di vincolo}
\end{longtable}

\subsection{Requisiti prestazionali}

Non sono stati individuati \citGloss{requisiti} prestazionali, in quanto la maggior parte delle attività, per essere concluse, necessitano di una interazione con una rete \citGloss{Ethereum}, e quindi non risultano costanti o prevedibili con precisione.\\
Qualsiasi operazione effettuata in una rete \citGloss{Ethereum} reale ha un tempo di soddisfacimento casuale influenzato dal carico della rete nel momento della richiesta e, nel caso di operazioni che vanno a modificare lo stato di un contratto, del quantitativo di Ether pagati per ogni unità di \citGloss{Gas}.\\
 
 
\newpage
\section{Tracciamento}
\subsection{Tracciamento fonti-requisiti}

\label{table:Tabella di tracciamento fonti-requisiti}


\rowcolors{2}{CRighePari}{CRigheDispari}
\renewcommand*{\arraystretch}{1.2}
\begin{longtable}[H]{p{2cm}p{5cm}p{5cm}}
	\rowcolor{CHeader} 
	\color{CHeaderText} \textbf{Fonte} & \color{CHeaderText} \textbf{Nome fonte} & \color{CHeaderText} \textbf{Requisiti} \\
	\endhead
	Capitolato & & \makecell[tl]{ R0F2.2 \\
	 R0F3 \\
	 R0F3.1 \\
	 R0F3.3 \\
	 R0F3.4 \\
	 R0F4 \\
	 R0F4.1 \\
	 R0F4.1.1 \\
	 R0F4.1.2 \\
	 R0F4.2 \\
	 R0F4.3 \\
	 R0F5 \\
	 R0F5.1 \\
	 R0F5.2 \\
	 R0F5.2.1 \\
	 R0F5.2.2 \\
	 R0F7 \\
	 R0F8 \\
	 R0F8.1 \\
	 R0F10 \\
	 R0F12\\
	 R0F12.1 \\
	 R0Q2 \\
	 R0Q3 \\
	 R0Q4 \\
	 R0Q7 \\
	 R0Q8 \\
	 R0V1 \\
	 R0V1.1 \\
	 R0V1.2 \\
	 R0V1.3 \\
	 R0V1.4 } \\
\rowcolor{CRigheDispari}
& & \makecell[tl]{ %TODO aggiustare alla fine
	 R0V1.5 \\
	 R0V1.6 \\
	 R0V1.7 \\
	 R0V2 \\
	 R0V3 \\
	 R0V3.1 \\
	 R0V3.2 \\
	 R2V3.3 \\
	 R0V4 \\
	 R1V7 \\
	 R0V8 \\
	 R0V9 \\
	 R0V10 } \\
	
	\rowcolor{CRighePari}
	Interno & & \makecell[tl]{ R0F1 \\
	 R0F1.1 \\
	 R0F1.1.1 \\
	 R0F1.1.2 \\
	 R0F1.1.3 \\
	 R0F1.1.4 \\
	 R0F1.1.5 \\
	 R0F1.2 \\
	 R0F1.3 \\
	 R0F1.4 \\
	 R0F1.4.1 \\
	 R0F1.4.2 \\
	 R0F1.4.3 \\
	 R0F1.4.4 \\
	 R0F2 \\
	 R0F2.1 \\
	 R0F2.1.1 \\
	 R0F2.1.2 \\
	 R0F2.1.3 \\
	 R0F3 \\
	 R0F3.1.1 \\
	 R0F3.2 \\
	 R0F3.2.1 \\
	 R0F3.2.2 \\
	 R0F3.2.3 \\
	 R0F3.4 
	} \\
	& & \makecell[tl]{ %TODO aggiustare alla fine
	 R0F3.4.1 \\
	 R0F3.4.2 \\
	 R0F3.4.3 \\
	 R0F3.4.4 \\
	 R0F3.4.4.1 \\
	 R0F3.4.5 \\
	 R0F3.4.6 \\
	 R0F4.2.1 \\
	 R0F4.2.2 \\
	 R0F4.2.3 \\
	 R0F5.1.1 \\
	 R0F5.1.2 \\
	 R0F5.1.3 \\
	 R0F5.1.4 \\
	 R0F5.1.5 \\
	 R0F6 \\
	 R0F6.1 \\
	 R1F9 \\
	 R0F11.3 \\
	 R0F11.3.1 \\
	 R0F12.3 \\
	 R0Q1 \\
	 R0V5 \\
	 R0V6 } \\
	
	VER-2017-11-22 & Verbale & \makecell[tl]{ R0Q2.1 \\
	 R0Q5 \\
	 R0Q6 } \\
	
	VER-2017-12-08 & Verbale & \makecell[tl]{ R0F11 \\
	 R0F11.1 \\
	 R0F11.2 } \\
	VER-2018-01-09 & Verbale & \makecell[tl]{ R0F10 \\
	R0V9 } \\ 
	
	UC1 & Breve guida & R1F9 \\
	UC2 & Login & R0F6 \\
	UC2.1 & Login automatico & R0F6.1 \\
	UC2.2 & Visualizzazione messaggio di errore riguardo a chiave assente & R0F6.1 \\
	UC2.3 & Visualizzazione messaggio di errore riguardo a MetaMask assente & R0F6.1 \\
	UC2.4 & Visualizzazione messaggio di errore riguardo a chiave non registrata & R0F6.1 \\
	UC3 & Registrazione & R0F8 \\
	UC3.1 & Inserimento nome & R0F8.1 \\
	UC3.2 & Inserimento cognome & R0F8.1 \\
	UC3.3 & Selezione categoria & R0F8.1 \\
	UC3.4 & Selezione corso di laurea & R0F8.1 \\
	UC3.5 & Visualizzazione errore campo non compilato & R0F8.1 \\
	UC3.6 & Visualizzazione errore utente già registrato & R0F8.1 \\
	UC3.7 & Visualizzazione errore chiave non presente & R0F8.1 \\
	UC3.8 & Visualizzazione errore MetaMask non installato & R0F8.1 \\
	UC4 & Logout & R0F7 \\
	UC5 & Amministrazione & \makecell[tl]{ R0F1 \\
		R0F2 \\
		R0F12 } \\
	UC5 & Gestione Utenti & R0F1 \\
	UC5.1 & Visualizzazione lista degli studenti approvati & R0F1.4 \\
	UC5.1.1 & Visualizzazione indirizzi degli studenti approvati & R0F1.4.1 \\
	UC5.1.2 & Visualizzazione nomi degli studenti approvati & R0F1.4.2 \\
	UC5.1.3 & Visualizzazione cognomi degli studenti approvati & R0F1.4.3 \\
	UC5.1.4 & Visualizzazione corsi degli studenti approvati & R0F1.4.4 \\
	UC5.2 & Visualizzazione lista dei professori approvati & R0F2.1 \\
	UC5.2.1 & Visualizzazione indirizzi dei professori approvati & R0F2.1.1 \\
	UC5.2.2 & Visualizzazione nomi dei professori approvati & R0F2.1.2 \\
	UC5.2.3 & Visualizzazione cognomi dei professori approvati & R0F2.1.3 \\
	UC5.3 & Visualizzazione lista utenti in attesa di abilitazione & R0F1.1 \\
	UC5.3.1 & Visualizzazione indirizzi degli utenti in attesa di abilitazione & R0F1.1.1 \\
	UC5.3.2 & Visualizzazione nomi degli utenti in attesa di abilitazione & R0F1.1.2 \\
	UC5.3.3 & Visualizzazione cognomi degli utenti in attesa di abilitazione & R0F1.1.3 \\
	UC5.3.4 & Visualizzazione ruolo degli utenti in attesa di abilitazione &R0F1.1.4  \\
	UC5.3.5 & Visualizzazione corsi degli utenti in attesa di abilitazione & R0F1.1.5 \\
	UC5.4 & Abilitazione utente & R0F1.2 \\
	UC5.5 & Rimozione utente & R0F1.3 \\
	UC6 & Gestione anni accademici & R0F12 \\
	UC6.1 & Aggiunta \citGloss{anno accademico} & R0F12.1 \\
	UC6.2 & Visualizzazione lista di tutti gli anni accademici & R0F12.3 \\
	UC6.2.1 & Visualizzazione anni solari relativi agli anni accademici & R0F12.3 \\
	UC6.3 & Visualizzazione messaggio anno malformato & R0F12.1 \\
	UC6.4 & Visualizzazione messaggio di \citGloss{anno accademico} già presente & R0F12.1 \\
	UC6.5 & Visualizzazione messaggio anno non compilato & R0F12.1 \\
	UC6.6 & Eliminazione anno accademico vuoto & R0F12.2 \\
	UC7 & Gestione corsi di laurea & R0F3 \\
	UC7.1 & Creazione \citGloss{corso di laurea} & \makecell[tl]{
		R0F3.1\\
		R0F3.1.1 } \\
	UC7.1.1 & Inserimento sigla corso di laurea & \makecell[tl]{
		R0F3.1\\
		R0F3.1.1 } \\
	UC7.2 & Visualizzazione errore sigla invalida & \makecell[tl]{
		R0F3.1\\
		R0F3.1.1 } \\
	UC7.3 & Visualizzazione lista completa dei corsi & R0F3.2 \\
	UC7.3.1 & Visualizzazione sigle dei \citGloss{corsi di laurea} & R0F3.2.2 \\
	UC7.3.2 & Visualizzazione anni accademici associati ai \citGloss{corsi di laurea} & R0F3.2.3 \\
	UC7.4 & Visualizzazione lista corsi di laurea per \citGloss{anno accademico} & R0F3.2.1  \\
	UC7.4.1 & Visualizzazione sigle dei corsi di laurea per \citGloss{anno accademico} & R0F3.2.2  \\
	UC7.5 & Creazione esame in un corso & R0F3.3\\
	UC7.5.1 & Inserimento nome esame & R0F3.3\\
	UC7.5.2 & Inserimento numero crediti esami & R0F3.3\\
	UC7.5.3 & Inserimento obbligatorietà esame & R0F3.3\\
	UC7.5.4 & Visualizzazione errore nome esame non valido & R0F3.3\\
	UC7.5.5 & Visualizzazione errore numero crediti esame non valido & R0F3.3\\
	UC7.6 & Visualizzazione errore nome esame non valido & R0F3.3  \\
	UC7.7 & Visualizzazione lista esami per \citGloss{corso di laurea} & R0F3.4 \\
	UC7.7.1 & Visualizzazione sigla esami per \citGloss{corso di laurea} & R0F3.4.1 \\
	UC7.7.2 & Visualizzazione crediti esami per \citGloss{corso di laurea} & R0F3.4.2 \\
	UC7.7.3 & Visualizzazione nomi dei professori associati agli esami per \citGloss{corso di laurea} se presenti & R0F3.4.3 \\
	UC7.7.4 & Visualizzazione cognomi dei professori associati agli esami per \citGloss{corso di laurea} se presenti & R0F3.4.3 \\
	UC7.8 & Visualizzazione lista esami & R0F3.4 \\
	UC7.8.1 & Visualizzazione sigla esami & R0F3.4.1 \\
	UC7.8.2 & Visualizzazione crediti esami & R0F3.4.2 \\
	UC7.8.3 & Visualizzazione sigle \citGloss{corsi di laurea} associati ad esami se presenti & R0F3.4.5 \\
	UC7.8.4 & Visualizzazione nomi dei professori associati agli esami se presenti & R0F3.4.3 \\
	UC7.8.5 & Visualizzazione cognomi dei professori associati agli esami se presenti & R0F3.4.3 \\
	UC7.9 & Visualizzazione dettagli esame & R0F3.4 \\
	UC7.9.1 & Visualizzazione sigla esame & R0F3.4.1 \\
	UC7.9.2 & Visualizzazione crediti esame & R0F3.4.2 \\
	UC7.9.3 & Visualizzazione corso di laurea associato all'esame se presente & R0F3.4.5 \\
	UC7.9.4 & Visualizzazione nome del professore associato all'esame se presente & R0F3.4.3 \\
	UC7.9.5 & Visualizzazione cognome del professore associato all'esame se presente & R0F3.4.3 \\
	UC7.9.6 & Visualizzazione numero di studenti iscritti all'esame & R0F3.4.6 \\
	UC7.10 & Visualizzazione di tutti gli anni accademici & R0F12.3 \\
	UC7.10.1 & Visualizzazione anni solari relativi agli anni accademici & R0F12.3 \\
	UC7.11 & Visualizzazione lista professori da associare ad un esame &  \\
	UC7.11.1 & Visualizzazione nome per ogni professore & R0F3.4.3 \\
	UC7.11.2 & Visualizzazione cognome per ogni professore & R0F3.4.3 \\
	UC7.11.3 & Visualizzazione indirizzo per ogni professore & R0F3.4.4.1 \\
	UC7.12 & Associazione professore all'esame & R0F3.4.4 \\
	
	
	
	
	
	
	%TODO ----------
	UC8 & Gestione aspetti relativi agli esami & R0F4 \\
	UC8.1 & Visualizzazione lista degli esami & R0F4.1 \\
	UC8.1.1 & Visualizzazione codici degli esami & R0F4.1.1 \\
	UC8.1.2 & Visualizzazione corsi degli esami & R0F4.1.2 \\
	UC8.2 & Visualizzazione lista degli studenti & R0F4.2  \\
	UC8.2.1 & Visualizzazione indirizzi degli studenti dell'esame & R0F4.2.1 \\
	UC8.2.2 & Visualizzazione nomi degli studenti dell'esame & R0F4.2.2 \\
	UC8.2.3 & Visualizzazione cognomi degli studenti dell'esame & R0F4.2.3 \\
	UC8.3 & Registrazione valutazione di un esame & R0F4.3 \\
	UC8.3.1 & Inserimento voto numerico & R0F4.3 \\
	UC8.3.2 & Visualizzazione errore voto inferiore a 18 & R0F4.3 \\
	UC8.3.2 & Visualizzazione errore voto non numerico & R0F4.3 \\
	UC9 & Gestione aspetti relativi allo studente & R0F5 \\
	UC9.1 & Visualizzazione lista degli esami & R0F5.1 \\
	UC9.1.1 & Visualizzazione dei crediti degli esami ai quali è iscritto & R0F5.1.1 \\
	UC9.1.2 & Visualizzazione della obbligatorietà degli esami ai quali è iscritto & R0F5.1.2 \\
	UC9.1.3 & Visualizzazione delle valutazioni degli esami ai quali è iscritto	& R0F5.1.3 \\
	UC9.2 & Visualizzazione degli esami opzionali e dei loro crediti & R0F5.2 \\
	UC9.3 & Iscrizione ad un esame opzionale & R0F5.2.2 \\
	UC9.4 & Visualizzazione delle informazioni relative ai crediti & R0F5.1.4 \\ 
	UC9.4.1 & Visualizzazione somma dei crediti & R0F5.1.4 \\ 
	UC9.4.2 & Visualizzazione crediti mancanti & R0F5.1.4 \\ 
	UC9.4.3 & Visualizzazione crediti ottenibili dagli esami a cui è iscritto & R0F5.1.4 \\ 
	& & R0F5.2.1 \\
	UC10 & Gestione degli amministratori & R0F11 \\
	UC10.1 & Aggiunta di un amministratore & R0F11.1 \\
	UC10.1.1 & Inserimento \citGloss{chiave pubblica} & R0F11.1 \\
	UC10.1.2 & Visualizzazione messaggio di errore riguardo a chiave mal formata & R0F11.1 \\
	UC10.1.3 & Visualizzazione messaggio di errore riguardo a chiave già registrata & R0F11.1 \\
	UC10.2 & Visualizzazione lista degli amministratori & R0F11.3 \\
	UC10.2.1 & Visualizzazione indirizzi degli amministratori & R0F11.3.1 \\
	UC10.3 & Rimozione amministratore & R0F11.2 \\	
	UC11 & Visualizzazione quantità di \citGloss{Gas}, \citGloss{Ether} e costo delle operazioni & R0F10 \\
	\hiderowcolors
	\caption{Tabella di tracciamento fonti-requisiti}
\end{longtable}

\newpage
\subsection{Tracciamento requisiti-fonti}

\label{table:Tabella di tracciamento requisiti-fonti}
\rowcolors{2}{CRighePari}{CRigheDispari}
\renewcommand*{\arraystretch}{1.2}
\begin{longtable}[H]{p{2cm}p{5.2cm}p{5cm}}
	\rowcolor{CHeader} 
	\color{CHeaderText} \textbf{Requisito} & \color{CHeaderText} \textbf{Descrizione requisito} & \color{CHeaderText} \textbf{Fonti} \\
	% REGEX for Notepad++
	% (.*)&(.*)&(.*)&(.*)      --->       \1& \3& \\\\ \ 
	\endhead
	R0F1 & L'amministratore può gestire gli utenti & \makecell[tl]{
		Interno \\
		UC5
	} \\
	R0F1.1 & L'amministratore può ottenere una lista di tutti gli utenti non abilitati & \makecell[tl]{
		Interno \\
		UC5.3
	} \\
	R0F1.1.1 & L'amministratore può ottenere l'indirizzo di tutti gli utenti non abilitati & \makecell[tl]{
		Interno \\
		UC5.3.1
	} \\
	R0F1.1.2 & L'amministratore può ottenere il nome di tutti gli utenti non abilitati & \makecell[tl]{
		Interno \\
		UC5.3.2
	} \\
	R0F1.1.3 & L'amministratore può ottenere il cognome di tutti gli utenti non abilitati & \makecell[tl]{
		Interno \\
		UC5.3.3
	} \\
	R0F1.1.4 & L'amministratore può ottenere il ruolo desiderato di tutti gli utenti non abilitati & \makecell[tl]{
		Interno \\
		UC5.3.4
	} \\
	R0F1.1.5 & L'amministratore può ottenere il corso desiderato, ove possibile, di tutti gli utenti non abilitati & \makecell[tl]{
		Interno \\
		UC5.3-5
	} \\

	R0F1.2 & L'amministratore può abilitare un utente & \makecell[tl]{
		Interno \\ 
		UC5 \\ 
		UC5.4
	} \\
	R0F1.3 & L'amministratore può rimuovere un utente & \makecell[tl]{
		Interno \\ 
		UC5.5
	} \\

	R0F1.4 & L'amministratore può ottenere una lista di tutti gli studenti & \makecell[tl]{
		Interno \\
		UC5.1
	} \\
	R0F1.4.1 & L'amministratore può ottenere l'indirizzo di tutti studenti & \makecell[tl]{
		Interno \\
		UC5.1.1
	} \\
	R0F1.4.2 & L'amministratore può ottenere il nome di tutti gli studenti & \makecell[tl]{
		Interno \\
		UC5.1.2
	} \\
	R0F1.4.3 & L'amministratore può ottenere il cognome di tutti gli studenti & \makecell[tl]{
		Interno \\
		UC5.1.3
	} \\
	R0F1.4.4  & L'amministratore può ottenere il corso di tutti gli studenti & \makecell[tl]{
		Interno \\
		UC5.1.4
	} \\

	R0F2 & L'amministratore può gestire i professori & \makecell[tl]{
		Interno
	} \\
	R0F2.1 & L'amministratore può ottenere una lista di tutti i professori & \makecell[tl]{
		Interno \\ 
		UC5.2 
	} \\
	R0F2.1.1 & L'amministratore può ottenere l'indirizzo di tutti i professori & \makecell[tl]{
		Interno \\ 
		UC5.2.1
	} \\
	R0F2.1.2 & L'amministratore può ottenere il nome di tutti i professori & \makecell[tl]{
		Interno \\ 
		UC5.2.2
	} \\
	R0F2.1.3 & L'amministratore può ottenere il cognome di tutti i professori & \makecell[tl]{
		Interno \\ 
		UC5.2.3
	} \\
	R0F2.2 & L'amministratore può assegnare un esame ad un determinato professore & \makecell[tl]{
		Capitolato
	} \\
	R0F12& L'università può gestire gli anni accademici & \makecell[tl]{
		Capitolato \\ 
		UC6
	} \\
	R0F12.1 & L'università può aggiungere un \citGloss{anno accademico} & \makecell[tl]{
		Capitolato \\ 
		UC6.1 \\ 
		UC6.3 \\ 
		UC6.4 \\ 
		UC6.5
	} \\
	R0F12.2 & L'università può rimuovere un \citGloss{anno accademico} & \makecell[tl]{
	Capitolato \\ 
	UC6.6
	} \\
	R0F12.3 &  L'università e gli amministratori possono ottenere una lista degli anni accademici, con indicato l'anno solare & \makecell[tl]{
		Interno \\ 
		UC6.2 \\
		UC6.2.1 \\
		UC7.10 \\
		UC7.10.1
	} \\ 
	R0F3 & L'amministratore può gestire i corsi di laurea relativi agli anni accademici & \makecell[tl]{
		Capitolato \\ 
		UC7 
	} \\
	R0F3.1 & L'amministratore può creare un nuovo \citGloss{corso di laurea} a partire da un \citGloss{anno accademico} & \makecell[tl]{
		Capitolato \\ 
		UC7.1 \\
		UC7.1.1 \\
		UC7.2
	} \\
	R0F3.1.1 & L'amministratore deve indicare un codice identificativo per ogni \citGloss{corso di laurea} & \makecell[tl]{
		Interno \\ 
		UC7.1 \\
		UC7.2
	} \\ 
	R0F3.2 & L'amministratore può ottenere una lista di tutti i corsi di laurea dato un \citGloss{anno accademico} & \makecell[tl]{
		Interno \\ 
		UC7.3
	} \\
	R0F3.2.1 & L'amministratore può ottenere una lista di tutti i corsi di laurea di un determinato \citGloss{anno accademico} & \makecell[tl]{
		Interno \\ 
		UC7.4
	} \\
	R0F3.2.2 & Ogni corso di laurea deve essere visualizzato tramite la sua sigla & \makecell[tl]{
		Interno \\ 
		UC7.3.1 \\
		UC7.4.1
	} \\
	R0F3.2.3 & Ogni corso di laurea mostrato deve essere accompagnato dall'anno di riferimento ove non indicato dall'utente stesso & \makecell[tl]{
		Interno \\ 
		UC7.3.2 \\
	} \\
	R0F3.3 & L'amministratore può creare nuovi esami & \makecell[tl]{
		Capitolato \\ 
		UC7.5 \\
		UC7.5.1 \\
		UC7.5.2 \\
		UC7.5.3 \\
		UC7.5.4 \\
		UC7.5.5 \\
		UC7.6
	} \\
	R0F3.4 & L'amministratore può ottenere una lista di tutti gli esami dato il \citGloss{corso di laurea} & \makecell[tl]{
		Interno \\ 
		UC7.7 \\
		UC7.8 \\
		UC7.9
	} \\
	R0F3.4.1 & Per ogni esame l'amministratore deve visualizzarne la sigla & \makecell[tl]{
		Interno \\
		UC7.7.1 \\
		UC7.8.1 \\
		UC7.9.1
	} \\
	R0F3.4.2 & Per ogni esame l'amministratore deve visualizzarne il numero di crediti & \makecell[tl]{
		Interno \\
		UC7.7.2 \\
		UC7.8.2 \\
		UC7.9.2
	} \\
	R0F3.4.3 & L'amministratore può ottenere il nome ed il cognome di quale professore è assegnato o sta per essere assegnato ad un determinato esame & \makecell[tl]{
		Interno \\ 
		UC7.7.3 \\
		UC7.7.4 \\
		UC7.9.4 \\
		UC7.8.5 \\
		UC7.8.4 \\
		UC7.9.5 \\
		UC7.11.1 \\
		UC7.11.2
	} \\
	R0F3.4.4  & L'amministratore può associare ad un esame il relativo professore & \makecell[tl]{
		Interno	 \\
		UC7.12 	
	} \\
	R0F3.4.4.1 & L'amministratore deve poter visualizzare l'indirizzo di un professore prima di assegnarlo ad un determinato esame per evitare problemi in caso di omonimi & \makecell[tl]{
		Interno	 \\ 
		UC7.10.3 \\
	} \\
	R0F3.4.5 & L'amministratore può ottenere la sigla del corso di laurea al quale appartiene un determinato esame ove non indicato dall'amministratore stesso oppure se richiesto & \makecell[tl]{
		Interno	 \\ 
		UC7.4.5
	} \\
	R0F3.4.6 & L'amministratore può ottenere il numero degli studenti iscritti a un determinato corso  & \makecell[tl]{
		Interno	 \\ 
		UC7.9.6
	} \\
	R0F4 & Un professore può gestire gli esami a lui assegnati & \makecell[tl]{
		Capitolato \\ 
		UC8
	} \\
	R0F4.1 & Un professore può ottenere la lista di tutti gli esami a lui assegnati & \makecell[tl]{
		Capitolato \\ 
		UC8.1
	} \\
	R0F4.1.1 & Un professore può ottenere il codice di tutti gli esami a lui assegnati & \makecell[tl]{
		Capitolato \\ 
		UC8.1.1
	} \\
	R0F4.1.2 & Un professore può ottenere il relativo corso di laurea di tutti gli esami a lui assegnati& \makecell[tl]{
		Capitolato \\ 
		UC8.1.2
	} \\
	R0F4.2 & Un professore può ottenere la lista di tutti gli studenti associati a un determinato esame & \makecell[tl]{
		Capitolato  \\ 
		UC8.2
	} \\
	R0F4.2.1 & Un professore può ottenere l'indirizzi di tutti studenti associati a un determinato esame & \makecell[tl]{
		Interno  \\ 
		UC8.2.1
	} \\
	R0F4.2.2 & Un professore può ottenere il nome di tutti gli studenti associati a un determinato esame & \makecell[tl]{
		Interno  \\ 
		UC8.2.2
	} \\
	R0F4.2.3 & Un professore può ottenere il cognome di tutti gli studenti associati a un determinato esame & \makecell[tl]{
		Interno  \\ 
		UC8.2.3
	} \\
	R0F4.3 & Un professore può registrare un esito ad un dato esame ad un dato studente registrato a quell'esame & \makecell[tl]{
		Capitolato \\ 
		UC8.3 \\
		UC8.3.1 \\
		UC8.3.2 \\
		UC8.3.3
	} \\
	R0F5 & Uno studente può gestire l'iscrizione agli esami & \makecell[tl]{
		Capitolato \\ 
		UC9
	} \\
	R0F5.1 & Uno studente può vedere l'elenco degli esami ai quali è iscritto & \makecell[tl]{
		Capitolato \\ 
		UC9.1
	} \\
	R0F5.1.1 & Uno studente può ottenere il numero di crediti assegnati agli esame & \makecell[tl]{
		Interno \\ 
		UC9.1.1
	} \\
	R0F5.1.2 & Uno studente può ottenere l'obbligatorietà degli esame & \makecell[tl]{
		Interno \\ 
		UC9.1.2
	} \\
	R0F5.1.3 & Uno studente può ottenere lo stato di superamento degli esame & \makecell[tl]{
		Interno \\ 
		UC9.1.3
	} \\
	R0F5.1.4 & Uno studente può ottenere il totale dei suoi crediti & \makecell[tl]{
		Interno \\ 
		UC9.4 \\
		UC9.4.1 \\
		UC9.4.3
	} \\
	R0F5.1.5 & Uno studente può ottenere il numero di crediti da raggiungere per la laurea & \makecell[tl]{
		Interno \\
		UC9.4.2
	} \\
	R0F5.2 & Uno studente può ottenere l'elenco degli esami opzionali & \makecell[tl]{
		Capitolato \\ 
		UC9.2
	} \\
	R0F5.2.1 & Uno studente può ottenere il numero di crediti degli esami opzionali & \makecell[tl]{
		Capitolato \\ 
		UC9.4
	} \\
	R0F5.2.2 & Uno studente può iscriversi ad un esame opzionale & \makecell[tl]{
		Capitolato \\ 
		UC9.3
	} \\
	R0F6 & L'utente può effettuare il login & \makecell[tl]{
		Interno \\ 
		UC2
	} \\
	R0F6.1 & Il login deve avvenire tramite il controllo delle chiavi, senza ulteriori azioni da parte dell'utente & \makecell[tl]{
		Interno \\ 
		UC2.1
	} \\
	R0F7 & L'utente può effettuare il logout & \makecell[tl]{
		Capitolato \\ 
		UC2.1  \\
		UC2.2 \\
		UC2.3 \\
		UC2.4 \\ 
		UC4
	} \\
	R0F8 & L'utente non ancora registrato può registrarsi & \makecell[tl]{
		Capitolato \\ 
		UC3
	} \\
	R0F8.1 & La registrazione necessita di nome, cognome, categoria (studente o professore) e selezione del corso di laura se si tratta di uno studente & \makecell[tl]{
		Capitolato \\
		UC3.1 \\
		UC3.2 \\
		UC3.3 \\
		UC3.4 \\
		UC3.5 \\
		UC3.6 \\
		UC3.7 \\
		UC3.8
	} \\
	R1F9 & L'utente può leggere una breve guida sull'uso di MetaMask e sul pagamento delle operazioni & \makecell[tl]{
		Interno \\ 
		UC1
	} \\
	R0F10 & L'utente deve poter vedere preventivamente il costo in \citGloss{Gas}, \citGloss{Ether} e Euro dell'operazione che sta per eseguire & \makecell[tl]{
		Capitolato \\
		UC11
	} \\
	R0F11 & L'università deve poter gestire gli amministratori & \makecell[tl]{
		VER-2017-12-08 \\
		UC10
	} \\
	R0F11.1 & L'università deve poter aggiungere amministratori attraverso l'inserimento del loro indirizzo & \makecell[tl]{
		VER-2017-12-08 \\
		UC10.1 \\
		UC10.1.1 \\ 
		UC10.1.2 \\
		UC10.1.3
	} \\
	R0F11.2 & L'università deve poter rimuovere amministratori & \makecell[tl]{
		VER-2017-12-08 \\
		UC10.3
	} \\
	R0F11.3 & L'università deve poter ottenere la lista di tutti gli amministratori & \makecell[tl]{
		Interno \\
		UC10.2
	} \\
	R0F11.3.1& L'università deve poter visualizzare l'indirizzo di ogni amministratore & \makecell[tl]{
		Interno \\
		UC10.2.1
	} \\
	R0Q1 & La progettazione e il codice devono seguire le norme e le metriche riportate nei documenti allegati X e Y & \makecell[tl]{
		Interno
	} \\
	R0Q2 & L'approccio di scrittura di \citGloss{JavaScript} deve essere promise Centric Approach & \makecell[tl]{
		Capitolato
	} \\
	R0Q2.1 & L'applicativo non deve fare uso di callback in presenza di alternative alle ultime & \makecell[tl]{
		VER-2017-11-22
	} \\
	R0Q3 & Il codice \citGloss{JavaScript} deve attenersi al \citGloss{AirBNB} \citGloss{JavaScript} style guide & \makecell[tl]{
		Capitolato
	} \\
	R0Q4 & Lo sviluppo deve essere supportato dall'utilizzo del tool ESLint & \makecell[tl]{
		Capitolato
	} \\
	R0Q5 & Dovrà essere fornito un manuale utente in lingua inglese che tratterà l'uso da parte di studenti e professori & \makecell[tl]{
		VER-2017-11-22
	} \\
	R0Q6 & Dovrà essere fornito un manuale di deploy e di utilizzo da parte degli amministratori in lingua inglese & \makecell[tl]{
		VER-2017-11-22
	} \\
	R0Q7 & Il codice sorgente deve essere pubblicato sulla piattaforma \citGloss{GitHub}, BitBucket o GitLab & \makecell[tl]{
		Capitolato
	} \\
	R0Q8 & Il codice deve attenersi il più possibile alle guide linea de "App a 12 Fattori" & \makecell[tl]{
		Capitolato
	} \\
	R0V1 & L'applicativo dovrà essere sviluppato attraverso l'uso di tecnologie web & \makecell[tl]{
		Capitolato
	} \\
	R0V1.1 & L'applicativo dovrà essere sviluppato con Node.js & \makecell[tl]{
		Capitolato
	} \\
	R0V1.2 & L'applicativo dovrà essere sviluppato con \citGloss{JavaScript} 8 (ES8) & \makecell[tl]{
		Capitolato
	} \\
	R0V1.3 & L'applicativo dovrà essere sviluppato con il boilerplate \citGloss{Redux} Minimal & \makecell[tl]{
		Capitolato
	} \\
	R0V1.4 & L'applicativo dovrà essere sviluppato utilizzando \citGloss{React} 15.x & \makecell[tl]{
		Capitolato
	} \\
	R0V1.5 & L'applicativo dovrà essere sviluppato utilizzando \citGloss{Redux} 3.x & \makecell[tl]{
		Capitolato
	} \\
	R0V1.6 & Il deploy del sito andrà eseguito utilizzando Surge.sh & \makecell[tl]{
		Capitolato
	} \\
	R0V1.7 & È desiderabile l'utilizzo di S\citGloss{CSS} in sostituzione a \citGloss{CSS} & \makecell[tl]{
		Capitolato
	} \\
	R0V2 & Gli \citGloss{smart contract} dovranno essere scritti in linguaggio \citGloss{Solidity} & \makecell[tl]{
		Capitolato
	} \\
	R0V3 & La connessione alla rete \citGloss{Ethereum} deve avvernire tramite MetaMask & \makecell[tl]{
		Capitolato
	} \\
	R0V3.1 & I test riguartandi gli \citGloss{smart contract} dovranno essere eseguiti in una rete locale ed almeno in una rete pubblica & \makecell[tl]{
		Capitolato
	} \\
	R0V3.2 & Il deploy degli \citGloss{smart contract} dovrà avvenire su rete locale testrpc e rete di test Ropsten & \makecell[tl]{
		Capitolato
	} \\
	R2V3.3 & È apprezzabile un deploy finale sulla rete principale di \citGloss{Ethereum} & \makecell[tl]{
		Capitolato
	} \\
	R0V4 & Lo sviluppo degli \citGloss{smart contract} dovrà avvenire utilizzando il framework Truffle & \makecell[tl]{
		Capitolato
	} \\
	R0V5 & L'applicativo deve essere accessibile ed utilizzabile dal \citGloss{browser} Mozilla \citGloss{Firefox} a partire dalla versione 52 & \makecell[tl]{
		Interno
	} \\
	R0V6 & L'applicativo deve essere accessibile ed utilizzabile dal \citGloss{browser} Google \citGloss{Chrome} a partire dalla versione 57 & \makecell[tl]{
		Interno
	} \\
	R1V7 & L'applicativo deve essere accessibile ed utilizzabile da un \citGloss{browser} mobile, per le versioni supportate fare riferimento alle controparti PC di \citGloss{Firefox} e \citGloss{Chrome} & \makecell[tl]{
		Capitolato
	} \\
	R0V8 & Un utente non deve poter compiere azioni sul sistema senza aver fatto l'accesso ad esso & \makecell[tl]{
		Capitolato
	}\\
	R0V9 & L'applicazione dei principi de "App a 12 Fattori" deve essere documentata & \makecell[tl]{
		Capitolato 
	}\\
	R0V10 & Il codice sorgente deve essere pubblicato con licenza MIT & \makecell[tl]{
		Capitolato
	}\\
	\hiderowcolors
	\caption{Tabella di tracciamento requisiti-fonti}
\end{longtable}

\newpage
\section{Riepilogo}

\label{table:Riepilogo del numero dei requisiti individuati}
\rowcolors{2}{CRighePari}{CRigheDispari}
\renewcommand*{\arraystretch}{1.2}
\begin{longtable}[H]{p{2.8cm}p{2.9cm}p{2.9cm}p{2.9cm}p{1.5cm}}
	\rowcolor{CHeader}
	\color{CHeaderText} \textbf{Tipologia} & \color{CHeaderText} \textbf{0 Obbligatori} & \color{CHeaderText} \textbf{1 Desiderabili} & \color{CHeaderText} \textbf{2 Opzionali} & \color{CHeaderText} \textbf{Totale} \\
	Funzionali & 68 & 1 & 0 & 69 \\
	Di qualità & 9 & 0 & 0 & 9 \\
	Di vincolo & 18 & 1 & 1 & 20 \\
	Prestazionali & 0 & 0 & 0 & 0 \\
	\hiderowcolors
	\caption{Riepilogo del numero dei requisiti individuati}
\end{longtable}


\end{document}