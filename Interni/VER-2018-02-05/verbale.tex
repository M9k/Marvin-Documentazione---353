\documentclass[VER-2018-05-02.tex]{subfiles}
\begin{document}

\chapter{Riunione}
\section{Informazioni generali}
\begin{itemize}
	\item \textbf{Motivo della riunione:} \'{E} stata indetta questa chiamata \citGloss{Skype} per decidere come modificare i documenti a seguito della valutazione ricevuta in Revisione dei Requisiti in data 26/01/2018, e di come organizzarci per redigere il \textit{Proof of Concept};
	\item \textbf{Luogo e data:} Chiamata \citGloss{Skype};
	\item \textbf{Orario:} 18.00-19.20;
	\item \textbf{Partecipanti:}Ogni membro del gruppo.
\end{itemize}

\chapter{Ordine del giorno}
Di seguito sono riportati i punti dell'ordine del giorno che sono stati discussi durante la riunione:
\begin{enumerate}
	\item Realizzazione del Proof of Concepts;
	\item Modifica dei documenti.
\end{enumerate}
\chapter{Resoconto}
\begin{enumerate}
	\item \textbf{Realizzazione del Proof of Concepts:} \'{E} stato deciso di creare un wireframe di Marvin che presenti un'architettura quanto più simile possibile a quella definitiva, composta dall'uso di \citGloss{React}, \citGloss{Redux} e di una rete \citGloss{Ethereum} locale.
	\item \textbf{Modifica dei documenti:} \'{E} stato deciso di inviare una mail al prof. Vardanega per chiarire i dubbi che sono sorti dalla valutazione della Revisione dei Requisiti.
	Sempre seguendo la valutazione sopracitata, è stato deciso di modificare i documenti nel modo seguente:
	\begin{itemize}
		\item \textbf{Tutti i documenti:} modificare il Diario delle Modifiche indicando il numero di paragrafo modificato;
		\item \textbf{Norme di Progetto:} \begin{itemize}
										     \item Espandere \S 2.2.2;
										     \item Descrivere le procedure in \S 3.2.4, \S 3.5.2 e \S 4.3.1 tramite diagrammi;
								           \end{itemize}
      \item \textbf{Piano di Progetto:} Inserire le ore di autoformazione pesate per il tipo di autoformazione;
      \item\textbf{Piano di Qualifica:} Spostare i contenuti di \S A, \S B e \S C nelle Norme di Progetto.
	\end{itemize}
\end{enumerate}

\section{Tracciamento delle decisioni}
\begin{table}[H]
	\begin{center}
		\begin{tabu} to \textwidth {
				>{\centering}m{0.25\linewidth}  
				>{\centering\arraybackslash}m{0.7\linewidth}
			}
			\tableHeaderStyle
			\textbf{Codice} & \textbf{Decisione} \\
			VER-2018-02-05.1 & Invio mail al prof. Vardanega. \\
			VER-2018-02-05.2 & Modifica dei documenti come da esito RR. \\
			VER-2018-02-05.3 & Modalità di realizzazione del Proof of Concepts. \\
		\end{tabu}
		\caption{Tracciamento delle decisioni del verbale}
	\end{center}
\end{table}
\end{document}