\documentclass[NormeDiProgetto.tex]{subfiles}

\begin{document}

\chapter{Processi primari}

\section{Fornitura}
\subsection{Studio di fattibilità}
\subsection[Rapporti di fornitura con la Proponente Red Babel]{Rapporti di fornitura con la Proponente \\ Red Babel}
\subsection{Documentazione fornita}

\section{Sviluppo}
\subsection{Analisi dei Requisiti}
\subsection{Codifica}
In questa sotto-sezione vengono elencate le norme alle quali i Programmatori devono attenersi durante l'attività di programmazione e implementazione.\\
All'inizio verrano elencate delle norme generali a cui i Programmatori devono attenersi con qualsiasi linguaggio di programmazione utilizzato, mentre di seguito verranno elencate delle norme specifiche per i linguaggi Javascript, React, Solidity e Scss.\\
Ogni norma è rappresentata da un paragrafo. Ciascuna ha un titolo, una breve descrizione e, se necessario, un esempio che illustra i modi accettati o meno. Alcune di esse includono inoltre una lista di possibili eccezioni d'uso. L'uso di norme e convenzioni è fondamentale per permettere la generazione di codice leggibile e uniforme, agevolare le fasi di manutenzione, verifica e validazione e migliorare la qualità del prodotto.

\paragraph*{Convenzioni per i nomi: }

\begin{itemize}
\item nomi da evitare perchè facilmente confondibili con i numeri \texttt{1} e \texttt{0}:
\begin{itemize}
\item \texttt{l} (lettera minuscola elle);
\item \texttt{O} (lettera maiuscola o);
\item \texttt{I} (lettera maiuscola i).
\end{itemize}
\item tutti i nomi devono essere \textbf{unici} ed \textbf{esplicativi} al fine di evitare al più possibile ambiguità e comprensione.
\item tutti i nomi e i commenti al codice per la documentazione vanno scritti in \textbf{inglese};
\item è possibile utilizzare in un commento la keyword \textbf{TODO} per indicare codice temporaneo e soluzioni a breve termine o evidentemente migliorabili;
\item ogni file contenente codice deve avere la seguente \textbf{intestazione} contenuta in un commento e posta all'inizio del file stesso:
\begin{center}{
	\begin{minipage}{12cm}
		\begin{Verbatim}[frame=single]
File : nome file
Version : versione file
Type : tipo file
Date : data di creazione
Author : nome autore /i
E- mail : email autore /i

License : tipo licenza

Advice : lista avvertenze e limitazioni

Changelog :
Autore || Data || breve descrizione delle modifiche
		\end{Verbatim}
	\end{minipage}
}
\end{center}
\item La \textbf{versione} del codice viene inserita all’interno dell’intestazione del file e rispetta il
seguente formalismo:
\begin{center}{\textbf{X.Y}}\end{center}	
\begin{itemize}
\item \textbf{X}: è l’indice di versione principale, un incremento di tale indice rappresenta un avanzamento della versione stabile, che porta il valore dell’indice Y ad essere azzerato;
\item \textbf{Y}: è l’indice di modifica parziale, un incremento di tale indice rappresenta una verifica o una modifica rilevante, come per esempio la rimozione o l’aggiunta di una istruzione.
\end{itemize}
La versione \textit{1.0} deve rappresentare la prima versione del file completo e stabile, cioè quando le sue funzionalità obbligatorie sono state definite e si considerano funzionanti. Solo dalla versione \textit{1.0} è possibile testare il file, con degli appositi test definiti, per	verificarne l’effettivo funzionamento.
\end{itemize}
Per i vari linguaggi verranno successivamente descritte altre norme per nomi di variabili, funzioni e altro per ogni linguaggio facendo riferimento ai seguenti stili:
\begin{itemize}
	\item \texttt{lowercase};
	\item \texttt{lower\_case\_with\_underscores};
	\item \texttt{UPPERCASE};
	\item \texttt{UPPER\_CASE\_WITH\_UNDERSCORES};
	\item \texttt{CapitalizedWords} (o \texttt{CapWords});
	\item \texttt{mixedCase};
	\item \texttt{Capitalized\_Words\_With\_Underscores}.
\end{itemize}

\paragraph*{Convenzioni per la documentazione: }


\subfile{Codifica/Javascript.tex}
\subfile{Codifica/React.tex}
\subfile{Codifica/Solidity.tex}
\subfile{Codifica/Scss.tex}

\subsection{Progettazione}

\end{document}