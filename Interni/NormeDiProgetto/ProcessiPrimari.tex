\documentclass[NormeDiProgetto.tex]{subfiles}
\begin{document}
\chapter{Processi primari}
\section{Fornitura}
In questa sezione vengono trattate le norme che i membri del gruppo \gruppo sono tenuti a rispettare al fine di proporsi e diventare fornitori nei confronti della Proponente Red Babel e dei committenti \Vardanega e \Cardin nell'ambito della progettazione, sviluppo e consegna del prodotto \progetto.


% TODO Rimuovere studio di fattibilità descrivendo cosa produrremo come grupo per la fine di questo periodo


\subsection{Studio di fattibilità}
In seguito alla presentazione ufficiale dei \textbf{Capitolati d'appalto} avvenuta Venerdì 10 novembre 2017 alle ore 10.30 presso l'aula 1C150 di Torre Archimede è stata convocata una riunione interna al gruppo per discutere in merito alla varie proposte presentate.\\
Una volta stabilita la scelta del capitolato per il quale proporsi come fornitori, gli analisti hanno condotto un'ulteriore e approfondita attività di analisi dei rischi e delle opportunità culminata con la redazione del documento \sdf \vruno. Tale documento include le motivazioni che hanno portato il gruppo \gruppo a proporsi come fornitore per il prodotto indicato e riporta per ciascun capitolato:
\begin{itemize}
	\item \textbf{Descrizione generale:} una sintesi del prodotto da sviluppare secondo quanto stabilito dal \citGloss{capitolato d'appalto};
	\item \textbf{Obbiettivo finale:} rappresenta il dominio Applicativo, cioè l'ambito di utilizzo del prodotto da sviluppare;
	\item \textbf{Tecnologie richieste:} rappresenta il dominio Tecnologico richiesto dal capitolato, raggruppando le tecnologie da impiegare nello sviluppo del progetto;
	\item \textbf{Valutazione finale:} racchiude le motivazioni, i rischi, le criticità evidenziate per le quali il capitolato in questione è stato respinto o accettato.
\end{itemize}

\section{Sviluppo}
\subsection{Analisi dei Requisiti}
L'analisi dei requisiti ha l'obbiettivo di individuare ed elencare tutti i \citGloss{requisiti} del capitolato. I requisiti possono essere estrapolati da più fonti:
\begin{itemize}
	\item \textbf{Capitolato d'appalto};
	\item \textbf{Verbali di riunioni esterne};
	\item \textbf{Casi d'uso}.
\end{itemize}
Il risultato di quest'analisi sarà il documento \adr \vruno redatto dagli \alisti, dopo essere stato verificato, al fine di:
\begin{itemize}
\item Descrivere l'obiettivo del lavoro del gruppo;
\item Fornire ai Progettisti riferimenti precisi ed affidabili;
\item Fissare le funzionalità e i \citGloss{requisiti} concordati col cliente;
\item Fornire una base per raffinamenti successivi al fine di garantire un
miglioramento continuo del prodotto e del processo di sviluppo;
\item Facilitare le revisioni del codice;
\item Fornire ai verificatori riferimenti per l'attività di test circa i casi d'uso principali e alternativi;
\item Stimare i costi.
\end{itemize}
Ogni requisito dovrà essere meno ambiguo possibile e rispettare le seguenti norme.

\subsubsection{Classificazione dei requisiti}
Ogni requisito è identificato da un codice, costruito come descritto di seguito:\\ \\
\textbf{\centerline{R[Importanza][Classificazione][Identificativo]}}\\
	\begin{itemize}
	\item La prima lettera (R) è l'abbreviazione di requisito;
	\item Il secondo valore indica l'\textbf{importanza}. Assume il valore:
		\begin{itemize}
				\item \textbf{Zero (0):} indica un requisito obbligatorio, il cui soddisfacimento
				dovrà necessariamente avvenire per garantire le funzioni base del
				sistema;
				\item \textbf{Uno (1):} se si tratta di un requisito desiderabile, cioè un requisito il cui
				soddisfacimento può dare maggiore completezza al sistema ma il
				non soddisfarlo non pregiudica alcuna funzione di base;
				\item \textbf{Due (2):} indica un requisito opzionale.
		\end{itemize}
	\item La terza lettera indica la \textbf{classificazione}. Assume valore F se si tratta di un Requisito funzionale, Q se di qualità, P se prestazionale e V se di vincolo;
	\item L'\textbf{identificativo} indica invece un numero progressivo.
	\end{itemize}

\subsubsection{Classificazione casi d'uso} Gli \alisti hanno anche il compito di identificare i casi d'uso, elencandoli con un grado di precisione che va dal generico verso il dettaglio. Ogni caso d'uso è descritto dalla seguente struttura:\\
\begin{itemize}
	\item \textbf{Codice identificativo e nome:} ogni caso d'uso è identificato da una serie di cifre separate dal punto. L'ultima cifra indica il numero del figlio, la penultima cifra indica il numero del padre e UC è l'abbreviazione di Use Case (caso d'uso in inglese). Queste cifre sono seguite dopo il trattino (-) dal nome del caso d'uso: \\\\
	\centerline{\textbf{UC[Codice padre].[Codice figlio] - Nome}}
	\item \textbf{Attori:} indica gli attori principali (ad esempio l'utente generico) e
	secondari (ad esempio ufficio universitario) del caso d'uso;
	\item \textbf{Scopo e descrizione:} riporta una breve descrizione del caso d'uso;
	\item \textbf{Scenario principale:} rappresenta il flusso degli eventi come lista
	numerata, specificando per ciascun evento il caso d'uso di riferimento;
	\item \textbf{Precondizione:} specifica le condizioni che sono identificate come vere
	prima del verificarsi degli eventi del caso d'uso;
	\item \textbf{Postcondizione:} specifica le condizioni che sono identificate come
	vere dopo il verificarsi degli eventi del caso d'uso;
	\item \textbf{Inclusioni (se presenti):} usate per non descrivere più volte lo stesso flusso di eventi,
	inserendo il comportamento comune in un caso d'uso a parte;
	\item \textbf{Estensioni (se presenti):} descrivono i casi d'uso che non fanno parte del flusso
	principale degli eventi, allo stesso modo di quanto descritto in “Scenario
	principale”.
\end{itemize}

\subsection{Progettazione}

\subsubsection{Scopo}
L'attività di Progettazione consiste nel descrivere una soluzione del problema
che sia soddisfacente per tutti gli stakeholders.
I \progi hanno il compito di svolgere tale attività, definendo l'architettura logica del prodotto identificando componenti chiare, riusabili e coese rimanendo nei costi fissati. La progettazione verrà fatta in due momenti: una prima parte ad alto livello sarà fatta durante il periodo di Progettazione della base tecnologica, dove verranno studiati i design pattern che potrebbero essere utilizzati durante il periodo seguente di Progettazione di dettaglio e codifica, durante il quale la progettazione diverrà atomica per ogni componente del sistema, in modo che i \progri possano sviluppare il codice eseguendo task focalizzate e singolari.\\
L'architettura definita dovrà avere le seguenti qualità:
\begin{itemize}
	\item \textbf{Sufficienza:} deve soddisfare i \citGloss{requisiti} definiti nel documento \adr;
	\item \textbf{Comprensibilità:} affinchè possa essere capita dagli stakeholder;
	\item \textbf{Modularità:} deve essere suddivisa in parti chiare e ben distinte, così che possa essere facilmente manutenibile;
	\item \textbf{Robustezza:} deve essere capace di sopportare ingressi diversi, sia da parte di utenti che dall'ambiente;
	\item \textbf{Flessibilità:} deve poter permettere modifiche al variare o all'aggiunta di requisiti senza perdite di performance o doverla restaurare profondamente;
	\item \textbf{Riusabilità:} le sue parti devono poter essere usate in altre applicazioni;
	\item \textbf{Efficienza:} deve essere pensata in modo da poter ridurre gli sprechi di tempo e spazio;
	\item \textbf{Affidabilità:} deve avere una elevata capacità di rispettare le specifiche nel tempo;
	\item \textbf{Disponibilità:} la manutenzione delle sue parti non dovrà inabilitare il funzionamento di tutto il sistema;
	\item \textbf{Sicurezza:} rispetto ad intrusioni e malfunzionamenti;
	\item \textbf{Semplicità:} ogni parte contiene solo il necessario e nulla di superfluo;
	\item \textbf{Incapsulazione:} le componenti dovranno essere progettate in modo che le informazioni interne siano nascoste;
	\item \textbf{Coesione:} in modo che le parti che hanno gli stessi obiettivi stanno insieme;
	\item \textbf{Basso accoppiamento:} parti distinte devono essere poco dipendenti l'una dalle altre, in modo da poter essere facilmente manutenibili.
\end{itemize}

Per svolgere al meglio il processo di progettazione, verrà utilizzato il software PlantUML\footnote{\nURI{http://plantuml.com/}}, che permette di creare diagrammi UML in maniera semplice e veloce. Infatti mette a disposizione strumenti adatti per creare tutti i tipi di diagrammi utili alla progettazione, come i diagrammi delle classi, di sequenza e di attività.
Per fare ciò, verrà seguito lo standard \citGloss{UML} 2.0, documentato nella versione 2.0 di Luglio 2005.

\subsubsection{Diagrammi delle classi}
Lo scopo dei diagrammi delle classi può essere sintetizzato in:
\begin{itemize}
	\item Analisi e progettazione della visione statica di un'applicazione;
	\item Descrizione delle responsabilità di un sistema;
	\item Base per diagrammi dei componenti e di rilascio;
	\item Ingegneria diretta e inversa.
\end{itemize}
Per creare facilmente i diagrammi delle classi, dovrà essere creato un file per ogni classe o raggruppamento di classi che dovrà avere estensione \texttt{.iuml}. In questo modo saranno più facilmente manutenibili e modificabili.\\
Per stabilire le relazioni tra le classi verrà creato un'altro file con estensione \texttt{.puml} in modo da avere il diagramma completo.\\
\subsubsection{React}
Ogni componente React verrà rappresentato attraverso un rettangolo 




\subsubsection{Diagrammi di sequenza}


\subsubsection{Sviluppo}
Lo sviluppo di \progetto avviene seguendo il modello incrementale, spiegato nel dettaglio nel documento \pdp \S 3. Sarà eseguita un'analisi comparativa di varie tecnologie esistenti al fine di capire quali siano le più consone allo sviluppo di \progetto. Nel caso si ritengano più consone tecnologie differenti da quelle proposte dalla proponente, il gruppo \gruppo la contatterà per discuterne insieme.\\
La proponente, tramite capitolato d'appalto, identifica queste tecnologie come obbligatorie per lo sviluppo:\\
\begin{itemize}
	\item \textbf{\citGloss{React};}
	\item \textbf{\citGloss{Redux};}
	\item \textbf{\citGloss{Metamask};}
	\item \textbf{Rete \citGloss{Ethereum}.}
\end{itemize}
\subsubsection{Integrazione}
L'attività di integrazione sarà effettuata utilizzando il servizio, compatibile con \citGloss{GitHub}, \textit{\citGloss{Travis}}.\\
Questo servizio implementa il modello di integrazione continua: ad ogni commit su repository \citGloss{GitHub}, \citGloss{Travis} crea una build ed esegue automaticamente i test di unità. In questo modo sarà possibile identificare immediatamente modifiche od errori che inficiano sul comportamento del prodotto.\\
Per quanto riguarda il test coverage, cioè il grado di esecuzione del codice sorgente durante l'esecuzione dei test, verrà utilizzato il tool \citGloss{JavaScript} \textit{IstanbulJS}. 
\subsubsection{Diagrammi}
Al fine di rendere più chiare le scelte progettuali adottate e
ridurre le possibili ambiguità, sarà necessario fare largo uso di vari tipi di diagrammi 
\textbf{UML 2.0}, realizzandone secondo le seguenti rappresentazioni:
\begin{itemize}
	\item \textbf{Diagrammi dei casi d'uso:} dedicati alla descrizione delle funzioni offerte dal sistema;
	\item \textbf{Diagrammi delle classi:} dedicati alla descrizione degli oggetti che fanno parte di un sistema e delle loro dipendenze;
	\item \textbf{Diagrammi dei package:} dedicati alla descrizione della dipendenza tra classi raggruppate in package;
	\item \textbf{Diagrammi di sequenza:} dedicati a descrivere la collaborazione nel tempo tra un gruppo di oggetti;
	\item \textbf{Diagrammi di attività:} dedicati a descrivere la logica procedurale.
\end{itemize}

\subsubsection{Obiettivi della progettazione}
Essa serve a garantire che il prodotto sviluppato soddisfi le proprietà e i bisogni specificati nell'attività di analisi ponendo i seguenti obiettivi:
\begin{itemize}
	\item Garantire la qualità di prodotto sviluppato, perseguendo la \textit{correttezza
		per costruzione};
	\item Organizzare e ripartire compiti implementativi, riducendo la
	complessità del problema originale fino alle singole componenti
	facilitandone la codifica da parte dei singoli \progri;
	\item Rendere chiara e comprensibile ogni parte dell' architettura ai differenti stakeholders;
	\item Mantenere nascosti i dettagli implementativi, seguendo il principio dell' \textit{information hiding};
	\item Mantenere bassa la dipendenza tra le varie parti del prodotto, seguendo il principio della \textit{modularità};
	\item Ottimizzare l'uso di risorse.
\end{itemize}

\subsection{Codifica}
In questa sotto-sezione vengono elencate le norme alle quali i \progri devono attenersi durante l'attività di programmazione e implementazione.\\
All'inizio verranno elencate delle norme generali a cui i \progri devono attenersi con qualsiasi linguaggio di programmazione utilizzato, mentre di seguito verranno elencate delle norme specifiche per i linguaggi \citGloss{JavaScript}, \citGloss{JavaScript} Syntax eXtension, Solidity e \citGloss{SCSS}.\\
Ogni norma è rappresentata da un paragrafo; ciascuna ha un titolo, una breve descrizione e, se necessario, un esempio che illustri le convenzioni da applicare. Alcune di esse includono inoltre una lista di possibili eccezioni d'uso.\\
L'uso di norme e convenzioni è fondamentale per permettere la generazione di codice leggibile e uniforme, agevolando così le fasi di manutenzione, \citGloss{verifica} e \citGloss{validazione} e migliorando la qualità del prodotto.

\paragraph*{Convenzioni per i nomi: }
\begin{itemize}
	\item Alcuni nomi sono da evitare in quanto facilmente confondibili con i numeri \texttt{1} e \texttt{0}, elencati di seguito:
	\begin{itemize}
		\item \texttt{l} (lettera minuscola elle);
		\item \texttt{O} (lettera maiuscola o);
		\item \texttt{I} (lettera maiuscola i).
	\end{itemize}
	\item Tutti i nomi devono essere \textbf{unici} ed \textbf{esplicativi} al fine di evitare al più possibile ambiguità e comprensione.
\end{itemize}
Per i vari linguaggi verranno successivamente descritte altre norme per nomi di variabili, funzioni e altro facendo riferimento ai seguenti stili:
\begin{itemize}
	\item \texttt{lowercase};
	\item \texttt{lower\_case\_with\_underscores};
	\item \texttt{UPPERCASE};
	\item \texttt{UPPER\_CASE\_WITH\_UNDERSCORES};
	\item \texttt{CapitalizedWords} (o \texttt{CapWords});
	\item \texttt{mixedCase};
	\item \texttt{Capitalized\_Words\_With\_Underscores};
	\item \texttt{lower-case-with-dashes}.
\end{itemize}

\paragraph*{Convenzioni per la documentazione: }
\begin{itemize}
	\item Tutti i nomi e i commenti al codice per la documentazione vanno scritti in \textbf{inglese};
	\item \`{E} possibile utilizzare in un commento la keyword \textbf{TODO} per indicare codice temporaneo e soluzioni a breve termine o evidentemente migliorabili;
	\item Ogni file contenente codice deve avere la seguente \textbf{intestazione} contenuta in un commento e posta all'inizio del file stesso:
\begin{center}{
\begin{minipage}{12cm}
\begin{Verbatim}[frame=single]
File : nome file
Version : versione file
Type : tipo file
Date : data di creazione
Author : nome autore/i
E- mail : email autore/i

License : tipo licenza

Advice : lista avvertenze e limitazioni

Changelog :
Autore || Data || breve descrizione delle modifiche
\end{Verbatim}
\end{minipage}
}
\end{center}
	\item La \textbf{versione} del codice viene inserita all'interno dell'intestazione del file e rispetta il
	seguente formalismo:
	\begin{center}{\textbf{X.Y}}\end{center}	
	\begin{itemize}
		\item \textbf{X}: è l'indice di versione principale, un incremento di tale indice rappresenta un avanzamento della versione stabile, che porta il valore dell'indice Y ad essere azzerato;
		\item \textbf{Y}: è l'indice di modifica parziale, un incremento di tale indice rappresenta una \citGloss{verifica} o una modifica rilevante, come per esempio la rimozione o l'aggiunta di una istruzione.
	\end{itemize}
	La versione \textit{1.0} deve rappresentare la prima versione del file completo e stabile, cioè quando le sue funzionalità obbligatorie sono state definite e si considerano funzionanti. Solo dalla versione \textit{1.0} è possibile testare il file, con degli appositi test predefiniti, per validarne la qualità.
\end{itemize}


\subfile{Codifica/Javascript.tex}
\subfile{Codifica/React.tex}
\subfile{Codifica/Solidity.tex}
\subfile{Codifica/Scss.tex}


\subsection{Procedure}

\subsubsection{Tracciamento componenti-requisiti}
	Si è scelto di utilizzare il software \citGloss{Trender} per il tracciamento automatizzato di tutte le informazioni ricavate dall'analisi dei \citGloss{requisiti}.\\
	Il database MySQL che garantisce la persistenza e l'accesso ai dati si trova su un server remoto, in modo tale da rendere sempre disponibile a tutti le informazioni che esso contiene.\\
	Qualsiasi componente del gruppo, utilizzando la parte \citGloss{front-end} dell'applicazione scritta in PHP e \citGloss{JavaScript}, può accedervi utilizzando l'account comune predisposto.\\
	Una volta effettuato l'accesso si ha a disposizione le funzionalità di modifica e di esportazione in codice \citGloss{LaTeX} dei dati presenti.\\
	
	%VERI_GIORAT aggiunta di Diagramma tracciamento componenti requisiti per rimpiazzare paragrafo precedente.
	
%	\citGloss{Trender} consente di mantenere il tracciamento di:\\
%	\begin{itemize}
%		\item \citGloss{requisiti};
%		\item Casi d'uso;
%		\item Attori;
%		\item Verbali;
%		\item Package;
%		\item Classi;
%		\item Test;
%		\item Voci del glossario.
%	\end{itemize}

\subsection{Strumenti}
Gli elementi indicati con \textbf{*} sono stati analizzati per l'utilizzo nelle fasi successive del progetto ma potrebbero essere cambiati in caso di necessità differenti da quanto pianificato.
\begin{itemize}
	\item \textbf{Tracciamento requisiti:} il gruppo \gruppo ha deciso di utilizzare il software \textbf{Trender}\footnote{\nURI{https://github.com/campagna91/Trender}} installato su server remoto condiviso;
	\item \textbf{Creazione diagrammi UML:} {UML Designer}\footnote{\nURI{http://www.umldesigner.org/}};
	\item \textbf{Scrittura del codice:} Webstorm by Jetbrains\footnote{\nURI{https://www.jetbrains.com/webstorm/}} per \citGloss{JavaScript} e Visual Studio Code\footnote{\nURI{https://code.visualstudio.com/}} per \citGloss{Solidity}; 
	\item \textbf{Toolkit sviluppo:} si utilizzano i pacchetti installabili tramite Node.js con gestore di pacchetti npm per risolvere dipendenze delle librerie e tool richiesti;
	\item \textbf{Esecuzione dei test:}
	\begin{itemize}
		\item \citGloss{React}: Mocha\footnote{\nURI{https://github.com/mochajs/mocha}}, Chai\footnote{\nURI{http://chaijs.com/}} e Enzyme\footnote{\nURI{https://github.com/airbnb/enzyme}};
		\item \citGloss{Redux}: Mocha e Chai;
		\item \citGloss{Solidity}: Mocha e Chai.
	\end{itemize}
	\item \textbf{Continuous Integration:} Travis\footnote{\nURI{http://travis-ci.com/}};
	\item \textbf{Tracciamento e copertura dei test:} 
	\begin{itemize}
		\item \textbf{Generazione copertura React e Redux:} Istanbul.JS \footnote{\nURI{https://istanbul.js.org}};
		\item \textbf{Generazione copertura Solidity:} Solidity-Coverage\footnote{\nURI{https://github.com/sc-forks/solidity-coverage}};
		\item \textbf{Visione online copertura test:} Codeclimate\footnote{\nURI{https://codeclimate.com/}}.
	\end{itemize}
	\item \textbf{Tool di supporto:} Scss-lint\footnote{\nURI{https://github.com/brigade/scss-lint}}. 	
\end{itemize}


\end{document}