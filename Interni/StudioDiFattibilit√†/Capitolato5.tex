\documentclass[StudioDiFattibilità.tex]{subfiles}
\begin{document}
\chapter{Capitolato C5 - IronWorks}
\section{Descrizione generale}
Il capitolato C5 propone lo sviluppo di un software
per la costruzione di robustness diagram UML e successiva conversione in codice Java.
\section{Obiettivo finale}
L'obiettivo finale consiste nella creazione di un'applicazione in grado di disegnare i diagrammi UML di "robustezza"(robustness diagram) e di generare il codice nel linguaggio di programmazione Java del diagramma creato, garantendo l'aumento della velocità di produzione e la qualità del software prodotto.
\section{Tecnologie richieste}
\begin{itemize}
	\item \textbf{HTML}, \textbf{CSS}, \textbf{Javascript}, stack standard per lo sviluppo frontend;
	\item \textbf{Apache Tomcat}, webserver per la realizzazione di applicazioni Java;
	\item \textbf{Node.js}, linguaggio per la generazione di applicativi server in Javascript, per il progetto è necessario scegliere tra queste due tecnologie.
\end{itemize}
\section{Valutazione finale}
Il capitolato è stato accolto positivamente dai membri del gruppo \gruppo: l'apparente bassa complessità di sviluppo e l'idea di poter avere una base di partenza "standardizzata" per la fase iniziale dello sviluppo di nuove applicazioni hanno suscitato grande interesse. Il capitolato non è stato scelto in quanto non presenta tecnologie innovative che possono fornire nuove capacità e conoscenze.

\end{document}