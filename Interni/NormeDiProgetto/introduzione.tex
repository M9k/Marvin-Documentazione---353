\documentclass[NormeDiProgetto.tex]{subfiles}

\begin{document}

\chapter{Introduzione}

\section{Scopo del documento}
Lo scopo di questo documento è fissare tutte le regole e le procedure fondamentali per assicurare al gruppo un modo di lavorare comune, le quali verranno discusse in tutte le sezioni seguenti.\\
Ciò garantirà una collaborazione efficiente tra tutti i diversi membri del gruppo.\\
Saranno inoltre elencati e discussi tutti gli strumenti software scelti internamente al gruppo per rispettare le regole e per attuare le procedure.

\section{Scopo del prodotto}
Lo scopo del prodotto è realizzare una piattaforma web chiamata \progetto che simuli le funzionalità di base per studenti, docenti e università di \citGloss{Uniweb}.\\
L'applicativo dovrà utilizzare al posto del database la rete \citGloss{Ethereum} interagendo con degli \citGloss{smart contract}.

\glossExpl

\section{Riferimenti utili}
\subsection{Riferimenti normativi}
\begin{itemize}
	\item Standard ISO/IEC 12207:1995\\ \nURI{http://www.math.unipd.it/~tullio/IS-1/2009/Approfondimenti/ISO\_12207-1995.pdf}
\end{itemize}
\subsection{Riferimenti informativi}
\begin{itemize}
	\item \textbf{\pdp \vrquattro}
	\item \textbf{\pdq \vrquattro}
	\item \textbf{Airbnb JavaScript Style Guide}\\
	\nURI{https://github.com/airbnb/javascript}
	\item \textbf{Documentazione officiale di Solidity}\\
	\nURI{http://solidity.readthedocs.io/en/develop/style-guide.html}
	\item \textbf{Airbnb CSS/Sass Style Guide}\\
	\nURI{https://github.com/airbnb/css}
	\item \textbf{Sito del corso di Ingegneria del Software - Regolamento Organigramma}\\
	\nURI{http://www.math.unipd.it/~tullio/IS-1/2017/Progetto/RO.html}
	\item \textbf{Sito del corso di Ingegneria del Software - Analisi dei Requisiti}\\
	\nURI{http://www.math.unipd.it/~tullio/IS-1/2017/Dispense/L08.pdf}\\
	\item \textbf{\textit{UML 2.0}}\\
	\nURI{https://www.omg.org/spec/UML/2.0/Superstructure/PDF}
\end{itemize}

\end{document}