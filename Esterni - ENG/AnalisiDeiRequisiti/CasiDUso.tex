\documentclass[AnalisiDeiRequisiti.tex]{subfiles}

\begin{document}

\chapter{Casi d'uso}
\section{Attori dei casi d'uso}
\subsection{Attori primari}
\begin{enumerate}
	\begin{figure}[h]
		\centering
		\includegraphics[width=0.8\linewidth]{attoriPrincipali.jpg}
		\caption{Gerarchia attori primari}
		\label{fig:Gerarchia attori primari}
	\end{figure}
	
	\item \textbf{Utente generico}\\
	Si riferisce ad un utente generico che accede al sito\\
	
	\item \textbf{Utente non autenticato}\\
	Ci si riferisce ad un utente generico con non ha ancora effettuato il login.\\
	
	\item \textbf{Utente autenticato}\\
	 Ci si riferisce ad un utente generico con chiave valida ed autenticato nel sistema tramite la procedura di login.\\
	
	\item \textbf{Amministratore}\\
	Ci si riferisce ad un utente autenticato nel sistema nel ruolo di amministratore\\
	
	\item \textbf{Professore}\\
	Ci si riferisce ad un utente autenticato nel sistema nel ruolo di professore.\\
	
	\item \textbf{Studente}\\
	Ci si riferisce ad un utente autenticato nel sistema nel ruolo di studente\\	
\end{enumerate}

\subsection{Attori secondari}
\begin{enumerate}
	\item \textbf{MetaMask}\\
	Plugin del browser MetaMask per interfacciarsi ad una rete Ethereum.\\
	
	\item \textbf{Ufficio Universitario}\\
	Entità fisica che consente l'immatricolazione e la registrazione dei professori.\\
\end{enumerate}

\section{Elenco dei casi d'uso}

\begin{figure}[H]
	\centering
	\includegraphics[width=0.8\linewidth]{UC.jpg} %TODO: cambiare relativa ai corsi a relativa agli esami
	\caption{Casi d'uso basilari}
	\label{fig:Casi d'uso basilari}
\end{figure}

%   -------------------------------------------------------------------------------------
%   ----------                 MODELLO PER GLI USER CASE              -------------------
%   -------------------------------------------------------------------------------------
\begin{comment}
\subsection{UCX - Nome}
\begin{itemize}
	\item \textbf{Attori primari:} ;\\
	\item \textbf{Attori secondari:} ;\\
	\item \textbf{Scopo e descrizione:} ;\\
	\item \textbf{Scenario principale:} ;\\
	\item \textbf{Scenario alternativo:} ;\\
	\item \textbf{Flusso principale degli eventi:}\\
	\begin{enumerate}
		\item L'utente... ;
		\item L'utente... ;
	\end{enumerate}
	\item \textbf{Estensioni:}\\
	\begin{enumerate}
		\item Se l'utente... ;[UCX.X.X]
	\end{enumerate}
	\item \textbf{Precondizione:} ;\\
	\item \textbf{Postcondizione:} .\\
\end{itemize}
\end{comment}
%
\subsection{UC1 - Breve guida}
\begin{itemize}
	\item \textbf{Attori primari:} Utente generico;\\
	\item \textbf{Scopo e descrizione:} L'utente visualizza una breve guida di introduzione su come installare il plugin MetaMask e su come gestire le chiavi in modo da istruirlo sulle modalità di accesso al sistema;\\
	\item \textbf{Scenario principale:} L'utente accede alla guida;\\
	\item \textbf{Precondizione:} Il sistema è raggiungibile e funzionante e l'utente desidera aprire la guida;\\
	\item \textbf{Postcondizione:} L'utente ha avuto delle nozioni riguardanti l'accesso al sistema.\\
\end{itemize}
\subsection{UC2 - Login}
\begin{itemize}
	\item \textbf{Attori primari} Utente non autenticato;\\
	\item \textbf{Scopo e descrizione:} L'utente richiede il login al sistema attraverso il plugin MetaMask;\\
	\item \textbf{Scenario principale:} L'utente non ancora riconosciuto dal sistema effettua il login;\\
	\item \textbf{Precondizione:} L'utente non è stato riconosciuto dal sistema;\\
	\item \textbf{Postcondizione:} L'utente viene riconosciuto da parte del sistema.\\
\end{itemize}

\begin{figure}[H]
	\centering
	\includegraphics[width=1.0\linewidth]{UC2.jpg}
	\caption{UC2 - Login}
	\label{fig:UC2 - Login}
\end{figure}

\subsection{UC2.1 - Login automatico}
\begin{itemize}
\item \textbf{Attori primari} Utente non autenticato;\\
\item \textbf{Attori secondari:} MetaMask;\\
\item \textbf{Scopo e descrizione:} L'utente attende il login da parte del sistema senza effettuare nessuna operazione aggiuntiva;\\
\item \textbf{Scenario principale:} L'utente non ancora riconosciuto dal sistema richiede il login;\\
\item \textbf{Estensioni:}\\
\begin{enumerate}
	\item Se l'utente non ha a disposizione una chiave viene avvisato con un errore a riguardo [UC2.2];
	\item Se l'utente ha a disposizione una chiave malformata viene avvisato con un errore a riguardo [UC2.3];
	\item Se l'utente ha a disposizione una chiave non registrata viene avvisato con un errore a riguardo [UC2.4];
\end{enumerate}
\item \textbf{Precondizione:} L'utente ha richiesto al sistema di venire riconosciuto;\\
\item \textbf{Postcondizione:} L'utente viene riconosciuto da parte del sistema.\\
\end{itemize}
\subsection{UC2.2 - Visualizzazione messaggio di errore riguardo a chiave assente}
\begin{itemize}
	\item \textbf{Attori primari:} Utente non autenticato;\\
	\item \textbf{Scopo e descrizione:} L'utente viene avvisato del fatto che non ha fornito nessuna chiave al sistema;\\
	\item \textbf{Scenario principale:} L'utente visualizza l'errore relativo all'assenza di una chiave per accedere al sistema;\\
	\item \textbf{Precondizione:} L'utente richiede il login senza fornire una chiave;\\
	\item \textbf{Postcondizione:} L'utente è consapevole di dover fornire una chiave. \\
\end{itemize}
\subsection{UC2.3 - Visualizzazione messaggio di errore riguardo a chiave malformata}
\begin{itemize}
	\item \textbf{Attori primari:} Utente non autenticato;\\
	\item \textbf{Scopo e descrizione:} L'utente viene avvisato del fatto che ha fornito una chiave malformata, quindi con lunghezza invalida o contenente caratteri non permessi;\\
	\item \textbf{Scenario principale:} L'utente visualizza il messaggio d'errore relativo alla chiave malformata;\\
	\item \textbf{Precondizione:} L'utente richiede il login utilizzando una chiave malformata;\\
	\item \textbf{Postcondizione:} L'utente è consapevole di cambiare la chiave o sistemare quella fornita.\\
\end{itemize}
\subsection{UC2.4 - Visualizzazione messaggio di errore riguardo a chiave non registrata}
\begin{itemize}
	\item \textbf{Attori primari:} Utente non autenticato;\\
	\item \textbf{Scopo e descrizione:} L'utente viene avvisato del fatto che ha fornito una chiave non registrata nel sistema;\\
	\item \textbf{Scenario principale:} L'utente viene informato che la sua chiave non risulta registrata, suggerendone la registrazione;\\
	\item \textbf{Precondizione:} L'utente richiede il login utilizzando una chiave non registrata;\\
	\item \textbf{Postcondizione:} L'utente è consapevole di dover effettuare la registrazione per accedere al sistema.\\
\end{itemize}

\subsection{UC3 - Registrazione}

\subsection{UC4 - Logout}

\subsection{UC5 - Amministrazione}

\subsection{UC6 - Gestione aspetti relativi agli esami}
\begin{itemize}
	\item \textbf{Attori primari:} Professore;\\
	\item \textbf{Scopo e descrizione:} L'utente è già riconosciuto dal sistema come professore e sceglie di utilizzare una funzionalità messa a sua disposizione da parte del sistema;\\
	\item \textbf{Precondizione:} Il professore già riconosciuto dal sistema desidera effettuare delle operazioni relative agli esami a lui assegnati;\\
	\item \textbf{Postcondizione:} Il professore ha visualizzato le operazioni messe a disposizione dal sistema e sceglie quale effettuare.\\
\end{itemize}

%TODO grafico UC6

\subsection{UC6.1 - Visualizzazione lista degli esami}
\begin{itemize}
	\item \textbf{Attori primari:} Professore;\\
	\item \textbf{Scopo e descrizione:} Il professore visualizza una lista di tutti gli esami ai quali è assegnato;\\
	\item \textbf{Scenario principale:} Il professore richiede al sistema la lista degli esami di sua competenza per poterla consultare;\\
	\item \textbf{Precondizione:} L'utente è già riconosciuto dal sistema come professore e richiede la visualizzazione della lista degli esami di sua competenza;\\
	\item \textbf{Postcondizione:} Il professore ottiene la lista degli esami ai quali è assegnato per poterla consultare.\\
\end{itemize}

\subsection{UC6.2 - Visualizzazione lista degli studenti}
\begin{itemize}
	\item \textbf{Attori primari:} Professore;\\
	\item \textbf{Scopo e descrizione:} Il professore visualizza una lista di tutti gli studenti iscritti ad un esame a lui assegnato;\\
	\item \textbf{Scenario principale:} Il professore richiede al sistema la lista degli studenti iscritti ad un esami di sua competenza per poterla consultare;\\
	\item \textbf{Flusso principale degli eventi:}\\
	\begin{enumerate}
		\item Il professore visualizza la lista degli esami di sua competenza [UC6.1];
		\item Il professore, una volta individuato l'esame al quale è interessato, ne richiede la lista degli studenti registrati;
		\item Il professore consulta la lista degli studenti dell'esame richiesto;
	\end{enumerate}
	\item \textbf{Precondizione:} L'utente è già riconosciuto dal sistema come professore e richiede la visualizzazione della lista degli studenti iscritti a un determinato esame di sua competenza;\\
	\item \textbf{Postcondizione:} Il professore ottiene la lista degli studenti iscritti all'esame richiesto per poterla consultare.\\
	%TODO discutere: nessuna estensione, ha la possibilità di richiederlo solo ai suoi corsi, e se manomette il JS comunque non accede a informazioni segrete
\end{itemize}

\subsection{UC6.3 - Registrazione valutazione di un esame}
\begin{itemize}
	\item \textbf{Attori primari:} Professore;\\
	\item \textbf{Scopo e descrizione:} Il professore inserisce nel sistema una valutazione;\\
	\item \textbf{Flusso principale degli eventi:}\\
	\begin{enumerate}
		\item Il professore visualizza la lista degli esami di sua competenza [UC6.1];
		\item Il professore, una volta individuato l'esame al quale è interessato, ne richiede la lista degli studenti registrati [UC6.2];
		\item Il professore consulta la lista degli studenti ed individua la persona alla quale deve inserire la valutazione;
		\item Il professore inserisce la valutazione allo studente;
	\end{enumerate}
	\item \textbf{Precondizione:} Il professore possiede una valutazione da assegnare a uno studente riguardante un determinato esame e desidera registrarlo;\\
	\item \textbf{Postcondizione:} Il voto è stato inserito nella blockchain universitaria.\\
%TODO discutere: nessuna estensione, vede solamente gli studenti dei suoi corsi, non può assegnare voti a studenti a caso di corsi a caso, se cerca di forzarlo modificando il js verrà bloccato dal contratto
\end{itemize}


\subsection{UC7 - Gestione aspetti relativi allo studente}
\begin{itemize}
	\item \textbf{Attori primari:} Studente;\\
	\item \textbf{Scopo e descrizione:} L'utente è già riconosciuto dal sistema come studente e sceglie di utilizzare una funzionalità messa a sua disposizione da parte del sistema;\\
	\item \textbf{Precondizione:} Lo studente già riconosciuto dal sistema desidera effettuare delle operazioni relative alla scelta di esami opzionali o alla visione delle valutazioni;\\
	\item \textbf{Postcondizione:} Lo studente ha visualizzato le operazioni messe a disposizione dal sistema e sceglie quale effettuare.\\
\end{itemize}

\subsection{UC7.1 - Visualizzazione delle informazione degli esami ai quali è iscritto}
\begin{itemize}
	\item \textbf{Attori primari:} Studente;\\
	\item \textbf{Scopo e descrizione:} Lo studente visualizza una lista di tutti gli esami ai quali è iscritto;\\
	\item \textbf{Scenario principale:} Lo studente richiede al sistema la lista degli esami ai quali è iscritto per poterla consultare;\\
	\item \textbf{Precondizione:} L'utente è già riconosciuto dal sistema come studente e richiede la visualizzazione della lista degli esami ai quali è iscritto ed le relative informazioni;\\
	\item \textbf{Postcondizione:} Lo studente ottiene la lista per poterla consultare.\\
\end{itemize}

\subsection{UC7.1.1 - Visualizzazione dei crediti degli esami ai quali è iscritto}
\begin{itemize}
\item \textbf{Attori primari:} Studente;\\
\item \textbf{Scopo e descrizione:} Lo studente visualizza il numero di crediti per ogni esame al quale è iscritto;\\
\item \textbf{Scenario principale:} Lo studente richiede al sistema il numero di crediti degli esami ai quali è iscritto per consultarli;\\
\item \textbf{Precondizione:} L'utente è già riconosciuto dal sistema come studente e richiede il numero di crediti per ogni esame ai quale è iscritto;\\
\item \textbf{Postcondizione:} Lo studente ottiene le informazioni richieste.\\
\end{itemize}

\subsection{UC7.1.2 - Visualizzazione della obbligatorietà degli esami ai quali è iscritto}
\begin{itemize}
	\item \textbf{Attori primari:} Studente;\\
	\item \textbf{Scopo e descrizione:} Lo studente visualizza il numero di crediti di per ogni esame al quale è iscritto;\\
	\item \textbf{Scenario principale:} Lo studente richiede al sistema il numero di crediti degli esami ai quali è iscritto per consultarli;\\
	\item \textbf{Precondizione:} L'utente è già riconosciuto dal sistema come studente e richiede il numero di crediti per ogni esame ai quale è iscritto;\\
	\item \textbf{Postcondizione:} Lo studente ottiene le informazioni richieste.\\
\end{itemize}

\subsection{UC7.1.3 - Visualizzazione delle valutazioni degli esami ai quali è iscritto}
\begin{itemize}
	\item \textbf{Attori primari:} Studente;\\
	\item \textbf{Scopo e descrizione:} Lo studente visualizza la valutazione, in formato numerico con l'indicazione di averlo superato se il voto è superiore a 18, per ogni esame al quale è iscritto;\\
	\item \textbf{Scenario principale:} Lo studente richiede al sistema le informazioni riguardanti le valutazioni degli esami ai quali è iscritto per consultarle;\\
	\item \textbf{Precondizione:} L'utente è già riconosciuto dal sistema come studente e richiede le informazioni riguardanti le valutazioni degli esami ai quali è iscritto;\\
	\item \textbf{Postcondizione:} Lo studente ottiene le informazioni richieste.\\
\end{itemize}

\subsection{UC7.2 - Visualizzazione degli esami opzionali e dei loro crediti}
\begin{itemize}
	\item \textbf{Attori primari:} Studente;\\
	\item \textbf{Scopo e descrizione:} Lo studente visualizza una lista di tutti gli esami opzionali ai quali ha possibilità di iscriversi ed i relativi crediti;\\
	\item \textbf{Scenario principale:} Lo studente richiede al sistema la lista di tutti gli esami opzionali ai quali ha possibilità di iscriversi per poterla consultare con i relativi crediti;\\
	\item \textbf{Precondizione:} L'utente è già riconosciuto dal sistema come studente e richiede la visualizzazione della lista di tutti gli esami opzionali ai quali ha possibilità di iscriversi;\\
	\item \textbf{Postcondizione:} Lo studente ottiene la lista per poterla consultare.\\
\end{itemize}

\subsection{UC7.3 - Registrazione ad un esame opzionale}
\begin{itemize}
	\item \textbf{Attori primari:} Studente;\\
	\item \textbf{Scopo e descrizione:} Lo studente si iscrive a un determinato esame opzionale;\\
	\item \textbf{Flusso principale degli eventi:}\\
	\begin{enumerate}
		\item Lo studente visualizza la lista degli esami opzionali [UC7.2];
		\item Lo studente individua l'esame al quale è interessato iscriversi;
		\item Lo studente compie l'operazione di iscrizione;
	\end{enumerate}
	\item \textbf{Precondizione:} Lo studente ha individuato l'esame opzionale al quale desidera iscriversi;\\
	\item \textbf{Postcondizione:} Lo studente inserisce nella blockchain universitaria l'iscrizione all'esame.\\
\end{itemize}

\subsection{UC7.4 - Visualizzazione delle informazioni relative ai crediti}
\begin{itemize}
	\item \textbf{Attori primari:} Studente;\\
	\item \textbf{Scopo e descrizione:} Lo studente visualizza un riepilogo del numero dei crediti che possiede e dell'obiettivo per poter conseguire la laurea;\\
	\item \textbf{Scenario principale:} Lo studente richiede al sistema le informazioni riguardanti i suoi crediti;\\
	\item \textbf{Precondizione:} L'utente è già riconosciuto dal sistema come studente e richiede la visualizzazione delle informazioni relative ai suoi crediti;\\
	\item \textbf{Postcondizione:} Lo studente ottiene le informazioni richieste per poterle consultare.\\
\end{itemize}

\end{document}