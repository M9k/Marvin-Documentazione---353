\documentclass[VER-2018-01-09.tex]{subfiles}
\begin{document}
\taburowcolors[2] 2{tableLineOne .. tableLineTwo}
\tabulinesep = ^3mm_2mm
\chapter{Informazioni sulla riunione}
\begin{itemize}
	\item \textbf{Motivo della riunione:} l'incontro è stato effettuato dopo il seminario "Il ciclo di sviluppo di \citGloss{DAPP} in \citGloss{Ethereum}" tenuto da \Proponente presso l'aula 1C150 di Torre Archimede.

	\item \textbf{Luogo e data:} Aula 1C150, Torre Archimede, Via Trieste 63, Padova 08-12-2107;
	\item \textbf{Ora di inizio:} 13:20;
	\item \textbf{Ora di fine:} 13:50;
	\item \textbf{Partecipanti:}
	\begin{itemize}
		\item Proponente:
		\begin{itemize}
			\item Alessandro Maccagnan;
		\end{itemize}
		\item Membri del gruppo 353:
		\begin{itemize}
			\item \Davide;
			\item \Elena;
			\item \Gianluca;
			\item \Mirco;
			\item \Parwinder;
			\item \Riccardo;
			\item \Valentina.
		\end{itemize}
	\end{itemize}
\end{itemize}
\chapter{Ordine del giorno}	
Di seguito sono riportati i punti dell'ordine del giorno che sono stati discussi insieme alla Proponente:
\begin{enumerate}
	\item Visione bozza contratto \citGloss{Solidity}; 
	\item Documentazione 12 Factors App;
	\item Utilizzo tecnologie differenti da boilerplate \citGloss{Redux} Minimal.
\end{enumerate}
\chapter{Resoconto}
\begin{enumerate}
	\item \textbf{Visione contratto Solidity:} l'incontro si è aperto con il gruppo \gruppo che ha esposto una bozza di contratto \citGloss{Solidity} per discutere del costo del caricamento di una serie di contratti su \citGloss{blockchain} \citGloss{Ethereum};
	
	\item \textbf{Documentazione 12 Factors App:} è stato chiarito che le 12 Factors App che dovranno solamente essere rispettate, documentando brevemente quali fattori abbiamo potuto applicare essendo in presenza di un'applicazione decentralizzata;
	
	\item \textbf{Tool differenti da Redux Minimal:} è stato affermato che nel corso dello sviluppo il gruppo potrà utilizzare tool e/o librerie differenti da quelli presenti su Redux Minimal.\\
	Questo sarà possibile dopo aver comunicato il cambio di tool alla proponente su \citGloss{Slack}, la quale comunicherà l'approvazione o bocciatura della modifica.
\end{enumerate}
\section{Argomenti discussi ma assenti dall'ordine del giorno}
\begin{itemize}
	\item \textbf{Pagina prezzi:} creazione di una pagina prezzi per mostrare il costo di operazioni lato amministratore, professori come aggiunta corsi, aggiunta responsabili con prezzi in ETH e in EUR, già individuato precedentemente e introdotto come UC9 dal requisito R0F10.
\end{itemize}
\section{Tracciamento delle decisioni}
\begin{table}[H]
	\begin{center}
		\begin{tabu} to \textwidth {
				>{\centering}m{0.25\linewidth}  
				>{\centering\arraybackslash}m{0.7\linewidth}
			}
			\tableHeaderStyle
			\textbf{Codice} & \textbf{Decisione} \\
			VER-2018-01-09.1 & 12 Factors app sarà documentato brevemente dal gruppo motivando il superamento dei fattori e le procedure per garantirli. \\
			VER-2018-01-09.2 & Durante lo sviluppo saranno possibile modifiche agli strumenti, librerie utilizzati per lo sviluppo previa approvazione su canale \citGloss{Slack} proponente. \\
		\end{tabu}
	\caption{Tracciamento delle decisioni del verbale}
	\end{center}
\end{table}
\end{document}