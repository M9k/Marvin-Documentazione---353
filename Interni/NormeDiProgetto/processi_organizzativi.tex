\documentclass[NormeDiProgetto.tex]{subfiles}

\begin{document}
	\chapter{Processi organizzativi}

	\section{Processi di coordinamento}
	
	\subsection{Comunicazione}
	In questa sezione vengono illustrate le norme che regolano la comunicazione sia tra i membri del gruppo 353 che con entità esterne, come committenti e Proponenti.
	\subsubsection{Comunicazioni interne}
	Le comunicazioni interne al gruppo vengono effettuate tramite \citGloss{Slack}, al cui interno sono stati predisposti dei canali tematici, suddivisi per argomento in questo modo:

	All'interno di \citGloss{Slack} sono stati predisposti dei canali tematici, suddivisi per argomento in questo modo:
	\begin{itemize}
		\item \textbf{\#General}: per discutere tutto ciò che riguarda l'organizzazione generale del progetto, la scelta degli strumenti di lavoro e per prendere decisioni in modo rapido. Inoltre qui vengono decisi gli argomenti principali da discutere nelle riunioni;
		\item \textbf{\#Calendario\underline{ }meeting}: all'interno del quale un \citGloss{bot} notifica giornalmente gli eventi facenti parte del calendario comune, come riunioni interne/esterne o eventi di formazione;
		\item \textbf{\#Daily\underline{ }standup}: il canale dedicato a un \citGloss{bot} che giornalmente chiede ai partecipanti notizie sulle attività svolte nel giorno precedente e quelle che sono programmate per il giorno corrente. Inoltre viene chiesto se si sono verificati problemi nelle attività in corso, così da aiutare il gruppo a organizzarsi e a concentrare gli sforzi su particolari attività;
		\item \textbf{\#Github\underline{ }notifications}: il canale creato affinché un \citGloss{bot} invii una notifica ogni qualvolta un membro del gruppo effettua un operazione su Github;
		\item \textbf{\#Gmail353}: il canale creato affinché un \citGloss{bot} invii una notifica ogni qualvolta la casella email del team riceva un messaggio;
		\item \textbf{\#Random}: Un canale generico, dove il gruppo discute liberamente di questioni non riguardanti il progetto e il suo sviluppo.	
	\end{itemize}

	Sono stati creati anche dei canali appositi per gestire meglio la stesura dei documenti, nello specifico:
	\begin{itemize}
		\item \textbf{\#Analisi\underline{ }requisiti}: per commentare i casi d'uso e i requisiti necessari per la stesura del documento \adr;
		\item \textbf{\#Piano\underline{ }progetto}: per confrontarsi sulla ripartizione dei ruoli e del monte ore a essi associato da includere nel \pdp;
		\item \textbf{\#Piano\underline{ }qualifica}: per discutere riguardo gli obiettivi e le strategie da applicare per garantire qualità e gestire \citGloss{verifica} e \citGloss{validazione}.
	\end{itemize}
	
	Sono stati creati i seguenti canali appositi per discutere di problemi e tenere traccia del lavoro svolto per le tecnologie più fondamentali:
	\begin{itemize}
		\item \textbf{\#React:} per discutere sulla tecnologia React;
		\item \textbf{\#Redux:} per discutere sulla tecnologia Redux;
		\item \textbf{\#Truffle\underline{ }solidity:} per discutere sulle tecnolgie Truffle e Solidity.
	\end{itemize}

	\subsubsection{Comunicazioni esterne}
	Questa sottosezione raccoglie le norme che regolano le comunicazioni con soggetti esterni al gruppo 353, nello specifico:
	\begin{itemize}
		\item La proponente \textbf{Red Babel} rappresentata da Alessandro Maccagnan e Milo Ertola, con i quali si intende stabilire un rapporto di collaborazione al fine di definire i \citGloss{requisiti} che permetteranno la realizzazione del prodotto;
		\item \textbf{Prof. Tullio Vardanega} e \textbf{Prof. Riccardo Cardin}, ai quali verrà fornita la documentazione richiesta in ciascuna revisione di progetto, con i quali si intende dialogare con fine il miglioramento continuo.
	\end{itemize}
	\paragraph{Comunicazioni esterne scritte}
	Le comunicazioni esterne scritte vengono effettuate utilizzando l'indirizzo email del gruppo:
	\begin{center}
		\mailgroup
	\end{center}
	a cui ogni membro ha accesso.\\
	Per comunicare con i Proponenti vengono utilizzati gli indirizzi email forniti durante la presentazione dei capitolati, rispettivamente 
	\begin{center}
		\mail{alessandro@redbabel.com} \\
		e\\
		\mail{milo@redbabel.com}
		
	\end{center}
	Inoltre è stato creato nel workspace \citGloss{Slack} della Proponente Red Babel un canale apposito per la comunicazione tra il gruppo 353 e i rappresentanti della Proponente, utile anche per accordarsi riguardo eventuali incontri e specifiche di progetto.\\
	Per comunicare con i committenti si utilizza l'indirizzo di posta elettronica, utilizzando un oggetto specifico e conciso per descrivere il contenuto del messaggio. Ci si rivolge ai committenti dando loro del Voi o del Lei.
	
	\subsection{Riunioni}

	Questa sotto-sezione definisce le regole che normano le riunioni, interne o esterne, e il loro svolgimento. Durante lo svolgimento di ogni riunione verrà nominato un segretario tra i membri del gruppo \gruppo, il cui compito verterà nel far rispettare l'ordine del giorno, annotare gli argomenti discussi e redarre il Verbale di Riunione.
	Se non è possibile effettuare una riunione in un luogo fisico, è stato deciso di utilizzare \citGloss{Skype}.

	\subsubsection{Verbale di riunione}
	Sarà compito del segretario redigere il Verbale di Riunione, il quale sarà strutturato secondo questo scheletro:
	\begin{itemize}
		\item \textbf{Verbale TIPO del DATA:} costituirà il frontespizio;
		\begin{itemize}
			\item \textbf{TIPO:} indica la tipologia d'incontro ossia se Interno o Esterno;
			\item \textbf{DATA:} indica il giorno in cui si è svolta la riunione mentre per 
		\end{itemize}
		\item \textbf{Informazioni sulla riunione}: questa sezione contiene:
		\begin{itemize}
			\item \textbf{Motivo della riunione}: illustra i motivi generali della riunione; 
			\item \textbf{Luogo e Data}: ad esempio, Padova 05 Dicembre 2017;
			\item \textbf{Orario}: di inizio e fine nel formato ventiquattro ore, separati da un trattino, ad esempio 15.15 - 16.45;
			\item \textbf{Partecipanti}: elencati iniziando da Proponente/Committenti se la riunione è esterna, seguiti dai membri del gruppo partecipanti.			
		\end{itemize}
			\item \textbf{Resoconto}: contiene il riassunto redatto dal segretario secondo i punti dell'ordine del giorno, precisando se sono stati discussi e le loro relative discussioni. In questa sezione potrà essere contenuto anche il resoconto di argomenti non presenti nell'ordine del giorno ma discussi durante la riunione e il tracciamento delle decisioni prese fatto nel seguente modo: ogni decisione sarà identificata da un codice \textbf{VER-DATA.X} dove VER-DATA è il nome del verbale e X è un numero sequenziale.
	\end{itemize}
	Nei verbali non è presente il diario delle modifiche.
	\paragraph{Nomenclatura e conservazione:} i file relativi ai verbali dovranno essere nominati come: \textbf{VER-DATA}.\\
	Ad esempio VER-2017-12-05, per permettere una facile organizzazione degli stessi.
	Essendo documenti ufficiali, devono essere redatti in \\citGloss{LaTeX}\ e inclusi nella \citGloss{repository} omonima.
	
	\subsubsection{Riunioni interne}
	La partecipazione alle riunioni interne è permessa ai soli membri del gruppo \gruppo.
	Il \respdiprog ha il compito di stilare l'ordine del giorno, fissare la data e inserirla nel calendario e approvare il verbale redatto dal Segretario.\\
	Di contro, i \textbf{partecipanti} devono presentarsi puntualmente alle riunioni, comunicare eventuali ritardi e partecipare attivamente alle discussioni.\\
	Affinché una riunione sia ritenuta valida, devono essere presenti almeno cinque membri del gruppo \gruppo.
	
	\subsubsection{Riunioni esterne}
	Le riunioni esterne vedono coinvolti sia i membri del gruppo \gruppo che la Proponente.
	Tutto ciò che viene discusso durante le riunioni, viene riportato nel verbale esterno corrispondente.\\
	A causa della locazione dell'azienda proponente, con sede a Amsterdam, le riunioni verranno principalmente effettuate tramite \citGloss{Skype}, ovviando così al problema della distanza fisica, dopo essersi accordati sul giorno e ora stabiliti utilizzando l'apposito canale \citGloss{Slack}.
	Quando possibile, verranno invece effettuate presso la Torre Tullio Levi Civita (ex Torre Archimede), previa disponibilità dei locali.\\
	I verbali esterni redatti dopo una riunione con la Proponente saranno consegnati in copia alla Proponente stessa, per un'ulteriore approvazione dei contenuti.
	
	\section{Processi di pianificazione}
	\subsection{Ruoli di progetto}
	La realizzazione del progetto è il risultato di un'attività collaborativa tra i membri del gruppo. Ad ogni persona infatti sarà attributo un ruolo, corrispondente a una figura aziendale, per un certo periodo di tempo. A rotazione, ogni membro del gruppo ricoprirà tutti i ruoli di seguito elencati:
	\begin{itemize}
		\item \textbf{Responsabile di progetto};
		\item \textbf{Amministratore};
		\item \textbf{Analista};
		\item \textbf{Progettista};
		\item \textbf{Programmatore};
		\item \textbf{Verificatore}.
	\end{itemize}
	Il cambio del ruolo sarà effettuato in modo da garantire continuità alle attività in corso e far sì che ognuno ricopra ogni ruolo per un tempo omogeneo.\\ L'assegnazione dei ruoli verrà effettuata \respdiprog  in modo da non creare condizioni contraddittorie, come ad esempio essere verificatori di documenti prodotti da se stessi, in modo da non compromettere la qualità del lavoro effettuato.
	\subsubsection{Responsabile di progetto}
	Il \respdiprog, o \textquotedblleft Project Manager", è il responsabile ultimo, per conto del suo gruppo, dei risultati del progetto. Partecipa al progetto per tutta la sua durata e accentra le responsabilità di scelta e approvazione. Inoltre rappresenta il gruppo di lavoro nei confronti di committenti e Proponenti. Egli:
	\begin{itemize}
		\item Elabora ed emana piani e scadenze;
		\item Approva l'emissione di documenti;
		\item Coordina le attività del gruppo;
		\item Si relaziona con il controllo di qualità interno al progetto;
		\item Approva l'Offerta e i relativi allegati.
	\end{itemize}

	\subsubsection{Amministratore}
	La figura dell'amministratore è indispensabile per permettere una buona produttività ed efficienza del gruppo, fornendo gli strumenti per adempiere ai propri compiti in maniera regolamentata. Deve gestire l'ambiente di lavoro redigendo i documenti che normano l'attività lavorativa e la loro \citGloss{verifica}. Infatti:
	\begin{itemize}
		\item È responsabile della redazione e attuazione di piani e procedure di Gestione per la Qualità;
		\item Controlla il versionamento e le configurazioni dei prodotti;
		\item Collabora alla redazione del \pdp;
		\item Redige le Norme di Progetto;
		\item Si assicura che la documentazione sia corretta, verificata, approvata e facilmente accessibile.
	\end{itemize}

	\subsubsection{Analista}
	L'attività dell'Analista è necessaria e fondamentale affinché il progetto possa essere realizzato. Il suo compito è quello di analizzare il dominio del problema per comprenderlo a pieno, abbassando le probabilità che vengano effettuati gravi problemi di progettazione. I suoi compiti comprendono:
	\begin{itemize}
		\item Lo studio e la definizione del problema da risolvere, per capire cosa deve essere realizzato e definire quindi gli accordi contrattuali in base ai \citGloss{requisiti} richiesti;
		\item La \citGloss{verifica} delle implicazioni di costo e qualità;
		\item La modellazione concettuale del sistema e la ripartizione dei \citGloss{requisiti};
		\item La realizzazione dello Studio di Fattibilità e dell'Analisi dei \citGloss{requisiti}.
	\end{itemize}
	\subsubsection{Progettista}
	Il Progettista è responsabile delle attività di progettazione, cioè della definizione di una soluzione soddisfacente per tutti gli stakeholder. Gli obiettivi del Progettista comprendono:
	\begin{itemize}
		\item La soddisfazione dei \citGloss{requisiti} con un sistema di qualità;
		\item La definizione dell'architettura logica del prodotto in modo che sia facile da mantenere, applicando soluzioni note e ottimizzate;
		\item La suddivisione del sistema in parti di complessità trattabile, per rendere il lavoro di codifica facilmente realizzabile e verificabile.
	\end{itemize} 

	\subsubsection{Programmatore}
	Il programmatore è responsabile delle attività di codifica che portano alla realizzazione del prodotto. Affinché questo avvenga, il suo compito consta solamente di implementare l'architettura definita dal Progettista. Per fare ciò:
	\begin{itemize}
		\item Scrive codice documentato, versionato e mantenibile secondo norme fissate;
		\item Crea le componenti necessarie a \citGloss{verifica} e \citGloss{validazione} del codice;
		\item Si occupa della stesura del Manuale Utente.
	\end{itemize}

	\subsubsection{Verificatore}
	Il verificatore è una figura presente per tutta la durata del progetto. Il compito principale è quello di responsabile delle attività di \citGloss{verifica}, che comprendono:
	\begin{itemize}
	\item L'accertamento che l'esecuzione delle attività di processo non abbia introdotto errori;
	\item La redazione del \pdq che illustra l'esito e la completezza di verifiche e prove effettuate secondo il piano.
	\end{itemize}
	
	\subsubsection{Rotazione dei ruoli}
	Ogni membro del gruppo dovrà ricoprire ciascuno dei ruoli del progetto.\\
	La pianificazione dovrà essere eseguita con precisione rispettando le seguenti regole:
	\begin{itemize}
		\item Ogni membro del gruppo non dovrà mai ricoprire un ruolo che preveda la \citGloss{verifica} dell'operato svolto da lui in precedenza poiché questo potrebbe	portare ad un conflitto di interesse;
		\item Bisogna tener conto dei possibili impegni o interessi dei singoli membri del gruppo;
		\item Ogni ruolo, prevedendo comportamenti e attività differenti, dovrà essere trasferito tra i vari membri in modo ottimale lasciando quindi scritto su un documento informale una lista di consigli e/o procedure da parte dell'ultimo assegnatario di quel ruolo;
		\item Ciascun membro dovrà assicurare l'esclusivo svolgimento del ruolo a lui assegnato.
	\end{itemize}
	
	
	\subsection{Ticketing}
	\subsubsection{Task list}
	Il \pdp prevede la suddivisione del modello di sviluppo in varie fasi.
	L'insieme delle attività da svolgere in una fase è contenuto in una task board.\\
	Il \respdiprog deve realizzare le task board per ogni fase del modello su \citGloss{Asana}, utilizzando il nome "\textbf{WORK X}", dove X corrisponde ad ogni fase di revisione del progetto, in ordine: RR, RP, RQ, RA.
	
	\subsubsection{Task}
	Ogni task viene creata dal \respdiprog oppure da un membro del gruppo. Nel secondo caso è richiesta l'approvazione della task da parte della maggioranza (4) dei rimanenti componenti del gruppo.
	Ogni task è quindi composta da un titolo significativo, due date( la prima d'inizio mentre la seconda di termine previsto), e un tag caratteristico per ogni documento a cui si riferisce. Ogni task inoltre può prevedere delle subtask, che possono essere assegnate a diversi membri del gruppo, in particolare se l'attività che la task rappresenta è complicata o tempisticamente lunga.
	
	\subsubsection{Ticket}
	I \citGloss{ticket} rappresentano l'associazione di una task ad un membro del gruppo.
	L'assegnazione dei \citGloss{ticket} ad uno specifico componente del gruppo può avvenire secondo due modalità:
	\begin{itemize}
		\item \textbf{Pro-attivamente:} l'assegnatario del task è già noto al momento della creazione dello stesso. In questo caso il \citGloss{ticket} viene assegnato direttamente allo specifico componente del gruppo da parte del \respdiprog; 
		\item \textbf{Retroattivamente:} nel caso di task a bassa priorità oppure di grandi quantità di lavoro da parte di tutti i componenti del gruppo, può non essere possibile assegnare direttamente un task ad un componente del gruppo. Questi tasks possono essere assegnati autonomamente qualora il componente finisse in anticipo i \citGloss{ticket} a lui già assegnati.
	\end{itemize}
	
	\section{Procedure}
	\subsection{Creazione e gestione dei task}	
	La procedura di creazione di un task è la seguente: \\
	\includegraphics[scale=0.3]{../../common/images/TaskCreation}		
	
	\subsection{Gestione dei ticket}
	 Ogni \citGloss{ticket} si rifà al seguente ciclo di vita a partire dalla sua creazione fino al completamento.\\
	\includegraphics[scale=0.3]{../../common/images/AsanaFlow}
	
	\subsection{Stesura del consuntivo}
	L'operazione di stesura del consuntivo può essere effettuata solamente dal \respdiprog della fase corrente.
	Le operazioni che il \respdiprog esegue sono le seguenti:
	\begin{enumerate}
		\item Esporta da \citGloss{Asana} tramite \citGloss{Instagantt} un foglio di calcolo, in formato Microsoft Excel, che mostra le ore rendicontate nella fase corrente;
		\item Inserisce nel foglio di calcolo precompilato relativo alla fase corrente, realizzato sulla base del template disponibile su \citGloss{GitHub};
		\item Inserisce le ore rendicontate nelle celle previste, ottenendo quindi la differenza di ore tra il preventivo e le ore rendicontate;
		\item Aggiunge all'interno del \pdp una sezione mostrante i valori ottenuti da questo confronto;
		\item Crea una tabella nella sezione descritta dal punto precedente mostrante la differenza tra le ore preventivate e quelle rendicontate, ricavando il budget effettivo rispetto a quello stimato;
		\item Trae infine delle conclusioni dai risultati avuti e scrive una valutazione complessiva del lavoro effettuato nella fase corrente.
	\end{enumerate}
	Per la revisione RR sono effettuati solamente i passaggi dall'uno al tre senza effettuare le differenze non avendo ore rendicontate per le fasi successive. Successivamente ad ogni revisione sarà aggiornato il \pdp con i valori ottenuti dal confronto tra ore preventivate e rendicontate. 
	
	\section{Strumenti}	
	\subsection{Pianificazione} Per quanto riguarda la pianificazione, la scelta del gruppo \gruppo è ricaduta sul servizio cloud \citGloss{Asana}.\\
	Questo servizio offre la possibilità di creare dei task ed assegnarli ai membri del gruppo 353, indicando anche la data di inizio e di fine in modo da consentire una pianificazione del lavoro da svolgere.\\
	Inoltre è possibile indicare delle dipendenze tra attività, favorendo così un controllo dei vincoli per lo svolgimento dei compiti.\\
	
	\subsection{Comunicazione}
	Per facilitare la comunicazione tra i membri del gruppo, è stato deciso di utilizzare \citGloss{Slack}, un'applicazione di messaggistica multi piattaforma con funzionalità specifiche per gruppi di lavoro. La decisione di utilizzare questo strumento è stata supportata dalle funzionalità di integrazione di \citGloss{bot} e di creazione di canali tematici, utili per poter classificare le conversazioni e permettere una comunicazione più efficace e mirata.\\
	Inoltre è stato deciso di affiancare \citGloss{Skype} come strumento per effettuare video chiamate in caso di impossibilità di riunirsi personalmente. \'{E} stato scelto questo strumento perché permette la video chiamata tra più persone contemporaneamente, è di facile utilizzo e accessibile anche da dispositivo mobile.\\

	\subparagraph{Standups:} per incentivare la continuità nel tempo dello sviluppo della commessa, si è scelto di utilizzare Standup Alice, un \citGloss{bot} per \emph{Slack} che consente la realizzazione giornaliera di \citGloss{standups}, permettendo al gruppo di seguire costantemente l'operato degli altri componenti e di confrontarsi nel caso sorgano dubbi o domande.

	\subsection{Creazione diagrammi di Gantt}
	Lo strumento scelto per	la realizzazione dei diagrammi di \citGloss{Gantt} è GanttProject, in quanto è gratuito, open source, multi piattaforma ed è stato considerato adatto ai nostri bisogni.
	\subsection{Calcolo del consuntivo}
	Gli strumenti utilizzati dal \respdiprog sono i seguenti:
	\begin{itemize}
		\item \textbf{Instagantt}\footnote{\nURI{https://instagantt.com}};
		\item \textbf{LibreOffice Calc}\footnote{\nURI{https://it.libreoffice.org/scopri/calc/}}.
	\end{itemize}
	
	\section{Formazione}
	\subsection{Formazione dei membri del gruppo}
		La formazione del personale è da realizzarsi in maniera autonoma. I membri del gruppo \gruppo sono tenuti a studiare individualmente le tecnologie che verranno utilizzate nel corso del progetto. \'{E} possibile che i membri del gruppo realizzino, in piena libertà, delle guide a carattere informale e relative ad un singolo argomento, allo scopo di facilitare la formazione ai restanti componenti del gruppo.

	\subsubsection{Guide e materiale utilizzato}
	 La documentazione di riferimento, oltre al materiale già citato nella sottosezione \emph{Riferimenti Informativi}, comprende:\\
	\begin{itemize}
		\item Per l'utilizzo di \LaTeX: \nURI{https://www.latex-project.org};\\
		\item Per l'utilizzo di GitHub: \nURI{https://github.com};\\
		\item Per l'utilizzo di React: \nURI{https://reactjs.org};\\
		\item Per l'utilizzo di Redux: \nURI{https://redux.js.org};\\
		\item Per l'utilizzo di Ethereum: \nURI{https://www.ethereum.org};\\
		\item Per l'utilizzo di Metamask, \nURI{https://metamask.io};\\ 
		\item Per l'utilizzo di Solidity: \nURI{https://solidity.readthedocs.io/en/develop}.\\
	\end{itemize}
	Il versionamento dei prodotti servirà anche per apprendere dall'operato
	altrui, in modo da integrare le conoscenze personali migliorando la qualità e
	l'efficienza delle attività.
	
\end{document}