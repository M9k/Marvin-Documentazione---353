\documentclass[PianoDiQualifica.tex]{subfiles}

\begin{document}

\chapter{Qualità di prodotto}

\section{Scopo}
Per garantire una buona qualità di prodotto, il gruppo \gruppo ha individuato dallo standard ISO/IEC 9126 le qualità che ritiene più importanti nell’arco del ciclo di vita del prodotto e le ha istanziate individuando obiettivi e metriche coerenti con i livelli di qualità perseguiti.

\section{Tabella delle metriche}
% ID METRICA
% M --> metrica
% PD --> prodotto
% D/S --> documenti/software (macrocategorie)
% 00n --> numero incrementale
Questa tabella indica i \textbf{range} di accettazione e di ottimalità per le metriche utilizzate per la qualità di prodotto:
\begin{table}[H]
	\begin{center}
		\begin{tabu} to \textwidth {
				>{\centering}m{0.15\linewidth}
				>{\centering}m{0.45\linewidth}
				>{\centering}m{0.18\linewidth} 
				>{\centering\arraybackslash}m{0.17\linewidth}
			}
			\tableHeaderStyle
			\textbf{ID} & \textbf{Nome} & \textbf{Range di accettazione} & \textbf{Range di ottimalità}\\
			\multicolumn{4}{c}{\textbf{Documenti}}\\
			\textlink{MPDD001}{MPDD001TAB}{MPDD001} & Indice di Gulpease & 50 - 100 & 60 - 100\\
			\textlink{MPDD002}{MPDD002TAB}{MPDD002} & Formula di Flesch & 40 - 60 & 50 - 60\\ 
			\textlink{MPDD003}{MPDD003TAB}{MPDD003} & Errori ortografici & 100\% corretti & 100\% corretti\\ 
		
			\hline
			\multicolumn{4}{c}{\textbf{Software}}\\
			\textlink{MPDS001}{MPDS001TAB}{MPDS001} & Copertura requisiti obbligatori & 100\% & 100\%\\
			\textlink{MPDS002}{MPDS002TAB}{MPDS002} & Copertura requisiti accettati & 60\% - 100\% & 80\% - 100\%\\
			\textlink{MPDS003}{MPDS003TAB}{MPDS003} & Accuratezza rispetto alle attese & 90\% - 100\% & 100\%\\
			\textlink{MPDS004}{MPDS004TAB}{MPDS004} & Densità di failure & 0\% - 10\% & 0\% \\
			\textlink{MPDS005}{MPDS005TAB}{MPDS005} & blocco di operazioni non corrette & 80\% - 100\% & 100\%\\
			\textlink{MPDS006}{MPDS006TAB}{MPDS006} & Comprensibilità delle funzioni offerte & 80\% - 100\% & 90\% - 100\%\\
			\textlink{MPDS007}{MPDS007TAB}{MPDS007} & Facilità di apprendimento delle funzionalità & 0 - 20 min & 0 - 10 min\\
			\textlink{MPDS008}{MPDS008TAB}{MPDS008} & Consistenza operazionale in uso & 80\% - 100\% & 90\% - 100\%\\  
			\textlink{MPDS009}{MPDS009TAB}{MPDS009} & Tempo di risposta & 0 - 8 sec & 0 - 3 sec \\
			\textlink{MPDS010}{MPDS010TAB}{MPDS010} & Capacità di analisi di failure & 60\% - 100\% & 80\% - 100\% \\
			\textlink{MPDS011}{MPDS011TAB}{MPDS011} & Impatto delle modifiche & 0\% - 20\% & 0\% - 10\% \\
			\textlink{MPDS012}{MPDS012TAB}{MPDS012} & Versioni di browser supportate & 70\% - 100\% & 100\%\\
			\textlink{MPDS013}{MPDS013TAB}{MPDS013} & Inclusione di funzionalità da altri prodotti & 80\% - 100\% & 90\% - 100\% \\
		\end{tabu}
		\caption{Tabella delle metriche della qualità di prodotto}
		\vspace{-1em}
	\end{center}
\end{table}

\section{The Twelve-Factor App}
Nel capitolato è stata citata la documentazione delle "The Twelve-Factor App", ed è stato chiarito con la Proponente, nell'incontro descritto nel \textbf{VER-2018-01-09}, che la documentazione sarà da ritenersi minimale e limitata a delle brevi descrizioni nella sezione seguente nelle prossime fasi.\\
Il lavoro su queste sarà concentrato sul rispetto di questi principi, analizzati e riassunti in appendice al documento.\\\\
Di questi non saranno applicabili totalmente o parzialmente i seguenti:
\begin{itemize}
	\item \textbf{5:} il server su cui gira l'applicazione non dovrà eseguire nulla se non inviare solamente file \citGloss{HTML} e \citGloss{JavaScript};
	\item \textbf{6:} l'applicazione non possiede processi non avendo uno stato;
	\item \textbf{7:} non è possibile il binding delle porte usando Metamask come unico protocollo verso il server esterno Infura;
	\item \textbf{8:} concorrenza non realizzabile sviluppando un prodotto decentralizzato;
	\item \textbf{9:} è inapplicabile, non c'è nulla da fermare o far partire, non c'è nulla che possa interrompere la sua esecuzione inavvertitamente sul lato server;
	\item \textbf{11:} il log non sarà perseguibile in remoto nativamente all'architettura decentralizzata dovendo quindi appoggiarsi a servizi di log esterno in caso di necessità, sarà possibile localmente nel broswer.
\end{itemize}

\end{document}