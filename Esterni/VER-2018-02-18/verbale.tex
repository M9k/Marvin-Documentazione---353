\documentclass[VER-2018-01-09.tex]{subfiles}
\begin{document}
\taburowcolors[2] 2{tableLineOne .. tableLineTwo}
\tabulinesep = ^3mm_2mm
\chapter{Riunione}
\section{Informazioni generali}
\begin{itemize}
	\item \textbf{Motivo della riunione:} \`{E} stata richiesta questa riunione per aggiornare la Proponente sul lavoro che stiamo svolgendo e per chiarire alcuni dubbi riguardanti le tecnologie che sono sorti lavorando sul \textit{Proof of Concept}.
	\item \textbf{Luogo e data:} Chiamata Skype, \nData{18}{02}{2018};
	\item \textbf{Orario:} 19.00 - 19.45;
	\item \textbf{Partecipanti:}
	\begin{itemize}
		\item Proponente: Alessandro Maccagnan e Milo Ertola;
		\item Gruppo \gruppo: tutti i membri tranne \Gianluca.
	\end{itemize}
\end{itemize}

\chapter{Resoconto}

\section{Argomenti}
Di seguito sono riportati i punti dell'ordine del giorno che sono stati discussi:
\begin{enumerate}
	\item \textbf{Problema con web3 e Redux:} realizzando il \textit{Proof of Concept}, l'inserimento di un oggetto web3 in Redux con la versione più recente di Metamask fallisce, in quanto l'oggetto creato è mutabile ed impossibile da modificare in un oggetto immutabile mantenendolo utilizzabile, quindi l'unica soluzione trovata dal team è quella di tenere l'oggetto di web3 sempre globale. La Proponente ha risposto che se viene utilizzato sempre lo stesso oggetto questo non porta a problemi;
	\item \textbf{Aggiornamenti delle componenti:} alla richiesta da parte del team se sia possibile aggiornare alcuni componenti, la Proponente ha risposto affermativamente, e ha consigliato di aggiornare un pacchetto alla volta. Però ha suggerito di risolvere prima i problemi con Redux e poi di concentrarsi sul resto;
	\item \textbf{Redux:} è la parte che al momento risulta essere più problematica per il team, perché presenta una logica complessa e non è facile capire se l'approccio utilizzato sia corretto. La Proponente ha consigliato un progetto\footnote{\nURI{https://github.com/reactjs/react-redux}} da studiare per capire l'interazione e la connessione tra React e Redux. Inoltre, ha suggerito di prendere un progetto semplice, come ad esempio una To-Do-List, e di capire con ciò l'iterazione tra le due tecnologie analizzandolo attraverso l'utilizzo dei plugin \textquotedblleft{}React Developer Tools\textquotedblright{} e \textquotedblleft{}Redux DevTools\textquotedblright{} disponibili per i browser Chrome e Firefox;
	\item \textbf{Documentazione in UML di Solidity:} la Proponente ha chiesto come verrà documentato Solidity in UML, non esistendo uno standard. Il team ha risposto che questa decisione deve ancora essere presa, e che l'argomento sarebbe stato oggetto di discussione con il \Vardanega il giorno seguente;
	\item \textbf{Variabili private in Solidity:} la Proponente ha chiesto se qualcuno del team aveva mai sentito parlare dell'esistenza di una tecnologia chiamata \textquotedblleft{}zkSNARKs\textquotedblright{}, introdotta ad ottobre in Ethereum e precedentemente presente sulla blockchain ZCash che permette di avere variabili private in Solidity. Il team non aveva ancora approfondito l'argomento e la Proponente ha aggiunto che è una tecnologia interessante ma non prevista nei requisiti, perché prevede un laborioso set-up iniziale;
	\item \textbf{Applicazione anche da mobile:} la Proponente ha chiesto se sarà possibile utilizzare l'applicativo da dispositivo mobile. Il team ha risposto che questa possibilità è stata inserita tra i requisti desiderabili, in quanto la versione mobile di Firefox supporta i plugin mentre quella di Chrome ancora no, quindi teoricamente solo gli utilizzatori di Firefox saranno in grado di utilizzare la DApp da mobile utilizzando Metamask;
	\item \textbf{Lentezza del sistema:} è possibile che l'interazione con il backend sia lenta, la Proponete ha quindi proposto al team di pensare ad un modo alternativo al solito spinner, dato che è probabile ci sia un'attesa di 30 secondi o anche superiore. Un'idea potrebbe essere quella di un banner che indichi l'avanzamento del processo oppure richiedere l'azione in background e consentire all'utente di svolgere altri compiti nel frattempo;
	\item \textbf{Altri utilizzi interessanti di una blockchain:} la Proponente ha dichiarato di avere proposto UniWeb come progetto perché era qualcosa che tutti noi conoscevamo, ma ha chiesto al team di riflettere su altri utilizzi interessanti di una blockchain o di un sistema distribuito che non potrebbero esistere in modo centralizzato, riflettendo anche sugli aspetti negativi che li caratterizzano.\\ Un utilizzo che è subito emerso è stato Zero-net (\nURI{www.zeronet.io}).
\end{enumerate}

\section{Tracciamento delle decisioni}
\begin{table}[H]
	\begin{center}
		\begin{tabu} to \textwidth {
				>{\centering}m{0.25\linewidth}  
				>{\centering\arraybackslash}m{0.7\linewidth}
			}
			\tableHeaderStyle
			\textbf{Codice} & \textbf{Decisione} \\
			VER-2018-02-18.1 & Aggiornare i pacchetti dopo aver risolto i problemi con Redux \\
			VER-2018-02-18.2 & Discutere con il \Vardanega dell'utilizzo di UML per Solidiy \\
			VER-2018-02-18.3 & Informarsi sulla tecnologia \textquotedblleft zkSNARKs\textquotedblright \\
			VER-2018-02-18.4 & Pensare ad una modalità alternativa al solito spinner per quanto riguarda le attese \\
			VER-2018-02-18.5 & Pensare ad altri utilizzi interessanti di una blockchain che ne sappiano sfruttare le caratteristiche peculiari \\
		\end{tabu}
	\caption{Tracciamento delle decisioni del verbale}
	\end{center}
\end{table}
\end{document}