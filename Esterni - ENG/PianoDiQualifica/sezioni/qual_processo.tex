\documentclass[PianoDiProgetto.tex]{subfiles}

\begin{document}

\chapter{Qualità di processo}

\section{Scopo}
Per garantire la qualità del prodotto è necessario perseguire la qualità dei processi che lo definiscono.
Per fare ciò si è deciso di adottare lo standard ISO/IEC 15504 denominato SPICE 
%TODO ISPIRARSI DA STANDARD per livelli di MATURITA' del Processo
%TODO che standard ISO e meglio che usiamo per qualità di processo?? 
http://www.praxiom.com/iso-90003.htm citato da Tullio può servire come base per capire processi meglio
http://www.math.unipd.it/~tullio/IS-1/2017/Dispense/L15.pdf

%template di un processo con seguenti sottosezioni
\section{Nome Processo}
	\subsection{Obiettivi di qualità}
	\subsection{Strategie}
	\subsection{Metriche}

%template 2 uguale al primo
\section{Nome Processo 2}
	\subsection{Obiettivi di qualità}
	\subsection{Strategie}
	\subsection{Metriche}

\end{document}