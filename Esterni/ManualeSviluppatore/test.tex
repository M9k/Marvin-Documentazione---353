\documentclass[ManualeSviluppatore]{subfiles}
\begin{document}

\chapter{Test}
\section{Chapter purpose}
This chapter has the purpose to indicate to the developers how to check the operation of their code and its syntax. \\

\section{Test on JavaScript Code}
To run the tests on the code, you must execute the following command: \\
\begin{ttfamily} yarn run test \end{ttfamily} \\
It will run all the test in the \sni{testnpm} directory. \\
Remember that if you prefer npm over yarn you just need to replace "\begin{ttfamily}yarn\end{ttfamily}" with "\begin{ttfamily}npm\end{ttfamily}". \\
To check if your code follows the style guide run: \\
\begin{ttfamily}yarn run eslint\end{ttfamily} \\
or \\
\begin{ttfamily} yarn run eslint-fix\end{ttfamily} \\
to also correct errors fixable by the ESLint utility. \\
Those scripts automatically check all the files inside the \begin{ttfamily}src, testnpm\end{ttfamily} and \begin{ttfamily}test\end{ttfamily} directories, using the rules specified by the \begin{ttfamily}.eslint\end{ttfamily} files placed in the root of each directory.\\
If you want to run the linter on a single file follow the hints written in the \href{https://eslint.org/docs/user-guide/getting-started}{eslint documentation}.

\section{Test on Solidity Code}
To run the tests on the contracts, you must execute the following command: \\
\begin{ttfamily} yarn run soltest \end{ttfamily} \\
It will create a local Ethereum blockchain and run all the test in the \sni{test} directory. \\
Like the JavaScript code, solidity files have to follow the \href{https://solidity.readthedocs.io/en/v0.3.1/style-guide.html}{strict style guide} provided by the official Solidity Documentation. To ensure you're following it just run: \\
\sni{yarn run solhint} \\
that will execute the \sni{solhint} utility on all your contract files placed inside the \sni{contracts} directory.

\section{Code coverage}
Marvin is equipped with code coverage utilities for both the Javascript and Solidity codebases.
To run the utilities separately run: \\
\sni{yarn run solidity-coverage} \\
or \\
\sni{yarn run js-coverage} \\
They will generate the reports on the terminal and also in the \sni{coverage} and \sni{js-coverage} directories. \\
If you want to have a single report for the two codbases run: \\
\sni{yarn run make-coverage \&\& yarn run merge-coverage} \\
It will create a file inside the root directory called \sni{report.lcov} that you can use to upload on any code coverage service (we use coveralls\footnote{\nURI{https://coveralls.io}}) that support lcov\footnote{\nURI{http://ltp.sourceforge.net/coverage/lcov.php}} report style. \\
Unfortunately the package we are using for solidity's coverage not work out of the box with Windows and is not really stable, so is possibile that it won't work on your machine. \\
To have more information about the package, visit \nURI{https://github.com/sc-forks/solidity-coverage}.

\end{document}