\documentclass[PianoDiProgetto.tex]{subfiles}

\begin{document}

\chapter{Analisi dei rischi}
Nella seguente tabella è stata effettuata un’analisi approfondita dei rischi atta ad ottimizzare l’avanzamento del progetto. Per migliorare la lettura della tabella, l'attualizzazione al periodo corrente viene riportata solo per i rischi che si sono veramente verificati.
%TODO aggiungere attualizzazione per il periodo di analisi

% per tabelle, alterna i colori delle righe		
\taburowcolors[2] 2{tableLineOne .. tableLineTwo}
\tabulinesep = ^3mm_2mm

\begin{longtabu} to \linewidth { >{\centering}m{.15\linewidth} >{\raggedright}m{.33\linewidth} >{\raggedright}m{0.25\linewidth} >{\centering}m{0.17\linewidth} }
	\caption[Tabella descrittiva dell'analisi dei rischi]{Tabella descrittiva dell'analisi dei rischi}
	\endlastfoot
	\rowfont{\bfseries\sffamily\leavevmode\color{white}}
	\rowcolor{tableHeader}
	\textbf{Nome} & \textbf{Descrizione} & \textbf{Rilevamento} & \textbf{Grado di rischio} \\
	\endhead
	
	\taburowcolors{tableLineOne..tableLineTwo}
	
	%Primo rischio
	\rowcolor{tableLineOne} \textbf{Scarsa esperienza} 
	&
	{\small Nessun membro del gruppo ha mai lavorato con un progetto cosi impegnativo, ciò potrebbe produrre dei ritardi.} 
	& 
	{\small Ogni membro comunicherà al Responsabile eventuali difficoltà.}
	 & \shortstack{Occorrenza: \\ \textbf{alta} \vspace{0.6em}\\ Pericolosità: \\ \textbf{alta} }\\
	\rowcolor{tableLineTwo} Piano di contingenza: 
	&
	 \multicolumn{3}{m{0.8075\linewidth}}{\small  I compiti ad elevata difficoltà verranno affidati a membri con maggiore esperienza. }\\
	\hhline{====}


	%Codice per Attualizzazione 
	\multicolumn{4}{c}{ \cellcolor{tableLineOne} \textbf{Attualizzazione}} \\*
	\rowcolor{tableLineTwo} \textbf{Periodo} 
	&
	 \multicolumn{3}{c}{\textbf{Mitigazione}} \\*
	\rowcolor{tableLineOne} Prova prova
	&
	 \multicolumn{3}{m{0.8075\linewidth}}{Prova prova prova prova prova prova prova prova prova prova prova prova prova.} \\
	\hhline{====}
	%Fine codice per Attualizzazione	
	
	%Secondo rischio
	\rowcolor{tableLineOne} \textbf{Contrasti nel gruppo} 
	&
	 {\small Quest'anno i gruppi sono stati formati in modo casuale. Quindi ogni membro si trova a fare il progetto con un gruppo di persone non conoscenti. Questo potrebbe portare alle tensioni e conflitti.}
	 &
	 {\small Sarà il compito del Responsabile monitorare la collaborazione tra i membri.}
	&
	 \shortstack{Occorrenza: \\ \textbf{bassa} \vspace{0.6em}\\ Pericolosità: \\ 
		\textbf{media} } \\
		\rowcolor{tableLineTwo} Piano di contingenza: 
	&
	\multicolumn{3}{m{0.8075\linewidth}}{\small  I ruoli saranno ruotati per minimizzare i contatti tra i membri in 
		conflitto.  }\\
	\hhline{====}
	
	
	%Terzo rischio
	\rowcolor{tableLineOne} \textbf{Disponibilità dei membri} 
	&
	{\small Alcuni membri del gruppo 353 sono anche lavoratori. Quindi il tempo dedicato al progetto da questi membri potrebbe essere limitato.}
	&
	{\small Ogni membro comunicherà in anticipo gli impegni che possono causare dei ritardi.}
	&
	 \shortstack{Occorrenza: \\ \textbf{media} \vspace{0.6em}\\ Pericolosità: 
		\\ \textbf{alta} }\\
		\rowcolor{tableLineTwo} Piano di contingenza:
	&
	\multicolumn{3}{m{0.8075\linewidth}}
	{\small Il carico di lavoro verrà ridistribuito tra i membri con maggiore 
		disponibilità.}\\
	\hhline{====}
	
	
	%Quarto rischio
	\rowcolor{tableLineOne} \textbf{Problemi hardware} 
	&
	{\small Ogni membro utilizza computer personale per lavorare al progetto, guasti hardware potrebbero causare perdite di dati e/o di tempo.}
	&
	{\small Ogni membro dovrà avvisare il gruppo in caso di comportamenti anomali del proprio computer.}
	&
	 \shortstack{Occorrenza: \\ \textbf{bassa} 
		\vspace{0.6em}\\ Pericolosità: \\ \textbf{medio-bassa} }\\
	\rowcolor{tableLineTwo} Piano di contingenza:
	&
	\multicolumn{3}{m{0.8075\linewidth}}{\small In caso di ingenti perdite di dati chi ha causato la perdita provvederà al ripristino.}\\
	\hhline{====}
	
	
	%Quinto rischio
	\rowcolor{tableLineOne} \textbf{Tecnologie da usare}
	&
	{\small Le tecnologie da studiare sono molto recenti e la documentazione fornita è spesso molto limitata e/o poco approfondita. Il tempo di apprendimento per queste tecnologie potrebbe causare dei ritardi nello svolgimento dei lavori.}
	&
	{\small Il Responsabile dovrà monitorare la preparazione dei vari membri rispetto al compito che devono svolgere.}
	&
	 \shortstack{Occorrenza: \\ \textbf{media} \vspace{0.6em}\\  
		Pericolosità: \\ \textbf{alta} } \\
	\rowcolor{tableLineTwo} Piano di contingenza:
	&
	\multicolumn{3}{m{0.8075\linewidth}}{\small In casi gravi i membri con più esperienza e famigliarità con quella tecnologia dovranno aiutare il membro in difficoltà, ridistribuendo il carico di lavoro.}\\
	\hhline{====}
	
	
	%Sesto rischio
	\rowcolor{tableLineOne} \textbf{Strumenti software} 
	&
	{\small Il gruppo si affida su software di terze parti e piattaforme online, eventuali disfunzioni potrebbero causare errori o perdite di dati.}
	&
	{\small Difficilmente sarà possibile rilevare il 
		problema poiché dipende da fattori esterni.}
	&
	 \shortstack{Occorrenza: \\ \textbf{bassa} 
		\vspace{0.6em}\\ Pericolosità: \\ \textbf{media} }\\
	\rowcolor{tableLineTwo} Piano di contingenza:
	&
	\multicolumn{3}{m{0.8075\linewidth}}{\small Si cercherà di mantenere backup periodici dei 
		dati e si conta sull'affidabilità degli strumenti scelti.}\\
	\hhline{====}
	
	
	%Settimo rischio
	\rowcolor{tableLineOne} \textbf{Costi delle attività}
	 &
	{\small La pianificazione prevede un costo per le attività. Essendo tutti membri del gruppo inesperti, è possibile che i tempi vendono calcolati in modo errato.}
	&
	{\small Il Responsabile verificherà periodicamente lo stato delle attività, in modo da evitare eventuali ritardi nello sviluppo delle attività.}
	&
	 \shortstack{Occorrenza: \\ \textbf{media} \vspace{0.6em}\\  
		Pericolosità: \\ \textbf{alta} }\\
	\rowcolor{tableLineTwo} Piano di contingenza:
	&
	\multicolumn{3}{m{0.8075\linewidth}}{\small Se il ritardo è ingente, il Responsabile dovrà ridistribuire il carico di lavoro fra gli altri membri avendo come obiettivo primario quello di non far slittare le mliestones.}\\
	\hhline{====}
	
	
	%Ultimo rischio
	\rowcolor{tableLineOne} \textbf{Modifica dei requisiti} 
	&
	{\small È possibile che la Proponente Red Babel decida di voler apportare delle modifiche. Questo porterebbe una parziale/totale riscrittura dell'Analisi dei Requisiti con conseguente rivalutazione della pianificazione delle attività.}
	&
	{\small È necessario lavorare a stretto contatto con la Proponente in modo da rilevare subito eventuali modifiche ai requisiti.}
	&
	 \shortstack{ Occorrenza: \\ \textbf{bassa} \vspace{0.6em}\\ 
		Pericolosità: \\ \textbf{molto alta} } \\
	\rowcolor{tableLineTwo} Piano di contingenza:
	&
	\multicolumn{3}{m{0.8075\linewidth}}{\small In caso di un cambiamento sostanzioso dei requisiti, il gruppo  discuterà tali cambiamenti con la Proponente in modo da trovare un punto di accordo.}\\
	\hhline{====}
	
	
\end{longtabu}
\end{document}