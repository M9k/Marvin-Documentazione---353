\documentclass[AnalisiDeiRequisiti.tex]{subfiles}

\begin{document}

\chapter{Descrizione generale}
\section{Obbiettivi del prodotto}
L'obiettivo finale del prodotto è la realizzazione di un simulatore di \citGloss{Uniweb}: portale che consente a studenti, professori e università la gestione dei propri dati e dei processi burocratici in forma telematica. \\
La peculiarità del sistema consiste nel suo dominio tecnologico: la quasi totalità della logica applicativa dovrà essere consegnata nella \citGloss{blockchain} pubblica "\citGloss{Ethereum}" ed esposta agli utenti tramite un'interfaccia web. \\
Lo scopo fondamentale è dimostrare la concreta possibilità di sviluppare applicazioni decentralizzate che non riguardino esclusivamente il mondo delle criptovalute.

\section{Funzioni del prodotto}
Le funzioni del prodotto si avvicinano a quelle basilari offerte dal sistema \citGloss{Uniweb} dell'Università di Padova:
\begin{itemize}
	\item La possibilità per gli studenti di iscriversi ad un \citGloss{corso di laurea} e gestire il proprio percorso accademico;
	\item La possibilità per i professori di gestire i propri corsi di laurea e gli esami erogati agli studenti;
	\item La possibilità per l'università di gestire il personale amministrativo che si occuperà di organizzare l'\citGloss{anno accademico} e gestire le interazioni di studenti e professori.
\end{itemize}
Data la particolarità della rete \citGloss{Ethereum}, in cui ogni operazione di scrittura ha un costo economico, l'applicativo fornirà all'utente, in maniera preventiva, la possibilità di visualizzare la stima del prezzo di ogni operazione da effettuare.


\section{Caratteristiche degli utenti}
Il prodotto si rivolge ad una classe d'utenza collusa con il mondo accademico.\\
L'utente, per accedere al sistema, dovrà possedere alcune nozioni base sul funzionamento della rete \citGloss{Ethereum} e del plugin Metamask, conoscenze che potranno essere acquisite tramite una breve guida fornita dall'applicazione.
\section{Piattaforma d'esecuzione}
Il prodotto finale può essere suddiviso in due macro parti con due diverse piattaforme d'esecuzione: il backend verrà ospitato dalla rete \citGloss{Ethereum}, l'interfacciamento ad esso (frontend) sarà fruibile tramite un qualsiasi \citGloss{browser} web predisposto all'installazione del plugin Metamask.
\section{Vincoli generali}
L'utente, per usufruire del servizio, dovrà possedere un \citGloss{browser} con installato il plugin Metamask, una connessione a internet e una coppia di chiavi (pubblica-privata) compatibile con la rete \citGloss{Ethereum}.	

\end{document}