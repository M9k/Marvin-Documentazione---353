\documentclass[PianoDiProgetto.tex]{subfiles}

\begin{document}
\chapter{Consuntivo di periodo e preventivo a finire}
Di seguito è riportato il resoconto del periodo d'investimento iniziale e i consuntivi per i periodi a carico del committente.\\\\
Nelle tabelle riassuntive vengono riportate le ore e i costi preventivati e a fianco tra parentesi una valore: 
\begin{itemize}
	\item \textbf{Positivo:} se sono state necessarie meno ore di lavoro rispetto al preventivo o se il costo è diminuito;
	\item \textbf{Negativo:} se sono state necessarie più ore di lavoro rispetto al preventivo o se il costo è aumentato;
\end{itemize}
Se il valore tra parentesi non è presente vuol dire che il numero di ore preventivate o il costo è stato rispettato.\\\\
Per una maggiore lettura delle tabelle, i valori pari a 0 (zero) sono stati omessi.

\newpage
\section{Periodo di Analisi dei requisiti}
\subsection{Resoconto}
La seguente tabella presenta il numero di ore di lavoro utilizzate dal gruppo \gruppo nei vari ruoli di progetto durante il periodo di Analisi dei requisiti:
\begin{table}[H]
	\begin{center}
		\begin{tabu} to \textwidth {
				>{\centering}m{0.3\linewidth} 
				>{\centering}m{0.05\linewidth}  
				>{\centering}m{0.05\linewidth}  
				>{\centering}m{0.05\linewidth}
				>{\centering}m{0.05\linewidth}
				>{\centering}m{0.07\linewidth}
				>{\centering}m{0.05\linewidth}
				>{\centering\arraybackslash}m{0.15\linewidth}
			}
			\tableHeaderStyle			
			\textbf{Nome} & \textbf{Re} & \textbf{Ad} & \textbf{An} & \textbf{Pj} & \textbf{Pr} & \textbf{Ve} & \textbf{Totale} \\
			\Davide &  & 11 (+2) & 4 (+1) &  & 0 (-10) & 10 (-6) & 35 (-13) \\
			\Elena & 10 (+2) & 8 (-2) & 3 (-1) &  & 0 (-10) & 4 (-2) & 35 (-13)\\
			\Gianluca &  & 2 (-4) & 15 &  & 0 (-10) & 8 (+1) & 35 (-13) \\
			\Mirco &  & 4 (-3) & 16 &  & 0 (-10) & 5 & 35 (-13) \\
			\Parwinder & 8 & 8 (+3) & 3 (-3) &  & 0 (-10) & 6 (-3) & 35 (-13) \\
			\Riccardo &  & 11 & 4 (+1) &  & 0 (-10) & 10 (-4) & 35 (-13) \\
			\Valentina & 5 & 8 (-6) & 4 (-2) &  & 0 (-10) & 8 (+5) & 35 (-13)\\
		\end{tabu}
		\caption{Resoconto orario - Resoconto Analisi dei requisiti}
		\vspace{-1em}
	\end{center}
\end{table}	
\newpage
La seguente tabella riassume i dati del resoconto per il periodo di Analisi dei \citGloss{requisiti}: 
\begin{table}[H]
	\begin{center}
		\begin{tabu} to \textwidth { 
				>{\centering}m{0.4\linewidth} 
				>{\centering}m{.2\linewidth}  
				>{\centering\arraybackslash}m{0.25\linewidth}  }
			\tableHeaderStyle
			\textbf{Ruolo} & \textbf{Ore} & \textbf{Costo in \euro} \\
			\resp & 23 (+2) & 690,00 (+60,00) \\
			\amme & 52 (-10) & 1.040,00 (-200,00)\\
			\alista & 49 (-4) & 1.225,00 (-100,00)\\
			\proga &  &  \\
			\progre & 70 (-70) & 1.050,00 (-1.050,00) \\
			\vere & 51 (-9) & 765,00 (-135,00) \\
			\hline
			\textbf{Totale} & \textbf{266} & \textbf{5.145,00} \\
			\textbf{Totale preventivato} & \textbf{175} & \textbf{3.720,00} \\
			\textbf{Differenza} & \textbf{-91 ore} & \textbf{-1.425,00 \euro} \\
		\end{tabu}
		\caption{Resoconto economico - Resoconto Analisi dei requisiti}
		\vspace{-1em}
	\end{center}
\end{table}

\subsection{Variazioni nella pianificazione}
Non si è incorsi in ritardi durante l'esecuzione delle varie attività previste.

\subsection{Conclusione}
Durante il periodo di Analisi dei \citGloss{requisiti} è stato necessario usare più ore rispetto a quelle preventivate per il ruolo di \amme e \alista perché le attività comprendenti la stesura del \pdq, il \pdp, l'\adr e delle \ndp hanno necessitato più tempo rispetto a quello preventivato. Sono state necessarie anche più ore per il ruolo di verificatore. Inoltre, tutti i membri del gruppo \gruppo hanno dovuto utilizzare 10 ore a testa per attività di auto formazione per strumenti e nuove tecnologie richieste dal progetto.\\
Il risultato complessivo del periodo è di 91 ore lavorative oltre il previsto e di una spesa aggiunta di 1.425,00 \euro, che però facendo parte del periodo di investimento non influirà sul totale rendicontato e inferiori alla soglia massima posta per l'investimento iniziale.\\

\newpage
\section{Periodo di Consolidamento dei requisiti}
\subsection{Resoconto}
La seguente tabella presenta il numero di ore di lavoro utilizzate dal gruppo \gruppo nei vari ruoli di progetto durante il periodo di Consolidamento dei requisiti:
\begin{table}[H]
	\begin{center}
		\begin{tabu} to \textwidth {
				>{\centering}m{0.3\linewidth} 
				>{\centering}m{0.05\linewidth}  
				>{\centering}m{0.05\linewidth}  
				>{\centering}m{0.05\linewidth}
				>{\centering}m{0.05\linewidth}
				>{\centering}m{0.05\linewidth}
				>{\centering}m{0.05\linewidth}
				>{\centering\arraybackslash}m{0.15\linewidth}
			}
			\tableHeaderStyle			
			\textbf{Nome} & \textbf{Re} & \textbf{Ad} & \textbf{An} & \textbf{Pj} & \textbf{Pr} & \textbf{Ve} & \textbf{Totale} \\
			\Davide &  & 3 & 4 (+4) &  &  &  & 7 (+4)\\
			\Elena &  &  & 2 (+2) &  &  & 5 (+2) & 7 (+4)\\
			\Gianluca &  &  &  &  &  & 7 (+4) & 7 (+4)\\
			\Mirco &  & 4 (+1) & 3 (+3) &  &  &  &  7 (+4)\\
			\Parwinder &  &  & 7 (+4) &  &  &  & 7 (+4)\\
			\Riccardo & 5 (+2) & 2 (+2) &  &  &  &  & 7 (+4)\\
			\Valentina &  &  &  &  &  & 7 (+4) & 7 (+4)\\
		\end{tabu}
		\caption{Resoconto orario - Resoconto Consolidamento dei requisiti}
		\vspace{-1em}
	\end{center}
\end{table}	
\newpage
La seguente tabella riassume i dati del resoconto per il periodo di Consolidamento dei \citGloss{requisiti}: 
\begin{table}[H]
	\begin{center}
		\begin{tabu} to \textwidth { 
				>{\centering}m{0.4\linewidth} 
				>{\centering}m{.2\linewidth}  
				>{\centering\arraybackslash}m{0.25\linewidth}  }
			\tableHeaderStyle
			\textbf{Ruolo} & \textbf{Ore} & \textbf{Costo in \euro} \\
			\resp & 5 (+2) & 120,00 (+60,00)  \\
			\amme & 9 (+3) & 180,00 (+60,00)\\
			\alista & 16 (+13) & 400,00 (+325,00) \\
			\proga &  &  \\
			\progre &  & \\
			\vere & 19 (+10) & 285,00 (+150,00) \\
			\hline
			\textbf{Totale} & \textbf{21} & \textbf{420,00} \\
			\textbf{Totale preventivato} & \textbf{49} & \textbf{1015,00} \\
			\textbf{Differenza} & \textbf{+28 ore} & \textbf{+595,00\euro} \\
		\end{tabu}
		\caption{Resoconto economico - Resoconto Consolidamento dei requisiti}
		\vspace{-1em}
	\end{center}
\end{table}

\subsection{Variazione nella pianificazione}
Non si è incorsi in ritardi durante l'esecuzione delle varie attività previste.

\subsection{Conclusione}
Durante il periodo di Consolidamento dei requisiti non è stato necessario usare il numero di ore preventivato in precedenza, poiché durante quel periodo l'unica attività eseguita dal gruppo è stata preparare le slide per la presentazione in aula e verificarle. Alcune piccole verifiche e modifiche sono state effettuate anche nell'\adr. Il risultato complessivo del periodo è di 28 ore lavorative in meno rispetto al previsto e di un risparmio di 595,00\euro.

\newpage
\section{Periodo di Progettazione della base tecnologica}
\subsection{Consuntivo}
La seguente tabella presenta il numero di ore di lavoro utilizzate dal gruppo \gruppo nei vari ruoli di progetto durante il periodo di Progettazione della base tecnologica:
\begin{table}[H]
	\begin{center}
		\begin{tabu} to \textwidth {
				>{\centering}m{0.3\linewidth} 
				>{\centering}m{0.05\linewidth}  
				>{\centering}m{0.05\linewidth}  
				>{\centering}m{0.05\linewidth}
				>{\centering}m{0.05\linewidth}
				>{\centering}m{0.05\linewidth}
				>{\centering}m{0.05\linewidth}
				>{\centering\arraybackslash}m{0.15\linewidth}
			}
			\tableHeaderStyle			
			\textbf{Nome} & \textbf{Re} & \textbf{Ad} & \textbf{An} & \textbf{Pj} & \textbf{Pr} & \textbf{Ve} & \textbf{Totale} \\
			\Davide & 8 &  & 10 (+2) &  & 6 (-2) & 6 & 30 \\
			\Elena &  &  &  & 14 (+4) & 6 (-4) & 10 & 30\\
			\Gianluca &  & 5 &  &  & 10 (-5) & 15 (+5) & 30 \\
			\Mirco &  &  &  & 12 & 10 (+2) & 8 (-2) & 30 \\
			\Parwinder &  &  &  & 6 & 10 (-4) & 14 (+4) & 30 \\
			\Riccardo & 5 &  &  & 9 (-1) & 6 & 10 (+1) & 30 \\
			\Valentina &  & 5 & 10 (+2) &  & 6 (-2) & 9 & 30\\
		\end{tabu}
		\caption{Resoconto orario - Consuntivo Progettazione della base tecnologica}
		\vspace{-1em}
	\end{center}
\end{table}	
\newpage
La seguente tabella riassume i dati del consuntivo per il periodo di Progettazione della base tecnologica: 
\begin{table}[H]
	\begin{center}
		\begin{tabu} to \textwidth { 
				>{\centering}m{0.4\linewidth} 
				>{\centering}m{.2\linewidth}  
				>{\centering\arraybackslash}m{0.25\linewidth}  }
			\tableHeaderStyle
			\textbf{Ruolo} & \textbf{Ore} & \textbf{Costo in \euro} \\
			\resp & 13 & 390,00  \\
			\amme & 10 & 200,00\\
			\alista & 20 (+4) & 500,00 (+100,00)\\
			\proga & 41 (+5) & 902,00 (+110,00) \\
			\progre & 54 (-19) & 810,00 (-285,00) \\
			\vere & 72 (+10) & 1080,00 (+150,00) \\
			\hline
			\textbf{Totale} & \textbf{210} & \textbf{3.807,00} \\
			\textbf{Totale preventivato} & \textbf{210} & \textbf{3.882,00} \\
			\textbf{Differenza} & \textbf{0 ore} & \textbf{+75,00\euro} \\
		\end{tabu}
		\caption{Resoconto economico - Consuntivo Progettazione della base tecnologica}
		\vspace{-1em}
	\end{center}
\end{table}


\subsection{Variazione nella pianificazione}
Non si è incorsi in ritardi durante l'esecuzione delle varie attività previste.

\subsection{Conclusione}
Durante il periodo di Progettazione della base tecnologica, il numero di ore preventivate si è dimostrato corretto. \`E stato necessario però ridistribuire l'orario a ruoli diversi, poiché per il periodo non è stato necessario utilizzare tutte le ore preventivate per la progettazione della base tecnologica, ma si sono rese necessarie più ore per la programmazione di essa. Inoltre, grazie a una buona qualità iniziale dei documenti, il numero di ore necessarie per la loro modifica non è stato elevato. Il numero ridotto di modifiche e una maggiore esperienza nella redazione dei documenti, ha reso meno dispendiosa l'attività di verifica. Il risultato complessivo del periodo è di 0 ore lavorative in meno rispetto al previsto e di un risparmio di 75,00\euro.

\section{Periodo di Progettazione di dettaglio e codifica}
\subsection{Consuntivo}
La seguente tabella presenta il numero di ore di lavoro utilizzate dal gruppo \gruppo nei vari ruoli di progetto durante il periodo di Progettazione della base tecnologica:
\begin{table}[H]
	\begin{center}
		\begin{tabu} to \textwidth {
				>{\centering}m{0.3\linewidth} 
				>{\centering}m{0.05\linewidth}  
				>{\centering}m{0.05\linewidth}  
				>{\centering}m{0.05\linewidth}
				>{\centering}m{0.05\linewidth}
				>{\centering}m{0.05\linewidth}
				>{\centering}m{0.05\linewidth}
				>{\centering\arraybackslash}m{0.15\linewidth}
			}
			\tableHeaderStyle			
			\textbf{Nome} & \textbf{Re} & \textbf{Ad} & \textbf{An} & \textbf{Pj} & \textbf{Pr} & \textbf{Ve} & \textbf{Totale} \\
			\Davide 	&  & 5 (-1) &  & 17 (-1) & 20 (+2) & 10 & 52 \\
			\Elena 		&  &  & 4 & 18 & 20 & 10 & 52 \\
			\Gianluca 	&  & 8 (-1) &  & 16 (-1) & 18 (+2) & 10 & 52 \\
			\Mirco		& 8 &  &  & 16 (-1) & 18 (+1) & 10 & 52 \\
			\Parwinder	&  &  &  & 24 & 18 & 10 & 52 \\
			\Riccardo 	&  &  & 5 & 16 & 21 & 10 & 52 \\
			\Valentina	& 3 & 2 (-2) &  & 27 & 20 (+2) &  & 52 \\
		\end{tabu}
		\caption{Resoconto orario - Consuntivo Progettazione di dettaglio e codifica}
		\vspace{-1em}
	\end{center}
\end{table}	
\newpage
La seguente tabella riassume i dati del consuntivo per il periodo di Progettazione della base tecnologica: 
\begin{table}[H]
	\begin{center}
		\begin{tabu} to \textwidth { 
				>{\centering}m{0.4\linewidth} 
				>{\centering}m{.2\linewidth}  
				>{\centering\arraybackslash}m{0.25\linewidth}  }
			\tableHeaderStyle
			\textbf{Ruolo} & \textbf{Ore} & \textbf{Costo in \euro} \\
			\resp & 11 & 330,00 \\
			\amme & 15 (-4) & 300,00 (-80) \\
			\alista & 9 & 225,00 \\
			\proga & 134 (-3) & 2.948,00 (-66) \\
			\progre & 135 (+7) & 2.025,00 (+105) \\
			\vere & 60 & 900,00 \\
			\hline
			\textbf{Totale} & \textbf{364} & \textbf{6.769,00} \\
			\textbf{Totale preventivato} & \textbf{364} & \textbf{6.728,00} \\
			\textbf{Differenza} & \textbf{0 ore} & \textbf{-41,00\euro} \\
		\end{tabu}
		\caption{Resoconto economico - Consuntivo Progettazione di dettaglio e codifica}
		\vspace{-1em}
	\end{center}
\end{table}


\subsection{Variazione nella pianificazione}
Non si è incorsi in ritardi durante l'esecuzione delle varie attività previste.

\subsection{Conclusione}
Durante il periodo di Progettazione di dettaglio e codifica, il numero di ore preventivate si è dimostrato corretto. \`E stato necessario però ridistribuire l'orario a ruoli diversi, poiché le ore preventivate necessarie alla programmazione in questo periodo sono state dedicate al ruolo di amministratore per la correzione dei documenti, e al ruolo di progettista, affinché potesse essere creata un'architettura solida per il progetto, così da poter avere un periodo di progettazione più corto ed efficace. Il risultato complessivo del periodo è di 0 ore lavorative in meno rispetto al previsto e di una spesa di 41,00\euro{} in più rispetto al previsto, che non porta però a uno sforamento del budget preventivato.

\section{Periodo di Validazione e collaudo}
Questa sezione verrà descritta alla fine di questo periodo.

\section{Preventivo a finire}
La seguente tabella presenta l'attuale preventivo a finire. Se il valore del consuntivo di un determinato periodo non è ancora presente, per il conteggio totale verrà utilizzato il valore presente nel preventivo.

\begin{table}[H]
	\begin{center}	
		\begin{tabu}to \textwidth  {
				>{\centering}m{0.45\linewidth} 
				>{\centering}m{0.2\linewidth}   
				>{\centering\arraybackslash}m{0.2\linewidth}}
			\tableHeaderStyle
			\textbf{Periodo} & \textbf{Preventivo \euro} & \textbf{Consuntivo \euro} \\
			Progettazione della base tecnologica & 3.882,00 & 3.807,00 \\
			Progettazione di dettaglio e codifica & 6.728,00 & 6.769,00 \\
			Validazione e collaudo & 2.436,00 & - \\
			\textbf{Totale rendicontato} & \textbf{13.046,00} & \textbf{13.012,00} \\
			
		\end{tabu}
	\end{center}
	\caption{Preventivo a finire}
\end{table}


\end{document}