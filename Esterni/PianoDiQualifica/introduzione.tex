\documentclass[PianoDiQualifica.tex]{subfiles}

\begin{document}

\chapter{Introduzione}

	\section{Scopo del documento}
	Lo scopo di questo documento consiste nel documentare le norme utilizzate dal Gruppo \gruppo adottate per la verifica e la \citGloss{validazione} dei prodotti e dei processi. Per ottenere lo scopo prefissato, i processi attuati e i prodotti realizzati saranno continuamente verificati, affinché non vengano introdotti errori che influiscano negativamente sul risultato finale tramite strategie e metriche qui descritte.
	
	\scopoProdotto
	
	\glossExpl
	
	\section{Riferimenti}
		\subsection{Riferimenti Normativi}
		\begin{itemize}
			\item \textbf{\ndp \vruno};
			\item \textbf{Standard ISO/IEC 9126}\\
			\nURI{https://it.wikipedia.org/wiki/ISO/IEC_9126}
			\begin{itemize}
				\item Modello di qualità.
			\end{itemize}
		\end{itemize}
		
		\subsection{Riferimenti Informativi}
		\begin{itemize}
			\item \textbf{Verifica e validazione: introduzione - Slide del corso di Ingegneria del Software}\\
			\nURI{http://www.math.unipd.it/~tullio/IS-1/2017/Dispense/L17.pdf}
			\item \textbf{Indice di Gulpese}\\
			\nURI{https://it.m.wikipedia.org/wiki/Indice_Gulpease}
			\begin{itemize}
				\item Descrizione e formula di calcolo.
			\end{itemize}
			\item \textbf{Formula di Flesch}\\
			\nURI{https://it.m.wikipedia.org/wiki/Formula_di_Flesch}
			\begin{itemize}
				\item Descrizione e formula di calcolo.
			\end{itemize}
			\item \textbf{Qualità del software - Slide del corso di Ingegneria del Software}\\
			\nURI{http://www.math.unipd.it/~tullio/IS-1/2017/Dispense/L13.pdf}
			\item \textbf{Qualità di processo - Slide del corso di Ingegneria del Software}\\
			\nURI{http://www.math.unipd.it/~tullio/IS-1/2017/Dispense/L15.pdf}
			\item \textbf{Processi SW - Slide del corso di Ingegneria del Software}\\
			\nURI{http://www.math.unipd.it/~tullio/IS-1/2017/Dispense/L03.pdf}
			\item \textbf{ISO/IEC 15504 - Pagina Wikipedia}\\
			\nURI{https://en.wikipedia.org/wiki/ISO/IEC_15504}
			\item \textbf{The Art of Software Testing, third edition, chapter 6: High-Order Testing} \\
			\nURI{http://shop.oreilly.com/product/9781118031964.do}
			\item \textbf{64 Test Metrics For Measuring Progress, Quality, Productivity \& More}\\
			\nURI{https://www.qasymphony.com/blog/64-test-metrics/}
		\end{itemize}
	


\end{document}