\usepackage{glossaries}
\makeglossaries
%INIZIO NOME
%\newglossaryentry{nome}{
%	name={nome},
%	description={descrizione}
%}
%FINE NOME

%ACRONIMI
\newacronym{BOT}{bot}{}
\newacronym{Bot}{bot}{}
\newacronym{Requisiti}{requisiti}{}
\newacronym{Requisito}{requisiti}{}
\newacronym{requisito}{requisiti}{}
\newacronym{stand-ups}{standups}{}
\newacronym{stand-up}{standups}{}
\newacronym{Stand-ups}{standups}{}
\newacronym{Stand-up}{standups}{}
\newacronym{Standups}{standups}{}
\newacronym{Branch}{branch}{}
\newacronym{Repository}{repository}{}
\newacronym{UniWeb}{Uniweb}{}
\newacronym{Anno accademico}{anno accademico}{}
\newacronym{Corso di laurea}{corso di laurea}{}
\newacronym{Esame universitario}{esame universitario}{}
\newacronym{Blockchain}{blockchain}{}
\newacronym{Block chain}{blockchain}{}
\newacronym{block chain}{blockchain}{}
\newacronym{Blocco}{blocco}{}
\newacronym{Blocchi}{blocco}{}
\newacronym{blocchi}{blocco}{}
\newacronym{Mining}{mining}{}
\newacronym{Mining farm}{mining farm}{}
\newacronym{Mining-farm}{mining farm}{}
\newacronym{mining-farm}{mining farm}{}
\newacronym{Miner}{miner}{}
\newacronym{Transazione}{transazione}{}
\newacronym{Transazioni}{transazione}{}
\newacronym{transazioni}{transazione}{}
\newacronym{Chiave privata}{chiave privata}{}
\newacronym{MetaMask}{metamask}{}
\newacronym{Metamask}{metamask}{}
\newacronym{Virtual DOM}{virtual DOM}{}
\newacronym{VirtualDOM}{virtual DOM}{}
\newacronym{virtualDOM}{virtual DOM}{}
\newacronym{Controllo di versione}{controllo di versione}{}
\newacronym{Ambiente di sviluppo integrato}{ambiente di sviluppo integrato}{}
\newacronym{Apache license}{Apache License}{}
\newacronym{Copyleft}{copyleft}{}
\newacronym{CopyLeft}{copyleft}{}
\newacronym{Copy left}{copyleft}{}
\newacronym{copy left}{copyleft}{}
\newacronym{Copy Left}{copyleft}{}
\newacronym{Milestone}{milestone}{}
\newacronym{Baseline}{baseline}{}
\newacronym{Stub}{stub}{}
\newacronym{Driver}{driver}{}
\newacronym{DApp}{DAPP}{}
\newacronym{Dapp}{DAPP}{}
\newacronym{Applicazioni Web}{applicazioni Web}{}
\newacronym{Applicazioni web}{applicazioni Web}{}
\newacronym{applicazioni web}{applicazioni Web}{}
\newacronym{webApp}{WebAPP}{}
\newacronym{WebApp}{WebAPP}{}
\newacronym{Browser}{browser}{}
\newacronym{Firefox}{firefox}{}
\newacronym{Chrome}{chrome}{}
\newacronym{Plug-in}{plug-in}{}
\newacronym{plugin}{plug-in}{}
\newacronym{plug in}{plug-in}{}
\newacronym{Plugin}{plug-in}{}
\newacronym{Plug in}{plug-in}{}
\newacronym{Capitolato d'appalto}{capitolato d'appalto}{}
\newacronym{Committente}{committente}{}
\newacronym{Airbnb}{airbnb}{}
\newacronym{AirBnb}{airbnb}{}
\newacronym{airBnb}{airbnb}{}
\newacronym{AirBNB}{airbnb}{}
\newacronym{airBNB}{airbnb}{}
\newacronym{Front-end}{front-end}{}
\newacronym{Front end}{front-end}{}
\newacronym{Frontend}{front-end}{}
\newacronym{front end}{front-end}{}
\newacronym{frontend}{front-end}{}
\newacronym{Back-end}{back-end}{}
\newacronym{Back end}{back-end}{}
\newacronym{Backend}{back-end}{}
\newacronym{back end}{back-end}{}
\newacronym{backend}{back-end}{}
\newacronym{Ciclo di Deming}{ciclo di Deming}{}
\newacronym{Validazione}{validazione}{}
\newacronym{Verifica}{verifica}{}
\newacronym{Commit}{commit}{}
\newacronym{Push}{push}{}
\newacronym{Pull}{pull}{}
\newacronym{Ticket}{ticket}{}
\newacronym{Chiave pubblica}{chiave pubblica}{}
\newacronym{Indirizzo}{indirizzo}{}
\newacronym{Smart contract}{smart contract}{}
\newacronym{VS Code}{VSCode}{}
\newacronym{VScode}{VSCode}{}
\newacronym{VS code}{VSCode}{}
\newacronym{White-box}{white-box}{}
\newacronym{Black-box}{black-box}{}

%INIZIO SLACK
\newglossaryentry{Slack}{
	name={Slack},
	description={Applicazione di messaggistica istantanea pensata per la collaborazione tra i membri di uno o più gruppi di lavoro. Offre la possibilità di creare canali privati tematizzati, gruppi privati e chat dirette. Tutto ciò che è presente in Slack può essere cercato, che sia un file, una conversazione o le persone stesse. Inoltre può essere esteso mediante l'uso di applicazioni di terze parti}
}
%FINE SLACK
%INIZIO BOT
\newglossaryentry{bot}{
	name={Bot},
	description={Programma che sostituisce un \citGloss{BOT}, \citGloss{bot} umano nello svolgere una attività usualmente ripetitiva, fornisce agli utenti servizi e informazioni in modo completamente automatizzato}
}
%FINE BOT
%INIZIO SKYPE
\newglossaryentry{Skype}{
	name={Skype},
	description={Software di messaggistica istantanea e VoIP, cioè permette di effettuare chiamate e videochiamate utilizzando la rete Internet}
}
%FINE SKYPE
%INIZIO ASANA
\newglossaryentry{Asana}{
	name={Asana},
	description={Applicazione web e mobile ideata per tracciare ed organizzare il lavoro di un gruppo. Permette la creazione di task e la loro assegnazione, la creazione di workspace che contengno i progetti e l'aggiunta e rimozione di commenti}
}
%FINE ASANA
%INIZIO INSTAGANTT
\newglossaryentry{Instagantt}{
	name={Instagantt},
	description={Applicazione integrabile con \citGloss{Asana} che permette di creare, modificare e visualizzare diagrammi di Gantt sincronizzati con le attività programmate in Asana}
}
%FINE INSTAGANTT
%INIZIO GANTT
\newglossaryentry{Gantt}{
	name={Diagrammi di Gantt},
	description={Strumento di supporto alla gestione dei progetti. \`{E} costituito partendo da un asse orizzontale a rappresentazione dell'arco temporale totale del progetto, suddiviso in fasi incrementali, e da un'asse verticale a rappresentazione delle mansioni o attività che costituiscono il progetto}
}
%FINE GANTT
%INIZIO REQUISITI
\newglossaryentry{requisiti}{
	name={Requisiti},
	description={Funzionalità che il prodotto software deve offrire}
}
%FINE REQUISITI
%INIZIO STANDUPS
\newglossaryentry{standups}{
	name={Stand-up meeting },
	description={Alcune metodologie di sviluppo del software utilizzano brevi riunioni tra i partecipanti al progetto per fare il punto della situazione, definiti stand-up meeting. A turno vengono esposti problemi e dubbi nati durante il lavoro di sviluppo e conseguentemente coordinati gli sforzi per risolvere i problemi individuati}
}
%FINE STANDUPS
%INIZIO GITHUB
\newglossaryentry{GitHub}{
	name={GitHub},
	description={Servizio di web hosting per lo sviluppo di progetti software che usa il sistema di \citGloss{controllo di versione} Git}
}
%FINE GITHUB

%INIZIO BRANCH
\newglossaryentry{branch}{
	name={Branch},
	description={Ramo di lavoro, favorisce uno sviluppo non lineare e incrementale, offrendo la possibilità di sviluppare degli incrementi sui branch che verranno poi uniti al branch principale, chiamato master}
}
%FINE BRANCH
%INIZIO REPOSITORY
\newglossaryentry{repository}{
	name={Repository},
	description={Ambiente di un sistema informativo in cui vengono gestiti i metadati attraverso tabelle relazionali. L'insieme di tabelle, regole e motori di calcolo tramite cui si gestiscono i metadati prende il nome di metabase}
}
%FINE REPOSITORY
%INIZIO UNIWEB
\newglossaryentry{Uniweb}{
	name={Uniweb},
	description={Portale dell'Università degli Studi di Padova disponibile all'indirizzo web \nURI{https://uniweb.unipd.it/Home.do} che permette a studenti di gestire la propria carriera universitaria, come visuailizzare il libretto e iscriversi agli appelli d'esame e ai docenti di predisporre le liste e di inserire i risultati degli appelli d'esame}
}
%FINE UNIWEB
%INIZIO ANNO
\newglossaryentry{anno accademico}{
	name={Anno accademico},
	description={Con il termine anno, o anno accademico, si definisce un periodo di tempo della durata approssimativamente di un anno nel quale l'università svolge le proprie attività}
}
%FINE ANNO
%INIZIO CORSO
\newglossaryentry{corso di laurea}{
	name={Corso di laurea},
	description={Con il termine corso, o corso di laurea, si definisce un percorso di studi che porta ad una determinata laurea}
}
%FINE CORSO
%INIZIO ESAME
\newglossaryentry{esame universitario}{
	name={Esame universitario},
	description={Con il termine esame, o esame universitario, si definisce un insieme di lezioni frontali ed eventualmente laboratori che veicolano delle informazioni agli studenti e si concludono con degli appelli}
}
%FINE ESAME
%INIZIO BLOCKCHAIN
\newglossaryentry{blockchain}{
	name={Blockchain},
	description={Registro pubblico e distribuito delle transizioni e dei contratti sulla quale si basa una cryptovaluta. Quando utilizzata nei documenti allegati si fa sempre riferimento alla blockchain alla base di \citGloss{Ethereum}}
}
%FINE BLOCKCHAIN
%INIZIO BLOCCO
\newglossaryentry{blocco}{
	name={Blocco},
	description={Parte di una \citGloss{blockchain} che contiene le transizioni. Per essere aggiunto alla blockchain deve avere l'approvazione da parte dei \citGloss{miner}}
}
%FINE BLOCCO
%INIZIO MINING
\newglossaryentry{mining}{
	name={Mining},
	description={Processo con il quale si va a confermare la validità dei blocchi e quindi ad inserirli nella \citGloss{blockchain}, in caso di esito negativo si procede all'eliminazione del blocco non ancora confermato}
}
%FINE MINING
%INIZIO MININGFARM
\newglossaryentry{mining farm}{
	name={Mining farm},
	description={Insieme, solitamente di grosse dimensione, di hardware dedicato ad attività di \citGloss{mining}}
}
%FINE MININGFARM
%INIZIO MINER
\newglossaryentry{miner}{
	name={Miner},
	description={Entità che svolge attività di \citGloss{mining}, può essere utilizzato per indicare l'hardware oppure la persona che possiede una \citGloss{mining farm}}
}
%FINE MINER
%INIZIO TRANSAZIONE
\newglossaryentry{transazione}{
	name={Transazione},
	description={Messaggio che richiede ai miner una operazione di alterazione della \citGloss{blockchain}, come ad esempio l'interazione con un contratto che ne modifica lo stato oppure un trasferimento di \citGloss{Ether}. Il costo dell'operazione viene pagato in \citGloss{Gas}}
}
%FINE TRANSAZIONE
%INIZIO GAS
\newglossaryentry{Gas}{
	name={Gas},
	description={Valuta per il pagamento delle transazioni, con un valore in \citGloss{Ether} variabile a seconda dello stato della \citGloss{blockchain}. Può indicare anche il costo per l'esecuzione di un contratto o di una parte di esso all'interno di una \citGloss{EVM}}
}
%FINE GAS
%INIZIO CHIAVEPRIVATA
\newglossaryentry{chiave privata}{
	name={Chiave privata},
	description={Anche abbreviata in chiave, è una stringa esadecimale di 65 caratteri utilizzata per firmare i messaggi, in modo da garantirne l'autenticità}
}
%FINE CHIAVEPRIVATA

%INIZIO METAMASK
\newglossaryentry{metamastk}{
	name={Metamask},
	description={Plug-in browser che consente un accesso alla blockchain Ethereum e che gestisce le chiavi}
}
%FINE METAMASK
%INIZIO VIRTUALDOM
\newglossaryentry{virtual DOM}{
	name={Virtual DOM},
	description={Astrazione del \citGloss{DOM} indipendente dal \citGloss{browser} che consente operazioni veloci senza necessità di manipolare realmente la pagina}
}
%FINE VIRTUALDOM
%INIZIO DOM
\newglossaryentry{DOM}{
	name={DOM},
	description={Document Object Model, astrazione strutturata degli elementi e del testo presenti in una pagina \citGloss{HTML}}
}
%FINE DOM
%INIZIO CONTROLLODIVERSIONE
\newglossaryentry{controllo di versione}{
	name={Controllo di versione},
	description={Software utilizzato per gestire lo sviluppo di programmi o documenti, mantenendo traccia di tutte le modifiche effetuate e permettendo così di analizzare la storia di un progetto}
}
%FINE CONTROLLODIVERSIONE
%INIZIO IDE
\newglossaryentry{ambiente di sviluppo integrato}{
	name={Ambiente di sviluppo integrato},
	description={detto anche IDE, Integrated development environment, è uno strumento software che offre un editor di testo accompagnato da tutti gli strumenti utili al programmatore per scrivere efficacemente codice}
}
%FINE IDE
%INIZIO APACHELICENSE
\newglossaryentry{Apache License}{
	name={Apache License},
	description={in italiano licenza Apache, è una licenza di software libero non copyleft scritta dall'Apache Software Foundation}
}
%FINE APACHELICENSE
%INIZIO COPYLEFT
\newglossaryentry{copyleft}{
	name={Colpyleft},
	description={Aggettivo indicante una licenza che non permette la ridistribuzione del software o di versioni modificate con vincoli aggiuntivi rispetto a quelli indicati dalla licenza stessa}
}
%FINE COPYLEFT
%INIZIO MILESTONE
\newglossaryentry{milestone}{
	name={Milestone},
	description={Termine inglese traducibile con pietra miliare, indica il raggiungimento di obbiettivi stabiliti con importanza strategica. Coincide con una o più \citGloss{baseline} e può derivare da un obbligo contrattuale o da un'opportunità decisa dal gruppo}
}
%FINE MILESTONE
%INIZIO BASELINE
\newglossaryentry{baseline}{
	name={Baseline},
	description={Base verifica e di appoggio dalla quale non si può retrocedere. \`{E} collocata in una \citGloss{repository} ed utilizzata per i successivi incrementi}
}
%FINE BASELINE
%INIZIO STUB
\newglossaryentry{stub}{
	name={Stub},
	description={Componente passivo fittizzio che serve a simulare una parte del sistema in modo tale da testare una porzione di codice singolarmente}
}
%FINE STUB
%INIZIO DRIVER
\newglossaryentry{driver}{
	name={Driver},
	description={Componente attivo fittizzio che serve a pilotare l'esecuzione di un codice in modo tale da permetterne il test e l'eventuale individuazione di errore}
}
%FINE DRIVER
%INIZIO DAPP
\newglossaryentry{DAPP}{
	name={DAPP},
	description={Decentralized application, indica una applicazione backend eseguita da una blockchain che supporta l'utilizzo di \citGloss{smart contract}}
}
%FINE DAPP
%INIZIO APPLICAZIONEWEB
\newglossaryentry{applicazioni Web}{
	name={Applicazioni Web},
	description={Applicazione usufruibile attraverso il browser, che non richiedono la necessità di installare programmi aggiuntivi sul proprio pc}
}
%FINE APPLICAZIONEWEB
%INIZIO WEBAPP
\newglossaryentry{webAPP}{
	name={WebAPP},
	description={Vedere \citGloss{applicazioni Web}}
}
%FINE WEBAPP
%INIZIO BROWSER
\newglossaryentry{browser}{
	name={Browser},
	description={Software che permette l'accesso a risorse web di varia natura, consente l'accesso alle WebAPP e usualmente supporta l'aggiunta di \citGloss{plug-in}}
}
%FINE BROWSER
%INIZIO FIREFOX
\newglossaryentry{firefox}{
	name={Firefox},
	description={Chiamato anche Mozilla Firefox, \citGloss{browser} open source sviluppato da Mozilla sulla base di Netscape, precedentemente conosciuto come Phoenix}
}
%FINE FIREFOX
%INIZIO CHROME
\newglossaryentry{chrome}{
	name={Chrome},
	description={Chiamato anche Google Chrome, \citGloss{browser} sviluppato da Google, disponibile in versione open source con caratteristiche simili sotto il nome di Chomium}
}
%FINE CHROME
%INIZIO PLUGIN
\newglossaryentry{plug-in}{
	name={Plug-in},
	description={Componente aggiuntivo che può essere installato in un software per ampliarne le funzionalità}
}
%FINE PLUGIN
%INIZIO CAPITOLATODAPPALTO
\newglossaryentry{capitolato d'appalto}{
	name={Capitolato d'appalto},
	description={\`{E} un documento tecnico per la definizione delle specifiche tecniche delle opere che andranno ad eseguirsi per effetto del relativo contratto, di cui è solitamente parte integrante}
}
%FINE CAPITOLATODAPPALTO
%INIZIO COMMITTENTE
\newglossaryentry{committente}{
	name={Committente},
	description={Figura che commissiona un lavoro, può essere una persona fisica o una figura giuridica. Ha potere decisionale e di spesa relativo alla gestione dell'appalto}
}
%FINE COMMITTENTE
%INIZIO CSS
\newglossaryentry{CSS}{
	name={CSS},
	description={Cascading Style Sheets, cioè fogli di stile a cascata, è un linguaggio utilizzato per definire l'aspetto grafico delle pagine web e dei documenti XML}
}
%FINE CSS
%INIZIO SCSS
\newglossaryentry{SCSS}{
	name={SCSS},
	description={Sassy CSS, estensione di \citGloss{CSS} che implementa nuove funzionalità. Viene compilato in CSS prima di essere utilizzato dai browser, questo garantisce la compatibilità con tutti i browser che supportano i fogli di stile}
}
%FINE SCSS
%INIZIO AIRBNB
\newglossaryentry{airbnb}{
	name={Airbnb},
	description={Compagnia operante a San Francisco fondata nel 2008, autrice di una guida di stile per la codifica in linguaggio JavaScript molto diffusa, denominata Airbnb \citGloss{JavaScript} Style Guide}
}
%FINE AIRBNB
%INIZIO ECMA
\newglossaryentry{ECMA}{
	name={ECMA},
	description={Associazione fondata nel 1961 che si occupa della standardizzazione nel settore informatico. Definisce il linguaggio ECMAScript, dal quale deriva il linguaggio \citGloss{JavaScript}}
}
%FINE ECMA
%INIZIO HTML
\newglossaryentry{HTML}{
	name={HTML},
	description={HyperText Markup Language, linguaggio di markup per la formattazione e la impaginazione di pagine web ipertestuali}
}
%FINE HTML
%INIZIO REACT
\newglossaryentry{React}{
	name={React},
	description={Chiamata anche React.js, è una libreria \citGloss{JavaScript} che favorisce la costruzione di interfacce grafice, dichiarativo e basato su l'utilizzo di componenti}
}
%FINE REACT
%INIZIO REDUX
\newglossaryentry{Redux}{
	name={Redux},
	description={Chiamata anche Redux.js, è una libreria JavaScript che favorisce la gestione dello stato delle applicazione web, spesso utilizzata assieme a \citGloss{React}}
}
%FINE REDUX
%INIZIO JAVASCRIPT
\newglossaryentry{JavaScript}{
	name={JavaScript},
	description={Linguaggio di scripting orientato agli oggetti e agli eventi, interpretabile dai browser. \`{E} standardizzato dalla \citGloss{ECMA} nel 1997 con il nome ECMAScript}
}
%FINE JAVASCRIPT
%INIZIO SOLIDITY
\newglossaryentry{Solidity}{
	name={Solidity},
	description={Linguaggio di scrittura per \citGloss{smart contract} su rete Ethereum ispirato da JavaScript e C++}
}
%FINE SOLIDITY
%INIZIO FRONTEND
\newglossaryentry{front-end}{
	name={Front-end},
	description={La parte del software visibile all'utente e con cui egli può interagire}
}
%FINE FRONTEND
%INIZIO BACKEND
\newglossaryentry{back-end}{
	name={Back-end},
	description={La parte del software che permette l'effettivo funzionamento delle iterazioni compiute dall'utente}
}
%FINE BACKEND
%INIZIO PDCA
\newglossaryentry{PDCA}{
	name={PDCA},
	description={Vedere \citGloss{ciclo di Deming}}
}
%FINE PDCA
%INIZIO CICLODIDEMING
\newglossaryentry{ciclo di Deming}{
	name={Ciclo di Deming},
	description={Modello per il miglioramento continuo della qualità in un'ottica a lungo raggio, anche detto PDCA (Plan, Do, Check, Act)}
}
%FINE CICLODIDEMING
%INIZIO TEXSTUDIO
\newglossaryentry{TexStudio}{
	name={TexStudio},
	description={Ambiente di scrittura integrato per favorire la creazione di documenti LaTeX, con numerose feature e personalizzazioni che rendono più efficace scrivere documenti}
}
%FINE TEXSTUDIO
%INIZIO WEBSTORM
\newglossaryentry{WebStorm}{
	name={WebStorm},
	description={Ambiente di sviluppo sviluppato da Jetbrains per lo sviluppo di applicazioni JavaScript, con numerose feature e personalizzazioni che rendono più efficace la scrittura del codice}
}
%FINE WEBSTORM
%INIZIO TRENDER
\newglossaryentry{Trender}{
	name={Trender},
	description={Sistema open source che permette il tracciamento di requisiti, casi d'uso, attori, verbali, package, classi e loro metodi.}
}
%FINE TRENDER
%INIZIO VALIDAZIONE
\newglossaryentry{validazione}{
	name={Validazione},
	description={Controllo effettuato per accertarsi che tutti i requisiti richiesti siano stati soddisfatti ed attestare la conformità con quanto richiesto dal committente}
}
%FINE VALIDAZIONE
%INIZIO VERIFICA
\newglossaryentry{verifica}{
	name={Verifica},
	description={Processo di supporto che accerta che le esecuzioni delle attività non abbiano introdotto errori}
}
%FINE VERIFICA
%INIZIO COMMIT
\newglossaryentry{commit}{
	name={Commit},
	description={Modifica o insieme di modifiche effettuate ai file contenuti nel \citGloss{repository}, richiede una breve descrizione su quanto aggiunto o modificato}
}
%FINE COMMIT
%INIZIO PUSH
\newglossaryentry{push}{
	name={Push},
	description={Azione che invia tutti i \citGloss{commit} presenti nel \citGloss{repository} locale a quello condiviso con gli altri membri del team di sviluppo}
}
%FINE PUSH
%INIZIO PULL
\newglossaryentry{pull}{
	name={Pull},
	description={Azione che scarica dal \citGloss{repository} remoto tutti i \citGloss{commit} effettuati dagli altri membri del team di sviluppo e li applica ai propri file}
}
%FINE PULL
%INIZIO PDF
\newglossaryentry{PDF}{
	name={PDF},
	description={Portable Document Format, formato sviluppato da Adobe per la rappresentazione di documenti in modo indipendente al dispositivo utilizzato}
}
%FINE PDF
%INIZIO LATEX
\newglossaryentry{LaTeX}{
	name={LaTex},
	description={Linguaggio per la stesura di documenti matematici e scientifici, permette il versionamento essendo puramente testuale}
}
%FINE LATEX
%INIZIO TICKET
\newglossaryentry{ticket}{
	name={Ticket},
	description={Notifica di segnalazione di una attività da svolgere per l'avanzamento del progetto oppure una segnalazione di errore all'interno del software da risolvere}
}
%FINE TICKET
%INIZIO CHIAVEPUBBLICA
\newglossaryentry{chiave pubblica}{
	name={Chiave pubblica},
	description={Anche abbreviata in indirizzo utente o account, è una stringa esadecimale di 41 carattere utilizzata per l'identificazione. Viene utilizzata anche per gli trasferimenti di \citGloss{Ether}}
}
%FINE CHIAVEPUBBLICA
%INIZIO INDIRIZZO
\newglossaryentry{indirizzo}{
	name={Indirizzo},
	description={\`{E} una stringa esadecimale di 41 carattere utilizzata per l'identificazione di contratti o di account}
}
%FINE INDIRIZZO
%INIZIO SMARTCONTRACT
\newglossaryentry{smart contract}{
	name={Smart contract},
	description={Protocolli per l'esecuzione di un contratto, il quale viene gestito in modo automatico senza bisogno dell'intervento umano, rimuovendo così la possibilità di violarlo o di manometterlo}
}
%FINE SMARTCONTRACT
%INIZIO ETHEREUM
\newglossaryentry{Ethereum}{
	name={Ethereum},
	description={Piattaforma decentralizzata per la gestione di \citGloss{smart contract} e di cryptovaluta \citGloss{Ether}}
}
%FINE ETHEREUM
%INIZIO ETHER
\newglossaryentry{Ether}{
	name={Ether},
	description={Moneta virtuale o cryptocurrency utilizzata nella rete Ethereum, convertibile in \citGloss{Gas}}
}
%FINE ETHER
%INIZIO EVM
\newglossaryentry{EVM}{
	name={Ethereum Virtual Machine},
	description={Macchina virtuale sulla quale vengono eseguiti gli \citGloss{smart contract}. Attualmente la loro esecuzione è assegnata prevalentemente a processori grafici}
}
%FINE EVM
%INIZIO TRAVIS
\newglossaryentry{Travis}{
	name={Travis},
	description={Servizio di integrazione continua distribuito e ospitato utilizzato per costruire e testare progetti software ospitati su \citGloss{GitHub}}
}
%FINE TRAVIS
%INIZIO LINTING
\newglossaryentry{Linting}{
	name={Linting},
	description={Analisi della qualità del codice statica individuando possibili errori prima della compilazione/esecuzione di linguaggi di programmazione.}
}
%FINE LINTING
%INIZIO LINTING
\newglossaryentry{VSCode}{
	name={VSCode},
	description={Chiamato anche Visual Studio Code, è un editor open suorce multipiattaforma leggero e supportato da molti plugin di terze parti che ne ampliano le funzionalità}
}
%FINE LINTING
%INIZIO QA
\newglossaryentry{QA}{
	name={QA},
	description={Acronimo per Quality Assurance, indica il team di verificatori che effettua l'analisi dinamica del software attraverso test ad alto livello.}
}
%FINE QA

%INIZIO WHITE-BOX
\newglossaryentry{white-box}{
	name={White-box},
	description={Tipologia di test atta a verificare la struttura del codice, controllando ogni azione logica che il codice può o deve compiere.}
}
%FINE WHITE-BOX

%INIZIO BLACK-BOX
\newglossaryentry{black-box}{
	name={Black-box},
	description={Tipologia di test atta a verificare il comportamento del codice fornendo dei parametri di ingresso e controllando che quelli di uscita corrispondano a quelli previsti, ignorando la logica all'interno.}
}
%FINE BLACK-BOX

%INIZIO TDD
\newglossaryentry{TDD}{
	name={TDD},
	description={Acronimo per Test Driven Development, è una filosifia di sviluppo che incoraggia la scrittura dei test per verificare il codice scritto. Nello specifico i test vengono scritti a priore mentre il codice come conseguenza dei test scritti.}
}
%FINE TDD

%INIZIO BDD
\newglossaryentry{BDD}{
	name={BDD},
	description={Variante del \citGloss{TDD} per i test ad alto livello, la B si riferisce a behaviour, cioè comportamento, indicando un focus, a livello di analisi, sul comportamento del codice. Vengono espressi in un linguaggio quasi informale.}
}
%FINE BDD

%INIZIO ATDD
\newglossaryentry{ATDD}{
	name={ATDD},
	description={Variante del TDD per i test di validazione.}
}
%FINE ATDD
%INIZIO PROOFOFCONCEPT
\newglossaryentry{Proof of Concept}{
	name={Proof of Concept},
	description={Un'incompleta realizzazione o abbozzo di un certo progetto o metodo, con lo scopo di dimostrarne la fattibilità o la fondatezza di alcuni principi o concetti costituenti}
}
%FINE PROOFOFCONCEPT
