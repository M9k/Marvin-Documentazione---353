\documentclass[PianoDiQualifica.tex]{subfiles}
\begin{document}

\chapter{ISO/IEC 15504}
Lo standard ISO/IEC 15504 contiene un modello di riferimento che definisce: 
\begin{itemize}
	\item Process dimension;
	\item Capability dimension.
\end{itemize}
La dimensione di processo divide i processi in cinque categorie:
\begin{itemize}
	\item Customer-supplier;
	\item Engineering;
	\item Supporting;
	\item Management;
	\item Organization.
\end{itemize}
Per ogni processo, lo standard ISO/IEC 15504 definisce dei livelli di capacità:
\begin{itemize}
	\item \textbf{Livello 0:}
	\begin{itemize}
		\item \textbf{Incomplete process}: il processo non è implementato o non raggiunge lo scopo stabilito;
	\end{itemize}
	\item \textbf{Livello 1:}
	\begin{itemize}
		\item \textbf{Performed process}: il processo è implementato e raggiunge lo scopo stabilito;
	\end{itemize}
	\item \textbf{Livello 2:}
	\begin{itemize}
		\item \textbf{Managed process}: il processo è gestito e i prodotti sono stabiliti, controllati e mantenuti;
	\end{itemize}
	\item \textbf{Livello 3:}
	\begin{itemize}
		\item \textbf{Established process}: un processo stabilito si basa su un processo standard;
	\end{itemize}
	\item \textbf{Livello 4:}
	\begin{itemize}
		\item \textbf{Predictable process}: il processo è adottato sistematicamente, entro limiti definiti;
	\end{itemize}
	\item \textbf{Livello 5:}
	\begin{itemize}
		\item \textbf{Optimizing process}: il processo è continuamente migliorato.
	\end{itemize}
\end{itemize}
La capacità dei processi viene misurata attraverso degli attributi di processo:
\begin{itemize}
	\item \textbf{Livello 1:}
	\begin{itemize}
		\item \textbf{Process performance:} capacità di un processo di raggiungere gli obiettivi trasformando input identificabili in output identificabili;
	\end{itemize}
	\item \textbf{Livello 2:}
	\begin{itemize}
		\item \textbf{Performance management:} capacità del processo di elaborare un prodotto coerente con gli obiettivi fissati;
		\item \textbf{Work product management:} capacità del processo di elaborare un prodotto documentato, controllato e verificato;
	\end{itemize}
	\item \textbf{Livello 3:}
	\begin{itemize}
		\item \textbf{Process definition:} l'esecuzione del processo si basa su standard di processo per raggiungere i propri obiettivi;
		\item \textbf{Process deployment:} capacità del processo di attingere a risorse tecniche e umane appropriate per essere attuato efficacemente;
	\end{itemize}
	\item \textbf{Livello 4:}
	\begin{itemize}
		\item \textbf{Process measurement:} gli obiettivi e le misure di prodotto e di processo vengono usati per garantire il raggiungimento dei traguardi definiti in supporto ai target aziendali;
		\item \textbf{Process control:} il processo viene controllato tramite misure di prodotto e processo per effettuare correzioni migliorative al processo stesso;
	\end{itemize}
	\item \textbf{Livello 5:}
	\begin{itemize}
		\item \textbf{Process innovation:} i cambiamenti strutturali, di gestione e di esecuzione vengono gestiti in modo controllato per raggiungere i risultati fissati;
		\item \textbf{Process optimization:} le modifiche al processo sono identificate e implementate per garantire il miglioramento continuo nella realizzazione degli obiettivi di business dell'organizzazione. 
	\end{itemize}
\end{itemize}
Ogni attributo consiste di una o più pratiche generiche che sono ulteriormente elaborate in indicatori pratici per aiutare la valutazione delle performance, sotto forma di indici N-P-L-F:
\begin{itemize}
	\item Non soddisfatto (0 - 15\%);
	\item Parzialmente soddisfatto ($ > $15\% - 50\%);
	\item Largamente soddisfatto ($ > $50\% - 85\%);
	\item Totalmente soddisfatto ($ > $85\% - 100\%).
\end{itemize}
\end{document}