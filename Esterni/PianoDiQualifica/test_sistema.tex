\documentclass[PianoDiQualifica.tex]{subfiles}

\begin{document}

\chapter{Test di sistema e validazione}
A causa della componente Metamask, non accessibile, non è possibile effettuare test di sistema.
Per i test di validazione, ogni scenario di setup presuppone che il server node di test sia avviato senza errori sulla porta 8080 e che il tester possegga diversi account metamask con crediti disponibili sulla blockchain fittizia.
Il tester inoltre possiede l’account che ha effettuato il deploy dei contratti sulla rete, avendo la possibilità di effettuare il login come utente Universitario.
La specifica degli Attori (utenti) e dei casi d’uso è descritta nel documento Analisi Dei Requisiti V4.0.0.

\rowcolors{2}{CRighePari}{CRigheDispari}
\begin{longtable}[H]{P{2cm}P{8.5cm}P{3cm}}
	\rowcolor{CHeader} 
	\color{CHeaderText}\textbf{Codice} & 
	\color{CHeaderText}\textbf{Descrizione} & 
	\color{CHeaderText}\textbf{Stato}\\
	\endhead
	TVnumerorequisito & Ciao & Superato \\ 
	\hiderowcolors
	\caption{Test di validazione}
\end{longtable}

\rowcolors{2}{CRighePari}{CRigheDispari}
\begin{longtable}[H]{P{6.5cm}P{7cm}}
	\rowcolor{CHeader} 
	\color{CHeaderText}\textbf{Codice} & 
	\color{CHeaderText}\textbf{Requisito} \\
	\endhead
	TVnumerorequisito & Ciao \\ 
	\hiderowcolors
	\caption{Tracciamento test di validazione - requisito}
\end{longtable}

\end{document}